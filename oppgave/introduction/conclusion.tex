\section{Summary and conclusions}


In this thesis, we have investigated the high isospin ground state of QCD---the pion conde condensed phase---and modeled how it might form astronomical objects, namely pion stars.
Pion stars were first proposed by \citeauthor{carignanoScrutinizingPionCondensed2017} in \autocite{carignanoScrutinizingPionCondensed2017} and investigated by \citeauthor{brandtNewClassCompact2018} in \autocite{brandtNewClassCompact2018}, in which they obtained the equations of states of the systems discussed in this text, using QCD lattice methods.
Pion stars have, furthermore, been explored using the linear sigma model and the $1/N$-expansion by \citeauthor{andersenBoseEinsteinCondensationPion2018}~\autocite{andersenBoseEinsteinCondensationPion2018}.

We have extended the investigation of pion stars using three-flavor chiral perturbation theory.
By exploiting the spontaneous breaking of the approximate $\Lie{SU}{3}_L\times\Lie{SU}{3}_R$ symmetry of the three lightest quarks, the up, down and strange quarks, one can construct an effective field theory of the resulting pseudo-Goldstone bosons.
We investigated the phase diagram in the $\mu_S-\mu_I$-plane at $T = 0$, where condensates of pions, charged kaons and neutral kaons are separated by first-order phase transitions, and how it is affected by electromagnetic interactions.
We calculated the equation of state of the pion-condensed phase, to leading order, next-to-leading order, and including electromagnetic interactions.
To create a more realistic material for astronomical objects, we further calculated the equation of state of the pion condensed phase including charged leptons and neutrinos to create an electrically neutral condensate.

These equations of state were used, in conjunction with the TOV equation of general-relativistic, gravitationally bound matter in hydrostatic equilibrium, to calculate the energy and pressure distribution of pion stars.
We showed analytically that the limiting radius of the pion star of a pure pion condensate with only strong interactions has a limiting radius of $R = 85.97\,\text{km}$, which becomes $R = 80.40\,\text{km}$ when including electromagnetic interactions.
We further showed that the equation of state of the $\pi\ell\nu_\ell$-system is closely approximated by that of ultrarelativistic fermions or electromagnetic radiation, $u = 3p$, as the neutrino contribution to the equation of state dominates.
The resulting mass-radius relation is therefore mostly defined by the pressure $p_\text{min}$ at the surface of the star, where the pion condensate vanishes and the neutrino atmosphere begins.

Large parts of the low-temperature QCD phase space are inaccessible for investigation from first principles.
The zero baryon number sector, however, can be simulated using QCD lattice methods.
This allows us to cross-check our effective theories, such as chiral perturbation theory.
Our results are in good agreement with further earlier the earlier work lattice QCD work, which gives more confidence to our theoretical understanding of pion stars.


\section{Outlook and further research}

By themselves, pions are unstable particles.
The neutral pion decays mainly via $\pio\rightarrow \gamma\gamma$, and has a mean lifetime of $\tau = 8.45\times 10^{-17}\,\text{s}$~\autocite{particledatagroupReviewParticlePhysics2020}---increadibly short.
The decay of the charged pions is suppressed due to the conservation of electric charge, and their main decay process is $\pipm\rightarrow \mu \nu_\mu$ as discussed in \autoref{section: charge neturality}.
Still, their mean lifetime is only $\tau = 2.60\times 10^{-8}\,\text{s}$~\autocite{particledatagroupReviewParticlePhysics2020}.
These timescales are much smaller than that of astronomical objects, which immediately casts some doubt on the research project of pion stars.
However, these results are for the vacuum state.
In the $\pi\ell\nu_\ell$-case the electrons and muons fill the Fermi-sphere up to their Fermi momentum, as we are considering $T = 0$, and the decay is thus Pauli-blocked.
Even if the final decay-states are available, in the pion condensate they are suppressed by $\mu_I^{-3}$~\autocite{brandtNewClassCompact2018}.

\todo[inline]{Higgs-fasen}
\todo[inline]{Hva med T!=0?}


It has been shown that high neutrino densities can cause the condensation of pions~\autocite{abukiPionCondensationDense2009}.

Early universe lepton asymmetry and pion condensation, gravitational signatures\autocite{vovchenkoPionCondensationEarly2021,wygasCosmicQCDEpoch2018}.



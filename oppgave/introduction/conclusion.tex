\section{Summary and conclusions}

Pion stars were first proposed by \citeauthor{carignanoScrutinizingPionCondensed2017} in \autocite{carignanoScrutinizingPionCondensed2017} and investigated by \citeauthor{brandtNewClassCompact2018} in \autocite{brandtNewClassCompact2018}, in which they obtained the equations of states of the systems discussed in this text, using QCD lattice methods.
Pion stars have, furthermore, been explored using the linear sigma model and the $1/N$-expansion~\autocite{andersenBoseEinsteinCondensationPion2018}.



\section{Outlook and further research}

By themselves, pions are unstable particles.
The neutral pion decays mainly via $\pio\rightarrow \gamma\gamma$, and has a mean lifetime of $\tau = 8.45\times 10^{-17}\,\text{s}$~\autocite{particledatagroupReviewParticlePhysics2020}---increadibly short.
The decay of the charged pions is suppressed due to the conservation of electric charge, and their main decay process is $\pipm\rightarrow \mu \nu_\mu$ as discussed in \autoref{section: charge neturality}.
Still, their mean lifetime is only $\tau = 2.60\times 10^{-8}\,\text{s}$~\autocite{particledatagroupReviewParticlePhysics2020}.
These timescales are much smaller than that of astronomical objects, which immediately casts some doubt on the research project of pion stars.
However, these results are for the vacuum state.
In the $\pi\ell\nu_\ell$-case the electrons and muons fill the Fermi-sphere up to their Fermi momentum, as we are considering $T = 0$, and the decay is thus Pauli-blocked.
Even if the final decay-states are available, in the pion condensate they are suppressed by $\mu_I^{-3}$~\autocite{brandtNewClassCompact2018}.

\todo[inline]{Higgs-fasen}
\todo[inline]{Hva med T!=0?}


It has been shown that high neutrino densities can cause the condensation of pions~\autocite{abukiPionCondensationDense2009}.

Early universe lepton asymmetry and pion condensation, gravitational signatures\autocite{vovchenkoPionCondensationEarly2021,wygasCosmicQCDEpoch2018}.



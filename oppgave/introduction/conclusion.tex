\section{Summary}


In this thesis, we have investigated the ground state of QCD for isospin chemical potential greater than the pion mass---the pion condensed phase.
The results have been applied to model how a pion condensate might form compact astronomical objects, pion stars.
These objects were first proposed by \citeauthor{carignanoScrutinizingPionCondensed2017} in \autocite{carignanoScrutinizingPionCondensed2017} and investigated by \citeauthor{brandtNewClassCompact2018} in \autocite{brandtNewClassCompact2018} using QCD lattice methods.


\subsection{Pion condensate}

We investigated the meson sector of zero-temperature quantum chromodynamics using three-flavor chiral perturbation theory.
By exploiting the spontaneous breaking of the approximate $\Lie{SU}{3}_L\times\Lie{SU}{3}_R$ symmetry of the three lightest quarks, the up, down and strange quarks, one can construct an effective field theory of the resulting pseudo-Goldstone bosons.
With this effective theory, we calculated the phase diagram in the $\mu_S-\mu_I$-plane at $T = 0$.
At $\mu_I = m_\pi$, a second-order phase transition from the vacuum phase to the pion-condensed phase occurs.
Similar second-order transitions occur between the vacuum phase and the charged kaon- and neutral kaon-condensed phase.
These condensed phases are separated by first-order phase transitions.
We calculated the effects of electromagnetic interactions on this phase diagram.

Focusing on the pion-condensed phase, we found the equation of state to leading order, next-to-leading order, and including the effects of electromagnetism.
Furthermore, we found the equations of state of systems of a pion condensate together with charged leptons.
This ensures charge-neutrality, which is more realistic as a basis for astronomical objects.
Lastly, we included the effects of weak interactions and calculated the equation of state of the $\pi\ell\nu_\ell$-system in which the pion condensate, charged leptons, and neutrinos are in chemical equilibrium at next-to-leading order.


\subsection{Pion stars}

The equations of state for the various configurations were used, in conjunction with the TOV equation, to model pion stars.
We calculated their energy and pressure profiles, and thus obtained the mass-radius relations.
The size and mass of pion stars vary greatly based on their composition, and so did how the radius changes as the mass increases.
We showed analytically that the limiting radius of a pion star composed of a pure pion condensate with only strong interactions is $R = 85.97\,\text{km}$, and becomes $R = 80.40\,\text{km}$ when including electromagnetic interactions.
We further showed that the equation of state of the $\pi\ell\nu_\ell$-system is closely approximated by that of ultrarelativistic fermions or electromagnetic radiation for $T>0$, $u = 3p$, as the neutrino contribution to the equation of state dominates.
The resulting mass-radius relation is therefore mostly defined by the pressure $p_\text{min}$ at the surface of the star, where the pion condensate vanishes and the neutrino atmosphere begins.


\subsection{Comparison wite data}

Large parts of the low-temperature QCD phase diagram are inaccessible for investigation from the first principles.
Asymptotic freedom allows for some results at asymptotically high temperatures or densities, such as the color superconducting phase at high densities.
However, due to the running of the strong coupling constant, perturbation theory breaks down below around $1 \text{GeV}$.
Currently, the only method available for this regime is QCD lattice methods, numerical schemes which calculate the path integral on a discrete lattice using the Metropolis-Hastings algorithm.
At finite baryon densities, this method fails due to the sign problem.
The zero baryon number sector, however, can be and has been investigated with QCD lattice methods.
This allows us to test our effective theories, such as chiral perturbation theory.
We compare our results with those of \citeauthor{brandtNewClassCompact2018}, in which they calculated the equation of state of the pion-condensed phase, as well as the resulting pion star mass-radius relations~\autocite{brandtNewClassCompact2018}.

When comparing with the results for the pion star mass-radius relations, we have to take into account the fact that the pion mass is slightly different in the lattice QCD system.
Using the lattice values, we find very good agreement in all cases.
Here too, the agreement improves using the next-to-leading order results, in comparison with the leading order results.
The good agreement between \chpt\, and QCD lattice methods increases our confidence in both methods, and as a consequence the confidence in our understanding of pion stars.



\section{Outlook and further research}

We have made several approximations in the calculation of the equation of state in this thesis.
Improving on these calculations will yield more accurate results.
Although we included the effects of electromagnetism in the pure pion condensate, we have not carried these effects over to the $\pi\ell\nu_\ell$-system.
The next-to-leading order Lagrangian including electromagnetic effects has been derived~\autocite{urechVirtualPhotonsChiral1995}, which could in principle be used to calculate the NLO equation of state, although this might prove to be prohibitively difficult.
The phase diagram could also be improved by a next-to-leading order treatment.
A fully consistent power counting entails, in the case of the leading order pion condensate, calculating electromagnetic effects in the lepton sector to and including $\Oh(e^2)$.
In the case of the next-to-leading order pion condensate, one should include one-loop effects.
A further improvement to the accuracy of the pion condensate calculation would be to take into account the effects due to $\Delta m \neq 0$.
This leads to different masses for the charged and neutral kaons, as well as a mixing of the neutral pion and the $\eta$-particle.


By themselves, pions are unstable particles.
The neutral pion decays mainly via $\pio\rightarrow \gamma\gamma$, and has a mean lifetime of $\tau = 8.45\times 10^{-17}\,\text{s}$~\autocite{zylaReviewParticlePhysics2020}---incredibly short.
The decay of the charged pions is suppressed due to the conservation of electric charge, and their main decay process is $\pipm\rightarrow \mu \nu_\mu$ as discussed in
%
~\autoref{section: charge neturality}.
%
Still, their mean lifetime is only $\tau = 2.60\times 10^{-8}\,\text{s}$
%
~\autocite{zylaReviewParticlePhysics2020}.
%
These timescales are much smaller than that of astronomical objects, which immediately casts some doubt on the research project of pion stars.
However, these results are for the vacuum state.
In the $\pi\ell\nu_\ell$-case the electrons and muons fill the Fermi-sphere up to their Fermi momentum, as we are considering $T = 0$, and the decay is thus Pauli-blocked.
Even if the final decay-states are available, in the pion condensate they are suppressed by $\mu_I^{-3}$~\autocite{brandtNewClassCompact2018}. 
For a more complete understanding of pion stars, research into time evolution is required.
An investigation of the decay of the pion condensate is then necessary.
The most realistic case, the $\pi\ell\nu_\ell$-system, has a neutrino atmosphere.
In the creation of neutron stars, the escape of neutrinos plays a vital role in cooling the system down~\autocite{glendenningCompactStarsNuclear2012}.
We similarly expect the neutrino atmosphere of the pion star to evaporate.
This will change the conditions at the surface of the star, such as the surface pressure $p_\text{min}$.
We found that this was crucial for the mass-radius relation, so a more realistic model must take this into account.
Here, finite temperature effects might play a vital role.


Pion stars, like neutron stars, are compact objects.
This allows us to approximate them as having zero temperature, as $T\ll m_\pi$ while $\mu \approx m_\pi$.
A more realistic model would take into account thermal effects.
This is especially important in the case of the $\pi\ell\nu_\ell$-system.
Due to the dominating contribution to the equation of state from the neutrinos, the isospin density remains small even at the core of the most massive star.
The isospin chemical potential is just $0.02\,m_\pi$ above the pion condensation transition.
Calculations of thermodynamic properties of the pion condensate at non-zero temperatures using chiral perturbation theory have shown good agreement with lattice results~\autocite{adhikariCondensatesPressureTwoflavor2021}.
This could allow for the modeling of pion stars at non-zero temperature, which is a more realistic model.


For now, the pion star has been a purely theoretical research project.
For any such object to form, a strong isospin asymmetry must be present.
Such circumstances, as we saw in the section on electric neutrality, can arise due to lepton asymmetry through the weak interaction.
It has been shown that high neutrino densities can cause the condensation of pions~\autocite{abukiPionCondensationDense2009}.
The lepton asymmetry of the universe, the abundance of leptons over anti-leptons, is not well known.
It has been shown that, within current experimental limits, the trajectory of the early universe through the QCD phase diagram as it cooled and expanded might have entered the pion-condensed phase.
If that is the case, pion stars might have formed and left observable traces in cosmological data such as the cosmological background gravitational radiation~\autocite{vovchenkoPionCondensationEarly2021,wygasCosmicQCDEpoch2018}.


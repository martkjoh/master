\vspace*{.5cm}
Recently, a new form of compact stars called pion stars has been proposed.
These are massive, gravitationally bound, astrophysical objects composed of a pion condensate.
In this thesis, we employ three-flavor chiral perturbation theory to calculate the thermodynamic properties of the pion condensate which we use to model pion stars.
We survey the theoretical foundations of chiral perturbation and the construction of the effective Lagrangian of the pseudo-scalar mesons, which includes the pions.
With this Lagrangian, we investigate its thermodynamics at zero temperature and non-zero isospin and strangeness chemical potential, $\mu_I$ and $\mu_S$.
We map out the phase diagram in the $\mu_I-\mu_S$-plane, where the vacuum phase transitions into condensed phases.
We furthermore calculate the equation of state of the pion condensate, parameterized by $\mu_I$ for a pure condensate to leading and next-to-leading order, and including the effects of electromagnetic interactions.
To model more realistic astrophysical objects, we additionally calculate the equation of state of a $\pi\ell\nu_\ell$-system---a composite system including charged leptons and neutrinos.


The equations of state we obtained are then used as inputs to the Tolman-Oppenheimer-Volkoff equation, which we use to calculate the pressure and energy distribution of pion stars, as well as the stellar mass and radius as a function of the central pressure---the mass-radius relation.
We compare our results for the mass-radius relation to earlier results from lattice QCD calculations and find good agreement.
Analytical results allow for discussion of their physical causes and derived various limits.
We give a closed-form expression for the limiting radius of a pion star composed of a pure pion condensate and investigate how the mass-radius relation of the $\pi\ell\nu_\ell$-system is determined by its surface pressure.



Muligheten for en ny type kompakte stjerner, navngitt pionstjerner, er nylig foreslått på teoretisk grunnlag.
Pionstjerner er massive astrofysiske objekter bestående av et pionkondensat og holdt sammen av tyngdekreften.
I denne avhandlingen bruker vi kiral perturbasjonsteori med tre kvarktyper for å regne ut de termodynamiske egenenskapene til pionkondensatet, som så blir brukt til å modelere pionstjerner.
Vi gjennomgår det teoretiske fundamentet til kiral perturbasjonsteori og konstruksjonen av en effektiv Lagrange-tettheten bestående av pseudoskalare mesoner, blandt annet pionene.
Vi undersøker de termodynamiske egenskapene til denne modelen ved temperatur lik null, og med kjemisk potensial for isospin og strangeness(særhet?), $\mu_I$ og $\mu_S$, forskjellige fra null.
Vi kartlegger fasediagrammet i $\mu_I-\mu_S$-flaten, hvor det skjer en overgang fra vakuumfasen til kondenserte faser.
Videre beregner vi pionkondensatets tilstandsligning til første og andre orden, samt inkludert effektene av elektromagnetiske vekselvirkninger.
For å modellere mer realistiske astrofysiske objekt regner vi ut tilstandsligningen til et pionkondensate som inkluderer ladde leptoner samt neutrinoer, et $\pi\ell\nu_\ell$-system.
Disse resultatene brukes sammen med Tolman-Oppenheimer-Volkoffligningen for å regne ut trykk- og energidistribusjonen i pionstjerner, samt stjernenes masse og radius som funksjon av trykket i sentrum, kalt masseradiusrelasjonen.
Vi sammenligner våre resultater for masseradiusrelasjonen med gitterberegninger av kvantekromodynamikk og finner god overenstemmelse.
Den analytiske naturen til resultatene våre gjør det mulig å finne grenseverdier og gi utforsk fysiske årsaker.
Vi finner et utrykk på lukket form for grennseverdien til radiusen for en pionstjerne bestående av et rent pionkondensat, og utforsker hvordan masseradiusrelasjonen til $\pi\ell\nu_\ell$-systemet avhenger av overflatetrykkjet til stjernen.

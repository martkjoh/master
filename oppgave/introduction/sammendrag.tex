Muligheten for en ny type kompakte stjerner navngitt pionstjerner har nylig blitt foreslått.
Pionstjerner er massive astrofysiske objekter samensatt av et pionkondensat og holdt sammen av tyngdekrefter.
I denne avhandlingen bruker vi kiral perturbasjonsteori med tre kvarktyper for å regne ut de termodynamiske egenenskapene til pionkondensate, som blir brukt til å modelere pionstjerner.
Vi gjennomgår de teoretiske fundamentene til kiral perturbasjonsteori og konstruksjonen av Lagrange-tettheten bestående av pseudoskalare mesoner, blandt annet pionene.
Vi undersøker de termodynamiske egenskapene til denne modelen ved null grader og når det kjemisk potensial for isospin og strangeness(særhet?), $\mu_I$ og $\mu_S$, er forskjellig fra null.
Vi kartlegger fasediagrammet på $\mu_I-\mu_S$-flaten, hvor det skjer en overgang fra vakuumfasen til kondenserte faser.
Videre beregner vi  pionkondensatets tilstandsligning til første og andre orden, samt med effektene av elektromagnetiske vekselvirkninger.
For å modellere et mer realistisk astrofysisk objekt regner vi ut tilstandsligningen til et pionkondensate som inkluderer ladde leptoner samt neutrinoer, et $\pi\ell\nu_\ell$-system.
Disse resultatene brukes sammen med Tolman-Oppenheimer-Volkoffligningen for å regne ut trykk- og energidistribusjonen i pionstjerner, samt stjernenes masse og radius som funksjon av trykket i sentrum.
Vi sammenligner våre resultater for masseradiusrelasjonen med gitterberegninger av kvantekromodynamikk, og finner god overenstemmelse.
Den analytiske formen til resultatene våre gjør det mulig å diskturere deres grenser og fysiske årsaker.
Vi finner et lukket utrykk for grennseverdien til radiusen til en pionstjerne bestående av et rent pionkondensat, og utforsker hvordan masseradiusrelasjonen til $\pi\ell\nu_\ell$-systemet avhenger av overflatetrykkjet til stjernen.


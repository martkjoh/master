\vspace*{1.5cm}
Muligheten for en ny type kompakte stjerner, navngitt pionstjerner, er nylig foreslått på teoretisk grunnlag.
Pionstjerner er massive astrofysiske objekter bestående av et pionkondensat og holdt sammen av tyngdekraften.
I denne masteroppgaven bruker vi kiral perturbasjonsteori med tre kvarktyper for å regne ut de termodynamiske egenskapene til pionkondensatet, som så blir brukt til å modellere pionstjerner.
Vi gjennomgår det teoretiske fundamentet til kiral perturbasjonsteori og konstruksjonen av en effektiv Lagrange-tetthet bestående av pseudoskalare mesoner, blant annet pionene.
Vi undersøker de termodynamiske egenskapene til denne modellen ved null temperatur, og med kjemisk potensial for isospin og strangeness, $\mu_I$ og $\mu_S$, forskjellige fra null.
Vi kartlegger fasediagrammet i $\mu_I-\mu_S$-planet, hvor det skjer en overgang fra vakuumfasen til kondenserte faser.
Videre beregner vi pionkondensatets tilstandsligning til første og andre orden, samt inkludert effekten av elektromagnetiske vekselvirkninger.
For å modellere mer realistiske astrofysiske objekter regner vi ut tilstandsligningen til et pionkondensat som inkluderer ladde leptoner og neutrinoer, et $\pi\ell\nu_\ell$-system.

Tilstandsligningene vi har kommet frem til brukes sammen med Tolman-Oppenheimer-Volkoffligningen for å regne ut trykk- og energiprofilen i pionstjerner, samt stjernenes masse og radius som funksjon av trykket i sentrum, kalt masseradiusrelasjonen.
Vi sammenligner våre resultater for masseradiusrelasjonen med gitterberegninger av kvantekromodynamikk og finner god overenstemmelse.
Den analytiske naturen til resultatene våre gjør det mulig å finne grenseverdier og gi fysiske tolkninger.
Vi finner et utrykk på lukket form for grenseverdien til radiusen for en pionstjerne bestående av et rent pionkondensat, og utforsker hvordan masseradiusrelasjonen til $\pi\ell\nu_\ell$-systemet avhenger av overflatetrykket til stjernen.

\section{Units}

In this thesis, we employ \emph{natural units}, defined by
%
\begin{equation}
    \hbar = c = k_B = 1,
\end{equation}
%
where $\hbar$ is Planck's reduced constant, $c$ is the speed of light, and $k_B$ is Boltzmann's constant.
Dimensionfull results are often given in $\text{MeV}$.
To obtain results in the SI-system, we use the following conversion factors, as given by~\cite{particledatagroupReviewParticlePhysics2020}
%
\begin{align}
    \label{speed of ligh}
    c       &= 2.998 \cdot 10^8     \, \text{m} \, \text{s}^{-1}, \\
    \label{hbar}
    \hbar   &= 1.055 \cdot 10^{-34} \, \text{J} \, \text{s}, \\
    \label{Boltzmanns constat}
    k_B     &= 1.380 \cdot 10^{-23} \, \text{J} \, \text K^{-1}, \\
    \label{Newtons gravitational constant}
    G       &= 6.674 \cdot 10^{-11} \, \text m^3 \, \text{kg}^{-1} \, \text s^{-2},
\end{align}
%
where $G$ is Newton's gravitational constant.
The conversion factor between $\text{MeV}$ and SI-units is
%
\begin{equation}
    \label{electronvolt}
    1 \, \text{MeV} = 1.60218\, \cdot 10^{-19} \, \text{J}. 
\end{equation}
%
In the calculation in \autoref{section: cold fermi star}, the value for the neutron mass is~\autocite{particledatagroupReviewParticlePhysics2020}
%
\begin{equation}
    \label{mass of neutron}
    m_N = 939.57 \, \text{MeV} = 1.674\cdot 10^{-27} \, \text{kg}.
\end{equation}
%
In astronomical calculation, the solar mass is used, which is
%
\begin{equation}
    \label{solar mass}
    M_\odot = 1.988 \cdot 10^{30} \, \text{kg}.
\end{equation}


\section{Structure of thesis}

To make this thesis as self contained as possible, we have included some parts from the earlier specialization project\todo{citation}, with minor modifications.
These sections are marked with an asteriks in the table of content.

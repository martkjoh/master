\section{Units}
\label{section: units}

In this thesis, we employ \emph{natural units}, defined by
%
\begin{equation}
    \hbar = c = k_B = 1,
\end{equation}
%
where $\hbar$ is Planck's reduced constant, $c$ is the speed of light, and $k_B$ is Boltzmann's constant.
Dimensionfull results are often given in $\text{MeV}$.
Uncertainties are indicated when the preciscion is less than four significant figures, however, the central value is always used in calcualtions.
Uncertainties are indicated in parantethesis, and the addition and subtraction of this value to the least significant result denote the confidence interval of one standard deviation.
That is, for a result $123.456(7)$, the range $123.456\pm0.007$ cover a confidence interval of $68.3\%$~\autocite{particledatagroupReviewParticlePhysics2020}.
All values in this section are from the Particle Data Group~\cite{particledatagroupReviewParticlePhysics2020}.
To obtain results in the SI-system, we use the following conversion factors, as given by
%
\begin{align}
    \label{speed of ligh}
    c       &= 2.998 \cdot 10^8     \, \text{m} \, \text{s}^{-1}, \\
    \label{hbar}
    \hbar   &= 1.055 \cdot 10^{-34} \, \text{J} \, \text{s}, \\
    \label{Boltzmanns constat}
    k_B     &= 1.380 \cdot 10^{-23} \, \text{J} \, \text K^{-1}, \\
    \label{Newtons gravitational constant}
    G       &= 6.674 \cdot 10^{-11} \, \text m^3 \, \text{kg}^{-1} \, \text s^{-2},
\end{align}
%
where $G$ is Newton's gravitational constant.
The conversion factor between $\text{MeV}$ and SI-units is
%
\begin{equation}
    \label{electronvolt}
    1 \, \text{MeV} = 1.60218\, \cdot 10^{-19} \, \text{J}. 
\end{equation}
%
The fine structure constant and the elementary charge is
%
\begin{align}
    \label{Fine structure constant}
    \alpha &= 7.297 \cdot 10^{-3}, \\
    \label{Elementary charge}
    e &:= \sqrt{4 \pi \alpha} =  3.028\cdot 10^{-1}.
\end{align}
%
In the calculation in \autoref{section: cold fermi star}, the value for the neutron mass is
%
\begin{equation}
    \label{mass of neutron}
    m_N = 939.57 \, \text{MeV} = 1.674\cdot 10^{-27} \, \text{kg}.
\end{equation}
%
In astronomical calculation, the solar mass is used, which is
%
\begin{equation}
    \label{solar mass}
    M_\odot = 1.988 \cdot 10^{30} \, \text{kg}.
\end{equation}
%
When working with chiral perturbation theory, we use
%
\begin{align}
    \label{pion decay constant}
    f_\pi & = \frac{1}{\sqrt{2}}130.2 (8) \, \text{MeV} = 92.1(6) \, \text{MeV}, \\
    \label{pion mass}
    m_\pi & = 134.98 \, \text{MeV} = 2.406 \cdot 10^{-28} \, \text{kg}, \\
    \label{charged pion mass}
    m_{\pi_\pm} &= 139.57 \, \text{MeV} = 2.488 \cdot 10^{-28}\, \text{kg},
\end{align}
%
where $f_\pi$ is the pion decay constant, and $m_\pi$ the mass of the neutral pion, $\pi^0$.

\section{Structure of thesis}

To make this thesis as self-contained as possible, we have included some parts from the earlier specialization project\todo{citation}, with minor modifications.
These sections are marked with an asterisk in the table of content.

This thesis is the concluding work for a 5-year Master of Science (``Sivilingeniør'') degree in Applied Physics (``Teknisk Fysikk'') at the Norwegian University of Science and Technology (NTNU), under the supervision of Professor Jens Oluf Andersen.
It is the result of 20 weeks of work in the spring of 2022 and builds on work done in the specialisation project (``Prosjektoppgave'') in the autumn of 2021.
The topic of the thesis is the thermodynamic properties of the pion-condensed phase of quantum chromodynamics, which is investigated using three-flavor chiral perturbation theory, and its applications to the study of pion stars.

\subsection*{Conventions}

In this thesis, we employ \emph{natural units}, defined by $\hbar = c = k_B = 1$.
Here, $\hbar$ is Planck's reduced constant, $c$ is the speed of light, and $k_B$ is Boltzmann's constant.
Dimensionfull results are given in $\text{MeV}$ or SI units and, unless otherwise stated, computed using the values given in \autoref{section: units} are used.
The metric is in the ``mostly minus'' convention, in which $g_{\mu \nu} = \text{diag}(1, -1, -1, -1)$.
We employ Einstein's summation convation in which repeated indices are summed over, $a_i b_i = {\sum}_i a_i b_i = a_1 b_1 +  a_2 b_2 +\dots$.
Spacetime indices are denoted by $\mu$, $\nu$, $\rho$, $\eta$, or $\lambda$ and should only be repeated once as sub- and superscripts, $a_\mu b^\mu = a^\mu b_\mu = g_{\mu\nu}a^\mu b^\nu$.
The placement of other indices does not have any importance and is only chosen for readability.


\subsection*{Acknowledgements}

\todo[inline]{Skrive acknowledgement}
% Supervision, guidance and patience with bad writing of Jens Oluf Andersen
% Brandt et. al. for providing their data
% Family in Bø for always supporting, friends in Trondheim for discussion and help forgetting when needed(?), and Nabla (?)
% Martin for his thesis (?)


{\flushright{
    \todo[]{signere?}
    Martin Kjøllesdal Johnsrud\\
    June 2022, Trondheim, Norway\\
}}
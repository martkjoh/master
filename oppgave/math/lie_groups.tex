\section[Lie groups${}^*$]{Lie groups}
This section is adapted from (PROSJEKTOPPGAVE).
\todo{Make this section fit better into this thesis}

Lie groups are a natural structure to capture the symmetries of a theory.
A Lie group is a smooth manifold, i.e., a space that is locally diffeomorphic to $\R^N$.
This means that we can locally parametrize the space by $N$ real numbers $\eta_\alpha$, using smooth invertible functions.
A Lie group is also equipped with group structure.
A group is a set, $G$, together with a map
%
\begin{align}
    (\cdot, \cdot):  G \times G &\longmapsto G ,\\
    (g_1, g_2) &\longmapsto g_3,
\end{align}
% 
called group multiplication. This map obeys the group axioms, which are the existence of an identity element $\one$, associativity and the existence of an inverse element $g^{-1}$ for all $g\in G$.
These can be written as
\begin{table}[!h]
    \centering
    \begin{tabular}{l l}
        $\forall g \in G, $&$ (g, \one) = g, $\\
        $\forall g_1, g_2, g_3 \in G, $ & $ (g_1, (g_2, g_3)) = ((g_1, g_2), g_3), $\\
        $\forall g \in G,\, \exists g^{-1} \in G,\, \text{s.t.}, $ & $ (g, g^{-1}) = \one.$
    \end{tabular}
\end{table}

In addition, we require that both the multiplication map and the inverse map, $g \mapsto g^{-1}$, are smooth.
We describe the set of continuous symmetry transformations,
%
\begin{equation}
    G = \setbuilder{g}{g \varphi = \varphi', \, S[\varphi'] = S[\varphi], \D \varphi' = \D \varphi },
\end{equation}
%
as a Lie group.
The group $G$ might act on $\varphi$ linearly, so $(g\varphi)_i = g_{ij}\varphi_j$, or in a more complicated matter.
In this case, the group multiplication is composition, i.e., performing transformations in succession.
This map is closed, as the composite of two symmetry transformations is another symmetry transformation.
The identity map is a symmetry transformation, and composition is associative.
This means that invertible symmetry transformations form a group.

We will focus on connected Lie groups, in which all elements $g \in G$ are in the same connected piece as the identity map $\one \varphi = \varphi$.
This means that for each $g\in G$, one can find a continuous path $\gamma(t)$ in the manifold, such that $\gamma(0) = \one$ and $\gamma(1) = g$.
Given such a path, we can study transformations close to the identity element.
As the Lie group is a smooth manifold, we can write\footnote{
    The factor $i$ is a physics convention and differs from how mathematicians define generators of a Lie group.
    }
\begin{equation}
    \gamma(\epsilon) = \one + i \epsilon V + \Oh{\epsilon}.
\end{equation}
%
$V$ is a generator, and is defined as
\begin{equation}
    iV = \odv{\gamma}{t}\Big|_{t=0}.
\end{equation}
%
We can define a path $\gamma$ in $G$ by its path through parameter space $\R^n$, $\gamma(t) = g(\eta(t))$.
Here, $\eta_\alpha(t)$ is a path through $\R^N$, the coordinates of $G$, such that $\eta_\alpha(0) = 0$ and $g(0) = \one$.
We can thus write the generator as
%
\begin{equation}
    V = \odv{\gamma}{t}\Big|_{t=0} = \odv{\eta_\alpha}{t}\Big|_{t=0} \pdv{g}{\eta_\alpha}\Big|_{\eta=0}
    = v_\alpha T_\alpha, \quad 
    T_\alpha = \odv{\eta_\alpha}{t}\Big|_{t=0}, \quad
    \pdv{g}{\eta_\alpha}\Big|_{\eta=0}.
\end{equation}
%
One can show that the generators form a vector space, with the basis $T_\alpha$, induced by the coordinates $\eta_\alpha$~\cite{leeSmoothManifolds2012}.
This vector space is called the tangent space of the identity element, $T_\one G$.
Infinitesimal transformations can therefore be written as
\begin{equation}
    \gamma(\epsilon) = \one + i \epsilon v_\alpha T_\alpha + \Oh{\epsilon}.
\end{equation}
%
The tangent space, together with the additional operation
\begin{align}
    [T_\alpha, T_\beta] = iC_{\alpha\beta}^\gamma T_\gamma,
\end{align}
%
called the Lie bracket, form a Lie algebra denoted $\mathfrak{g}$.
$C_{\alpha \beta}^\gamma$ are called structure constants.
They obey the Jacobi identity,
\begin{equation}
    \label{jacobi identity}
    C_{\alpha \beta}^\gamma + C_{\beta\gamma}^\alpha +  C_{\gamma\alpha}^\beta = 0,
\end{equation}
%
which mean that they are totally antisymmetric.
For matrix groups, which we deal with in this text, the Lie bracket is the commutator.
A subset of the original Lie group, $H \subset G$, closed under the group action, is called a subgroup.
$H$ then has its own Lie algebra $\mathfrak{h}$, with a set of $m = \dim H$ generators, $t_a$, which is a subset of the original generators $T_\alpha$.
We denote the remaining set of generators $x_i$, such that $t_a$ and $x_i$ together span $\mathfrak{g}$.
The commutators of $t_a$ must be closed, which means that we can write
%
\begin{align}
    [t_a, t_b] &= i C_{ab}^{c} t_c,\\
    [t_a, x_i] &= i C_{ai}^k x_k, \\
    [x_i, x_j] &= i C_{ij}^k x_k + i C_{ij}^c t_c,
\end{align}
%
where $abc$ runs over the generators of $\mathfrak h$, and $ijk$ runs over the rest.
The second formula can be derived using the Jacobi identity REF:{jacobi identity}, which implies that $C_{ab}^k = 0 = -C_{ak}^b$.
This is called a Cartan decomposition.

One parameter subgroups are one special case of Lie subgroups.
If a curve $\gamma(t)$ through $G$ obey
%
\begin{equation}
    \gamma(t)\gamma(s) = \gamma(t + s), \quad \gamma(0) = \one,
\end{equation}
%
then all the points on this curve from a one parameter subgroup of $G$.
This path is associated with a generator, 

\begin{equation}
    \odv{\gamma}{t} \Big|_{t=0} = i \eta_\alpha T_\alpha.
\end{equation}
%
This association is one-to-one, and allows us to define the exponential map,
\begin{equation}
    \exp{i \eta_\alpha T_\alpha} := \gamma(1).
\end{equation}
%
For connected and compact Lie groups, all elements of the Lie group $g \in G$ can be written as an exponential of elements in the corresponding Lie algebra $\eta_\alpha T_\alpha \in \mathfrak g$.
For matrix groups, the exponential equals the familiar series expansion~\cite{leeSmoothManifolds2012}
%
\begin{equation}
    \exp{X} = \sum_n \frac{1}{n!} X^n.
\end{equation}
%
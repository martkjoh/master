In this chapter, we will take the general knowledge from the general theory in \autoref{chapter: QFT} and apply it to the specific case of quantum chromodynamics, which results in \emph{chiral perturbation theory}, or \chpt.

\section{QCD}

Quantum chromodynamics (QCD) is the theory of how quarks interact via the strong force.
%
\begin{equation}
    \Ell_{\text{QCD}} 
    = \sum_{f=1}^{N_f} \sum_{c=1}^{N_c} 
    \bar q_{fc} \left(i \gamma^\mu D_{\mu fc} - m_f\right) q_{fc}
    = \bar q (i \slashed D - m)q.
\end{equation}
%
The covariant derivative is defined as
%
\begin{equation}
    [D_\mu q]_{fc} 
    = 
    \partial_\mu q_{fc} 
    - i g A^\alpha_{\mu} \lambda^\alpha_{cc'} \tau_{a} q_{fc'} 
\end{equation}
%
Here, $A^\alpha_\mu$ are the eight gluon fields.

\subsection{Yang-Mills theory and Gauge symmetry}


\subsection{Chiral symmetry}

The massless QCD-Lagrangian is the basis for chiral perturbation theory, and is
%
\begin{equation}
    \Ell_0 = i \bar q \slashed \partial q.
\end{equation}
The Lagrangian containing external currents are
%
\begin{equation}
    \Ell_{\text{ext}}
    = - \bar q \left(s + i \gamma^5 p \right) q
    + \bar q \left(v_\mu + \gamma^5 a_\mu\right)\gamma^\mu q.
\end{equation}
%
The external sources are defined as
%
\begin{equation}
    s = s_0\one + s_a \tau_a, \quad
    p = p_0\one + p_a \tau_a, \quad
    v^\mu = v_0^\mu \one + v_a^\mu \tau_a, \quad
    a^\mu = a_0^\mu \one + a^\mu_a \tau_a.
\end{equation}
%
which are the scalar, pseudo-scalar, vector and pseudo-vector currents, respectivly.
We denote these currents collectively as $j = (s, p, v^\mu_a, a_a^\mu)$.
The electromagnetic interactions of quarks are captured by 
%
\begin{equation}
    i\bar q \slashed \partial q 
    \rightarrow i \sum_f \bar q_f \gamma^\mu \left( \delta_{ff'}\partial_\mu - Q_{ff'} \mathcal A_\mu\right) q_{f'},
\end{equation}
where $\mathcal A_\mu$ is the photon field, while charge matrix $Q$ is
%
\begin{equation}
    Q = \frac{q_e}{3}
    \begin{pmatrix}
        2 & 0 \\
        0 & -1
    \end{pmatrix}
    = 
    q_e \left(\frac{1}{6} \one + \frac{1}{2} \tau_3\right).
\end{equation}
%
In \chpt we account for this with the a external vector current
%
\begin{equation}
    v_\text{EM}^{\mu} = -i q_e \left(\frac{1}{6} \one + \frac{1}{2}\tau_3\right) \mathcal{A}^\mu.
\end{equation}
%

In this chapter, we will take the general knowledge from the general theory in \autoref{chapter: QFT} and apply it to the specific case of quantum chromodynamics, which results in \emph{chiral perturbation theory}, or \chpt.

\section{QCD}
This section is based on~\autocite{peskinIntroductionQuantumField1995,schererIntroductionChiralPerturbation2002,schwartzQuantumFieldTheory2013}

\subsection{Yang-Mills theory and Gauge symmetry}


In our discussion on global symmetries, we considered the global transformation of fields by some group $G$.
In gauge theories, we will consider local transformations.
That is, the transformations are themselves functions of spacetime, $U = U(x)$, and take on some value in $G$ for all points in space.
With this, however, we encounter a problem with comparing the value of a field at different points.
As the symmetry is local, a gauge transformation will generally affect the field at two points differently.
We must find a way to compare fields at different points independent of gauge transformations.
This is similar to a problem we have encountered before.
In differential geometry, as described in \autoref{section: differential geometry}, we needed a connection $\Gamma^\rho_{\mu \nu}$ to compare vectors in different tangent spaces in a coordinate independent way.
In gauge theories, we generalize this by defining a connection, $A_\mu$, to compare field values at different points in a gauge-independent way.

Consider a set of $N_c$ fields $\psi_c$, which the symmetry group $\Lie{SU}{N}$ acts linearly on as $\psi_c \rightarrow U_{cc'} \psi_{c'}$.
We can write $U = \exp{i \eta_\alpha T_\alpha}$, where $T_\alpha$ are the generators of $\lie{su}{N_c}_c$, and can therefore be written $A_\mu = A_\mu^\alpha T_\alpha$.
The transformation is then made local by letting the coordinates of $\Lie{SU}{N}$ be functions of spacetime, $\eta_\alpha = \eta_\alpha(x)$.
As we did in \autoref{section: differential geometry}, we define the covariant derivative $D_\mu$ to transform as the thing on which it acts.
It has the form
%
\begin{equation}
    \label{covariant derivative Yang-Mills}
    D_\mu^{cc'} \psi_{c'} = (\delta_{cc'}\partial_\mu - i g A_\mu^{cc'} )\psi_{c'},
\end{equation}
%
where $A_\mu^{cc'}$ is a new, dynamic field, the gauge field.
This field takes values in the Lie algebra of the gauge group, $\lie{su}{N}$.
We will suppress the $c$-indices for cleaner notation.
This field also transform under the gauge group.
By enforcing the transformation rule $D_\mu A_\nu \rightarrow U D_\mu A_\nu$, we can deduce the transformation properties of the gauge field, 
%
\begin{equation}
    \label{Gauge transformation gauge field}
    A_\mu\rightarrow U \left(A_\mu + \frac{i}{g} \partial_\mu\right) U^\dagger
\end{equation}
%
With the covariant derivative, we can create gauge-invariant terms, such as $\bar \psi D_\mu \psi$.
In \autoref{section: differential geometry} we introduced the Riemann tensor as the commutator of covaraint derivatives, \autoref{Riemann tensor}.
This ensures that it transforms as a tensor and gives us the interpretation as a quantity that measures the amount vectors curved when parallel transported in a small loop.
In analogy, we define the \emph{field strength tensor},
%
\begin{equation}
    G_{\mu \nu} := \frac{i}{g} [D_\mu, D_\nu]
    = \partial_\mu A_\nu - \partial_\nu A_\mu - i g[A_\mu, A_\nu].
\end{equation}
%
$A_\mu$ is an element of a Lie algebra, so the commutator is given by the structure constants of that algebra, \autoref{structure constants}.
The field strength tensor transforms as $G_\mu \rightarrow U G_{\mu \nu}U^\dagger$.
This allows us to create gauge-invariant terms of only this tensor, which, as with the Ricci scalar in general relativity, are the building blocks of the Lagrangian of the gauge field.
The lowest order terms are
%
\begin{equation}
    G^{\mu \nu}_\alpha G_{\mu \nu}^\alpha, \quad
    \epsilon^{\mu \nu \rho \sigma} G_{\mu \nu}^\alpha G_{\rho \sigma}^\alpha.
\end{equation}
%
Here, $\alpha$ is the index in $\lie{su}{N}$-space.

\subsection{The QCD Lagrangian}

Quantum chromodynamics, or QCD, is the specific gauge theory of quarks $q_{fc}$, spin-$\frac{1}{2}$ particles, interacting via the strong force, a $\Lie{SU}{3}_c$ gauge field denoted $A_\mu$.
There are six quarks $q$, called flavors and indexed by $f$, with an additional quantum number called color indexed by $c$.
The quarks, labeled u, d, s, c, t, and b, have different masses.
This thesis will include the two or three lightest quarks at different times.
Therefore, we denote the number of flavors by $N_f$.
The Lagrangian of QCD, including only the strong force, is
%
\begin{equation}
    \Ell_{\text{QCD}} 
    = \bar q (i \slashed D - m)q - \frac{1}{4} G_{\mu \nu}^\alpha G^{\mu \nu}_\alpha.
\end{equation}
%
We have suppressed color and flavor indices.
$\slashed D q = \gamma^\mu (\partial_\mu - i g A_\mu) q$ is the covariant derivative associated with the $\Lie{SU}{3}_c$ gauge group with coupling constant $g$, and $\gamma^\mu$ are the Dirac matrices, as described in \autoref{section: algebra bases}.
The quark mass matrix, $m$, acts on the flavor indexes as the flavor states are mass eigenstates.
There are no known symmetries that forbid a $\epsilon^{\mu \nu \rho \sigma} G_{\mu \nu}^\alpha G_{\rho \sigma}^\alpha$-term, and its absence is dubbed the strong CP problem~\autocite{schwartzQuantumFieldTheory2013}.



\subsection{Chiral symmetry}

If we consider the massless QCD Lagrangian, $m = 0$, it has an additional symmetry of rotation in its flavour indices.
We can project the quarks down to their \emph{chiral} components by introducing projection operators
\begin{equation}
    P_R = \frac{1}{2}\left(1 + \gamma^5\right), \quad
    P_L = \frac{1}{2}\left(1 - \gamma^5\right).
\end{equation}
%
Here, $\gamma^5$ is the ``fifth gamma-matrix'', as described in \autoref{section: algebra bases}.
As good projection operators, they obey
%
\begin{equation}
    P_R + P_L = 0, \quad P_R P_L = P_L P_R = 0, \quad P_I^2 = P_I, \, I=R, L.
\end{equation}
%
By the properties of $\gamma^5$ and $\bar q = q^\dagger\gamma^0$, these operators project out the opposite chirality of $q$ and $q$,
%
\begin{equation}
    P_I q = q_I, \quad \bar q  P_I  = q_{\bar I},\quad
    I = R, L, \quad \bar I = L, R. 
\end{equation}
%
With this, we can write the quark-sector of massless QCD as
%
\begin{equation}
    i \bar q \slashed D q
    = i \bar q \slashed D (P_R + P_L)^2 q
    = i \bar q_L \slashed D q_L + i \bar q_R \slashed D q_R.
\end{equation}
%
This operator is invariant under the transformations
%
\begin{equation}
    q_R \rightarrow U_R q_R, \quad
    q_L \rightarrow U_L q_L,
\end{equation}
%
where $U_L$ and $U_R$ are Hermitian matrices that act on the flavor indices.
These transformations form the Lie group $\Lie{U}{N_f}_R \times \Lie{U}{N_f}_L = \Lie{U}{1}_R \times \Lie{SU}{N_f}_R \times \Lie{U}{1}_L \times \Lie{SU}{N_f}_L $.
This transformation can also be described in terms of the diagonal subgroup.
This subgroup is made up of transformations where $U_R = U_L$, called vector transformations, and the remaining subgroup of transformations where $U_L =  U_R^\dagger$, called axial transformations.
These together form $\Lie{U}{N_f}_A \times \Lie{U}{N_f}_V= \Lie{U}{1}_V \times \Lie{SU}{N_f}_V \times \Lie{U}{1}_A \times \Lie{SU}{N_f}_A $.
The currents corresponding to these transformations are
%
\begin{equation}
    \label{conserved currents qcd}
    J_V^\mu = \bar q_R \gamma^\mu q_R, \quad
    V^\mu_\alpha =  \bar q T_\alpha \gamma^\mu q, \quad
    J_A^\mu = \bar q_L \gamma^\mu \gamma^5 q_L, \quad
    A^\mu_\alpha = \bar q T_\alpha \gamma^\mu \gamma^5 q.
\end{equation}
%
Here, $T_\alpha$ and $T_\alpha \gamma^5$ are the generators of $\Lie{SU}{N_f}_V$ and $\Lie{SU}{N_f}_A$.
This symmetry, though, is broken in several ways.
Firstly, transformations of the form $e^{i \alpha \gamma^5} \in \Lie{U}{1}_A$ are subject to the \emph{axial anomaly}.
As mentioned in \autoref{section: symmetry and goldstone's theorem}, in a quantum theory not only the action has to be invariant but the integration measure as well, and $\D q \D \bar q$ is not.
This is encoded in the Schwinger-Dyson equation~\autocite{schwartzQuantumFieldTheory2013}\todo[]{write more about SD-eqs, ward identities}
%
\begin{equation}
    \partial_\mu \ex{J_A^\mu} = -\frac{e^2}{(4 \pi)^2} \ex{\varepsilon^{\mu \nu \rho \sigma}F_{\mu \nu} F_{\rho \sigma}},
\end{equation}
%
whose right side would vanish if the quantum theory was invariant under $\Lie{U}{1}_A$.
The remaining symmetry is $G =  \Lie{U}{1}_V \times \Lie{SU}{N_f}_V \times \Lie{SU}{N_f}_A$.
Next, the mass term explicitly breaks this symmetry.
If we consider all quarks to have the same mass $m_q$, so that $m = m_q \one$, only $\Lie{U}{N_f}_A$ is broken.
This is called the \emph{chiral limit}.  
However, when we include the fact that the masses of the quarks are different, we break the symmetry further.
Other external currents, chemical potentials, and the electromagnetic interaction also break the symmetry.
We will discuss how to incorporate this in the next chapter.
Lastly, the $G$-symmetry is broken spontaneously by the ground state quark condensate,
%
\begin{equation}
    \ex{\bar q_f q_f} = -f^2 B_0 \neq 0, \, f \in \{ u, d, s\}.
\end{equation}
%
The scalar quark operator is not invariant under $\Lie{SU}{N_f}_A$, and as discussed in \autoref{section: symmetry and goldstone's theorem}, this leads to the spontaneous symmetry breaking pattern.
%
\begin{equation}
    \Lie{SU}{N_f}_L \times \Lie{SU}{N_f}_R 
    \longrightarrow \Lie{SU}{N_f}_L \times \Lie{SU}{N_f}_R \big/ \Lie{SU}{N_f}_A 
    = \Lie{SU}{N_f}_V.
\end{equation}
%
This pattern enables us to construct an effective low energy theory for QCD physics.
We will take this symmetry breaking as an axiom and use it to construct \chpt.



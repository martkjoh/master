In this chapter, we will take the theory in \autoref{chapter: QFT} and apply it to the specific case of quantum chromodynamics to derive \emph{chiral perturbation theory} (\chpt).
Chiral perturbation theory is an effective description in which the \emph{pseudoscalar mesons}, such as the pions, are the degrees of freedom.
They are pseudo-Goldstone bosons due to the spontaneous breaking of the QCD vacuum and allow for a perturbative description of the strong force even at low energies.
We begin with a description of quantum chromodynamics.


\section{Quantum chromodynamics}
\label{section:qcd}

This section is based on~\autocite{peskinIntroductionQuantumField1995,schererIntroductionChiralPerturbation2002,schwartzQuantumFieldTheory2013}.
Quantum chromodynamics, QCD, is the theory of quarks, $q_{fc}$, interacting via the strong force.
Quarks are spin-$\frac{1}{2}$ spinors, and thus fermions.
There are six \emph{flavors} of quarks divided into three \emph{generations}.
The first generation is the up and down quarks, the second is the charm and strange quarks, and the third generation is the top and bottom quarks.
The flavors are labeled by the index $f$.
The second index, $c$, labels the quantum number associated with the strong force, called color.
The strong force is mediated by gluons, denoted $A_\mu^c$, and is an example of a Yang-Mills theory.
We will give a short overview of Yang-Mills theory before we detail QCD further.


\subsection{Yang-Mills theory and Gauge symmetry}
\label{subsection: yang-mills theory and gauge symmetry}


In quantum electrodynamics, two systems are related by a gauge transform where the photon field transforms as $\mathcal A_\mu \rightarrow \mathcal A_\mu + \partial_\mu \alpha(x)$, are physically indistinguishable.
This principle of gauge invariance dictates the form of interactions of the theory.
Yang-Mills theory generalizes this approach.
In our discussion on global symmetries, we considered the global transformation of fields by some group $G$.
As we did in \autoref{subsection: ward identities}, we now promote this to a local transformation.
That is, the transformations are themselves functions of spacetime, $U = U(x)$, and take on some value in $G$ for all points in space.
With this, however, we encounter a problem with comparing the value of a field at different points.
As the symmetry is local, a gauge transformation will generally affect the field at two points differently.
We must find a way independent of gauge choice to compare fields at different points in space.
This is similar to a problem we have encountered before.
In differential geometry, as described in \autoref{section: differential geometry}, we needed a connection $\Gamma^\rho_{\mu \nu}$ to compare vectors in different tangent spaces in a coordinate independent way.
In gauge theories, we generalize this by defining a connection, $A_\mu$, to compare field values at different points in a gauge-independent way.

Consider a set of $N$ fields $\psi_c$, which the symmetry group $\Lie{SU}{N}$ acts linearly on as $\psi_c \rightarrow U_{cc'} \psi_{c'}$.
We can write $U = \exp{i \eta_\alpha T_\alpha}$, where $T_\alpha$ are the generators of $\lie{su}{N}$.
The transformation is then made local by letting the coordinates of $\Lie{SU}{N}$ be functions of spacetime, $\eta_\alpha = \eta_\alpha(x)$.
As we did in \autoref{section: differential geometry}, we define the covariant derivative $D_\mu$ to transform as the thing on which it acts.
Assume this covariant derivative has the form
%
\begin{equation}
    \label{covariant derivative Yang-Mills}
    D_\mu^{cc'} \psi_{c'} = (\delta_{cc'}\partial_\mu - i g A_\mu^{cc'} )\psi_{c'},
\end{equation}
%
where $A_\mu^{cc'}(x)$ is a new, dynamic field.
This is similar to what we did in differential geometry, \autoref{covariant derivative diff geom}.
We will suppress the $c$-indices for cleaner notation.
By enforcing the transformation rule $D_\mu \psi \rightarrow U D_\mu \psi$, we can deduce the transformation properties of the gauge field, 
%
\begin{equation}
    \label{Gauge transformation gauge field}
    A_\mu\rightarrow U \left(A_\mu + \frac{i}{g} \partial_\mu\right) U^\dagger
\end{equation}
%
For an infinitesimal transformation, this becomes
%
\begin{equation}
    A_\mu \rightarrow A_\mu + i \eta_\alpha [T_\alpha, A_\mu] + \frac{1}{g} \partial_\mu \eta_\alpha T_\alpha.
\end{equation}
%
From \autoref{subsection: lie algebras}, we see that $A_\mu$ transform under the adjoint representation of global $\lie{su}{N}$.
We can therefore write $A_\mu = A_\mu^\alpha T_\alpha$.
With the covariant derivative, we can create gauge-invariant terms, such as $\bar \psi D_\mu \psi$.
In \autoref{section: differential geometry}, we introduced the Riemann tensor as the commutator of covariant derivatives, \autoref{Riemann tensor}, which ensures that it transforms as a tensor.
We found that this tensor measures how much a vector is rotated by the curvature of space when parallel transported around in a small loop.
In analogy, we define the \emph{field strength tensor},
%
\begin{equation}
    G_{\mu \nu} := \frac{i}{g} [D_\mu, D_\nu]
    = \partial_\mu A_\nu - \partial_\nu A_\mu - i g[A_\mu, A_\nu].
\end{equation}
%
$A_\mu$ is an element of a Lie algebra, so the commutator is given by the structure constants of that algebra, \autoref{structure constants}.
The field strength tensor transforms as $G_{\mu\nu} \rightarrow U G_{\mu \nu}U^\dagger$.
This allows us to create gauge-invariant terms of only this tensor, which, as with the Ricci scalar in general relativity, are the building blocks of the Lagrangian of the gauge field.
The lowest order terms are
%
\begin{equation}
    G^{\mu \nu}_\alpha G_{\mu \nu}^\alpha, \quad
    \epsilon^{\mu \nu \rho \sigma} G_{\mu \nu}^\alpha G_{\rho \sigma}^\alpha.
\end{equation}
%
Here, $\alpha$ is the index in $\lie{su}{N}$-space.



\subsection{The QCD Lagrangian}

The Lagrangian of QCD, including only the strong force, is
%
\begin{equation}
    \Ell_{\text{QCD}} 
    = \bar q (i \slashed D - m)q - \frac{1}{4} G_{\mu \nu}^\alpha G^{\mu \nu}_\alpha.
\end{equation}
%
We have suppressed color and flavor indices, which are summed over, and $\bar q = q^\dagger \gamma^0$.
The gauge group of the strong force is $\Lie{SU}{3}_c$ with the coupling constant $g$, and the corresponding covariant derivative is
%
\begin{equation}
    (D_\mu q)_{cf} =  (\delta_{cc'} \partial_\mu - i g A_\mu^{cc'}) q_{c'f},
\end{equation}
%
We employ the Feynman slash-notation, where $\slashed D = \gamma^\mu D_\mu$, and $\gamma^\mu$ are the Dirac matrices, as described in \autoref{section: algebra bases}.
The quark mass matrix, $m$, acts on the flavor indexes as the flavor states are mass eigenstates.
There are no known symmetries that forbid an $\epsilon^{\mu \nu \rho \sigma} G_{\mu \nu}^\alpha G_{\rho \sigma}^\alpha$-term, however it is an empirical fact that it is either not present or highly suppressed. 
Its absence is dubbed the strong CP problem~\autocite{schwartzQuantumFieldTheory2013}.



\subsection{Chiral symmetry}
\label{subsection: chiral symmetry}

If we consider the massless QCD Lagrangian, $m = 0$, it has an additional symmetry of rotation in its flavour indices.
We can project the quarks down to their \emph{chiral} components by introducing projection operators
\begin{equation}
    P_R = \frac{1}{2}\left(1 + \gamma^5\right), \quad
    P_L = \frac{1}{2}\left(1 - \gamma^5\right).
\end{equation}
%
Here, $\gamma^5$ is the ``fifth gamma-matrix'', as described in \autoref{section: algebra bases}.
As good projection operators, they obey
%
\begin{equation}
    P_R + P_L = 0, \quad P_R P_L = P_L P_R = 0, \quad P_I^2 = P_I, \, I=R, L.
\end{equation}
%
By the properties of $\gamma^5$ and $\bar q = q^\dagger\gamma^0$, these operators project out the opposite chirality of $q$ and $\bar q$,
%
\begin{equation}
    P_I q = q_I, \quad \bar q  P_I  = q_{\bar I},\quad
    I = R, L, \quad \bar I = L, R. 
\end{equation}
%
With this, we can write the quark-sector of massless QCD as
%
\begin{equation}
    i \bar q \slashed D q
    = i \bar q \slashed D (P_R + P_L)^2 q
    = i \bar q_L \slashed D q_L + i \bar q_R \slashed D q_R.
\end{equation}
%
This operator is invariant under the transformations
%
\begin{equation}
    q_R \rightarrow U_R q_R, \quad
    q_L \rightarrow U_L q_L,
\end{equation}
%
where $U_L$ and $U_R$ are Hermitian matrices that act on the flavor indices.
These transformations form the Lie group $\Lie{U}{N_f}_R \times \Lie{U}{N_f}_L = \Lie{U}{1}_R \times \Lie{SU}{N_f}_R \times \Lie{U}{1}_L \times \Lie{SU}{N_f}_L $.
This transformation can also be described in terms of the diagonal subgroup.
This subgroup is made up of transformations where $U_R = U_L$, called vector transformations, and the remaining subgroup of transformations where $U_L =  U_R^\dagger$, called axial transformations.
These together form $\Lie{U}{N_f}_A \times \Lie{U}{N_f}_V= \Lie{U}{1}_V \times \Lie{SU}{N_f}_V \times \Lie{U}{1}_A \times \Lie{SU}{N_f}_A $.
The currents corresponding to these transformations are
%
\begin{equation}
    \label{conserved currents qcd}
    J_V^\mu = \bar q \gamma^\mu q, \quad
    V^\mu_\alpha =  \bar q T_\alpha \gamma^\mu q, \quad
    J_A^\mu = \bar q \gamma^\mu \gamma^5 q, \quad
    A^\mu_\alpha = \bar q T_\alpha \gamma^\mu \gamma^5 q.
\end{equation}
%
Here, $T_\alpha$ and $T_\alpha \gamma^5$ are the generators of $\Lie{SU}{N_f}_V$ and $\Lie{SU}{N_f}_A$.
This symmetry, though, is broken in several ways.
Firstly, transformations of the form $e^{i \alpha \gamma^5} \in \Lie{U}{1}_A$ are subject to the \emph{axial anomaly}.
As mentioned in \autoref{section: symmetry and goldstone's theorem}, in a quantum theory not only the action has to be invariant but the integration measure as well, and $\D q \D \bar q$ is not.
This is encoded in the Schwinger-Dyson equation~\autocite{schwartzQuantumFieldTheory2013}
%
\begin{equation}
    \partial_\mu \ex{J_A^\mu} = -\frac{e^2}{(4 \pi)^2} \ex{\varepsilon^{\mu \nu \rho \sigma}F_{\mu \nu} F_{\rho \sigma}}.
\end{equation}
%
The remaining symmetry is $\Lie{U}{1}_V \times \Lie{SU}{N_f}_V \times \Lie{SU}{N_f}_A$.
Next, the mass term explicitly breaks this symmetry.
If we consider all quarks to have the same mass $m_q$, so that $m = m_q \one$, only $\Lie{U}{N_f}_A$ is broken.
This is called the \emph{chiral limit}.  
However, when we include the fact that the masses of the quarks are different, we break the symmetry further.
Other external currents, chemical potentials, and electromagnetic interactions also break the symmetry.

Lastly, the symmetry is broken spontaneously by the ground state quark condensate, $\braket{\bar q q} \neq 0$.
In the chiral limit, one can show that the scalar bilinear for all three quarks $u$, $d$, and $s$ must be equal, and we define $\braket{\bar q q} = \braket{\bar uu} = \braket{\bar dd} = \braket{\bar ss}$~\autocite{schererIntroductionChiralPerturbation2002}.
The scalar quark operator is not invariant under $\Lie{SU}{N_f}_A$, and as discussed in \autoref{section: symmetry and goldstone's theorem}, this leads to the spontaneous symmetry breaking pattern.
%
\begin{equation}
    \Lie{SU}{N_f}_L \times \Lie{SU}{N_f}_R 
    \longrightarrow \Lie{SU}{N_f}_L \times \Lie{SU}{N_f}_R \big/ \Lie{SU}{N_f}_A 
    = \Lie{SU}{N_f}_V.
\end{equation}
%
This pattern enables us to construct an effective low energy theory for QCD physics.
We will take this symmetry breaking as an axiom and use it to construct \chpt.
The scalar quark condensate is quantified by the constant $B_0$ via the relation
%
\begin{equation}
    \ex{\bar q q} = - f_\pi^2 B_0.
\end{equation}
%
With the spontaneous symmetry breaking of the quark condensate, we get the resulting Goldstone bosons $\varphi_\alpha$.
The pion decay constant $f_\pi$ is defined by the matrix element of current corresponding to the broken symmetry, $A_\alpha^\mu$ between the vacuum $\ket{0}$ and Goldstone boson momentum states, $\ket{\varphi_a(p)}$.
One can show that this element, by Lorentz covariance, must have the form~\autocite{schererIntroductionChiralPerturbation2002,schwartzQuantumFieldTheory2013}
%
\begin{equation}
    \braket{0| A^\mu_\alpha (x) | \varphi_\beta(p)} 
    = i  f_\pi \delta_{\alpha\beta}  \, p^\mu e^{i p_\mu x^\mu}.
\end{equation}
%
We are now ready to construct \chpt.

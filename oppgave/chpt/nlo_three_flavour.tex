\section{Next-to-leading order Lagrangian}
\label{section: nlo chpt}


To construct the next-to-leading order Lagrangian, one follows the same approach of combining the building blocks we found in \autoref{chapter: chpt} into locally $\Lie{SU}{3}_R\times\Lie{SU}{3}_L$-invariant terms.
We do not need to include terms with the field strength tensors $f_{\mu\nu}^{(r)}$ and $f_{\mu\nu}^{(l)}$, as in our case, they will vanish.
We will not consider electromagnetic interactions.
With this, the NLO Lagrangian is~\autocite{gasserChiralPerturbationTheory1985}
%
\begin{align}
    \label{NLO Lagrangian three flavor}
    \nonumber
    \Ell_4 
    ={} &
    L_1\Tr{\nabla_\mu \Sigma \nabla^\mu \Sigma^\dagger}
    + L_2 \Tr{\nabla_\mu \Sigma \nabla_\nu \Sigma^\dagger} 
    \Tr{\nabla^\mu \Sigma \nabla^\nu \Sigma^\dagger}
    + L_3 \Tr{\left(\nabla_\mu \Sigma \nabla^\mu \Sigma^\dagger\right)^2} \\ \nonumber
    & + L_4 \Tr{\nabla_\mu \Sigma \nabla^\mu \Sigma^\dagger} 
    \Tr{\chi \Sigma^\dagger + \Sigma \chi^\dagger}
    + L_5 \Tr{
        \nabla_\mu \Sigma \nabla^\mu \Sigma^\dagger 
        \left(\chi \Sigma^\dagger + \Sigma \chi^\dagger\right)
    }\\
    &+ L_6 \Tr{\chi \Sigma^\dagger + \Sigma \chi^\dagger}^2 
     + L_7 \Tr{\chi \Sigma^\dagger - \Sigma \chi^\dagger}^2
    + L_8 \Tr{ \left(\chi^\dagger \Sigma \right)^2 + \left(\chi \Sigma^\dagger \right)^2}
    + H_2 \Tr{\chi \chi^\dagger}.
\end{align}
%
Here, $L_i$ and $H_i$ are coupling constants called low energy constants (LEO).
The static Lagrangian in the pion-condensed phase is
%
\begin{align}
    \nonumber
    \Ell_4
    = {}&
    2(2L_1 + 2L_2 + L_3) \mu_I^4 \sin^4\alpha
    + 4  L_4 \left( 2 \bar m^2 \cos\alpha + m_S^2 \right) \mu_I^2\sin^2\alpha
    + 4 L_5 \bar m^2 \mu_I^2 \cos\alpha \sin^2 \alpha 
    \\ & 
    + 4 L_6 (2\bar m\cos\alpha + m_S)^2
    + 2 L_8 \left(2 \bar m^4 \cos2\alpha + 2 \Delta m^4 + m_S^4\right)
    + H_2 \left(2\bar m^4 + 2\Delta m^4 + m_S^4 \right).
\end{align}
%
All the terms in the NLO Lagrangian up to and including $\Oh((\pi/f)^2)$, calculated using CAS-software, are shown in \autoref{section: symbolic calculations}.

$L_i$ and $H_i$  are bare coupling constants, which are unobservable, but they are related to the observable renormalized coupling constants.
The renormalized coupling constants in the $\overline{\mathrm{MS}}$-scheme, as explained in \autoref{section:free scalar field}, are related to the bare couplings through
%
\begin{align}
    L_i 
    & = 
    L_i^r 
    -
    \frac{1}{2} \frac{\mu^{-2\epsilon}} {(4 \pi)^2}
    \left(\frac{1}{\epsilon} + 1 \right) \Gamma_i, \quad
    H_i = 
    H_i^r
    -  \frac{1}{2}  \frac{\mu^{-2\epsilon}}{(4 \pi)^2} 
    \left(\frac{1}{\epsilon} + 1 \right) \Delta_i .
\end{align}
%
Here, $\Gamma_i$ and $\Delta_i$ are numerical constants which are used to match the divergences.
These have been calculated for $\mu_I = 0$~\autocite{gasserChiralPerturbationTheory1985}.
They are independent of $\mu_I$, and we can therefore use them in this calculation.
They are
\begin{equation}
    \Gamma_1 = \frac{3}{32}, \quad
    \Gamma_2 = \frac{3}{16}, \quad
    \Gamma_3 = 0, \quad
    \Gamma_4 = \frac{1}{8}, \quad
    \Gamma_5 = \frac{3}{8}, \quad
    \Gamma_6 = \frac{11}{144}, \quad
    \Gamma_8 = \frac{5}{48}, \quad
    \Delta_2 = \frac{5}{24}.
\end{equation}
%
The bare coupling constants $L_i$ and $H_i$ are independent of our renormalization scale $\mu$.
From this we obtain the renormalization group equations for the running coupling constants,
\begin{equation}
    \mu \diff{L_i^r}{\mu } 
    = - \frac{  \mu^{-2\epsilon} }{(4 \pi)^2} \Gamma_i + \mathcal{O}(\epsilon), \quad
    \mu \diff{H_i^r}{\mu } 
    = - \frac{ \mu^{-2\epsilon}}{(4 \pi)^2} \Delta_i + \mathcal{O}(\epsilon).
\end{equation}
%
Inserting the solutions back into the bare coupling constants, we get to the first order in $\epsilon$
%
\begin{equation}
    L_i
    = 
    L_i^r(M)
    - \frac{1}{2} \frac{\mu^{-2\epsilon}} {(4 \pi)^2}
    \left(\frac{1}{\epsilon} + 1 + \ln{\frac{\tilde \mu^2}{M^2}}\right) \Gamma_i,
    \quad
    H_i
    = 
    H_i^r (M)
    - \frac{1}{2} \frac{\mu^{-2\epsilon}} {(4 \pi)^2}
    \left(\frac{1}{\epsilon} + 1 + \ln{\frac{\tilde \mu^2}{M^2}}\right) \Delta_i.
\end{equation}
%
$L_i^r(M)$ and $H_i^r(M)$ are constants of integration and must be measured.
This only applies if the numerical constants $\Gamma_i$/$\Delta_i$ are non-zero.
If they are zero, then the coupling is not running, and the measured value can be applied at all energies.
The values used in this text are given in \autoref{table: coupling constants}.
They are given at the $\rho$-meson mass, $m_\rho = 770\, \text{MeV}$.
%
\begin{table}
    \centering
    \def\arraystretch{1.2}
    \caption{The renormalized coupling constants, measured at $M = m_\rho$.}
    \label{table: coupling constants}
    \begin{tabular}{c c c}
        \hline \hline
        constant & value [$\times 10^{-3}$] & source \\
        \hline
        $L_1^r(m_\rho)$ & $\phantom{-}1.0 \pm 0.1 $ & \autocite{bijnensMesonicLowEnergyConstants2014} \\
        $L_2^r(m_\rho)$ & $-1.6 \pm 0.2 $ & \autocite{bijnensMesonicLowEnergyConstants2014} \\
        $L_3^r(m_\rho)$ & $-3.8 \pm 0.3 $ & \autocite{bijnensMesonicLowEnergyConstants2014} \\
        $L_4^r(m_\rho)$ & $\phantom{-}0.0 \pm 0.3 $ & \autocite{bijnensMesonicLowEnergyConstants2014} \\
        $L_5^r(m_\rho)$ & $\phantom{-}1.2 \pm 0.1 $ & \autocite{bijnensMesonicLowEnergyConstants2014} \\
        $L_6^r(m_\rho)$ & $\phantom{-}0.0 \pm 0.4 $ & \autocite{bijnensMesonicLowEnergyConstants2014} \\
        $L_8^r(m_\rho)$ & $\phantom{-}0.5 \pm 0.2 $ & \autocite{bijnensMesonicLowEnergyConstants2014} \\
        $H_2^r(m_\rho)$ & $-3.4 \pm 1.5 $ & \autocite{jaminFlavoursymmetryBreakingQuark2002} 
    \end{tabular}
\end{table}
%
After inserting the renormalized coupling constants into the Lagrangian, it can be written
%
\begin{align}
    \label{nlo static lagrangian}
    \nonumber
    \Ell_4
    = {}&
    2(2L_1^r + 2L_2^r + L_3^r) \mu_I^4 \sin^4\alpha
    + 4  L_4^r \left( 2 \bar m^2 \cos\alpha + m_S^2 \right) \mu_I^2\sin^2\alpha
    + 4 L_5^r (\bar m^2 \cos\alpha) (\mu_I^2  \sin^2 \alpha )
    \\ \nonumber
    & 
    + 4 L_6^r (2\bar m\cos\alpha + m_S)^2
    + 2 L_8^r \left(2 \bar m^4 \cos2\alpha + 2 \Delta m^4 + m_S^4\right)
    + H_2^r \left(2\bar m^4 + 2\Delta m^4 + m_S^4 \right)
    \\ \nonumber
    &
    - \frac{1}{2}\frac{\mu^{-2\epsilon}}{(4\pi)^2}
    \left( \frac{1}{\epsilon} + 1 + \ln\frac{\tilde \mu^2}{M^2} \right)
    \Bigg[
        \frac{37}{15} (\bar m^2 \cos\alpha)^2
        +\frac{5}{2} (\bar m^2 \cos\alpha)(\mu_I^2\sin^2\alpha)
        +\frac{9}{8} (\mu_I^2\sin^2\alpha)^2 \\
        &
        \quad\quad\quad\quad\quad\quad\quad\quad\quad\quad\quad\quad\quad
        +\frac{11}{9} (\bar m^2 \cos\alpha)m_S^2
        +\frac{1}{2} (\mu_I^2\sin^2\alpha)m_S^2
        +\frac{5}{6} \Delta m^4
        +\frac{13}{18} m_S^2
    \Bigg].
\end{align}




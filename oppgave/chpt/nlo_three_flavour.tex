\section{Next-to-leading order}
\label{section: nlo chpt}


To construct the next-to-leading order Lagrangian, one follows the same appraoch of combining the terms we found in \autoref{chapter: chpt} into locally $\Lie{SU}{3}_R\times\Lie{SU}{3}_L$-invariant terms.
We do not need to include terms with the field strength tensors $f_{\mu\nu}^{(r)}$ and $f_{\mu\nu}^{(l)}$, as in our case they will vanish, and we do not include the electromagnetic interaction.
With this, the NLO Lagrangian is~\autocite{gasserChiralPerturbationTheory1985}
%
\begin{align}
    \label{NLO Lagrangian three flavor}
    \nonumber
    \Ell_4 
    ={} &
    L_1\Tr{\nabla_\mu \Sigma \nabla^\mu \Sigma^\dagger}
    + L_2 \Tr{\nabla_\mu \Sigma \nabla_\nu \Sigma^\dagger} 
    \Tr{\nabla^\mu \Sigma \nabla^\nu \Sigma^\dagger}
    + L_3 \Tr{\left(\nabla_\mu \Sigma \nabla^\mu \Sigma^\dagger\right)^2} \\ \nonumber
    & + L_4 \Tr{\nabla_\mu \Sigma \nabla^\mu \Sigma^\dagger} 
    \Tr{\chi \Sigma^\dagger + \Sigma \chi^\dagger}
    + L_5 \Tr{
        \nabla_\mu \Sigma \nabla^\mu \Sigma^\dagger 
        \left(\chi \Sigma^\dagger + \Sigma \chi^\dagger\right)
    }\\
    &+ L_6 \Tr{\chi \Sigma^\dagger + \Sigma \chi^\dagger}^2 
     + L_7 \Tr{\chi \Sigma^\dagger - \Sigma \chi^\dagger}^2
    + L_8 \Tr{ \left(\chi^\dagger \Sigma \right)^2 + \left(\chi \Sigma^\dagger \right)^2}
    + H_2 \Tr{\chi \chi^\dagger}.
\end{align}
%
Here, $L_i$ and $H_i$ are the necessary free parameters, called low energy constants (LEO).
\todo[inline]{Include values and running of these couplings}.
Expanding to zeroth order in the pion condensate, this becomes
%
\begin{align}
    \nonumber
    \Ell_4
    = {}&
    2(2L_1 + 2L_2 + L_3) \mu_I^4 \sin^4\alpha
    + 4  L_4 \left( 2 \bar m^2 \cos\alpha + m_S^2 \right) \mu_I^2\sin^2\alpha
    + 4 L_5 \mu_I^2\sin^2 \alpha 
    \\ & 
    + 4 L_6 (2\bar m\cos\alpha + m_S)^2
    + 2 L_8 \left(2 \bar m^4 \cos2\alpha + 2 \Delta m + m_S\right)
    + H_1 \left(2\bar m^4 + 2\Delta m^4 + m_S^4 \right).
\end{align}
%
All the terms in the NLO Lagrangian, up and to including  $\Oh((\pi/f)^2)$, calculated using cas-software, is shown in \autoref{section: symbolic calculations}.


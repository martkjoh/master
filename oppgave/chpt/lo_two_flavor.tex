In this appendix, we give a short summary of some of the results from the specialization project on which this thesis builds.
The specialization project focused on the thermodynamics of \chpt\, but included only two flavors, i.e., quark types.
In addition, we give examples of some derivations which is readily generalized to three flavors.


\section{*Two-flavour \chpt\ to leading order}
\label{section: two-flavor chpt to leading order}


In this section, we will assume $N_f = 2$, which means the generators are $T_a = \frac{1}{2} \tau_a$, where $\tau_a$ are the Pauli Matrices, as described in \autoref{section: algebra bases}.
The leading order Lagrangian in Winberg's power counting scheme, with $e = 0$, is
%
%
\begin{equation}
    \label{leading order two-flavor chpt Lagrangian}
    \Ell_2 = 
    \frac{1}{4} f^2 \Tr{\nabla_\mu \Sigma (\nabla^\mu \Sigma)^\dagger}
    + \frac{1}{4} f^2 \Tr{\chi^\dagger \Sigma + \Sigma^\dagger \chi}.
\end{equation}
%
The covariant derivative is
%
\begin{align}
    \nabla_\mu \Sigma &= \partial_\mu \Sigma - i [v_\mu, \Sigma],
    \quad v_\mu = \frac{1}{2} \mu_I \delta_\mu^0 \tau_3,
\end{align}
%
and we have defined 
%
\begin{equation}
    \chi = 2 B_0 
    \begin{pmatrix}
        m_u & 0\\
        0 & m_d
    \end{pmatrix}
    =
    \begin{pmatrix}
        \bar m - \Delta m & 0\\
        0 & \bar m + \Delta m
    \end{pmatrix}.
\end{equation}
%
To incorporate a finite isospin density, we must parametrize the Goldstone manifold differently than in the vacuum.
We follow the analysis in~\autocite{adhikariTwoflavorChiralPerturbation2019}.
We assume the ground state is independent of space, $\pi_a(x) = \pi_a^0$, and write it as
%
\begin{equation}
    \Sigma_\alpha 
    :=
    \exp{i \alpha n_a \tau_a}
    = 
    \cos \alpha + i n_a \tau_a \sin \alpha,
\end{equation}
%
where
%
\begin{equation}
    \alpha = \frac{1}{f} \sqrt{\pi^0_a \pi^0_a}, \quad
    n_a = \frac{\pi^0_a}{\sqrt{\pi^0_a \pi^0_a}}.
\end{equation}
%
With this, the covariant derivative is $\nabla_\mu \Sigma_\alpha = - iv^a_\mu [\tau_a, \Sigma_\alpha]$, and the two terms in the first order Lagrangian are
%
\begin{align}
    \Tr{\nabla_\mu \Sigma_\alpha  (\nabla^\mu \Sigma_\alpha)^\dagger}
    & = 2 \mu_I^2 (n_1^2 + n_2^2) \sin^2 \alpha, \quad
    \Tr{\chi^\dagger \Sigma_\alpha + {\Sigma_\alpha}^\dagger \chi}
    = 4 \bar m^2 \cos \alpha.
\end{align}
%
%
We see that, to first order, all results are independent of $\Delta m$.
To find the new ground state, we minimize the Hamiltonian density.
With the assumption that the fields are constant, the first order Hamiltonian density is
%
\begin{equation}
    \He_2 = - \Ell_2 = 
    - f^2 
    \left[
        \bar m^2 \cos\alpha 
        + \frac{1}{2} \mu_I^2 (n_1^2 + n_2^2 ) \sin^2 \alpha
    \right].
\end{equation}
%
For $\mu_I = 0$, this is independent of $n_a$, and minimized by $\alpha = 0$.
Now, as $n_i n_i = 1$, we have that $n_1^2 + n_2^2 = 1 - n_3^2$.
This means that, for $\mu_I \neq 0$, the energy is minimized by $n_3 = 0$.
We can write $n_1 = \cos \phi$, $n_2 = \sin \phi$, for some real number $\phi$, which gives the ground state
%
\begin{equation}
    \Sigma_\alpha 
    = \one \cos \alpha  + i ( \tau_1 \cos \phi + \tau_2 \sin \phi) \sin \alpha .
\end{equation}
%
We can choose, without loss of generality, $\phi = 0$~\autocite{sonQCDFiniteIsospin2001}.
This corresponds to a change of basis of $\lie{su}{2}$, $\tau_1 \rightarrow \tilde \tau_1 = \tau_1 \cos \phi + \tau_2 \sin \phi$ and $\tau_2 \rightarrow \tilde \tau_2 = - \tau_1 \sin \phi + \tau_2 \cos \phi$.
With this, the new ground state is
%
\begin{equation}
    \label{general groundstate}
    \Sigma_\alpha = \exp{i \alpha \tau_1}.
\end{equation}
%
Any exited state is a transformation of the ground state by $\Lie{SU}{2}_A$.
For $\mu_I = 0$, this corresponds to 
%
\begin{equation}
    \Sigma(x) = U_R(x) \Sigma_0 U_L^\dagger(x) = U(x) \Sigma_0 U(x).
\end{equation}
%
where
%
\begin{equation}
    U(x) = \exp{i \frac{\tau_a\pi_a(x)}{2f}}.
\end{equation}
%
We see that this recovers the parametrization vacuum parametrization.
For $\mu_I \neq 0$, the ground state may be shifted, and so $U(x)$ must be too.
The groundstate transforms as
%
\begin{equation}
    \Sigma_0 \rightarrow \Sigma_\alpha 
    = \hat U_L \Sigma_0 \hat U_R^\dagger = A_\alpha \Sigma_0 A_\alpha.
\end{equation}
%
where
%
\begin{equation}
    A_\alpha : = \exp{i \frac{1}{2} \alpha \tau_1} 
    = \cos \frac{\alpha}{2} + i \tau_1 \sin\frac{\alpha}{2}.
\end{equation}
%
This induces the following transformations for the fluctuations,
%
\begin{align}
    U_L & \rightarrow \hat U_L U_L \hat U_L^\dagger = A_\alpha U_L A_\alpha^\dagger, \\
    U_R & \rightarrow \hat U_R U_R \hat U_R^\dagger = A_\alpha^\dagger U_R A_\alpha.
\end{align}
%
%
The new parameterization is thus
%
\begin{align}
    \label{sigma}
        \Sigma(x) = A_\alpha [U(x) \Sigma_0 U(x)] A_\alpha.
\end{align}
%
%
With this, we can expand the first order Lagrangian, \autoref{leading order two-flavor chpt Lagrangian}, in powers of $\pi/f$
We will use this expansion to calculate the free energy density.
Expanding $\Sigma$ to $\Oh \left((\pi/f)^5\right)$, we get
%
\begin{align}
    \nonumber
    \Sigma &=
     \left(
        1 
        - \frac{\pi_a^2}{2f^2}
        + \frac{\pi_a^2\pi_b^2}{24f^4}
    \right)
    (\cos{\alpha} + i \tau_1 \sin{\alpha})\\
    &\quad+
    \left(
        \frac{\pi_a}{f} 
        - \frac{\pi_b^2\pi_a}{6f^3} 
    \right)
    \left(
        i\tau_a - 2i \delta_{a1}\tau_1\sin^2{\frac{\alpha}{2}} - \delta_{a1} \sin{\alpha}
    \right).
    \label{expansion of sigma}
\end{align}
%
%

The kinetic term in the \chpt\, Lagrangian is
%
\begin{align}
    \nabla_\mu \Sigma (\nabla^\mu \Sigma)^\dagger 
    = \partial_\mu \Sigma \partial^\mu \Sigma^\dagger 
    - i \left(\partial_\mu \Sigma [v^\mu, \Sigma^\dagger] - \hc \right)
    - [v_\mu,\Sigma][v_\mu, \Sigma^\dagger].
    \label{kinetic term}
\end{align}
%
Using \autoref{expansion of sigma} we find the expansion of the constitutive parts of the kinetic term to be
%
\begin{align}
    \notag
    \partial_\mu \Sigma 
   & = 
    \left[
        \left(
            \frac{-1}{f^2}
            + \frac{\pi_b^2}{6f^4}
        \right)
        (\pi_a \partial_\mu \pi_a)
        \cos{\alpha}
        - 
        \left(
            \frac{\partial_\mu \pi_1}{f} 
            - \frac{\pi_b^2 \partial_\mu\pi_1
            + 2 \pi_1 \pi_b \partial_\mu\pi_b}{6f^3} 
        \right)
        \sin{\alpha}
    \right]
    \\ \notag 
    & \quad -
    \left[
        \left(
            \frac{-1}{f^2}
            + \frac{\pi_b^2}{6f^4}
        \right)
        (\pi_a \partial_\mu \pi_a)
        \sin{\alpha}
        - \left(
        \frac{\partial_\mu \pi_1}{f} 
        - \frac{\pi_b^2 \partial_\mu\pi_1
        + 2 \pi_1 \pi_b \partial_\mu\pi_b}{6f^3}
        \right)
        2 \sin^2{\frac{\alpha}{2}}
    \right]
    i \tau_1 
    \\ \label{Sigma derivative}
    & \quad  + 
    \left(
        \frac{\partial_\mu \pi_a}{f} 
        - \frac{\pi_b^2 \partial_\mu\pi_a 
        + 2 \pi_a \pi_b \partial_\mu\pi_b}{6f^3} 
    \right)
    i \tau_a,
\end{align}
%
%
and
%
\begingroup
\allowdisplaybreaks
\begin{align}
    \notag
    [v_\mu,\Sigma]
    =
    -\mu_I \delta^0_\mu
    \Bigg\{&
        \left[
        \left(
            1 
            - \frac{\pi_a^2}{2f^2}
            + \frac{\pi_a^2\pi_b^2}{24f^4}
        \right)
        \sin{\alpha}
        + 
        \left(
            \frac{\pi_1}{f} 
            - \frac{\pi_b^2\pi_1}{6f^3} 
        \right) \cos{\alpha}
        \right]
        \tau_2\\
        &-
        \left(
            \frac{\pi_2}{f} 
            - \frac{\pi_b^2\pi_2}{6f^3} 
        \right)
        \tau_1
    \Bigg\}.
    \label{sigma commutator}
\end{align}
\endgroup
%
Combining \autoref{Sigma derivative} and \autoref{sigma commutator} gives the following terms
%
\begin{align}
    % Term 1
    & \Tr{\partial_\mu \Sigma \partial^\mu \Sigma^\dagger}
    = \frac{2}{f^2} \partial_\mu \pi_a \partial^\mu \pi_a
    + \frac{2}{3f^4}
    \left[
        (\pi_a\partial_\mu \pi_a)(\pi_b\partial^\mu \pi_b)
        -        
        (\pi_a\partial_\mu \pi_b)(\pi_b\partial^\mu \pi_a)
    \right], \\
    % Term 2
    \nonumber
    -i  &\Tr{\partial^\mu\Sigma[v_\mu,\Sigma^\dagger] - \hc}
    \\\nonumber & \quad\quad\quad\quad
    =
    4 \mu_I \frac{\partial_0\pi_2}{f}
    + 8 \mu_I \frac{\pi_3}{3f^3}\sin{\alpha}(
        \pi_2 \partial_0 \pi_3 - \pi_3 \partial_0 \pi_2
        ) \sin{\alpha}
    \\ & \quad \quad \quad \quad
    \quad
    +
    \left(
        \frac{4\mu_I}{f^2} \cos{\alpha}
        - \frac{8 \mu_I\pi_1}{3f^3} \sin{\alpha}
        - \frac{4 \mu_I \pi_a \pi_a} {3f^4}\cos{\alpha} 
    \right) 
    (\pi_1\partial_0 \pi_2 - \pi_2 \partial_0 \pi_1), \\
    % Term 3
    \nonumber
    - & \Tr{[v_\mu,\Sigma][v^\mu,\Sigma^\dagger]}
    \\ & \quad\quad\quad\quad
    = \mu_I{}^2
    \bigg[
        2 \sin^2{\alpha}
        +
        \left(
            \frac{2}{f} 
            - \frac{4\pi_a \pi_a}{3 f^3} 
        \right)
        \pi_1  \sin{2\alpha}
        + \left(
            \frac{2}{f^2}
            - \frac{2 \pi_a \pi_a}{3 f^4} 
        \right)
        \pi_a \pi_b k_{ab}
    \bigg], 
    \\
    % Mass Term
    \hspace{-1.5cm}
    & \Tr{\chi^\dagger \Sigma + \Sigma^\dagger\chi}
    \\ \nonumber & \quad\quad\quad\quad
    = 
    \bar m^2 
    \Bigg(
        4 \cos{\alpha} 
        - \frac{4 \pi_1}{f} \sin{\alpha} 
        - \frac{2 \pi_a \pi_a}{f^2} \cos{\alpha}
        % \\ & \quad\quad\quad\quad\quad\quad\quad\quad\quad\quad
        + \frac{2 \pi_1 \pi_a \pi_a}{3 f^3} \sin{\alpha}
        + \frac{(\pi_a \pi_a)^2}{6 f^4}\cos{\alpha}
    \Bigg), 
\end{align}
%
where $k_{ab} =\delta_{a1} \delta_{b1} \cos{2\alpha}  + \delta_{a2}\delta_{b2}\cos^2{\alpha} - \delta_{a3}\delta_{b3} \sin^2{\alpha}$.
Notice that the mass term is independent of the difference in quark masses, $\Delta m$.
If we write the Lagrangian \autoref{leading order two-flavor chpt Lagrangian} as $\Ell_2 = \Ell_2^{(0)} + \Ell_2^{(1)} + \Ell_2^{(2)} +...$, where $\Ell_2^{(n)}$ contains all terms of order $(\pi/f)^n$, then the result of the series expansion is
%
\begin{align}
%%%%%%%%%%%%%%%%%%
%% zeroth-order %%
%%%%%%%%%%%%%%%%%%
\Ell_2^{(0)}
&  =
    f^2   
    \left(
        \bar m^2 \cos{\alpha}
        + \frac{1}{2} \mu^2 \sin^2{\alpha}
    \right),
    \label{L0}
\\
%%%%%%%%%%%%%%%%%%
%% first order %%
%%%%%%%%%%%%%%%%%%
\label{L1}
\Ell_2^{(1)}
& =
    f 
    (
        \mu_I^2\cos{\alpha}
        - \bar m^2
    ) \pi_1 \sin{\alpha}
    + f \mu_I \partial_0\pi_2 \sin{\alpha},
\\
%%%%%%%%%%%%%%%%%%
%% second-order %%
%%%%%%%%%%%%%%%%%%
\Ell_2^{(2)}
& =
    \frac{1}{2} \partial_\mu\pi_a\partial^\mu\pi_a
    + \mu_I \cos{\alpha} \left( \pi_1 \partial_0\pi_2 - \pi_2\partial_0\pi_1 \right)
    - \frac{1}{2} \bar m^2 \pi_a \pi_a \cos{\alpha}
    + \frac{1}{2} \mu_I ^2 \pi_a \pi_b k_{ab},
\label{L2}
\\
%%%%%%%%%%%%%%%%%%
%% third-order %%
%%%%%%%%%%%%%%%%%%
\notag
\Ell_2^{(3)}
& =
    \frac{\pi_a\pi_a \pi_1}{6f}
    (\bar m^2 \sin{\alpha}-2\mu_I{}^2 \sin{2\alpha})\\ \label{L3}
    &\quad
    -
    \frac{2 \mu_I}{3 f}
    \left[
        \pi_1(\pi_1 \partial_0\pi_2 - \pi_2\partial_0\pi_1)
        +
        \pi_3(\pi_3\partial_0\pi_2-\pi_2 \partial_0\pi_3)
    \right]
    \sin{\alpha},
\\
%%%%%%%%%%%%%%%%%%
%% fourth-order %%
%%%%%%%%%%%%%%%%%%
\notag
\Ell_2^{(4)}
& =
\frac{1}{6f^2}
\curly{
    \frac{1}{4} \bar m^2 (\pi_a\pi_a)^2 \cos{\alpha}
    -
    \left[
        (\pi_a \pi_a) (\partial_\mu \pi_b \partial^\mu \pi_b )
        - (\pi_a \partial_\mu \pi_a)(\pi_b \partial^\mu \pi_b )
    \right]
}
\\ &\quad
- \frac{\mu_I \pi_a\pi_a}{3f^2}
\left[
    \left(\pi_1\partial_0 \pi_2 - \pi_2 \partial_0 \pi_1\right)
    \cos{\alpha}
    + \frac{1}{2} \mu_I \pi_a \pi_b k_{ab}
\right].
\label{L4}
\end{align}
%



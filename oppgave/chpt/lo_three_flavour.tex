\section{Three-flavor \chpt\ to leading order}
\label{section: three-flavor chpt to leading order}



\subsection{Ground state}

For $N_f = 3$, the generators are $T_\alpha = \frac{1}{2} \lambda_\alpha$, where $\lambda_\alpha$ are the Gell-Mann matrices, as shown in \autoref{section: algebra bases}.
The mass matrix is now
%
\begin{equation}
    m = 
    \begin{pmatrix}
        m_u & 0 & 0 \\
        0 & m_d & 0 \\
        0 & 0 & m_s
    \end{pmatrix},
\end{equation}
%
and the charge matrix is
%
\begin{equation}
    Q = \frac{1}{3}
    \begin{pmatrix}
        2 & 0 & 0\\
        0 & -1 & 0\\
        0 & 0 & -1
    \end{pmatrix}
    = \frac{1}{2} \left( \lambda_3 + \frac{1}{\sqrt{3}} \lambda_8 \right).
\end{equation}
%
The leading order Lagrangian still has the same form,
%
\begin{equation}
    \Ell_2 
    = \frac{1}{4}f^2 \Tr{\nabla_\mu \Sigma \nabla^\mu \Sigma^\dagger}
    + \frac{1}{4}f^2 \Tr{\chi \Sigma^\dagger + \Sigma \chi^\dagger}
    + e^2 C \Tr{\Sigma Q \Sigma^\dagger Q},
\end{equation}
%
where
%
\begin{equation}
    \chi = 2B_0 m =  \frac{2}{3} M_1^2 \one + M_2^2 \lambda_3 + M_3^2 \lambda_8,
\end{equation}
%
and
%
\begin{equation}
    M_1^2 = B_0 (m_u + m_d + m_s), \quad
    M_2^2 = B_0 (m_u - m_d), \quad
    M_3^2 = \frac{1}{\sqrt 3} B_0 (m_u + m_d - 2m_S).
\end{equation}
%
To find the ground state, we start with $e = 0$.
The covariant derivative is then
%
\begin{equation}
    \nabla_\mu \Sigma = \partial_\mu \Sigma - i [v_\mu, \Sigma], \quad 
    v_\mu = \mu \delta^0_\mu,
\end{equation}
%
Here, $\mu$ is the chemical potential matrix,
%
\begin{equation}
    \mu = 
    \begin{pmatrix}
        \mu_u & 0 & 0 \\
        0 & \mu_d & 0 \\
        0 & 0 & \mu_s
    \end{pmatrix}
    = 
    \begin{pmatrix}
        \frac{1}{3}\mu_B + \frac{1}{2}\mu_I & 0 & 0 \\
        0 & \frac{1}{3}\mu_B - \frac{1}{2}\mu_I & 0 \\
        0 & 0 & \frac{1}{3}\mu_B - \frac{1}{2}\mu_S
    \end{pmatrix}
    = \frac{1}{3}(\mu_B - \mu_S) \one 
    + \frac{1}{2} \mu_I \lambda_3
    + \frac{1}{\sqrt{3}}\mu_S\lambda_8,
\end{equation}
%
where $\mu_B = \frac{3}{2}(\mu_u + \mu_d)$, $\mu_I = \mu_u - \mu_d $ and $\mu_S = \frac{1}{2}(\mu_u + \mu_d)-\mu_s$.
Here, $\mu_u$, $\mu_d$ and $\mu_s$ is the up, down and strange quark chemical potentials, while $\mu_B$, $\mu_I$ and $\mu_S$ is baryon, isospin and strangeness chemical potentials.

We need to parametrize the ground state, as we did in \autoref{subsection: parametrization}, and define
Let\todo[color=red]{Dette er feil!}
%
\begin{equation}
    \Sigma_\alpha 
    = \exp{i \alpha n_a \lambda_a} = \cos \alpha + i n_a \lambda_a \sin \alpha,
    \quad \alpha = \frac{1}{f} \sqrt{\pi_a^0 \pi_a^0}, \quad n_a = \frac{\pi_a^0}{\sqrt{\pi_b^0 \pi_b^0}}. 
\end{equation}
%
The relevant terms are then
%
\begin{align}
    \frac{1}{4} \Tr{\nabla_\mu \Sigma_\alpha \nabla^\mu \Sigma_\alpha^\dagger}
    & = \frac{1}{2} \sin^2\alpha
    \left[
        \mu_I^2 (n_1^2 + n_2^2) 
        + \frac{1}{4} (\mu_I + 2\mu_s)^2(n_4^2 + n_5^2)
        + \frac{1}{4} (\mu_I - 2\mu_s)^2(n_6^2 + n_7^2)
    \right]\\
    \frac{1}{4} \Tr{\chi \Sigma^\dagger + \Sigma \chi^\dagger} 
    & = M_1^2 \cos \alpha
\end{align}
%
We notice that both terms are independent of $\mu_B$.\todo[]{hvorfor?}
With this, the static Hamiltonian is
%
\begin{align}
    \He_0
    = -\frac{1}{2}f^2\sin^2\alpha
    \left[
        \mu_I^2a^2 + \mu_\Kpm^2 b^2  + \mu_\Ko^2 c^2
    \right]
    - f^2M_1^2 \cos\alpha
\end{align}
%
We have defined the chemical potentials $\mu_{K^{\pm}} = \frac{1}{2}(\mu_I + 2 \mu_S) = \mu_u - \mu_s$ and $\mu_{K^{\pm}} = \frac{1}{2}(\mu_I - 2 \mu_S) = -\mu_d + \mu_s$, 
and
%
\begin{equation}
    a^2 = n_1^2 + n_2^2, \quad
    b^2 = n_4^2 + n_5^2, \quad
    c^2 = n_6^2 + n_7^2, \quad
    a^2 + b^2 + c^2 = 1 - n_3^2 - n_8^2.
\end{equation}
%
All the terms with a square chemical potential factors are positive definite, which means that the Hamiltonian will always be minimized by $n_3 = n_8 = 0$.
Furhtermore, we can without loss of generality chose $n_1 = n_4 = n_6 = 0$.
This corresponds to changing basis of $\lie{su}{3}$.
Depending on the signs of $\mu_I$ and $\mu_S$, we must have either $b = 0$ or $c = 0$
If $\sgn(\mu_I) = \sgn(\mu_S)$, then $\mu_\Kpm > \mu_\Ko$, and $c = 0$.
Likewise, if $\sgn(\mu_I) = - \sgn(\mu_S)$, then $\mu_\Kpm < \mu_\Ko$, and $b = 0$.
To begin with, we assume the former.
Define $a^2 = \cos^2\beta$, which implies $b^2 = \sin^2\beta$.
The Hamiltonian density is then
%
\begin{equation}
    \He_0 =
    -\frac{1}{2} f^2
    \left[
        \mu_I^2 \cos^2\beta + \mu_\Kpm^2\sin^2\beta
    \right]
    \sin^2\alpha
    -
    f^2 M_1^2 \cos\alpha.
\end{equation}


The $\beta$ parameter is set, as $\alpha$, by minimizing $\He$.
We have
%
\begin{equation}
    \pdv{\He}{\beta} = \frac{1}{2} (\mu_I^2 - \mu_\Kpm^2) f^2 \sin^2\alpha\cos2\beta, \quad
    \pdv[2]{\He}{\beta} = f^2 (\mu_I^2 - \mu_\Kpm^2) \sin^2\alpha\sin2\beta.
\end{equation}
%
We see that, if we are in the pion condensate phase where $\alpha \neq 0$, the stationary points for $\beta$ are $0$ and $\pi/2$.
However, which one these that is a minimum depends on the sign of $\mu_I^2 - \mu_\Kpm^2$, as this determines the sign of the second derivative.
For $\mu_I^2 > \mu_\Kpm^2$, $\beta = 0$, while for $\mu_I^2 < \mu_\Kpm^2$ we have $\beta = \pi/2$.
The analysis for $\sgn(\mu_I) = -\sgn(\mu_S)$ is the same, only with $\mu_\Kpm$ changed to $\mu_\Ko$.
The different ground states are then
%
\begin{equation}
    \Sigma_0 = \one, \quad
    \Sigma_\pi = \exp{i\alpha \lambda_2}, \quad
    \Sigma_\Kpm = \exp{i\alpha \lambda_5}, \quad
    \Sigma_\Ko = \exp{i\alpha \lambda_7}.
\end{equation}
%
As we found for two flavors, this corresponds to a transformation of the vacuum to a new ground state by, $\Sigma_0 \rightarrow A_\alpha \Sigma_0 A_\alpha$.
We must therefore transform the excitations around ground state in the same way.
However, now the transformation depend on which phase we are in.
We therefore parametrize the fields as
%
\begin{align}
    \Sigma = A^i_\alpha [U(x) \Sigma_0 U(x)] A_\alpha^i, \quad
    U(x) = \exp{i \frac{\pi_a \lambda_a}{2 f}}, \quad
    A_\alpha^i = \exp{i \alpha \lambda_i}.
\end{align}
%
Here, there is no sum over $i$.
Rather, $i = 2$, $5$, or $7$, dependent on if we are in the pion condensate, the charged kaon condensate or neutral kaon condensate.




\subsection{Leading order}


We work in the pion condensate, with $e = 0$.
The relevant terms are then
%
\begin{align}
    \nonumber
    \frac{f^2}{8B_0}\Tr{\chi \Sigma^\dagger + \Sigma \chi^\dagger}
    & =
    - \frac{1}{4}(m_u + m_d)\cos\alpha (\pi_1^2 + \pi_2^2 + \pi_3^2)
    - \frac{1}{4} 
    \left[
        (m_u + m_s)\cos^2\frac{\alpha}{2} - m_d \sin^2\frac{\alpha}{2}
    \right](\pi_4^2 + \pi_5^2)\\ \nonumber
    &- 
    \left[
        (m_d + m_s)\cos^2\frac{\alpha}{2} - m_u \sin^2\frac{\alpha}{2}
    \right](\pi_6^2 + \pi_7^2) 
    + \frac{1}{12} 
    \left[
        (m_u + m_d + 2m_s) \cos\alpha + 2m_s
    \right] \pi_8^2 \\
    &-\frac{1}{2 \sqrt{3}} (m_u - m_d) \pi_3 \pi_8
    - \frac{1}{2}(m_u + m_d)\sin\alpha \pi_2
    + \frac{1}{2}(m_u + m_d)\cos\alpha + \frac{1}{4}m_s(\cos\alpha + 1)
\end{align}


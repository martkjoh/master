\section{Three-flavor \chpt\ to leading order}
\label{section: three-flavor chpt to leading order}



\subsection{Ground state}

For $N_f = 3$, the generators are $T_\alpha = \frac{1}{2} \lambda_\alpha$, where $\lambda_\alpha$ are the Gell-Mann matrices, as shown in \autoref{section: algebra bases}.
The mass matrix is now
%
\begin{equation}
    m = 
    \begin{pmatrix}
        m_u & 0 & 0 \\
        0 & m_d & 0 \\
        0 & 0 & m_s
    \end{pmatrix},
\end{equation}
%
and the charge matrix is
%
\begin{equation}
    Q = \frac{1}{3}
    \begin{pmatrix}
        2 & 0 & 0\\
        0 & -1 & 0\\
        0 & 0 & -1
    \end{pmatrix}
    = \frac{1}{2} \left( \lambda_3 + \frac{1}{\sqrt{3}} \lambda_8 \right).
\end{equation}
%
The leading order Lagrangian still has the same form,
%
\begin{equation}
    \Ell_2 
    = \frac{1}{4}f^2 \Tr{\nabla_\mu \Sigma \nabla^\mu \Sigma^\dagger}
    + \frac{1}{4}f^2 \Tr{\chi \Sigma^\dagger + \Sigma \chi^\dagger}
    + e^2 C \Tr{\Sigma Q \Sigma^\dagger Q},
\end{equation}
%
where
%
\begin{equation}
    \chi = 2B_0 m =  \frac{2}{3} M_1^2 \one + M_2^2 \lambda_3 + M_3^2 \lambda_8,
\end{equation}
%
and
%
\begin{equation}
    M_1^2 = B_0 (m_u + m_d + m_s), \quad
    M_2^2 = B_0 (m_u - m_d), \quad
    M_3^2 = \frac{1}{\sqrt 3} B_0 (m_u + m_d - 2m_S).
\end{equation}
%
To find the ground state, we start with $e = 0$.
The covariant derivative is then
%
\begin{equation}
    \nabla_\mu \Sigma = \partial_\mu \Sigma - i [v_\mu, \Sigma], \quad 
    v_\mu = \mu \delta^0_\mu,
\end{equation}
%
Here, $\mu$ is the chemical potential matrix,
%
\begin{equation}
    \mu = 
    \begin{pmatrix}
        \mu_u & 0 & 0 \\
        0 & \mu_d & 0 \\
        0 & 0 & \mu_s
    \end{pmatrix}
    = 
    \begin{pmatrix}
        \frac{1}{3}\mu_B + \frac{1}{2}\mu_I & 0 & 0 \\
        0 & \frac{1}{3}\mu_B - \frac{1}{2}\mu_I & 0 \\
        0 & 0 & \frac{1}{3}\mu_B - \frac{1}{2}\mu_S
    \end{pmatrix}
    = \frac{1}{3}(\mu_B - \mu_S) \one 
    + \frac{1}{2} \mu_I \lambda_3
    + \frac{1}{\sqrt{3}}\mu_S\lambda_8,
\end{equation}
%
where $\mu_B = \frac{3}{2}(\mu_u + \mu_d)$, $\mu_I = \mu_u - \mu_d $ and $\mu_S = \frac{1}{2}(\mu_u + \mu_d)-\mu_s$.
Here, $\mu_u$, $\mu_d$, and $\mu_s$ are the up, down, and strange quark chemical potentials, while $\mu_B$, $\mu_I$, and $\mu_S$ are the baryon, isospin, and strangeness chemical potentials.
The baryon number of all mesons, the $\pi_a$'s, is zero, and $\Sigma$ transforms as $\Sigma \rightarrow \Sigma$ under $U(1)_V$, the corresponding symmetry group.
As a consequence, any result will be independent of $\mu_B$.
We can also see this from the fact that $\mu_B$ only appears with the identity matrix $\one$ in $\mu$.
As a consequence, any dependence on $\mu_B$ in $\nabla_\mu \Sigma$ will vanish as $\one$ commutes with everything.



\subsection{Parametrization}

From the structure constants of $\lie{su}{3}$, \autoref{structure constants su(3)}, we see that we can create three independent $\lie{su}{2}$ sub-algebras.
We introduce the matrices
%
\begin{equation}
    \lambda_Q = \lambda_3 + \frac{1}{\sqrt{3}}\lambda_8, \quad
    \lambda_K = \lambda_3 - \frac{1}{\sqrt{3}}\lambda_8,
\end{equation}
%
which commute, i.e., $[\lambda_Q, \lambda_K] = 0$.
With this, we get the commutation relations
%
\begin{equation}
    [\lambda_i, \lambda_j] = 2i \epsilon_{ijk} \lambda_k,\quad
    ijk \in
    \begin{cases}
        &\{1, 2, 3\}\\ &\{4, 5, \lambda_Q\}\\ &\{6, 7, \lambda_K\}.
    \end{cases}
\end{equation}
%
We here define the Levi-Civita symbol by $\epsilon_{123} = \epsilon_{34Q} =\epsilon_{67K} = 1$.
To find the ground state, we define
%
\begin{equation}
    \Sigma_\alpha 
    = \exp{i \alpha n_a \lambda_a} = \cos \alpha + i n_a \lambda_a \sin \alpha,
    \quad \alpha = \frac{1}{f} \sqrt{\pi_a^0 \pi_a^0}, \quad n_a = \frac{\pi_a^0}{\sqrt{\pi_b^0 \pi_b^0}}. 
\end{equation}
%
For $\mu_S = 0$, we expect to recover the results from the two-flavor case, which corresponds to $n_1^2 + n_2^2 =1$, $n_a = 0$ for $i>2$.
As argued earlier, we may chose $n_1 = 0$ without loss of generality, in which case the ground state becomes
%
\begin{equation}
    \Sigma_\alpha^{\pipm} = \exp{i \alpha \lambda_2} = (\one - \lambda^2_2) + \lambda_2^2 \cos\alpha + i \lambda_2\sin\alpha.
\end{equation}
%
If we define $\mu_\Kpm = \frac{1}{2}(\frac{1}{2}\mu_I + \mu_S)$ and $\mu_\Ko = \frac{1}{2}(\frac{1}{2}\mu_I - \mu_S)$, then we can write the external currents corresponding to $\mu_I$ and $\mu_S$ as
%
\begin{equation}
    \frac{1}{2}\mu_I Q_I + \mu_S Q_8 = \mu_\Kpm Q_{\Kpm} + \mu_\Ko Q_\Ko.
\end{equation}
%
Analogously to how turning up $\mu_I$ leads to a condensate in the first $\lie{su}{2}$ subalgebra, we can expect these chemical potentials to lead to a condensation in their respective subalgebra.
We therefore choose the corresponding vacuums as
%
\begin{align}
    \Sigma_\alpha^{\Kpm} = \exp{i \alpha \lambda_5} = (\one - \lambda^2_5) + \lambda_5^2 \cos\alpha + i \lambda_5\sin\alpha, \\
    \Sigma_\alpha^{\K0} = \exp{i \alpha \lambda_7} = (\one - \lambda^2_7) + \lambda_7^2 \cos\alpha + i \lambda_7\sin\alpha.
\end{align}
%




\subsection{Leading order}


We work in the pion condensate, with $e = 0$.
The relevant terms are then
%
\begin{align}
    \nonumber
    \frac{f^2}{8B_0}\Tr{\chi \Sigma^\dagger + \Sigma \chi^\dagger}
    & =
    - \frac{1}{4}(m_u + m_d)\cos\alpha (\pi_1^2 + \pi_2^2 + \pi_3^2)
    - \frac{1}{4} 
    \left[
        (m_u + m_s)\cos^2\frac{\alpha}{2} - m_d \sin^2\frac{\alpha}{2}
    \right](\pi_4^2 + \pi_5^2)\\ \nonumber
    &- 
    \left[
        (m_d + m_s)\cos^2\frac{\alpha}{2} - m_u \sin^2\frac{\alpha}{2}
    \right](\pi_6^2 + \pi_7^2) 
    + \frac{1}{12} 
    \left[
        (m_u + m_d + 2m_s) \cosalpha + 2m_s
    \right] \pi_8^2 \\
    &-\frac{1}{2 \sqrt{3}} (m_u - m_d) \pi_3 \pi_8
    - \frac{1}{2}(m_u + m_d)\sin\alpha \pi_2
    + \frac{1}{2}(m_u + m_d)\cos\alpha + \frac{1}{4}m_s(\cos\alpha + 1)
\end{align}


\section{Chiral perturbation theory}
\label{section: chiral perturbation theory}

We now apply the theory we developed in \autoref{chapter: QFT}.
The systematics of chiral perturbation theory, or \chpt, was laid out by Gasser and Leutwyler~\autocite{gasserChiralPerturbationTheory1984,gasserChiralPerturbationTheory1985} and is based on Weinberg's idea that quantum field theories on their own does not contain more information than the bare minimum~\autocite{weinbergPhenomenologicalLagrangians1979}.
In addition to these paper, this section is based on~\autocite{eckerChiralPerturbationTheory1995,fearingExtensionChiralPerturbation1996,schererIntroductionChiralPerturbation2002}.


\subsection{*Non-linear realization}

To construct the Lagrangian of chiral perturbation theory we start with the Lagrangian of massless QCD, 
%
\begin{equation}
    \Ell^0_\text{QCD} = i \bar q \slashed D q - \frac{1}{4} G^\alpha_{\mu \nu} G_\alpha^{\mu \nu}
\end{equation}
%
As discussed in last section, this Lagrangian is invariant under the full symmetry group $G = \Lie{SU}{N_f}_R \times \Lie{SU}{N_f}_L$, but the system undergoes spontaneous symmetry breaking to the smaller group $H = \Lie{SU}{N_f}_V$.
As we found in \autoref{seciton: ccwz construction}, the low energy dynamics will therefore be described by a $G/H = \Lie{SU}{N_f}_A$-valued field $\Sigma$.
Let $g \in G$.
We write $g = (U_L, U_R)$, where $U_R \in \Lie{SU}{N_f}_R$, $U_L \in \Lie{SU}{N_f}_L$.
Elements in $H$ are then of the form $(U, U)$, while elements in $G$ are of the for $(U, U^\dagger)$.
A general element $g$ can be written as
%
\begin{equation}
    g = (U_L, U_R) = (1, U_R {U_L}^\dagger) (U_L, U_L).
\end{equation}
%
Since $(U_L, U_L) \in H$, this means that we can write the coset $g H$ as $(1, U_R {U_L}^\dagger)H$, which gives a way to choose a representative element for each coset.
We identify
%
\begin{equation}
    \Sigma = U_R {U_L}^\dagger. 
\end{equation}
%
This is our standard form for elements in $gH$.
As we saw in \autoref{seciton: ccwz construction}, it therefore implicitly define transformation properties of the Goldstone bosons, which is given by the function $h(g, \xi)$.
For $\tilde g \in G$, we have
%
\begin{equation}
    \tilde g (1, \Sigma)
    = (\tilde U_L, \tilde U_R) (1, U_R {U_L}^\dagger)
    = (1, \tilde U_R (U_R {U_L}^\dagger) \tilde {U_L}^\dagger) (\tilde U_L, \tilde U_L)
    = (1, \tilde U_R \Sigma \tilde U_L) \tilde h.
\end{equation}
%
This gives the transformation rule
\begin{equation}
    \Sigma \rightarrow \Sigma' = U_R \Sigma {U_L}^\dagger.
\end{equation}
%
This gives simple transformation rules for $(U, U) \in H$ and $(U, U^\dagger) \in G/H$,
\begin{align}
    \label{sigma transform under H}
    H:& \quad \Sigma \rightarrow \Sigma' = U \Sigma U^\dagger, \\
    \label{sigma transform under G/H}
    G/H:& \quad \Sigma \rightarrow \Sigma' = U \Sigma U.
\end{align}
%
Due to how $G$ factors into two Lie groups, the constituents of the Mauer-Cartan form are 
\todo[]{hvorfor?}
%
\begin{equation}
    d_\mu = i \Sigma(x)^\dagger \partial_\mu \Sigma(x),\quad
    e_\mu = 0.
\end{equation}
%
We can now create $G$-invariant terms by taking traces of $d_\mu$'s.
As we will discuss in \autoref{subsection: Weinberg's power counting scheme}, the order of a term in the Lagrangian will be dependent on the number of $d_\mu$'s.
As $d_\mu \in \lie{su}{N_f}$, which we represent by the traceless matrices, the lowest order term is trivial,
%
\begin{equation}
    \Tr{d_\mu} = 0.
\end{equation}
%
Using $\partial_\mu [\Sigma(x)^\dagger\Sigma(x)] = 0 $, we can write
\begin{equation}
    d_\mu d_\nu = 
    - \Sigma(x)^\dagger [\partial_\mu \Sigma(x)] \Sigma(x)^\dagger [\partial_\nu \Sigma(x)]
    =\Sigma(x)^\dagger [\partial_\mu \Sigma(x)] [\partial_\nu \Sigma(x)^\dagger] \Sigma(x).
\end{equation}
%
This leaves us with the single Lorentz invariant leading order term,
\begin{equation}
    \Tr{d_\mu d^\mu} = \Tr{\partial_\mu \Sigma (\partial^\mu \Sigma)^\dagger},
\end{equation}


However, constructing the effective Lagrangian out of terms invariant under $G$ is too restrictive to get the most general effective action.
This only allows for an even number of $d_\mu$'s, and observed processes such as the decay of the neutral pion through $\pi^0 \rightarrow \gamma \gamma$ would not be possible~\cite{schererIntroductionChiralPerturbation2002}.
This is because we have not allowed for terms that change the Lagrangian with a divergence term, as discussed in \autoref{section: symmetry and goldstone's theorem}.
Terms of this type are called Wess-Zumino-Witten (WZW) terms~\cite{weinbergQuantumTheoryFields1996}.
We will not consider these here, as they do not affect the thermodynamic quantities in question~\cite{adhikariTwoflavorChiralPerturbation2019}.

\subsection{External currents}


As discussed in \autoref{section: effective field theories}, we can incorporate external currents and symmetry breaking terms by promoting the symmetry $G$ to a gauge symmetry, treating the external currents as gauge fields, and demanding gauge invariance of the effective Lagrangian.
The external currents may couple to conserved currents, \autoref{conserved currents qcd}, or the other bilinears we can create out of quarks, $\bar q q$, $\bar q\gamma^5 q$, $\bar q T_\alpha q$, and $\bar q T_\alpha \gamma^5 q$.
The Lagrangian of these external currens is
%
\begin{align}
    \Ell_{\text{ext}}
    = - \bar q \left(s - i \gamma^5 p \right) q
    + \bar q \gamma^\mu  \left(v_\mu + \gamma^5 a_\mu\right)q.
\end{align}
%
\todo{Hva er grunnen til valgene av fortegn?}
Here, $s$, $p$, $v_\mu$ and $a_\mu$ are all $N_f\times N_f$ matrices acting on the flavor indices.
They are, respectively, the scalar, pseudo-scalar, vector, and pseudo-vector currents.
We denote these currents collectively as $j = (s, p, v^\mu, a^\mu)$.
The masses of the quarks are accounted for by setting the scalar current $s = m + \tilde s$.
Here, $m$ is the mass matrix of the quarks, while $\tilde s$ are possible other scalar currents.
Other examples of external currents are chemical potentials, such as the isospin chemical potential, which regulate conserved charges in the system.
We now need to find the transformation properties of these currents under $G$.
We define
%
\begin{equation}
    r_\mu = v_\mu + a_\mu, \quad l_\mu = v_\mu - a_\mu, \quad  
    \chi = 2 B_0 (s +ip), \quad
    \chi^\dagger = 2 B_0(s - ip).
\end{equation}
%
By making a local $G$-transformation and enforcing gauge-invariance, we find that these transform as
%
\begin{align}
    r_\mu &\rightarrow U_R (r_\mu + i\partial_\mu) U_R^\dagger, \\
    l_\mu &\rightarrow U_L (l_\mu + i\partial_\mu) U_L^\dagger, \\
    \chi &\rightarrow U_R \chi {U_L}^\dagger.
\end{align}
%
As in Yang-Mills theory, we can now create field strength tensors of the gauge fields, to build more gauge-invariant terms.
We define
%
\begin{equation}
    f_{\mu \nu}^{(r)} 
    = 
    \partial_\mu r_\nu - \partial_\nu r_\mu - i[r_\mu, r_\nu], 
    \quad f_{\mu \nu}^{(l)} 
    = \partial_\mu l_\nu - \partial_\nu l_\mu - i[l_\mu, l_\nu].
\end{equation}
%
Including dynamical fields, such as the photon field $\mathcal A_\mu$, is slightly more complicated.
Quantum electrodynamics, or QED, is a gauge theory with a $\Lie{U}{1}_\text{EM}$ gauge group, where the covariant derivative acing on quarks is
%
\begin{equation}
    \label{EM covariant derivative on quarks}
    i\bar q \slashed D' q 
    = 
    i \bar q \gamma^\mu \left( \one \partial_\mu - i e Q \mathcal A_\mu\right) q
    =
    i \bar q \slashed \partial q - e \mathcal A_\mu J^\mu,
\end{equation}
%
Here, $\mathcal A_\mu$ is the photon field corresponding to the gauge group, $e = |e|$ is the elementary charge as given in \autoref{Elementary charge}, $J^\mu = - \bar q Q \gamma^\mu q$ is the electromagnetic charge current, and $Q$ is the quark charge matrix.
This matrix is the generator of $\Lie{U}{1}_\text{EM}$.
In the case of $N_f=3$,  $Q = \text{diag}(\frac{2}{3}, -\frac{1}{3}, -\frac{1}{3})$.
From \autoref{EM covariant derivative on quarks}, we see that $eQ\mathcal A_\mu$ is a vector current.
Although the transformation of the quarks under the electromagnetic gauge group can be seen as a subgroup of $G$, we \emph{do not} transform external currents to enforce gauge invariance, this is instead done by $\mathcal{A}_\mu$.
As $\mathcal{A}_\mu$ is a dynamical field, we can not use it to enforce $G$-gauge invariance.
However, if we treat the charge matrix $Q$ as an external field, then we can restore the invariance.
This gives the transformation rule
%
\begin{equation}
    Q_I \rightarrow U_I Q_I U_I^\dagger, \, I = R, L.
\end{equation}
%
Here, $Q_I = P_I Q$ are the chiral charge matrices.
With these external fields, we must introduce a covariant derivative acting on $\Sigma$ to enforce local $G$ invariance.
This is
%
\begin{equation}
    \nabla_\mu\Sigma = \partial_\mu \Sigma - ir'_\mu \Sigma + i \Sigma l'_\mu,
\end{equation} 
%
where $r'_\mu = r_\mu + eQ\mathcal{A}_\mu$ and $l'_\mu = l_\mu + eQ\mathcal{A}_\mu$.
We do not strictly \emph{need} to include the electromagnetic field in the gauge derivative, we could just build $G$ invariant terms of $eQ$, $\mathcal A_\mu$ and $\Sigma$, however this is the most economical way to achieve this.

Lastly, we must include terms from quantum electrodynamics involving only the photon field, which are
%
\begin{equation}
    \Ell^0_\text{QED}[\mathcal A] 
    = -\frac{1}{4} F^{\mu \nu}F_{\mu \nu}, \quad
    F_{\mu \nu} = 2 \partial_{[\mu}\mathcal A_{\nu]}.
\end{equation}
%
The full Lagrangian is then
%
\begin{equation}
    \Ell_\text{QCD}[q, \bar q, A,\mathcal A, j] = \Ell_\text{QCD}^0[q, \bar q, A] + \Ell^0_\text{QED}[\mathcal A] + \Ell_\text{ext}[\mathcal A, j].
\end{equation}
%
We now define the effective Lagrangian of $\chpt$, $\Ell_\text{eff}$ as
%
\begin{equation}
    \label{definition effective Lagrangian chpt}
    Z[j]
    = 
    \int \D q \D \bar q \D A \D \mathcal A\,
    \exp{i\int \dd^4 x \Ell_\text{QCD}[q, \bar q, A, \mathcal A, j] }
    = 
    \int \D \pi \D \mathcal A\,
    \exp{i\int \dd^4 x \Ell_\text{eff}[\varphi, \mathcal{A}, j] }.
\end{equation}


\subsection{Weinberg's power counting scheme}
\label{subsection: Weinberg's power counting scheme}
\todo[inline]{Skriv/kopier tekst om Weinberg's power counting scheme}

\section{Chiral perturbation theory}
\label{section: chiral perturbation theory}

The systematics of chiral perturbation theory, or \chpt, was laid out by Gasser and Leutwyler~\autocite{gasserChiralPerturbationTheory1984,gasserChiralPerturbationTheory1985} and is based on Weinberg's idea that quantum field theories on their own does not contain more information than the bare minimum~\autocite{weinbergPhenomenologicalLagrangians1979}.
In addition to these paper, this section is base on~\autocite{eckerChiralPerturbationTheory1995,fearingExtensionChiralPerturbation1996,schererIntroductionChiralPerturbation2002}.


\subsection{*Non-linear realization}

To construct the Lagrangian of chiral perturbation theory we start with the Lagrangian of massless QCD, 
%
\begin{equation}
    \Ell^0_\text{QCD} = i \bar q \slashed D q - \frac{1}{4} G^\alpha_{\mu \nu} G_\alpha^{\mu \nu}
\end{equation}
%
As discussed in last section, this Lagrangian is invariant under the full symmetry group $G = \Lie{U}{1}_V \times \Lie{SU}{N_f}_R \times \Lie{SU}{N_f}_L$, but the system undergoes spontaneous symmetry breaking to the smaller group $H = \Lie{SU}{N_f}_V$.
As we found in \autoref{seciton: ccwz construction}, the low energy dynamics will therefore be described by a $G/H = \Lie{SU}{N_f}_A$-valued field $\Sigma$.
Let $g \in G$.
We write $g = (U_L, U_R)$, where $U_R \in \Lie{SU}{N_f}_R$, $U_L \in \Lie{SU}{N_f}_L$.
Elements $h \in H$ are then of the form $h = (U, U)$.
A general element $g$ can be written as
%
\begin{equation}
    g = (U_L, U_R) = (1, U_R U_L^\dagger) (U_L, U_L).
\end{equation}
%
Since $(U_L, U_L) \in H$, this means that we can write the coset $g H$ as $(1, U_R U_L^\dagger)H$, which gives a way to choose a representative element for each coset.
We identify
%
\begin{equation}
    \Sigma = U_R U_L^\dagger. 
\end{equation}
%
This is our standard form for elements in $gH$.
As we saw in \autoref{seciton: ccwz construction}, it therefore implicitly define transformation properties of the Goldstone bosons, which is given by the function $h(g, \xi)$.
For $\tilde g \in G$, we have
%
\begin{equation}
    \tilde g (1, \Sigma)
    = (\tilde U_L, \tilde U_R) (1, U_R U_L^\dagger)
    = (1, \tilde U_R (U_R U_L^\dagger) \tilde U_L^\dagger) (\tilde U_L, \tilde U_L)
    = (1, \tilde U_R \Sigma \tilde U_L) \tilde h.
\end{equation}
%
This gives the transformation rule
\begin{equation}
    \Sigma \rightarrow \Sigma' = U_R \Sigma U_L^\dagger.
\end{equation}
%
Under transformations by $h = (U, U^\dagger) \in H$, we have
\begin{equation}
    \label{sigma transform under H}
    \Sigma \rightarrow \Sigma' = U \Sigma U^\dagger.
\end{equation}
%
Due to how $G$ factors into two Lie groups, the constituents of the Mauer-Cartan form are 
\todo[]{hvorfor?}
%
\begin{equation}
    d_\mu = i \Sigma(x)^\dagger \partial_\mu \Sigma(x),\quad
    e_\mu = 0.
\end{equation}
%
Using $\partial_\mu [\Sigma(x)^\dagger\Sigma(x)] = 0 $, we can write
\begin{equation}
    d_\mu d^\mu = 
    - \Sigma(x)^\dagger [\partial_\mu \Sigma(x)] \Sigma(x)^\dagger [\partial^\mu \Sigma(x)]
    =\Sigma(x)^\dagger [\partial_\mu \Sigma(x)] [\partial^\mu \Sigma(x)^\dagger] \Sigma(x).
\end{equation}
%
In \autoref{seciton: ccwz construction}, we found the lowest order terms, \autoref{first order terms CCWZ}.
As $d_\mu \in \lie{su}{N_f}$, which we represent by the traceless matrices, we have
%
\begin{equation}
    \Tr{d_\mu} = 0.
\end{equation}
%
This leaves us with the single leading order term
\begin{equation}
    \Tr{d_\mu d^\mu} = \Tr{\partial_\mu \Sigma (\partial^\mu \Sigma)^\dagger},
\end{equation}
%
where we have used the cyclic property of the trace.


However, constructing the effective Lagrangian out of terms invariant under $G$ is too restrictive to get the most general effective action.
This only allows for an even number of $d_\mu$'s, and observed processes such as the decay of the neutral pion through $\pi^0 \rightarrow \gamma \gamma$ would not be possible~\cite{schererIntroductionChiralPerturbation2002}.
This is because we have not allowed for terms that change the Lagrangian with a divergence term, as discussed in \autoref{section: symmetry and goldstone's theorem}.
Terms of this type are called Wess-Zumino-Witten (WZW) terms~\cite{weinbergQuantumTheoryFields1996}.
We will not consider these here, as they do not affect the thermodynamic quantities in question~\cite{adhikariTwoflavorChiralPerturbation2019}.

\subsection{External currents}


To incorporate other fields or terms that break $G$, such as the quark masses, we add a Lagrangian containing external currents.
These can include either couple to the conserved currents, \autoref{conserved currents qcd}, or the other bilinears we can create out of quarks, $\bar q q$, $\bar q\gamma^5 q$, $\bar q T_\alpha q$, and $\bar q T_\alpha \gamma^5 q$.
The Lagrangian is
%
\begin{align}
    \Ell_{\text{ext}}
    = - \bar q \left(s - i \gamma^5 p \right) q
    + \bar q \gamma^\mu  \left(v_\mu + \gamma^5 a_\mu\right)q.
\end{align}
%
\todo{Hva er grunnen til valgene av fortegn?}
Here, $s$, $p$, $v_\mu$ and $a_\mu$ are all $N_f\times N_f$ matrices acting on the flavor indices.
They are, respectively, the scalar, pseudo-scalar, vector, and pseudo-vector currents.
We denote these currents collectively as $j = (s, p, v^\mu, a^\mu)$.
The masses of the quarks are accounted for by setting the scalar current $s = m + \tilde s$.
Here, $m$ is the mass matrix of the quarks, while $\tilde s$ are possible other scalar currents.
Other examples of external currents are chemical potentials, such as the isospin chemical potential, which regulate conserved charges in the system.

We define the right handed and left handed currents as
\begin{equation}
    r_\mu = v_\mu + a_\mu, \quad l_\mu = v_\mu - a_\mu
\end{equation}
%

When we 


Including dynamical fields, such as the photon field $\mathcal A_\mu$, is slightly more complicated.
The electromagnetic interactions is a gauge theory with gauge group  $\Lie{U}{1}_\text{EM}$, a subgroup of $G$.
The electromagnetic covariant derivative acing on quarks is
%
\begin{equation}
    \label{EM covariant derivative on quarks}
    i\bar q \slashed D' q 
    = 
    i \bar q \gamma^\mu \left( \one \partial_\mu - i e Q \mathcal A_\mu\right) q
    =
    i \bar q \slashed \partial q - e \mathcal A_\mu J^\mu,
\end{equation}
where $\mathcal A_\mu$ is the photon field corresponding to the gauge group, $e = |e|$ is the elementary charge as given in \autoref{Elementary charge}, $J^\mu = - \bar q Q \gamma^\mu q$ is the electromagnetic charge current, and $Q$ is the quark charge matrix.
This matrix is the generator of $\Lie{U}{1}_\text{EM}$, and thus is a part of Lie algebra corresponding to $G$. 
In the case of $N_f=3$,  $Q = \text{diag}(\frac{2}{3}, -\frac{1}{3}, -\frac{1}{3})$, while the last term is not included for $N_f=2$.
From \autoref{EM covariant derivative on quarks}, we see that $eQ\mathcal A_\mu$ is a vector current.
We therefore include it by setting $ v^{\mu} = e Q \mathcal{A}^\mu + \tilde v^\mu$, where again $\tilde v^\mu$ are possible other vector currents.

\todo[inline]{skriv om infføringen av $Q_I$, $I = R, L$}

Lastly, we must include terms from quantum electrodynamics involving only the photon field, which are
%
\begin{equation}
    \Ell^0_\text{QED}[\mathcal A] 
    = -\frac{1}{4} F^{\mu \nu}F_{\mu \nu}, \quad
    F_{\mu \nu} = 2 \partial_{[\mu}\mathcal A_{\nu]}.
\end{equation}
 

The full Lagrangian is then
%
\begin{equation}
    \Ell_\text{QCD}[q, \bar q, A,\mathcal A, j] = \Ell_\text{QCD}^0[q, \bar q, A] + \Ell^0_\text{QED}[\mathcal A] + \Ell_\text{ext}[\mathcal A, j].
\end{equation}
%
We now define the effective Lagrangian of $\chpt$, $\Ell_\text{eff}$ as
%
\begin{equation}
    \label{definition effective lagrangian chpt}
    Z[j]
    = 
    \int \D q \D \bar q \D A \D \mathcal A\,
    \exp{i\int \dd^4 x \Ell_\text{QCD}[q, \bar q, A, \mathcal A, j] }
    = 
    \int \D \pi \D \mathcal A\,
    \exp{i\int \dd^4 x \Ell_\text{eff}[\pi, \mathcal{A}, j] }.
\end{equation}


\subsection{Winberg's power counting scheme}
\todo[inline]{Skriv/kopier tekst om Weinberg's power counting scheme}



\subsection{Building blocks}

Covariant derivative
%
\begin{equation}
    \nabla_\mu\Sigma = \partial_\mu \Sigma - ir_\mu \Sigma + i \Sigma l_\mu.
\end{equation}
%
Scalar
%
\begin{equation}
    \chi = 2 B_0 (s + ip), \quad \chi^\dagger = 2 B_0 (s - ip)  
\end{equation}
%
Field strength tensor
%
\begin{equation}
    f_{\mu \nu}^{(r)} = \partial_\mu r_\nu - \partial_\nu r_\mu - i[r_\mu, r_\nu], 
    \quad f_{\mu \nu}^{(l)} = \partial_\mu l_\nu - \partial_\nu l_\mu - i[l_\mu, l_\nu].
\end{equation}
%
Transformations under $G = \Lie{SU}{N_f}_R \times \Lie{SU}{N_f}_L \times \Lie{U}{1}_V$, where
\begin{align}
    U_I = P_I \exp{-i \eta_\alpha T_\alpha }, \quad I = R, L,  \quad U_V = \exp{- i \theta}
\end{align}
%
%
\begin{align}
    \Sigma &\rightarrow U_R \Sigma U_L^\dagger, \\
    r_\mu &\rightarrow U_V U_R (r_\mu + i\partial_\mu) U_R^\dagger U_V^\dagger
    = U_R^\dagger (r_\mu + i \partial_\mu) U_R^\dagger - \partial_\mu \theta, \quad 
    r, R \rightarrow l, L. \\
    \chi &\rightarrow U_R \chi U_L^\dagger \\
    Q_I &\rightarrow U_I Q_I U_I^\dagger, \, I = R, L.
\end{align}
%
We count $\chi$ as order 2, $e$ as order 2 and $\nabla_\mu\Sigma$ as order 1.
Notice that $e$ and $Q$ must always appear as $e Q$, as the orignal Lagrangian \autoref{definition effective lagrangian chpt} is invariant under the transformation $e \rightarrow e/\lambda$ and $Q \rightarrow \lambda Q$~\autocite{pencoIntroductionEffectiveField2020}.
\section{Chiral perturbation theory}

Chiral perturbation theory, or \chpt, is an effective field theory which exploits the chiral symmetries of QCD to describe its low energy dynamics.
The basis of \chpt\ for chiral perturbation theory and is the massless QCD Lagrangian,
%
\begin{equation}
    \Ell^0_\text{QCD} = i \bar q \slashed D q - \frac{1}{4} G^\alpha_{\mu \nu} G_\alpha^{\mu \nu}
\end{equation}
%
This Lagrangian is invariant under the full symmetry group $G = \Lie{U}{1}_V \times \Lie{SU}{N_f}_V \times \Lie{SU}{N_f}_A$.
To incorporate other fields or terms that break $G$, such as the quark masses, we add a Lagrangian containing external currents are\todo{Hva er grunnen til valgene av fortegn?}
%
\begin{align}
    \Ell_{\text{ext}}
    = - \bar q \left(s - i \gamma^5 p \right) q
    + \bar q \gamma^\mu  \left(v_\mu + \gamma^5 a_\mu\right)q.
\end{align}
%
The external sources are defined as
%
\begin{equation}
    s = s_0\one + s_a \tau_a, \quad
    p = p_0\one + p_a \tau_a, \quad
    v^\mu = v_0^\mu \one + \frac{1}{2}v_a^\mu \tau_a, \quad
    a^\mu = a_0^\mu \one + \frac{1}{2}a^\mu_a \tau_a.
\end{equation}
%
\todo{Generaliser til SU(N)}
which are respectively the scalar, pseudo-scalar, vector, and pseudo-vector currents.
We denote these currents collectively as $j = (s, p, v^\mu_a, a_a^\mu)$.

\todo[inline]{fiks resten av seksjon}
The masses of the quarks are accounted for by setting the scalar current $s_0$ equal the mass matrix of the quarks,
%
\begin{equation}
    \label{mass matrix quarks}
    m_q =
    \begin{pmatrix}
        m_u & 0  \\
        0 & m_d
    \end{pmatrix}
\end{equation}
%
These field can be static background fields, as is the case for the mass contribution to $s_0$, or a dynamical field such as the photon field.
Let $j_s$ denote static fields, while $j_s$ denote dynamical fields, so that $j = j_s + j_d$.
The dynamical fields migh have their own Lagrangian with terms independent of quarks, $\Ell_d[j_d]$.
In the case where the photon field is included as a dynamical field, this will have a $-\frac{1}{4}F_{\mu \nu}F^{\mu \nu}$ term.
\todo{Er dette riktig?}
The full Lagrangian is then
%
\begin{equation}
    \Ell_\text{QCD}[q, \bar q, A, j] = \Ell_\text{QCD}^0[q, \bar q, A] + \Ell_\text{ext}[j] + \Ell_d[j_d].
\end{equation}


In the grand canonical ensemble, as discussed in (ref termisk feltteori), we introduce a chemical potential $\mu$ and couple it to a conserved charge.
We are interested in the case where the chemical potential of the third component of isospin is, denoted $\mu_I$, is non-zero.
This corresponds to a modification of the Lagrangian by
%
\begin{equation}
    \Ell \rightarrow \Ell + \mu_I \frac{1}{2} \bar q \gamma_0\tau_3 q,
\end{equation}
%
which corresponds to an external vector current
%
\begin{equation}
    v^\mu_I = \frac{1}{2} \mu_I  \delta^\mu_0 \tau_3.
\end{equation}
%

The electromagnetic interactions is a gauge theory as well, and the electromagnetic covariant derivative acing on quarks is
%
\begin{equation}
    i\bar q \slashed D' q 
    = 
    i \bar q \gamma^\mu \left( \one \partial_\mu - i e Q \mathcal A_\mu\right) q
    =
    i \bar q \slashed \partial q - e \mathcal A_\mu J^\mu,
\end{equation}
where $\mathcal A_\mu$ is the photon field corresponding to the $\Lie{U}{1}_\text{EM}$ gauge group, $e = |e|$ is the elementary charge as given in \autoref{Elementary charge}, $J^\mu = - \bar q Q \gamma^\mu q$ is the electromagnetic charge current, and $Q$ is the quark charge matrix,
%
\begin{equation}
    \label{quark charge matrix}
    Q = \frac{1}{3}
    \begin{pmatrix}
        2 & 0 \\
        0 & -1
    \end{pmatrix}
    = 
    \frac{1}{6} \one + \frac{1}{2}\tau_3.
\end{equation}
%
As with the chemical potential, this is accounted for by an external current vector current, \todo{Er dette riktig fortegn}
%
\begin{equation}
    v_\text{EM}^{\mu} = e Q \mathcal{A}^\mu.
\end{equation}
%
We define the right handed and left handed currents as
\begin{equation}
    r_\mu = v_\mu + a_\mu, \quad l_\mu = v_\mu - a_\mu
\end{equation}
%
We now define the effective Lagrangian of $\chpt$ as
%
\begin{equation}
    \label{definition effective lagrangian chpt}
    Z[j_s]
    = 
    \int \D q \D \bar q \D A \D j_d
    \exp{i\int \dd^4 x \Ell_\text{QCD}[q, \bar q, A, j] }
    = 
    \int \D \pi \D j_d
    \exp{i\int \dd^4 x \Ell_\text{eff}[\pi, \mathcal{A}, j] }.
\end{equation}


\subsection{Building blocks}

\begin{align}
    \Sigma = A_\alpha [U(x) \Sigma_0 U(x)] A_\alpha, \quad
    \Sigma_0 = \one,\,
    U = \exp{i\frac{\pi_a \tau_a}{2 f}},\,
    A_\alpha = \exp(i \alpha \tau_1).
\end{align}
%
Covariant derivative ($a_\mu = 0$)
%
\begin{equation}
    \nabla_\mu\Sigma = \partial_\mu \Sigma - ir_\mu \Sigma + i \Sigma l_\mu.
\end{equation}
%
Scalar:
%
\begin{equation}
    \chi = 2 B_0 (s + ip), \quad s = m_q, \, p = 0.
\end{equation}
%
Define $\bar m^2 = B_0 (m_u + m_d)$ and $\Delta m^2 = B_0(m_u - m_d)$, so that
%
\begin{equation}
    \label{chi definition}
    \chi = \bar m^2 \one + \Delta m^2 \tau_3,
\end{equation}
%
Field strength tensor
%
\begin{equation}
    f_{\mu \nu}^{(r)} = \partial_\mu r_\nu - \partial_\nu r_\mu - i[r_\mu, r_\nu], 
    \quad r\rightarrow l.
\end{equation}
%
EM + chemical potential:
%
\begin{equation}
    r_\mu = l_\mu = v_\mu 
    = 
    \frac{1}{2} \mu_I \delta^0_\mu \tau_3
    + e Q \mathcal{A}_\mu.
    \frac{1}{6} e \mathcal A^\mu 
    + \frac{1}{2}(e \mathcal A^\mu + \mu_I\delta^\mu_I) \tau_3.
\end{equation}
%
$G = \Lie{SU}{N_f}_R \times \Lie{SU}{N_f}_L \times \Lie{U}{1}_V$
Transformations:
\begin{align}
    g\in G&, \quad  g = g_R \times g_L \times g_V, \\
    g_I(q) &= (P_I U_I + P_{\bar{I}}) q = U_I q_I \quad
    g_I(\bar q) = \bar q (P_{\bar{I}} U_I^\dagger + P_I), \\
    g_V(q) & = U_V q, \, g_V(\bar q) = \bar q U_V^\dagger \\
    U_I &= P_I \exp{-i \eta_\alpha \frac{\tau_\alpha}{2} }, \quad I = R, L, \\
    U_V &= \exp(- i \theta)
\end{align}
Promoting $G$ to gauge group, to get gauge invariance we must have \todo{Check Q}
%
\begin{align}
    \Sigma &\rightarrow U_R \Sigma U_L^\dagger, \\
    r_\mu &\rightarrow U_V U_R (r_\mu + i\partial_\mu) U_R^\dagger U_V^\dagger
    = U_R^\dagger (r_\mu + i \partial_\mu) U_R^\dagger - \partial_\mu \theta, \quad 
    r, R \rightarrow l, L. \\
    \chi &\rightarrow U_R \chi U_L^\dagger \\
    Q_I &\rightarrow U_I Q_I U_I^\dagger, \, I = R, L.
\end{align}
%
We count $\chi$ as ortder 2, $e$ as order 2 and $\nabla_\mu$ as order 1.
Notice that $e$ and $Q$ must always appear as $e Q$, as the orignal Lagrangian \autoref{definition effective lagrangian chpt} is invariant under the transformation $e \rightarrow e/\lambda$ and $Q \rightarrow \lambda Q$~\autocite{pencoIntroductionEffectiveField2020}.



\section{Electromagnetic effects}

In this section, we will explore the effects of electromagnetism on chiral perturbation theory.
The leading order Lagrangian is then~\autocite{eckerRoleResonancesChiral1989,urechVirtualPhotonsChiral1995}
%
\begin{equation}
    \label{leading order lagrangian EM}
    \Ell_2^{\text{EM}}
    = 
    \frac{1}{4}f^2 
    \Tr{
        \nabla_\mu \Sigma \nabla^\mu \Sigma^\dagger
    }
    +
    \frac{1}{4}f^2 
    \Tr{
        \chi \Sigma^\dagger + \Sigma\chi^\dagger
    }
    +
    e^2 C
    \Tr{Q \Sigma Q \Sigma^\dagger}
\end{equation}
%
$Q$ is the quark charge matrix, \autoref{quark charge matrix}, $C$ and dimensionfull constant, and $\chi = 2B_0 m$, where $m$ is the quark mass matrix \autoref{mass matrix quarks}.



\subsection{Contribution to mass}


To find the electromagnetic effect on the pion mass, we assume $\mu_I = 0$.
We use the parametrization $\Sigma = \exp{i \pi_a \tau_a / f}$, and the covariant derivative is in this case
%
\begin{equation}
    \nabla_\mu \Sigma = \partial_\mu \Sigma - i e \mathcal A_\mu [Q, \Sigma].
\end{equation}
%
We expand to second order in $\pi_a/f$, which gives
%
\begin{align}
    f^2 \Tr{\nabla_\mu \Sigma \nabla^\mu \Sigma}
    &=
    2\partial_\mu \pi_a\partial^\mu \pi_a
    + 4 e \mathcal A^\mu (\pi_1 \partial_\mu \pi_2 -\pi_2 \partial_\mu \pi_1)
    + 4 e^2 \mathcal A^2 (\pi_1^2 + \pi_2^2), \\
    \Tr{\chi \Sigma^\dagger + \Sigma \chi^\dagger}
    & = 4 \bar m^2\left(1 - \frac{1}{2} \frac{\pi_a \pi_a}{f^2}\right), \\
    \Tr{Q \Sigma Q \Sigma^\dagger}
    & = \frac{5}{9} - \frac{\pi_1^2 + \pi_2^2}{f^2}.
\end{align}
%
Inserting this into \autoref{leading order lagrangian EM}, we get
%
\begin{equation}
    \Ell_2^\text{EM}
    = \bar m^2 f^2 + \frac{5}{9}e^2 C
    + \frac{1}{2}\partial_\mu \pi_a \partial^\mu \pi_a
    - \frac{1}{2} \bar m_\pm^2 (\pi_1^2 + \pi_2^2) 
    - \frac{1}{2}\bar m^2 \pi_3^2
    + e \mathcal A^\mu (\pi_1 \partial_\mu \pi_2 -\pi_2 \partial_\mu \pi_1)
    + e^2 \mathcal A^2 (\pi_1^2 + \pi_2^2).
\end{equation}
%
where
\begin{equation}
    \bar m_\pm^2 = \bar m^2 + 2\frac{e^2}{f^2}C.
\end{equation}
%
This is the electromagnetic contribution to the mass at leading order.
It only affects the $\pi_1, \pi_2$ pions, which are a linear combination of $\pi_\pm$, the charged pions.
To leading order, $\bar m = m_\pi$, the neutral pion mass, and $\bar m_{\pm} = m_{\pi_{\pm}}$
From the values listed in \autoref{section: units}, we find
%
\begin{equation}
    \label{EM mass contribtuion leading order}
    \Delta m_{\pm} := 2\frac{e}{f}\sqrt{C} = \sqrt{m_{\pi_\pm}^2 - m_{\pi}^2} = 35.50 \, \text{MeV}.
\end{equation}
%
This corresponds to $C = 0.3771 \, u_0 = 1.1649\cdot 10^8 \, \text{MeV}^4$.


\subsection{Tree level free energy at non-zero isospin density} 

\todo{Kan vi være sikre på at denne parametriseringen fremdeles er riktig?}
The zeroth-order expansion in $\pi/f$ is
%
\begin{equation}
    \Sigma = e^{i \alpha \tau_1} = \sin \alpha + i \tau_1 \cos \alpha.
\end{equation}
%
This gives the contributions
%
\begin{align}
    \Tr{\nabla_\mu \Sigma \nabla^\mu \Sigma^\dagger}
    & = 2 \sin^2\alpha\left(\mu_I^2 + 2 e \mu \mathcal A_0 + e^2\mathcal A^2 \right), \\
    \Tr{\chi \Sigma^\dagger + \Sigma \chi^\dagger}
    & =4 \bar m^2 \cos \alpha,\\
    \Tr{Q \Sigma Q \Sigma^\dagger}
    & =  \cos^2 \alpha - \frac{4}{9}.
\end{align}
%
The leading order contribution to the free energy is due to the stationary Lagrangian, that is $\pi_a = \mathcal A_\mu = 0$.
Inserting this into \autoref{leading order lagrangian EM}, we get \todo[color=red]{FIKS FAKTOR 2!!!}
%
\begin{equation}
    \label{static lagrangian with EM}
    \Ell 
    = f^2 \left[
        \frac{1}{2}\mu_I^2 \sin^2 \alpha + \bar m^2 \cos \alpha 
        + \Delta m^2_{\pi_\pm} \left(\cos^2 \alpha - \frac{4}{9}\right)
    \right].
\end{equation}
%


\section{Chiral perturbation theory}
\label{section: chiral perturbation theory}

We now apply the theory we developed in \autoref{chapter: QFT}.
The systematics of chiral perturbation theory, or \chpt, was laid out by Gasser and Leutwyler~\autocite{gasserChiralPerturbationTheory1984,gasserChiralPerturbationTheory1985} and is based on Weinberg's idea that quantum field theories on their own do not contain more information than the bare minimum~\autocite{weinbergPhenomenologicalLagrangians1979}.
In addition to these papers, this section is based on~\autocite{eckerChiralPerturbationTheory1995,fearingExtensionChiralPerturbation1996,schererIntroductionChiralPerturbation2002}.


\subsection{*Weinberg's power counting scheme}
\label{subsection: Weinberg's power counting scheme}


Our plan is now to use the results from \autoref{seciton: ccwz construction} to construct the most general Lagrangian of the Goldstone bosons due to the breaking of the QCD vacuum.
This, however, will result in a theory with an infinite number of free parameters, making it unwieldy.
We need an expansion scheme in order to compute observable perturbatively.
We are working in the low-energy limit, so it is natural to expand in pion momenta.
As we saw in \autoref{seciton: ccwz construction}, the terms in the Lagrangian will be made up of combinations of the terms $e_\mu$ and $d_\mu$ of the Maurer-Cartan form, $i\Sigma \partial_\mu \Sigma = e_\mu + d_\mu$.
Therefore, all terms in the effective Lagrangian will be proportional to a certain number of derivatives of the Goldston bosons, which Lorentz invariance demands to be even.

Consider the matrix element $\mathcal M$ for a given Feynman diagram with external pion lines with momenta $q$, where both the energies and momenta are less than or equal to some energy scale $Q$.
If we scale $Q\rightarrow tQ$, and consequently also the external momenta $q \rightarrow tq$, momentum conservation at each vertex ensures that each internal momentum $p$ of the diagram scales as $p \rightarrow tp$.
Assume this diagram is made up of $V_i$ copies of the vertex $i$, which contain $d_i$ derivatives.
Each of these vertices scale as $t^{d_i}$.
The propagators contribute a factor $p^{-2}$ and will therefore scale as $t^{-2}$, and the integration measure $\dd^4 p$ scales as $t^4$.
This means that a matrix element with $L$ loops and $I$ internal lines scales as
%
\begin{equation}
    \mathcal M(q) \rightarrow \mathcal M(t q) = t^D \mathcal M(q),
\end{equation}
%
where 
\begin{equation}
    D = \sum_i V_i d_i - 2 I + 4 L.
\end{equation}
%

$D$ is called the \emph{chiral dimension} of $\mathcal M$.
Using the formula \autoref{Number of loops} for number of loops in a Feynman diagram, we get
\begin{equation}
    D = \sum_i V_i(d_i - 2) + 2 L + 2.
\end{equation}
%
For low energy scales $Q$, the largest contribution will come from the matrix elements of the smallest chiral dimension $D$.
A general process will consist of a sum of matrix elements of different chiral dimensions.
We can expand this element in powers of the pion momenta by using $t$ as the expansion parameter.
The leading order term will be those where $L = 0$ and $d_i = 2$ so that $D = 2$.
This means that all tree-level contributions of the lowest chiral dimension are from terms in the Lagrangian with exactly two derivatives.
Next is $D = 4$, which contains both tree-level contributions from terms with $d_i = 4$ and a one-loop contribution from $d_i = 2$.
We therefore expand the effective Lagrangian as
\begin{equation}
    \Ell_\text{eff} = \Ell_2 + \Ell_4 + ...,
\end{equation}
%
where $\Ell_{2n}$ contains $2n$ derivatives.
This is equivalent to scaling the space-time coordinates as $x^\mu \rightarrow tx^\mu$, and expanding the Lagrangian in powers of $t$.

We must also allow for the fact that pions have non-zero mass, interact in a symmetry-breaking way with the electromagnetic field, and the possibility of finite external currents.
This is all implemented by external currents, which we also have to assume are small.
Any external pions are on-shell, so the pion mass $m_\pi$ must be less than the energy scale $Q$.
As we will see, this corresponds to scaling the quark masses as $m_q \rightarrow t^2 m_q$.
Similarly, $\mu_I$ must also be less than $Q$, which means that we scale it as $\mu_I\rightarrow t \mu_I$.
We include electromagnetic interaction by scaling the fundamental electric charge $e$ as $e\rightarrow te$~\autocite{urechVirtualPhotonsChiral1995}.
Following these rules, each term in the effective Lagrangian will have a well-defined chiral dimension $D$, ensuring a consistent series expansion.
The term $\Ell_{D}$ then contain all allowed terms that scale as $t^D$~\autocite{schererIntroductionChiralPerturbation2002,weinbergPhenomenologicalLagrangians1979,weinbergQuantumTheoryFields1996}.


\subsection{*Non-linear realization}

To construct the Lagrangian of chiral perturbation theory, we start with the Lagrangian of massless QCD,
%
\begin{equation}
    \Ell^0_\text{QCD} = i \bar q \slashed D q - \frac{1}{4} G^\alpha_{\mu \nu} G_\alpha^{\mu \nu}
\end{equation}
%
As discussed in last section, this Lagrangian is invariant under the full symmetry group $G = \Lie{SU}{N_f}_R \times \Lie{SU}{N_f}_L$, but the system undergoes spontaneous symmetry breaking to the smaller group $H = \Lie{SU}{N_f}_V$.
As we found in \autoref{seciton: ccwz construction}, the low energy dynamics will therefore be described by a $G/H = \Lie{SU}{N_f}_A$-valued field $\Sigma$.
Let $g \in G$.
We write $g = (U_L, U_R)$, where $U_R \in \Lie{SU}{N_f}_R$, $U_L \in \Lie{SU}{N_f}_L$.
Elements in $H$ are then of the form $(U, U)$, while elements in $G$ are of the for $(U, U^\dagger)$.
A general element $g$ can be written as
%
\begin{equation}
    g = (U_L, U_R) = (1, U_R {U_L}^\dagger) (U_L, U_L).
\end{equation}
%
Since $(U_L, U_L) \in H$, this means that we can write the coset $g H$ as $(1, U_R {U_L}^\dagger)H$, which gives a way to choose a representative element for each coset.
We identify
%
\begin{equation}
    \Sigma = U_R {U_L}^\dagger. 
\end{equation}
%
This is our standard form for elements in $gH$.
As we saw in \autoref{seciton: ccwz construction}, it therefore implicitly define transformation properties of the Goldstone bosons, which is given by the function $h(g, \xi)$.
For $\tilde g \in G$, we have
%
\begin{equation}
    \tilde g (1, \Sigma)
    = (\tilde U_L, \tilde U_R) (1, U_R {U_L}^\dagger)
    = (1, \tilde U_R (U_R {U_L}^\dagger) \tilde {U_L}^\dagger) (\tilde U_L, \tilde U_L)
    = (1, \tilde U_R \Sigma \tilde U_L) \tilde h.
\end{equation}
%
This gives the transformation rule
\begin{equation}
    \Sigma \rightarrow \Sigma' = U_R \Sigma {U_L}^\dagger.
\end{equation}
%
This gives simple transformation rules for $(U, U) \in H$ and $(U, U^\dagger) \in G/H$,
\begin{align}
    \label{sigma transform under H}
    H:& \quad \Sigma \rightarrow \Sigma' = U \Sigma U^\dagger, \\
    \label{sigma transform under G/H}
    G/H:& \quad \Sigma \rightarrow \Sigma' = U \Sigma U.
\end{align}
%
Due to how $G$ factors into two Lie groups, the constituents of the Mauer-Cartan form are 
\todo[]{hvorfor?}
%
\begin{equation}
    d_\mu = i \Sigma(x)^\dagger \partial_\mu \Sigma(x),\quad
    e_\mu = 0.
\end{equation}
%
We can now create $G$-invariant terms by taking traces of $d_\mu$'s.
As we will discuss in \autoref{subsection: Weinberg's power counting scheme}, the order of a term in the Lagrangian will be dependent on the number of $d_\mu$'s.
As $d_\mu \in \lie{su}{N_f}$, which we represent by the traceless matrices, the lowest order term is trivial,
%
\begin{equation}
    \Tr{d_\mu} = 0.
\end{equation}
%
Using $\partial_\mu [\Sigma(x)^\dagger\Sigma(x)] = 0 $, we can write
\begin{equation}
    d_\mu d_\nu = 
    - \Sigma(x)^\dagger [\partial_\mu \Sigma(x)] \Sigma(x)^\dagger [\partial_\nu \Sigma(x)]
    =\Sigma(x)^\dagger [\partial_\mu \Sigma(x)] [\partial_\nu \Sigma(x)^\dagger] \Sigma(x).
\end{equation}
%
This leaves us with the single Lorentz invariant leading order term,
\begin{equation}
    \Tr{d_\mu d^\mu} = \Tr{\partial_\mu \Sigma (\partial^\mu \Sigma)^\dagger},
\end{equation}


However, constructing the effective Lagrangian out of terms invariant under $G$ is too restrictive to get the most general effective action.
This only allows for an even number of $d_\mu$'s, and observed processes such as the decay of the neutral pion through $\pi^0 \rightarrow \gamma \gamma$ would not be possible~\cite{schererIntroductionChiralPerturbation2002}.
This is because we have not allowed for terms that change the Lagrangian with a divergence term, as discussed in \autoref{section: symmetry and goldstone's theorem}.
Terms of this type are called Wess-Zumino-Witten (WZW) terms~\cite{weinbergQuantumTheoryFields1996}.
We will not consider these here, as they do not affect the thermodynamic quantities in question~\cite{adhikariTwoflavorChiralPerturbation2019}.

\subsection{External currents}


As discussed in \autoref{section: effective field theories}, we can incorporate external currents and symmetry breaking terms by promoting the symmetry $G$ to a gauge symmetry, treating the external currents as gauge fields, and demanding gauge invariance of the effective Lagrangian.
The external currents may couple to conserved currents, \autoref{conserved currents qcd}, or the other bilinears we can create out of quarks, $\bar q q$, $\bar q\gamma^5 q$, $\bar q T_\alpha q$, and $\bar q T_\alpha \gamma^5 q$.
The Lagrangian of these external currens is
%
\begin{align}
    \Ell_{\text{ext}}
    = - \bar q \left(s - i \gamma^5 p \right) q
    + \bar q \gamma^\mu  \left(v_\mu + \gamma^5 a_\mu\right)q.
\end{align}
%
Here, $s$, $p$, $v_\mu$ and $a_\mu$ are all $N_f\times N_f$ matrices acting on the flavor indices.
They are, respectively, the scalar, pseudo-scalar, vector, and pseudo-vector currents.
We denote these currents collectively as $j = (s, p, v^\mu, a^\mu)$.
The masses of the quarks are accounted for by setting the scalar current $s = m + \tilde s$.
Here, $m$ is the mass matrix of the quarks, while $\tilde s$ are possible other scalar currents.
Other examples of external currents are chemical potentials, such as the isospin chemical potential, which regulate conserved charges in the system.
We now need to find the transformation properties of these currents under $G$.
We define
%
\begin{equation}
    r_\mu = v_\mu + a_\mu, \quad l_\mu = v_\mu - a_\mu, \quad  
    \chi = 2 B_0 (s +ip), \quad
    \chi^\dagger = 2 B_0(s - ip).
\end{equation}
%
By making a local $G$-transformation and enforcing gauge-invariance, we find that these transform as
%
\begin{align}
    r_\mu &\rightarrow U_R (r_\mu + i\partial_\mu) U_R^\dagger, \\
    l_\mu &\rightarrow U_L (l_\mu + i\partial_\mu) U_L^\dagger, \\
    \chi &\rightarrow U_R \chi {U_L}^\dagger, \\
    \chi^\dagger &\rightarrow U_L \chi^\dagger {U_R}^\dagger.
\end{align}
%
As in Yang-Mills theory, we can now create field strength tensors of the gauge fields, to build more gauge-invariant terms.
We define
%
\begin{equation}
    f_{\mu \nu}^{(r)} 
    = 
    \partial_\mu r_\nu - \partial_\nu r_\mu - i[r_\mu, r_\nu], 
    \quad f_{\mu \nu}^{(l)} 
    = \partial_\mu l_\nu - \partial_\nu l_\mu - i[l_\mu, l_\nu].
\end{equation}
%
\todo[inline]{Include transformation properties of field strength tensors.}
Including dynamical fields, such as the photon field $\mathcal A_\mu$, is slightly more complicated.
Quantum electrodynamics, or QED, is a gauge theory with a $\Lie{U}{1}_\text{EM}$ gauge group, where the covariant derivative acing on quarks is
%
\begin{equation}
    \label{EM covariant derivative on quarks}
    i\bar q \slashed D' q 
    = 
    i \bar q \gamma^\mu \left( \one \partial_\mu - i e Q \mathcal A_\mu\right) q
    =
    i \bar q \slashed \partial q - e \mathcal A_\mu J^\mu,
\end{equation}
%
Here, $\mathcal A_\mu$ is the photon field corresponding to the gauge group, $e = |e|$ is the elementary charge as given in \autoref{Elementary charge}, $J^\mu = - \bar q Q \gamma^\mu q$ is the electromagnetic charge current, and $Q$ is the quark charge matrix.
This matrix is the generator of $\Lie{U}{1}_\text{EM}$.
In the case of $N_f=3$,  $Q = \text{diag}(\frac{2}{3}, -\frac{1}{3}, -\frac{1}{3})$.
From \autoref{EM covariant derivative on quarks}, we see that $eQ\mathcal A_\mu$ is a vector current.
Although the transformation of the quarks under the electromagnetic gauge group can be seen as a subgroup of $G$, we \emph{do not} transform external currents to enforce gauge invariance, this is instead done by $\mathcal{A}_\mu$.
As $\mathcal{A}_\mu$ is a dynamical field, we can not use it to enforce $G$-gauge invariance.
However, if we treat the charge matrix $Q$ as an external field, then we can restore the invariance.
This gives the transformation rule
%
\begin{equation}
    Q_I \rightarrow U_I Q_I U_I^\dagger, \, I = R, L.
\end{equation}
%
Here, $Q_I = P_I Q$ are the chiral charge matrices.
With these external fields, we must introduce a covariant derivative acting on $\Sigma$ to enforce local $G$ invariance.
This is
%
\begin{equation}
    \nabla_\mu\Sigma = \partial_\mu \Sigma - ir'_\mu \Sigma + i \Sigma l'_\mu,
\end{equation} 
%
where $r'_\mu = r_\mu + eQ\mathcal{A}_\mu$ and $l'_\mu = l_\mu + eQ\mathcal{A}_\mu$.
We do not strictly \emph{need} to include the electromagnetic field in the gauge derivative, we could just build $G$ invariant terms of $eQ$, $\mathcal A_\mu$ and $\Sigma$, however, this is the most economical way to achieve this.


Lastly, we must include terms from quantum electrodynamics involving only the photon field, which are
%
\begin{equation}
    \Ell_\text{QED}[\mathcal A] 
    = -\frac{1}{4} F^{\mu \nu}F_{\mu \nu}, \quad
    F_{\mu \nu} = \partial_{\mu}\mathcal A_{\nu} - \partial_{\nu}\mathcal A_{\mu}.
\end{equation}
%
Furthermore, the covariant derivative is extended to include the photon field in addition to the gluon field.
The full Lagrangian is then
%
\begin{equation}
    \Ell_\text{QCD}[q, \bar q, A,\mathcal A, j] 
    = \Ell_\text{QCD}^0[q, \bar q, A, \mathcal A] 
    + \Ell_\text{QED}[\mathcal A] + \Ell_\text{ext}[q, \bar q, j].
\end{equation}
%
We now define the effective Lagrangian of $\chpt$, $\Ell_\text{eff}$ as
%
\begin{equation}
    \label{definition effective Lagrangian chpt}
    Z[j]
    = 
    \int \D q \D \bar q \D A \D \mathcal A\,
    \exp{i\int \dd^4 x \Ell_\text{QCD}[q, \bar q, A, \mathcal A, j] }
    = 
    \int \D \varphi \D \mathcal A\,
    \exp{i\int \dd^4 x \Ell_\text{eff}[\varphi, \mathcal{A}, j] }.
\end{equation}
%
We know the symmetries of $\Ell_\text{QCD}$, and we have found constituent terms of $\Ell_\text{eff}$ and their transformation properties under these symmetries.
Weinberg's ``theorem'' now tells us that we can construct $\Ell_\text{eff}$ by including all terms that obey the symmetry principles of $\Ell_\text{QCD}$, and with chiral dimension, we have a way of ordering them in a series expansion.
This is all we need to start doing calculations.


\section{*NLO analysis}

The one-loop contribution to the free energy density is
%
\begin{equation}
    \label{one loop free energy}
    \Eff^{(1)}
    = - \frac{i}{V T} \frac{1}{2}
    \Tr{\ln\left( -\fdv{S[\pi = 0]}{\pi_a(x), \pi_b(y)} \right)}.
\end{equation}
%
This can be evaluated using the rules for functional differentiation given in \autoref{appendix: Functional derivatives}.
To leading order,
%
\begin{align}
    \fdv{S[\pi = 0]}{\pi_a(x), \pi_b(y)}
    = \fdv{}{\pi_a(x), \pi_b(y)}
    \int \dd^4 x \, \Ell^{(2)}_2
    = D_{ab}^{-1}(x - y).
\end{align}
%
Here, $\Ell^{(2)}_2$ is the quadratic part of the Lagrangian, as given in \autoref{quadratic Lagrangian}, and $D^{-1}$ is the corresponding inverse propagator of the pion fields,
\begin{equation}
    D_{ab}^{-1}(x-y) = 
    \left[
        - \delta_{ab}(\partial_x^2 + m^2_a)
        + m_{12}(\delta_{a1} \delta_{b2} - \delta_{a2}\delta_{b1}) \partial_{x, 0}
    \right] \delta(x-y)
\end{equation}
%
The inverse propagator is a matrix, which means that the determinant in \autoref{one loop free energy} is both a matrix determinant, over the three pion indices, and a functional determinant.
In \autoref{section: propagator} we found the matrix part of the determinant in momentum space, which we can write using the dispersion relations of the pion fields
\begin{equation}
    \det(- D^{-1}) = \det(-p_0^2 + E_0^2) \det(-p_0^2 + E_+^2) \det(-p_0^2 + E_-^2).
\end{equation}
%
These dispersion relations are functions of the three-momentum $\vec p$, and are given in \autoref{dispresion relation pi 0} and \autoref{dispresion relation pi pm}.
The functional determinant can therefore be evaluated as
%
\begin{align}
    \nonumber
    \Tr{\ln\left( -\fdv{S[\pi = 0]}{\pi_a(x), \pi_b(y)} \right)}
    & = \ln \det(-p_0^2 + E_0^2) + \ln \det(-p_0^2 + E_+^2) + \ln \det(-p_0^2 + E_-^2) \\
    \nonumber
    & = \Tr{ \ln(-p_0^2 + E_0^2) + \ln(-p_0^2 + E_+^2)+  \ln(-p_0^2 + E_-^2) } \\
    & = (VT) \int \frac{\dd^4 p}{(2 \pi)^4} 
    \left[ \ln(-p_0^2 + E_0^2) + \ln(-p_0^2 + E_+^2) + \ln(-p_0^2 + E_-^2)  \right],
\end{align}
%
where we have used the identity $\ln\det M = \Tr \ln M $.


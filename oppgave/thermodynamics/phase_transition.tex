\section{*Phase transition}
\label{section: phase transition}


Our leading-order analysis showed that $\alpha$ is zero for $\mu_I \leq \bar m$ and then increases continuously for $\mu_I>\bar m$.
Furthermore, $\bar m = m_\pi$ to leading-order.
This behavior is illustrated in \autoref{fig: alpha}.
This is the hallmark of a phase transition, where $\alpha$ is the order parameter.
The behavior of systems near points of phase transition is described by Landau theory~\autocite{peskinIntroductionQuantumField1995}.
Using \autoref{leading order contribution free energy}, we can expand the leading-order free energy in $\alpha$,
%
%
\begin{align}
    \nonumber
    \Eff
    & = -f^2 \bar m^2 + f^2 \frac{1}{2}(\bar m^2 - \mu_I^2)\alpha^2
    - \frac{1}{24} f^2 (\bar m^2 - 4 \mu_I^2) \alpha^4 + \Oh(\alpha^5)\\
    & = \Eff(\alpha=0) + a(\mu_I)\alpha^2 + \frac{1}{2} b(\mu_I)\alpha^4 + \Oh(\alpha^5).
\end{align}
%
Notice that near $\mu_I = \bar m$, $b > 0$.
As earlier, the equation that governs $\alpha$ is
%
\begin{equation}
    \label{landau ginsburg lo}
    \pdv{\Eff}{\alpha} = 2 [a(\mu_I) + b(\mu_I) \alpha^2] \alpha = 0.
\end{equation}
%
If $a>0$, then $\alpha = 0$ will be the only solution, which gives us the criterion for a phase transition
%
\begin{equation}
    a(\mu_I) = 0.
\end{equation}
%
As expected, this criterion is fulfilled at $\mu_I = \bar m$.
Near $\mu_I = \bar m$, we can write
%
\begin{equation}
    a = - a_0 (\mu_I - \bar m), \quad b = b_0,
\end{equation}
%
where $a_0$ and $b_0$ are positive constants, so the solution to \autoref{landau ginsburg lo} for $\mu_I>\bar m$ is
%
%
\begin{equation}
    \alpha(\mu_I) = \sqrt{\frac{a_0}{b_0}} (\mu_I - \bar m)^{1/2}.
\end{equation}
%
The free energy around the phase transition is illustrated in \autoref{fig: phase transition}.
\todo[]{fiks figur}

\begin{figure}[h]
    \centering
    % \includegraphics[width=0.9\textwidth]{../scripts/numerikk/plots/phase_transition.pdf}
    \caption{
        The plot shows normalized free energy density as a function of $\alpha$, in the two different phases. Each line is a constant $\mu_I$ slice of the surface in \autoref{fig: free energy surface}.
        }
    \label{fig: phase transition}
\end{figure}

The order parameter $\alpha$ changes continuously as the system transitions between phases.
This means we have a \emph{second order} phase transition.
The power-law behavior, $\alpha \propto (\mu_I - \mu_I^c)^\beta$, is typical of systems near a phase transition.
The exponent $\beta$, which in this case equals $1/2$, is called a \emph{critical exponent}.
If we have a system where $b < 0$ near $\mu_I = \bar m$, then we must expand $\Eff$ further to show if the phase transition is continuous or not.
\autoref{fig: free energy surface 2} shows the free energy surface but modified so that $b_0 < 0$, together with the corresponding value of $\alpha$, which now changes discontinuously at the point of phase transition.
\todo[]{fiks fig}

\begin{figure}[h]
    \centering
    % \includegraphics[width=0.7\textwidth]{../scripts/numerikk/plots/free_energy_surface2.pdf}
    \caption{The surface is free energy density, only modified so that $b_0<0$. The black line traces out the minimum for each value of $\mu_I$, which jumps discontinuously at the point of phase transition.}
    \label{fig:free energy surface 2}
\end{figure}


In the vacuum phase, $\alpha = 0$, the ground state is given by 
%
\begin{equation}
    \Sigma(\pi = 0) = \Sigma_0 = \one,
\end{equation}
%
where we have used \autoref{sigma}.
Under $H = SU(2)_V$, $\Sigma$ transforms as
%
\begin{equation}
    \Sigma(x) \rightarrow \Sigma'(x) = U_V \Sigma(x) U_V^\dagger,
    \quad
    V = \exp{i \frac{1}{2} \theta_a \tau_a},
\end{equation}
%
We see that the vacuum phase ground state is invariant under $H$.
However, for $\alpha \neq 0$, the ground state is shifted to
%
\begin{equation}
    \Sigma(\pi=0) = A_\alpha \Sigma_0 A_\alpha = \exp{i \alpha \tau_1}.
\end{equation}
%
This is not, in general, invariant under transformations in $H$. 
The generators $\tau_2$ and $\tau_3$ are broken.
In \autoref{fig: masses}, we saw that the mass of the $m_-$ particle vanishes, so we identify this particle with the corresponding Goldstone mode.
There is only one Goldstone mode. 
However, this is not a Lorentz invariant system, in which case we cannot guarantee one massless mode per broken generator.
In \autoref{section: leading order}, we defined
%
\begin{equation}
    \alpha = \frac{1}{f} \sqrt{\pi_a^0 \pi_a^0},
\end{equation}
%
where $\pi_a^0$ was the ground state expectation value of the pion fields, as defined in the vacuum phase, when $\mu_I \neq 0$.
The new ground state thus corresponds to a condensate of pions.
The isospin symmetry $\Lie{SU}{2}_V$ is not a perfect symmetry of the QCD Lagrangian, but is explicitly broken by the mass term
\begin{equation}
    \bar q m q = \frac{1}{2} \bar q [(m_u + m_d) \one + (m_u - m_d)\tau_3] q.
\end{equation}
%
This term, however, is invariant under the subgroup $\Lie{U}{1}_{I_3} \subset \Lie{SU}{2}_V$, generated by $\tau_3$.
If we write out a generic element from this subgroup,
%
\begin{equation}
    U = e^{i \theta \tau_3} = 
    \begin{pmatrix}
        e^{i\theta} & 0 \\
        0 & e^{-i \theta}
    \end{pmatrix},
\end{equation}
%
we see that this corresponds to rotating the phase of the up and down quark but not rotating them into each other.
Thus, the pion condensate spontaneously breaks an exact symmetry of the two-flavor QCD Lagrangian, and we expect the corresponding Goldstone mode to remain massless outside the chiral limit.

To find the value of $\mu_I$ to next-to-leading order, we must expand the NLO free energy in powers of $\alpha$
When we expand the static, second-order Lagrangian to $\alpha^2$, we get
%
\begin{align}
    \nonumber
    \Eff_4^{(0)}
    &= - (l_3 + l_4)\bar m^4 + [(l_3 + l_4)\bar m^4 -l_4 \bar m^2\mu_I^2]\alpha^2
    \\
    & =
    \const + 
    \frac{\mu^{-2\epsilon}}{(4\pi)^2}
    \left[
        \left(
            \bar l_4 - \frac{1}{4}\bar l_3
        \right)\bar m^4
        -\bar l_4\bar m^2\mu_I^2
        -\left(
            1 + \frac{1}{\epsilon} + \ln\frac{\tilde \mu^2}{M^2}
        \right)
        \left(\frac{3}{4}\bar m^2 - \mu_I^2\right)\bar m^2
    \right]\alpha^2 + \mathcal{O}(\epsilon),
\end{align}
%
where $\const$ is independent of $\alpha$, and thus not of interest.
From the one-loop correction, we have the contributions
%
\begin{equation}
    \Eff_2^{(1)} = i \frac{1}{2}\int \frac{\dd^4 p}{(2\pi)^2} \ln(-p^2 + m_3^2)
    +  i \frac{1}{2} \int \frac{\dd^4 p}{(2\pi)^2} \ln[(-p^2 + m_1^2)(-p^2 + m_2^2) - p_0^2 m_{12}^2].
\end{equation}
%
The first integral is the same free energy contribution from the $\pi_0$-particle as we have calculated earlier in \autoref{Free energy pi 0}, and it reads
%
\begin{equation}
    \Eff_{\pi_0}^{(1)}
    = i \frac{1}{2}\int \frac{\dd^4 p}{(2\pi)^2} \ln(-p^2 + m_3^2)
    = - \mu^{-2\epsilon}\frac{1}{4} \frac{m_3^4}{(4 \pi)^2}
    \left(\frac{1}{\epsilon} + \frac{3}{2} + \ln \frac{\tilde \mu^2}{m_3^2}\right) + \mathcal{O}(\epsilon).
\end{equation}
%
The mass $m_3$ is dependent on $\alpha$, and has the expansion
%
\begin{align*}
    m_3^4
    &= \bar m^4 + \bar m^2(2\mu_I^2 - \bar m^2)\alpha^2+\Oh(\alpha^4), \\
    \ln \frac{\mu^2}{m_3^2}
    &=
    \ln \frac{\mu^2}{\bar m_3^2} - \frac{1}{2} \frac{(2\mu_I^2 - \bar m^2)}{\bar m^2}+ \Oh(\alpha^4).
\end{align*}
%
In the second integral, we rewrite the argument of the logarithm as
%
\begin{equation}
    (-p^2 + m_1^2)(-p^2 + m_2^2) - p_0^2 m_{12}^2
    =  \left[-p^2 + \frac{1}{2}(m_1^2 + m_2^2)\right]^2 - p_0^2 m_{12}^2 - \frac{1}{4}(m_1^2 - m_2^2)^2.
\end{equation}
%
When we calculate the $\alpha$ dependence of the last term, we get  $(m_1^2 - m_2^2)^2 = \mu^4 \sin^4\alpha = \Oh(\alpha^4)$, which means that for our purposes, we can ignore this term.
We further rewrite the remaining expression by factoring it,
%
\begin{equation}
    \left[-p^2 + \frac{1}{2}(m_1^2 + m_2^2)\right]^2 - p_0^2 m_{12}^2
    = \left[-p^2 + \frac{1}{2}(m_1^2 + m_2^2) - p_0 m_{12} \right]
    \left[-p^2 + \frac{1}{2}(m_1^2 + m_2^2) + p_0 m_{12} \right].
\end{equation}
%
We then complete the square in each of the factors,
%
\begin{equation}
    - p^2 + \frac{1}{2}(m_1^2 + m_2^2) \pm p_0 m_{12}
    = - \left(p_0 \mp \frac{1}{2}m_{12}\right)^2 + |\vv p|^2 + m_4^2,
\end{equation}
%
where
%
\begin{align}
    m_4^2 
    & = \frac{1}{2}
    \left(
        m_1^2 +m_2^2 +\frac{1}{2}m_{12}
     \right)
    = \bar m^2 \cos\alpha + \frac{1}{2} \mu_I^2 \sin^2 \alpha, \\
    m_4^4
    & = \bar m^4 - \bar m^2(m^2 + \mu_I^2) \alpha^2 + \Oh(\alpha^4), \\
    \ln \frac{\mu^2}{m_4^2} 
    & = \ln \frac{\mu_I^2}{\bar m^2} 
    + \frac{1}{2}\frac{\bar m^2 + \mu_I^2}{\bar m^2}\alpha^2
    + \Oh(\alpha^4).
\end{align}
%
After a shift of variables, the integral has the same form as the logarithmic integrals we have calculated earlier, which gives us the result
%
\begin{equation}
    \Eff_{\pi_\pm}
    = i \int \frac{\dd^4}{(2\pi)^4}
    \ln(-p^2 + m_4^2)
    = 
    - \mu^{-2\epsilon}\frac{1}{2} \frac{m_4^4}{(4 \pi)^2}
    \left(\frac{1}{\epsilon} + \frac{3}{2} + \ln \frac{\tilde \mu^2}{m_4^2}\right)+ \mathcal{O}(\epsilon).
\end{equation}
%
Combining these two contributions to the one-loop correction of the free energy gives
%
\begin{align*}
    \Eff_2^{(1)}
    &=
    \mathrm{const.}
    +
    \frac{\mu^{-2 \epsilon } }{(4 \pi)^2} 
    \left(1 + \frac{1}{\epsilon} + \ln \frac{\tilde \mu^2}{m^2}\right)
    \left(\frac{3}{4}m^2 - \mu_I^2\right)
    \bar m^2 \alpha^2+ \mathcal{O}(\epsilon).
\end{align*}
%
We see that again, the $\epsilon^{-1}$ will cancel when we combine the NLO-terms.
Setting $\epsilon = 0$, the NLO correction to the free energy becomes
%
\begin{align}
    \tilde \Eff_1
    & = 
    \const+ 
    \frac{1}{(4\pi)^2}
    \left[
        \left(
            \bar l_4 - \frac{1}{4}\bar l_3
        \right)\bar m^4
        -\bar l_4\bar m^2\mu_I^2
        + \ln\frac{M^2}{\bar m^2}
        \left(\frac{3}{4}\bar m^2 - \mu_I^2\right)\bar m^2
    \right]\alpha^2
    + \Oh(\alpha^4).
\end{align}
%
All coupling constants are measured at $M = m_\pi$.
Using this in the logarithm gives,
%
\begin{equation}
    \ln \frac{M^2}{\bar m^2}
    = \ln \frac{m_\pi^2}{\bar m^2}
    = \ln \left[1 + \Oh[2]{(\bar m/f)} \right]
    = \Oh[2]{(\bar m/f)}.
\end{equation}
%
The term proportional to the logarithm thus vanishes to next-to-leading order.
Combining these expressions give total NLO free energy, up to second order in $\alpha$, is
%
\begin{equation}
    \Eff_{\mathrm{NLO}}
    =
    \Eff_{\mathrm{NLO}}(\alpha = 0)
    +
    \frac{1}{2} f^2 \bar m^2
    \left(
        1
        -
        \frac{1}{2}
        \frac{\bar l_3 - 4 \bar l_4}{(4 \pi)^2} \frac{\bar m^2}{f^2}
    \right)\alpha^2
    - \frac{1}{2}f^2 \mu_I^2
    \left(
        1
        +
        \frac{2 \bar l_4}{(4 \pi)^2}
        \frac{\bar m^2}{f^2}
    \right) \alpha^2
    + \Oh[4]{\alpha}.
\end{equation}
%
We now insert the physical constants $f_\pi$ and $m_\pi$, by using the next-to-leading order expressions \autoref{equation bare mass} and \autoref{equation bare decay constant}.
Notice that
%
\begin{equation}
    f_\pi^2 m_\pi^2
    = f^2 \bar m^2
    \left[
        1 - \frac{1}{2} \frac{\bar l_3 - 4 \bar l_4}{(4 \pi)^2} \frac{\bar m^2}{f^2}
        +
        \Oh[]{\frac{\bar m^4}{f^4}}
    \right],
\end{equation}
%
which means that the NLO free energy has the same structure as the leading order expression,
%
\begin{equation}
    \Eff_{\mathrm{NLO}}
    =
    \Eff_{\mathrm{NLO}}(\alpha = 0)
    + \frac{1}{2}f_\pi^2 (m_\pi^2 - \mu_I^2 )\alpha^2
    + \Oh(\alpha^4z).
\end{equation}
%
This shows that the critical isospin chemical potential is $\mu_I^c = m_\pi$, also at next-to-leading order.
We expect this to hold to all orders in perturbation theory.

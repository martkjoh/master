\section{Electric charge neutrality}


A pion condensate will have an electric charge.
For three flavor QCD, the electric charge matrix, \autoref{three-flavor charge matrix}, will appear in the QCD Lagrangian as $\mu_Q \bar q Q q$, where $\bar q Q q$ is the electric charge density, and $\mu_Q = \mu_I + 2 \mu_S$ is the electric charge chemical potential.
In the case of $\mu_S = 0$, this is thus the 
The charge density is therefore
%
\begin{equation}
    n_Q = - \pdv{\Eff}{\mu_Q} = n_I.
\end{equation}
%
A realistic astrophysical object will not have a macroscopic electric charge.
We will therefore model pion stars with the additional constraint of charge neutrality, by including charged leptons, muons or electrons.
These leptons are free fermions, with an electric charge of $- e$, and we may therefore use the results from \autoref{section: cold fermi star} and \autoref{section: fermions}.
The electric charge density of the leptons is given by minus particle number $n_\ell$, which we found in \autoref{Fermi gas particle density},
%
\begin{equation}
    n_{\ell} = \frac{8}{3} \frac{u_{\ell, 0}}{m_\ell} x_f^3,
\end{equation}
%
where $x_f = \sqrt{\mu^2_\ell - m_\ell^2}/m_\ell$ is the dimensionless Fermi momentum, $m_\ell$ the lepton mass, and $\mu_\ell$ the lepton chemical potential.
We have introduced the characteristic energy density of the lepton, which is defined by
%
\begin{equation}
    u_{\ell, 0} = \frac{m^4_\ell}{3 \pi^2}
\end{equation}
%
The criterion for charge neutrality is thus
%
\begin{equation}
    n_I = n_\ell.
\end{equation}
%
This determines the lepton chemical potential $\mu_\ell$, given $\mu_I$.
To leading order and excluding electromagnetic contribution to the pion interaction, we can use \autoref{pressure leading order chpt} write this explicitly as
%
\begin{equation}
    A \left(\frac{\mu_\ell^2 }{m_\ell^2} - 1 \right)^{3/2}
    = \frac{\mu_I}{m_\pi}\left( 1 - \frac{m_\pi^4}{\mu_I^4}  \right),
\end{equation}
%
The right and left side vanish at $(\mu_I, \mu_\ell) = (m_\pi, m_\ell)$, which we have seen earlier is the point where the pressure and energy density of both the Fermi gas and the pion condensate vanish.
This shows that our pion star will have a well-defined radius.
We have introduced the constant
%
\begin{equation}
    A = \frac{3}{8} \frac{m_\ell} {u_{0, \ell}} \frac{u_0}{m_\pi}
    = 3\pi^2 \frac{m_\pi f_\pi^2}{m_\ell^3}  
\end{equation}
%
Setting the lepton mass to the electron mass or muon mass gives, respectively, $A = 1.409\times10^{6}$ and $A = 1.167\times 10^{-1}$.
Inserting $\mu_\ell^2/m_\ell^2 = 1 + \epsilon^2$ and $\mu_I^2 / m_\pi^2 = 1 + \epsilon'^2$, then this relationship is, to the lowest order in $\epsilon$ and $\epsilon'$, $A \epsilon^3 = \epsilon'^4$.



The total pressure and energy density will now be the sum of the contributions from the pion condensate and the leptons.
The lepton contribution to these, which we found in \autoref{Fermi gas energy density} and \autoref{Fermi gas pressure}, is
%
\begin{align}
    u_\ell 
    &= u_{\ell,0} 
    \left[
        (2x_f^3 + x_f) \sqrt{1+x_f^2} - \arcsinh(x_f)
    \right]. \\
    p_\ell 
    &=\frac{1}{2} u_{\ell,0} 
    \left[
        (2x_f^3 - 3x_f) \sqrt{1+x_f^2} + 3\arcsinh(x_f)
    \right].
\end{align}
%

\section{Electric charge neutrality}


A pion condensate will have an electric charge.
In the grand canonical ensemble, the QCD Lagrangian will have the term  $\mu_Q \bar q Q q$, where $\bar q Q q$ is the electric charge density, $Q$ the quark charge matrix \autoref{three-flavor charge matrix}, and $\mu_Q = \mu_I + 2 \mu_S$ is the electric charge chemical potential.
In the case of $\mu_S = 0$, the charge density is thus
%
\begin{equation}
    n_Q = - \pdv{\Eff}{\mu_Q} = n_I.
\end{equation}
%
A realistic astrophysical object will not have a macroscopic electric charge.
We will therefore model pion stars with the additional constraint of charge neutrality, by including charged leptons in the form of muons or electrons.
These leptons are free fermions, with an electric charge of $- e$.
We may therefore use the results from \autoref{section: cold fermi star} and \autoref{section: fermions}.
The electric charge density of the leptons is given by minus particle number $n_\ell$, which we found in \autoref{Fermi gas particle density},
%
\begin{equation}
    n_{\ell} = \frac{8}{3} 
    \frac{u_{\ell, 0}}{m_\ell} x_f^3,
\end{equation}
%
where $x_f = \sqrt{ {\mu_\ell^2}/{m_\ell^2} - 1}$, is the dimensionless fermi momentum, $m_\ell$ the lepton mass, and $\mu_\ell$ the lepton chemical potential.
This formula is valid for $\mu_ell \geq m_\ell$.
We have introduced the characteristic energy density of leptons,
%
\begin{equation}
    u_{\ell, 0} = \frac{m^4_\ell}{3 \pi^2}
\end{equation}
%
The criterion for charge neutrality is thus
%
\begin{equation}
    \label{criterion charge neutrality}
    n_I = n_\ell.
\end{equation}
%
With this, we can determine the lepton chemical potential given a isospoin chemical potential, $\mu_\ell = \mu_\ell(\mu_I)$.
The leading order result for the pion condensate is given in \autoref{pressure leading order chpt}.
Inserting these results into \autoref{criterion charge neutrality}, we get
%
\begin{equation}
    A \left(\frac{\mu_\ell^2 }{m_\ell^2} - 1 \right)^{3/2}
    = \frac{\mu_I}{m_\pi}\left( 1 - \frac{m_\pi^4}{\mu_I^4}  \right),
\end{equation}
%
The right and left side vanish at $(\mu_I, \mu_\ell) = (m_\pi, m_\ell)$, which we have seen earlier is the point where the pressure and energy density of both the Fermi gas and the pion condensate vanish.
This shows that our pion star will have a well-defined radius.
We have introduced the dimensionless constant
%
\begin{equation}
    A = \frac{3}{8} \frac{m_\pi} {m_\ell} \frac{u_{0, \ell}}{u_0}
    = \frac{1}{8 \pi^2} \frac{m_\ell^3}{m_\pi f_\pi^2}.
\end{equation}
%
Setting the lepton mass to the electron mass or muon mass gives, respectively, $A = 1.3599 \times10^{- 7}$ and $A = 5.7092 \times 10^{3}$.
The explicit form of the functional relation is
%
\begin{equation}
    \label{mu ell from mu I}
    \frac{\mu_\ell}{m_\ell}
    =
    \sqrt{
        1 + A^{-2/3}
        \left[
            \frac{\mu_I}{m_\pi}\left( 1 - \frac{m_\pi^4}{\mu_I^4}  \right)
        \right]^{2/3}
    }.
\end{equation}
%

The total pressure and energy density will now be the sum of the contributions from the pion condensate and the leptons.
The lepton contribution to these, which we found in \autoref{Fermi gas energy density} and \autoref{Fermi gas pressure}, is
%
\begin{align}
    u_\ell 
    &= u_{\ell,0} 
    \left[(2x_f^3 + x_f) \sqrt{1 + x_f^2} - \arcsinh\left(x_f\right)\right], \\
    p_\ell
    &=\frac{1}{3} u_{\ell,0}
    \left[(2x_f^3 - 3x_f) \sqrt{1 + x_f^2} + 3\arcsinh\left(x_f\right)\right].
\end{align}
%
The contribution from the pion condensate is, as before we found in \autoref{pressure leading order chpt} and \autoref{energy density leading order chpt},
%
\begin{align}
    u_\pi &= \frac{1}{2} u_0 \left( \frac{\mu_I}{m_\pi} - \frac{m_\pi}{\mu_I}\right)^2 \\
    p_\pi &= \frac{1}{2} u_0 \left( 2 + \frac{\mu_I^2}{m_\pi^2} - 3 \frac{m_\pi^2}{\mu_I^2}  \right)
\end{align}
%
This leads to a total pressure and energy
%
\begin{equation}
    p = p_\pi + p_\ell, \quad u = u_\pi + u_\ell.
\end{equation}
%
As \autoref{criterion charge neutrality} gives $\mu_\ell$ as a function of $\mu_I$, these are both parametrized by the isospin chemical potential, and as before we can thus extract the equation of state $u = u(p)$.


We can study the limit of the combined system, by again letting $\mu_I^2/m_\pi^2 = 1 + \epsilon$.
Inserting this into \autoref{mu ell from mu I} and expanding to first order in $\epsilon$, we get that $\mu_\ell \approx 1 + (2 A^{-1} \epsilon)^{2/3} $.
This is equivalent to $x_f \approx (2 A^{-1} \epsilon)^{1/3} $.
In \autoref{section: cold fermi star}, we found the non-relativistic limit of the pressure and energy of the Fermi gas, i.e., the lowest order contribution in $x_f$, as $x_f \rightarrow 0$.
Inserting the new result, we get the leading low energy limits of the pressure and energy, in units of $u_0$, 
%
\begin{equation}
    u_{\ell, \text{nr}} = \frac{8}{3} \frac{u_{\ell,0}}{u_0} A^{-1} \epsilon, \quad
    p_{\ell, \text{nr}} = \frac{8}{15} \frac{u_{\ell,0}}{u_0} \left(\frac{2}{A} \right)^{5/3} \epsilon^{5/3}.
\end{equation}
%
From \autoref{section: thermodynamics leading order}, we have the equivalent expression for the pion condensate,
%
\begin{equation}
    u_{\pi, \text{nr}} = 2 u_0 \epsilon, \quad p_{\pi, \text{nr}} = \frac{1}{2} u_0 \epsilon^2.
\end{equation}
%
As we see, to leading order, both the pion condensate and the leptons will contribute to the leading order energy density, however \emph{only} the leptons will contribute to the leading order pressure.
At low enough isospin chemical potential, then, the leading order behavior of the combined system is
%
\begin{equation}
    u_{\text{nr}} = \left(2 u_0 + \frac{8}{3} u_{\ell,0} A^{-1} \right)\epsilon ,\quad
    p_{\text{nr}} = \frac{8}{15} u_{\ell,0} \left(\frac{2}{A} \right)^{5/3} \epsilon^{5/3}.
\end{equation}
%
The equation of state is now a polytrope with $\gamma = \frac{5}{3}$, which is different from the $\gamma = 2$ polytrope that the pion condensate is on its own.

In an intermediate range, however, the pressure of a \emph{light} lepton will be suppressed by a factor $u_{\ell,0}/ u_0 A^{-5/3}$, and the pion contribution might be dominant for a while.
In this regime, then, the equation of state is then still a polytrope with $\gamma = 2$, but the constant is changed due to the lepton contribution to the energy density.
The pressure in the intermediate range is
%
\begin{equation}
    p_i = \frac{1}{2} u_0 \epsilon^2,
\end{equation}
%
and the equatin of state is thus
%
\begin{equation}
    p_i = K u^\gamma, \quad \gamma = 2, \quad 
    K = \frac{1}{2} \left(2 u_0 + \frac{8}{3} u_{\ell,0} A^{-1} \right)^2
\end{equation}
%



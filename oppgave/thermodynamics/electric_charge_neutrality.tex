\section{Electric charge neutrality}
\label{section: charge neturality}


A pion condensate will have an electric charge.
In the grand canonical ensemble, the QCD Lagrangian will have the term  $\mu_Q e\bar q \gamma^0 Q q$, where $e \bar q \gamma^\mu Q q$ is the electric current density, $Q$ the quark charge matrix \autoref{three-flavor charge matrix}, and $e \mu_Q = \mu_I + 2 \mu_S$ is the electric charge chemical potential.
In the case of $\mu_S = 0$, the charge density is
%
\begin{equation}
    n_Q = - \pdv{\Eff}{\mu_Q} = e n_I.
\end{equation}
%
A realistic astrophysical object will not have a macroscopic electric charge.
Therefore, we will model pion stars with the additional constraint of charge neutrality by including charged leptons, of muons or electrons, as free fermions.
These leptons have an electric charge of $- e$.
We may therefore use the results from \autoref{section: cold fermi star} and \autoref{section: fermions}.
The electric charge density of the leptons is given $- e n_\ell$, where $n_\ell$ is the particle number, the lepton number.
In \autoref{Fermi gas particle density}, we found the expression
%
\begin{equation}
    \label{lepton density}
    n_{\ell} = \frac{8}{3} 
    \frac{u_{\ell, 0}}{m_\ell} x_f^3,
\end{equation}
%
where $x_f = \sqrt{ {\mu_\ell^2}/{m_\ell^2} - 1}$ is the dimensionless Fermi momentum, $m_\ell$ the lepton mass, and $\mu_\ell$ the lepton chemical potential.
This formula is valid for $\mu_\ell \geq m_\ell$.
We have introduced the characteristic energy density of leptons,
%
\begin{equation}
    u_{\ell, 0} = \frac{m^4_\ell}{8 \pi^2}.
\end{equation}
%
The criterion for charge neutrality is then
%
\begin{equation}
    \label{criterion charge neutrality}
    n_I = n_\ell.
\end{equation}
%
With this, we can determine the lepton chemical potential as a function of the isospin chemical potential, $\mu_\ell = \mu_\ell(\mu_I)$.
The leading order result for the isospin density of the pion condensate is given in \autoref{isospin density}.
Inserting the expressions for these densities into \autoref{criterion charge neutrality}, we get
%
\begin{equation}
    \label{equation mu ell mu I}
    A \left(\frac{\mu_\ell^2 }{m_\ell^2} - 1 \right)^{3/2}
    = \frac{\mu_I}{m_\pi}\left( 1 - \frac{m_\pi^4}{\mu_I^4}  \right).
\end{equation}
%
Both densities vanish at the point $(\mu_I, \mu_\ell) = (m_\pi, m_\ell)$, which we have seen earlier is the point where the pressure and energy density of both the Fermi gas and the pion condensate vanish.
We have introduced the dimensionless constant
%
\begin{equation}
    A = \frac{8}{3} \frac{m_\pi} {m_\ell} \frac{u_{0, \ell}}{u_0}
    = \frac{1}{3 \pi^2} \frac{m_\ell^3}{m_\pi f_\pi^2}.
\end{equation}
%
Setting the lepton mass equal the electron mass or muon mass gives, respectively, $A = 3.9 \times10^{- 9}$ and $A = 3.5 \times 10^{-2}$.
In this case, we can write the expression for $\mu_\ell(\mu_I)$ as an explicit function,
%
\begin{equation}
    \label{mu ell from mu I}
    \frac{\mu_\ell}{m_\ell}
    =
    \sqrt{
        1 + A^{-2/3}
        \left[
            \frac{\mu_I}{m_\pi}\left( 1 - \frac{m_\pi^4}{\mu_I^4}  \right)
        \right]^{2/3}
    }.
\end{equation}
%
These relationships are illustrated in \autoref{fig: chemical potentials}.
The plot on the left is the electron chemical potential as a function of isospin chemical potential, both normalized to the masses of their corresponding particles, while the muon chemical potential is on the right.
We see that both lepton potentials are suppressed, relative to the isospin chemical potential, by the $A$ constant in \autoref{equation mu ell mu I}.
The electron is much lighter than the pion, and therefore grows much faster than the muon chemical potential.

\begin{figure}[!htb]
    \centering
    \begin{subfigure}{0.49 \textwidth}
        \includegraphics[width=\textwidth]{../scripts/figurer/charge_neutrality/chemical_potential_e.pdf}
    \end{subfigure}
    \begin{subfigure}{0.49\textwidth}
        \includegraphics[width=\textwidth]{../scripts/figurer/charge_neutrality/chemical_potential_mu.pdf}
    \end{subfigure} 
    \caption{
        The lepton chemical potentials as functions of the isospin chemical potential normalized to their respective particles masses.
        The electron chemical potential is shown to the left, the muon to the right.
    } 
    \label{fig: chemical potentials}
\end{figure}


The total pressure and energy density will now be the sum of the contributions from the pion condensate and the leptons.
The lepton contribution to these, which we found in \autoref{Fermi gas energy density} and \autoref{Fermi gas pressure}, is
%
\begin{align}
    u_\ell 
    &= u_{\ell,0} 
    \left[(2x_f^3 + x_f) \sqrt{1 + x_f^2} - \arcsinh\left(x_f\right)\right], \\
    p_\ell
    &=\frac{1}{3} u_{\ell,0}
    \left[(2x_f^3 - 3x_f) \sqrt{1 + x_f^2} + 3\arcsinh\left(x_f\right)\right].
\end{align}
%
The contribution from the pion condensate is, as we found in \autoref{pressure leading order chpt} and \autoref{energy density leading order chpt},
%
\begin{align}
    u_\pi &= \frac{1}{2} u_0 \left( \frac{\mu_I}{m_\pi} - \frac{m_\pi}{\mu_I}\right)^2 \\
    p_\pi &= \frac{1}{2} u_0 \left( 2 + \frac{\mu_I^2}{m_\pi^2} - 3 \frac{m_\pi^2}{\mu_I^2}  \right)
\end{align}
%
This leads to a total pressure and energy
%
\begin{equation}
    p = p_\pi + p_\ell, \quad u = u_\pi + u_\ell.
\end{equation}
%
As \autoref{criterion charge neutrality} gives $\mu_\ell$ as a function of $\mu_I$, these are both parametrized by the isospin chemical potential, and we can extract the equation of state $u = u(p)$.
The full equation of state in different regimes is shown in \autoref{fig: eos with leptons}.
On the top left, the addition of the electron results in a much stiffer equation of state than the addition of the muon.
We have seen that the low-pressure energy density of the pion is $m_\pi n_I $, while for the lepton, it is $m_\ell n_\ell$.
As $n_I = n_\ell$, the low-pressure limit of the energy density is $(m_\pi + m_\ell)n_ I$, which explains why the muon, where $m_\mu \approx m_\pi$, contributes a lot more to the energy density than the electron, for which $m_e \ll m_\pi$.
We see that the pion contribution dominates the high-pressure regime in the top right plot.
However, as chiral perturbation assumes small pion energies and small external currents, this result must be used carefully.
The equation of state in an intermediate-range is shown on the bottom.
It is not very sensitive to the lepton mass, at least as long as it is less than the pion mass.

\begin{figure}
    \centering
    \begin{subfigure}{0.49\textwidth}
        \includegraphics[width=\textwidth]{../scripts/figurer/charge_neutrality/eos_nr.pdf}
    \end{subfigure}
    \begin{subfigure}{0.49\textwidth}
        \includegraphics[width=\textwidth]{../scripts/figurer/charge_neutrality/eos_ur.pdf}
    \end{subfigure}
    \includegraphics[width=\textwidth]{../scripts/figurer/charge_neutrality/eos_I.pdf}
    \caption{
        The equation of state including a lepton is compared with the equation of state of only the pion condensate in two different regimes.
        The pressure and energy density is normalized to $u_0 = f_\pi^2 m_\pi^2$.
    }
    \label{fig: eos with leptons}
\end{figure}

We can find the ultrarelativistic limit by letting $\mu_I^2/m_\pi^2 = y$, $y \gg 1$.
From \autoref{mu ell from mu I}, we find that the lepton chemical potential to leading order in $y$ is $\mu_\ell^2 \propto y^{1/3}$.
In \autoref{section: cold fermi star}, we found that in the ultrarelativistic limit of non-interacting fermions, both the pressure and energy is proportional to $x_f^4 \propto y^{2/3}$.
In the case of the pion, however, both are proportional to $\mu_I^2 \propto y$.
Therefore, the ultrarelativistic limit of the combined system is, to leading order, given by the ultrarelativistic limit of the pion condensate alone.

As before, we can find the non-relativisitic limit by letting $\mu_I^2/m_\pi^2 = 1 + \epsilon$.
Inserting this into \autoref{mu ell from mu I} and expanding to first order in $\epsilon$, we get $\mu_\ell^2/m_\ell^2 = 1 + (2 A^{-1} \epsilon)^{2/3} $.
This is equivalent to $x_f = (2 A^{-1} \epsilon)^{1/3} $.
In \autoref{section: cold fermi star}, we found the non-relativistic limit of the pressure and energy of the Fermi gas, i.e., the lowest order contribution in $x_f$, as $x_f \rightarrow 0$.
Inserting this new result into these limits, we get the leading low energy limits of the pressure and energy, 
%
\begin{equation}
    u_{\ell, \text{nr}} = \frac{8}{3} \frac{2}{A} u_{\ell,0} \epsilon, \quad
    p_{\ell, \text{nr}} = \frac{8}{15} \left(\frac{2}{A} \right)^{5/3}  u_{\ell,0}  \epsilon^{5/3}.
\end{equation}
%
From \autoref{section: thermodynamics leading order}, we have the equivalent expressions for the pion condensate,
%
\begin{equation}
    u_{\pi, \text{nr}} = 2 u_0 \epsilon, \quad p_{\pi, \text{nr}} = \frac{1}{2} u_0 \epsilon^2.
\end{equation}
%
As we see, the energy density of the pion condensate and the leptons are of the same order, and both will therefore contribute to the leading order energy density.
However, the lepton pressure is of a lower order, and \emph{only} this will contribute to the leading order pressure.
At low enough isospin chemical potential, then, the leading order behavior of the combined system is
%
\begin{equation}
    u_{\text{nr}} 
    = 2 u_0 \left(1 + \frac{1}{2}\frac{m_\ell}{m_\pi} \right)  \epsilon, \quad
    p_{\text{nr}} = \frac{8}{15} u_{\ell,0} \left(\frac{2}{A} \right)^{5/3} \epsilon^{5/3}.
\end{equation}
%
The equation of state is now a polytrope with $\gamma = \frac{5}{3}$, different from the $\gamma = 2$ polytrope of only the pion condensate.
For a heavy lepton, $m_\ell \gg m_\pi$, the factor $\smash{A^{-\frac{5}{3}}}$ would suppress the leading order of the lepton contribution to the pressure.
Thus, the equation of state would have an intermediate range in which it would be well approximated as a polytrope with $\gamma = 2$, and a modified poly tropic constant $K'^{-1} = 8 (1 + \frac{1}{2} m_\ell/m_\pi)^2$.
The full equation of state is compared to the non-relativistic limit in \autoref{fig: lepton eos limit}.

\begin{figure}[!htb]
    \centering
    \includegraphics[width=\textwidth]{../scripts/figurer/charge_neutrality/eos_lim.pdf}
    \caption{
        The full equation of state of the pion + lepton systems, compared with the non-relativistic limit.
        Both the energy density and the pressure is given in units of $u_0 = m_\pi^2 f_\pi^2$.
    }
    \label{fig: lepton eos limit}
\end{figure}


Results for the equation of state of the pion condensate, as well as the system with pions and electrons or muons, were obtained by \citeauthor{brandtNewClassCompact2018} in \autocite{brandtNewClassCompact2018} using computational QCD lattice methods.
Their results are shown in \autoref{fig: brandt eos}, and compared with our results.
There is a good agreement between the results, although the computational results tend to give a slightly less stiff equation of state.

\begin{figure}[!htb]
    \centering
    \includegraphics[width=\textwidth]{../scripts/figurer/brandt_eos.pdf}
    \caption{
        The equation of state of only pions, pions and electrons, and pions and muons.
        The results of \citeauthor{brandtNewClassCompact2018}, solid color, are compared with our results, dashed lines.
        The result of \citeauthor{brandtNewClassCompact2018} incorporate statistical and systematic uncertainty in both $u$ and $p$ in the width of the lines.
        Both energy density and pressure are shown in units of $u_0 = f_\pi^2 m_\pi^2$.
    }
    \label{fig: brandt eos}
\end{figure}




\subsection{Neutrinos}


The primary decay mode of charged pions is via the strong force, which couples quarks to a charged lepton, $\ell$, and its corresponding neutrino, $\nu_\ell$.
Neutrinos are very light leptons which only interact via the weak force.
The exact nature of the mass of neutrinos are still an open problem~\Autocite{schwartzQuantumFieldTheory2013}.
In this text, we will regard the neutrinos as massless.
The leading order diagrams for the relevant interaction is
%
\begin{equation*}
    \feynmandiagram [horizontal=a to b]{
        u [particle=$u$] 
        -- [fermion] a -- [fermion]
        d [particle=$\bar b$], 
        a -- [photon, edge label=$W^+ $] b,
        ell [particle=$\ell^+$] -- [fermion] b -- [fermion]
        nu [particle=$\nu_\ell$]
        };
    \quad\quad\quad\quad\quad\quad\quad
    \feynmandiagram [horizontal=a to b]{
        u [particle=$\bar u$]
        -- [anti fermion] a -- [anti fermion]
        d [particle=$b$], 
        a -- [photon, edge label=$W^-$] b,
        ell [particle=$\ell^-$] -- [anti fermion] b -- [anti fermion]
        nu [particle=$\bar \nu_\ell$]
        };
\end{equation*}
%
The $W^\pm$-boson, the mediator of the weak force, only couples to left-handed fermions.
That is, the coupling term has the form $W^\pm_\mu \bar \psi_L \gamma^\mu \psi_L$, where $\psi_L = P_L \psi$, as discussed in \autoref{subsection: chiral symmetry}.
This will result in effective interactions of the form
%
\begin{equation*}
    \feynmandiagram [horizontal=a to b]{
        pi [particle=$\pi^+$] -- [fermion] a  [blob], 
        a -- [photon, edge label=$W^+ $] b,
        ell [particle=$\ell^+$] -- [fermion] b -- [fermion]
        nu [particle=$\nu_\ell$]
        };
        \quad\quad\quad\quad\quad\quad\quad
        \feynmandiagram [horizontal=a to b]{
            pi [particle=$\pi^-$] -- [fermion] a  [blob], 
            a -- [photon, edge label=$W^- $] b,
            ell [particle=$\ell^-$] -- [anti fermion] b -- [anti fermion]
            nu [particle=$\bar \nu_\ell$]
            };
\end{equation*}
%
where the blob represents the effective interaction between the pions and the $W$-bosons.
The decay is dominated by $\pi \rightarrow \mu \nu_\mu$, and not $\pi \rightarrow e \nu_e$ as one might first expect, as a lighter particle have a large phase space.
This is due to \emph{helicity suppression}.
For light fermions such as the electrons, the helicity, i.e., the alignment of spin and momentum, is approximately given by the chirality.
This process gives to left-handed particles, and in the mass-less limit this is equivalent with particles with helicity $h = -1$ and antiparticles with $h = 1$.
However, this is forbidden on grounds of conservation of linear and angular momentum~\autocite{griffithsIntroductionElementaryParticles2008}.

We treat neutrinos $\nu_\ell$ as free, massless fermions, which then has its own chemical potential, $\mu_{\nu_e\ll}$.
The process we described above can be written as $\pi^+\ell \rightarrow \nu_\ell $ using crossing symmetry.
In chemical equilibrium the chemical potentials will therefore obey
%
\begin{equation}
    \label{chemical equillibrium weak interaction}
    \mu_I + \mu_\ell = \mu_{\nu_\ell}.
\end{equation}
%
In chemical equilibrium, there will be created both electrons and muons.
Lepton-numbers $L_\ell$, the sum of the particle number of the charged lepton $\ell$ and its neutrino $\nu_\ell$, is not a conserved quantity in the real world.
This is due to neutrino oscillations, processes on the form $\nu_\ell \rightarrow \nu_{\ell'}$.
These oscillations happen constantly, as the flavor states of neutrinos are not the same as the mass eigenstates, and there will thus also be a chemical equilibrium between electrons and muons too, defined by
%
\begin{equation}
    \mu_e = \mu_\mu.
\end{equation}
%
The equation for charge neutrality now becomes $n_I = n_e + n_\mu$, or
%
\begin{equation}
    \frac{\mu_I}{m_\pi} \left(1 - \frac{m_\pi^4}{\mu_I^4}\right)
    =
    A_e \sqrt{\frac{\mu_e^2}{m_e^2} - 1}
    +
    \theta(\mu_e - m_\mu) 
    A_\mu \sqrt{a^2 \frac{\mu_e^2}{m_e^2} - 1}.
\end{equation}
%
where $a = m_e/m_\mu \approx 1 / 200$.
Thus, for low densities there will be no muons.
The pressure and energy from the neutrino is given by the ultrarelativistic limit of the free fermion result, which we found in \autoref{section: cold fermi star}.
We have here divided by a factor two, as there are only left-handed neutrinos, and thus only half as many degrees-of-freedom as fermions where both charities appear.
The chemical potential and Fermi momentum of a massless particle are equal, which gives us
%
\begin{equation}
    p_{\nu_\ell} = \frac{\mu_{\nu_\ell}^4}{8 \pi^2}, \quad 
    u_{\nu_\ell} = \frac{\mu_{\nu_\ell}^4}{24 \pi^2}.
\end{equation}
%

The total pressure and energy density is now, as before, the sum of the individual contribution of each species.

As before, the criterion of charge neutrality ensures that the transition into the vacuum phase for the pion condensate, at $\mu_I = m_\pi$, happens simultaneously as the fermion density vanish, here at $\mu_e = m_e$.
However, due to the weak-interaction chemical equilibrium, \autoref{chemical equillibrium weak interaction}, the neutrino chemical potential and thus the neutrino pressure will be non-zero even with no pion condensate nor any leptons. 



% \begin{tikzpicture}
%     \begin{feynman}
%         \vertex (u) {$u$};
%         \vertex[right=1cm of u] (b1);
%         \vertex[right=1cm of b1] (b2);
%         \vertex[right=1cm of b2] (ell) {$\ell^+$};
        
%         \vertex[below=0.75cm of b1] (a1);
%         \vertex[below=0.75cm of b2] (a2);
        
%         \vertex[below=1.5cm of u] (d) {$\bar d$};
%         \vertex[below=1.5cm of ell] (nu) {$\nu_\ell$};

%         \diagram* {
%             {[edges=fermion]
%             (u) -- (a1) -- (d) 
%             },
%             {[edges=fermion]
%             (ell) -- (a2) -- (nu)
%             },
%         };
%     \end{feynman}
% \end{tikzpicture}
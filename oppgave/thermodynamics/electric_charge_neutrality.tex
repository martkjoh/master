\section{Electric charge neutrality}


A pion condensate will have an electric charge.
In the grand canonical ensemble, the QCD Lagrangian will have the term  $\mu_Q \bar q Q q$, where $\bar q Q q$ is the electric charge density, $Q$ the quark charge matrix \autoref{three-flavor charge matrix}, and $\mu_Q = \mu_I + 2 \mu_S$ is the electric charge chemical potential.
In the case of $\mu_S = 0$, the charge density is thus
%
\begin{equation}
    n_Q = - e \pdv{\Eff}{\mu_Q} = e n_I.
\end{equation}
%
A realistic astrophysical object will not have a macroscopic electric charge.
We will therefore model pion stars with the additional constraint of charge neutrality, by including charged leptons in the form of muons or electrons.
These leptons are free fermions, with an electric charge of $- e$.
We may therefore use the results from \autoref{section: cold fermi star} and \autoref{section: fermions}.
The electric charge density of the leptons is given $- e n_\ell$, where $n_\ell$ is the particle number, and in this case the lepton number.
In \autoref{Fermi gas particle density}, we found the expression
%
\begin{equation}
    n_{\ell} = \frac{8}{3} 
    \frac{u_{\ell, 0}}{m_\ell} x_f^3,
\end{equation}
%
where $x_f = \sqrt{ {\mu_\ell^2}/{m_\ell^2} - 1}$, is the dimensionless Fermi momentum, $m_\ell$ the lepton mass, and $\mu_\ell$ the lepton chemical potential.
This formula is valid for $\mu_\ell \geq m_\ell$.
We have introduced the characteristic energy density of leptons,
%
\begin{equation}
    u_{\ell, 0} = \frac{m^4_\ell}{3 \pi^2}.
\end{equation}
%
The criterion for charge neutrality is thus
%
\begin{equation}
    \label{criterion charge neutrality}
    n_I = n_\ell.
\end{equation}
%
With this, we can determine the lepton chemical potential as a function of the isospoin chemical potential, $\mu_\ell = \mu_\ell(\mu_I)$.
The leading order result for the isospin density of the pion condensate is given in \autoref{pressure leading order chpt}.
Inserting the expressions for the densities into \autoref{criterion charge neutrality}, we get
%
\begin{equation}
    \label{equation mu ell mu I}
    A \left(\frac{\mu_\ell^2 }{m_\ell^2} - 1 \right)^{3/2}
    = \frac{\mu_I}{m_\pi}\left( 1 - \frac{m_\pi^4}{\mu_I^4}  \right),
\end{equation}
%
The right and left side vanish at $(\mu_I, \mu_\ell) = (m_\pi, m_\ell)$, which we have seen earlier is the point where the pressure and energy density of both the Fermi gas and the pion condensate vanish.
We have introduced the dimensionless constant
%
\begin{equation}
    A = \frac{3}{8} \frac{m_\pi} {m_\ell} \frac{u_{0, \ell}}{u_0}
    = \frac{1}{8 \pi^2} \frac{m_\ell^3}{m_\pi f_\pi^2}.
\end{equation}
%
Setting the lepton mass to the electron mass or muon mass gives, respectively, $A = 1.360 \times10^{- 7}$ and $A = 5.709 \times 10^{3}$.
In this case, we can write the expression for $\mu_\ell(\mu_I)$ as an explicit function,
%
\begin{equation}
    \label{mu ell from mu I}
    \frac{\mu_\ell}{m_\ell}
    =
    \sqrt{
        1 + A^{-2/3}
        \left[
            \frac{\mu_I}{m_\pi}\left( 1 - \frac{m_\pi^4}{\mu_I^4}  \right)
        \right]^{2/3}
    }.
\end{equation}
%
These relationships are illustrated in \autoref{fig: chemical potentials}.
The plot on the left is the electron chemical potential as a function of isospin chemical potential, both normalized to the masses of their corresponding particles.
We see that the values of the electron chemical potential are far greater than those of the isospin chemical potential, as they are suppressed by the $A$-constnat in \autoref{equation mu ell mu I}.
The muon chemical potential, shown on the right and normalized to the muon mass, is promoted by $A$ and thus much smaller than the isospoin chemical potential.

\begin{figure}[!htb]
    \centering
    \begin{subfigure}{0.49 \textwidth}
        \includegraphics[width=\textwidth]{../scripts/figurer/charge_neutrality/chemical_potential_e.pdf}
    \end{subfigure}
    \begin{subfigure}{0.49\textwidth}
        \includegraphics[width=\textwidth]{../scripts/figurer/charge_neutrality/chemical_potential_mu.pdf}
    \end{subfigure} 
    \caption{
        The lepton chemical potentials as functions of the isospin chemical potential, all normalized to their respective particles mass.
        The electron chemical potential is shown to the left, the muon to the right.
    } 
    \label{fig: chemical potentials}
\end{figure}


The total pressure and energy density will now be the sum of the contributions from the pion condensate and the leptons.
The lepton contribution to these, which we found in \autoref{Fermi gas energy density} and \autoref{Fermi gas pressure}, is
%
\begin{align}
    u_\ell 
    &= u_{\ell,0} 
    \left[(2x_f^3 + x_f) \sqrt{1 + x_f^2} - \arcsinh\left(x_f\right)\right], \\
    p_\ell
    &=\frac{1}{3} u_{\ell,0}
    \left[(2x_f^3 - 3x_f) \sqrt{1 + x_f^2} + 3\arcsinh\left(x_f\right)\right].
\end{align}
%
The contribution from the pion condensate is, as we found in \autoref{pressure leading order chpt} and \autoref{energy density leading order chpt},
%
\begin{align}
    u_\pi &= \frac{1}{2} u_0 \left( \frac{\mu_I}{m_\pi} - \frac{m_\pi}{\mu_I}\right)^2 \\
    p_\pi &= \frac{1}{2} u_0 \left( 2 + \frac{\mu_I^2}{m_\pi^2} - 3 \frac{m_\pi^2}{\mu_I^2}  \right)
\end{align}
%
This leads to a total pressure and energy
%
\begin{equation}
    p = p_\pi + p_\ell, \quad u = u_\pi + u_\ell.
\end{equation}
%
As \autoref{criterion charge neutrality} gives $\mu_\ell$ as a function of $\mu_I$, these are both parametrized by the isospin chemical potential, and we can extract the equation of state $u = u(p)$.
The full equation of state in two different regimes is show in \autoref{fig: eos with leptons}.
The plot on the left is the low-pressure regime.
As the light electron almost immediately enters the ultrarelativistic regime, $\mu_e\gg m_e$, it will contribute mostly pressure.
The heavy muon, on the other hand, is non-relativistic, and thus contribute mostly to the energy density due to its rest mass.
On the right, we see that differences between the pion contribution to both the energy density and pressure, as we will see later.

\begin{figure}
    \centering
    \begin{subfigure}{0.49\textwidth}
        \includegraphics[width=\textwidth]{../scripts/figurer/charge_neutrality/eos_nr.pdf}
    \end{subfigure}
    \begin{subfigure}{0.49\textwidth}
        \includegraphics[width=\textwidth]{../scripts/figurer/charge_neutrality/eos_ur.pdf}
    \end{subfigure}
    \caption{
        The equation of state including a lepton is compared with the equation of state of only the pion condensate, in two different regimes.
        The pressure and energy density is normalized to $u_0 = f_\pi^2 m_\pi^2$.
    }
    \label{fig: eos with leptons}
\end{figure}


We can study the limit of the combined system, by again letting $\mu_I^2/m_\pi^2 = 1 + \epsilon$.
Inserting this into \autoref{mu ell from mu I} and expanding to first order in $\epsilon$, we get $\mu_\ell = 1 + (2 A^{-1} \epsilon)^{2/3} $.
This is equivalent to $x_f = (2 A^{-1} \epsilon)^{1/3} $.
In \autoref{section: cold fermi star}, we found the non-relativistic limit of the pressure and energy of the Fermi gas, i.e., the lowest order contribution in $x_f$, as $x_f \rightarrow 0$.
Inserting this new result into these limits, we get the leading low energy limits of the pressure and energy, 
%
\begin{equation}
    u_{\ell, \text{nr}} = \frac{8}{3} A^{-1} u_{\ell,0} \epsilon, \quad
    p_{\ell, \text{nr}} = \frac{8}{15} \left(\frac{2}{A} \right)^{5/3}  u_{\ell,0}  \epsilon^{5/3}.
\end{equation}
%
From \autoref{section: thermodynamics leading order}, we have the equivalent expressions for the pion condensate,
%
\begin{equation}
    u_{\pi, \text{nr}} = 2 u_0 \epsilon, \quad p_{\pi, \text{nr}} = \frac{1}{2} u_0 \epsilon^2.
\end{equation}
%
As we see, the pion condensate and the leptons energy density is of the same order, and will therefore both contribute to the leading order energy density.
However, the lepton pressure is of lower order, and \emph{only} this will contribute to the leading order pressure.
At low enough isospin chemical potential, then, the leading order behavior of the combined system is
%
\begin{equation}
    u_{\text{nr}} = \left(2 u_0 + \frac{8}{3} u_{\ell,0} A^{-1} \right)\epsilon ,\quad
    p_{\text{nr}} = \frac{8}{15} u_{\ell,0} \left(\frac{2}{A} \right)^{5/3} \epsilon^{5/3}.
\end{equation}
%
The equation of state is now a polytrope with $\gamma = \frac{5}{3}$, different from the $\gamma = 2$ polytrope of only the pion condensate.
The equation of state of the lepton is compared with this limit in \autoref{fig: lepton eos}
This figure is not dependent on the mass of the lepton.

\begin{figure}
    \centering
    \includegraphics[width=0.6\textwidth]{../scripts/figurer/charge_neutrality/eos_lepton.pdf}
    \caption{The equation of state of the lepton, compared with the non-relativisitic limit.
    Both energy density and pressure are normalized to the characteristic lepton density.}
    \label{fig: lepton eos}
\end{figure}


In an intermediate range, however, the pressure of a heavy lepton will be suppressed by a factor $u_{\ell,0}/ u_0 A^{-5/3}\propto (m_\pi^{1/3} f_\pi^{{2}/{3}})^5 (m_\pi f_\pi)^{æ2} m_\ell^{-1}$, which for $m_\ell \gg m_\pi$ and $m_\ell \gg f_\pi$ is $\ll 1$, and the pion contribution might be dominant for a while.
In this regime, then, the equation of state is then still a polytrope with $\gamma = 2$, but the constant is changed due to the lepton contribution to the energy density.
The pressure in the intermediate range is
%
\begin{equation}
    p_i = \frac{1}{2} u_0 \epsilon^2,
\end{equation}
%
and the equation of state is thus
%
\begin{figure}[!htb]
    \centering
    \begin{subfigure}{0.49\textwidth}
        \includegraphics[width=\textwidth]{../scripts/figurer/charge_neutrality/eos_lepton_limitse.pdf}
    \end{subfigure}
    \begin{subfigure}{0.49\textwidth}
        \includegraphics[width=\textwidth]{../scripts/figurer/charge_neutrality/eos_lepton_limitsmu.pdf}
    \end{subfigure}
    \caption{
        The full equation of state of the pion + lepton system, compared with intermediate and non-relativistic limit.
    On the right, the lepton is the electron, while on the right it is the muon.
    The full equation of state is compared to two different limits.
    }
    \label{fig: intermediate regime}
\end{figure}
%
\begin{equation}
    p_i = K u^2, \quad 
    K = \frac{1}{2} \left(2 u_0 + \frac{8}{3} u_{\ell,0} A^{-1} \right)^2
\end{equation}
%
This is illustrated in \autoref{fig: intermediate regime}.
In this figure, both the intermediate limit and the non-relativistic limit is shown.
To the left is the system with electrons, and we see that even though $m_e < m_\pi$, there is still a regime of validity of around five orders of magnitude for the intermediate limit.
For $p/u_0 < 10^{-10}$, we see that the non-relativistic limit is very good.
On the right, we see that the intermediate limit has a very large range of applicability, of more than ten orders of magnitude.
The equation of state does finally approach the non-relativistic limit around $p/u_0 < 10^{-16}$, however it is necessary with higher-than-normal precision in the numerical calculations to see this.
 


\subsection{Stars}


The mass-radius relation for pion stars including leptons is shown in \autoref{fig: mass radius relation with leptons}, and is compared to the relation for pion stars of only pions.
We see that the two leptons affect the relation in different ways.
The light electron does not contribute much to the energy density and the ultrarelativistic limit of the full equation of state quickly approaches that of only the pion. The maximum mass is therefore not changed much, and the corresponding radius is only slightly larger.
In the non-relativisitic regime, however, the electron results in a stiffer equation, and a different polytropic index.
There is therefore no upper limit to the radius, which only grows with lower total mass.
The muon, on the other hand, is much heavier and thus mostly contributes to the energy density.
This results in a less stiff equation of state, and a smaller and less massive star.
As the intermediate limit, where $\gamma = 2$, is such a good approximation for the equation of state, this relation seems to approach a maximum radius.
However, we can notice a slight curve towards higher radius at the lowest mass.
We expect this to continue in the non-relativisitic limit, as the pressure from the pion becomes negligible compared to the degeneracy from muons.
This effect, however, we would need much higher precision numerical computation to show.


\begin{figure}
    \centering
    \includegraphics[width=\textwidth]{../scripts/figurer/pion_star/mass_radius_lepton.pdf}
    \caption{
        The mass-radius relation of pion stars, with and without a lepton to enforce charge neutrality.
        The radius is given in km, and the mass in solar masses.
        }
        \label{fig: mass radius relation with leptons}
\end{figure}


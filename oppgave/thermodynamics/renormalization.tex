\section{*Next-to-leading order and renormalization}

We have now regularized the divergences, which allows them to be handled in a well-defined way.
However, they are still there.
To get rid of them, we need to use renormalization.
As laid out in \autoref{section: chiral perturbation theory}, the power counting scheme ensures that all terms in $\Ell_{2n}$ scales as $t^{2n}$ when the momenta $p$ are scaled as  $p \rightarrow t p$.\footnote{
    Remember that we scale pion mass $\bar m = B_0(m_u + m_d)$ as $t^2$, and the chemical potential as $t$.
    }
The tree-level free energy from $\Ell_{2n}$ is thus of order $p^{2n}$.
The $m$-loop correction to the tree level result is then suppressed by $p^{2m}$~\autocite{gasserChiralPerturbationTheory1984,weinbergPhenomenologicalLagrangians1979}.
Our one-loop calculation using $\Ell_2$ therefore contains divergences of order $p^{4}$. 
Since $\Ell_4$ is, by construction, the most general possible Lagrangian at order $p^4$, it contains coupling constants that can be renormalized to absorb all these divergences.

The renormalized coupling constants in $\Ell_4$ have been calculated for $\mu_I = 0$\autocite{gasserChiralPerturbationTheory1984}.
They are independent of $\mu_I$, and we can therefore use them in this calculation.
The renormalized coupling constants in the $\overline{\mathrm{MS}}$-scheme are related to the bare couplings through
%
\begin{align}
    l_i 
    & = 
    l_i^r 
    - \mu^{-2\epsilon}\frac{1}{2} \frac{\gamma_i }{(4 \pi)^2} 
    \left(\frac{1}{\epsilon} + 1 \right),
    \quad \quad
    i \in \{1, ... 7\},
    \\
    h_i 
    & = 
    h_i^r
    - \mu^{-2\epsilon} \frac{1}{2}  \frac{\delta_i }{(4 \pi)^2} 
    \left(\frac{1}{\epsilon} + 1 \right), 
    \quad \quad
    i \in \{1, ... 3\}.
\end{align}
%
Here, $\gamma_i$ and $\delta_i$ are numerical constants which are used to match the divergences.
The relevant terms are\footnote{Some authors~\autocite{adhikariTwoflavorChiralPerturbation2019,gerberHadronsChiralPhase1989} instead use $h_1' = h_1 - l_4$, with a corresponding $\delta_1' = \delta_1 - \gamma_1 = 0$.}
%
\begin{gather}
    \gamma_1 = \frac{1}{3}, \quad
    \gamma_2 = \frac{2}{3}, \quad
    \gamma_3 = - \frac{1}{2}, \quad
    \gamma_4 = 2, \\
    \delta_1 = 2, \quad
    \delta_3 = 0.
\end{gather}
%
The bare coupling constants $l_i$ and $h_i$, though infinite, are independent of our renormalization scale $\mu$.
From this we obtain the renormalization group equations for the running coupling constants,
\begin{equation}
    \mu \odv{l_i^r}{\mu } = - \mu^{-2\epsilon} \frac{\gamma_i }{(4 \pi)^2} + \mathcal{O}(\epsilon), \quad
    \mu \odv{h_i^r}{\mu } = -  \mu^{-2\epsilon}\frac{\delta_i}{(4 \pi)^2}+ \mathcal{O}(\epsilon).
\end{equation}
%
These have the general solutions
\begin{equation}
    l_i^r 
    = \frac{1}{2} \mu^{-2\epsilon} \frac{\gamma_i}{(4 \pi)^2} 
    \left( \bar l_i - \ln{\frac{\tilde \mu^2}{M^2}} \right)+ \mathcal{O}(\epsilon),
    \quad
    h_i^r 
    = \frac{1}{2} \mu^{-2\epsilon} \frac{\gamma_i}{(4 \pi)^2} 
    \left( \bar h_i - \ln{\frac{\tilde \mu^2}{M^2}} \right)+ \mathcal{O}(\epsilon),
\end{equation}
%
where $\bar l_i$ and $\bar h_i$ are the values of the coupling constants (times a constant) measured at the energy $M$.
This only applies if the numerical constants $\gamma_i$/$\delta_i$ are non-zero.
If they are zero, then the coupling is not running, and the measured value can be applied at all energies.
The bare couplings are thus given by
\begin{align}
    \label{bare couplings as functions of measured values}
    l_i &= \mu^{-2\epsilon} \frac{1}{2} \frac{\gamma_i}{(4 \pi)^2}
    \left(
        \bar l_i -1- \frac{1}{\epsilon} - \ln\frac{\tilde \mu^2}{M^2}
    \right)+ \mathcal{O}(\epsilon), \\
    h_i &= \mu^{-2\epsilon} \frac{1}{2} \frac{\delta_i}{(4 \pi)^2}
    \left(
        \bar h_i - 1 - \frac{1}{\epsilon} - \ln\frac{\tilde \mu^2}{M^2}
    \right)
    + \mathcal{O}(\epsilon).
\end{align}
%
The next-to-leading contribution to the free energy at tree-level is $\Eff_4^{0} = - \Ell_4^{(0)}$, which is given by \autoref{NLO-L0}.
When substituting \autoref{bare couplings as functions of measured values} into the bare couplings, we get
%
\begin{align*}
    \Eff^{(0)}_4
    & = 
    - (l_1 + l_2)\mu_I^4 \sin^4{\alpha}
    - (l_3 + l_4)\bar m^4 \cos^2{\alpha}
    - l_4 \bar m^2 \mu_I{}^2 \cos{\alpha} \sin^2{\alpha}
    -(h_1- l_4) \bar m^4
    - h_3 \Delta m^4
    \\
    & = 
    - \mu^{-2 \epsilon} \frac{1}{2} \frac{1}{(4 \pi)^2}
    \bigg[
        \frac{1}{3}
        \left( 
            \bar l_1 + 2 \bar l_2 - 3
        \right) \mu_I^4 \sin^4 \alpha
        +
        \frac{1}{2}
        \left(
            - \bar l_3 + 4 \bar l_4 - 3
        \right) \bar m^4 \cos^2\alpha
        \\
        & \quad \quad \quad \quad \quad \quad \quad
        + 2 \left(\bar l_4 - 1\right)
        \bar m^2 \mu_I^2 \cos\alpha \sin^2 \alpha
        + 2 (\bar l_4 - \bar h_1) \bar m^4
        + \bar h_3 \Delta m^4
        \\
        & \quad \quad \quad \quad \quad \quad \quad
        - 
        \left(\frac{1}{\epsilon} + \ln \frac{\tilde \mu^2}{M^2}\right) 
        \left(
            \mu_I^4\sin^4\alpha + \frac{3}{2} \bar m^4 \cos^2 \alpha
            + 2 \bar m^2 \mu_I^2 \cos\alpha \sin^2 \alpha
        \right) 
    \bigg] + \mathcal{O}(\epsilon).
\end{align*}
%
Notice that the term proportional to $\epsilon^{-1}$ cancel exactly with the divergent term from $\Eff^{(2)}$, as we expected.
Adding all the contribution to the free energy density, and taking the limit $\epsilon \rightarrow 0$, we get the next-to-leading order free energy density,
%
\begin{align}
    \nonumber
    \Eff_{\mathrm{NLO}} &=
    - f^2 \left(\bar m^2 \cos \alpha + \frac{1}{2}\mu_I^2 \sin^2 \alpha\right)
    + \Eff^{(1)}_{\mathrm{fin}, \pi_\pm} \\ \nonumber
    & - \frac{1}{2}\frac{1}{(4 \pi)^2}
    \bigg[
        \frac{1}{3}
        \left( 
            \bar l_1 + 2 \bar l_2 + \frac{3}{2} + 3 \ln \frac{M^2}{m_3^2}
        \right) \mu_I^4 \sin^4 \alpha
        +
        \frac{1}{2}
        \left(
            - \bar l_3 + 4 \bar l_4 + \frac{3}{2} + 2\ln \frac{M^2}{m_3^2}
            + \ln \frac{M^2}{\tilde m^2_2}
        \right) \bar m^4 \cos^2\alpha \\
        & \quad \quad \quad \quad 
        + 2 \left(\bar l_4 + \frac{1}{2} + \ln \frac{M^2}{m_3^2}\right)
        \bar m^2 \mu_I^2 \cos\alpha \sin^2 \alpha
        \label{NLO free energy}
    \bigg].
\end{align}
%
We have dropped the terms proportional to $\bar l_4 - \bar h_1$ and $\bar h_3$, as they only add an unobservable constant value to the free energy.
With next to leading order results, we must use the next-to-leading order values of the masses and pion decay constants, given by~\autocite{gasserChiralPerturbationTheory1984}
%
\begin{align}
    \label{equation bare mass}
    m_\pi^2 & = \bar m^2 + \frac{1}{2}\bar l_3 \frac{\bar m^4}{(4\pi)^2 f^2}, \\
    \label{equation bare decay constant}
    f_\pi^2 & = f^2 + 2\bar l_4\frac{\bar m^2}{(4\pi)^2f^2}.
\end{align}
%



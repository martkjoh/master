\section{*Next-to-leading order and renormalization}

We have now regularized the divergences, which allows them to be handled in a well-defined way.
However, they are still there.
To get rid of them, we need to use renormalization.
As laid out in \autoref{section: chiral perturbation theory}, the power counting scheme ensures that all terms in $\Ell_{2n}$ scales as $t^{2n}$ when the momenta $p$ are scaled as  $p \rightarrow t p$.\footnote{
    Remember that we scale pion mass $\bar m = B_0(m_u + m_d)$ as $t^2$, and the chemical potential as $t$.
    }
The tree-level free energy from $\Ell_{2n}$ is thus of order $p^{2n}$.
The $m$-loop correction to the tree level result is then suppressed by $p^{2m}$~\autocite{gasserChiralPerturbationTheory1984,weinbergPhenomenologicalLagrangians1979}.
Our one-loop calculation using $\Ell_2$ therefore contains divergences of order $p^{4}$. 
Since $\Ell_4$ is, by construction, the most general possible Lagrangian at order $p^4$, it contains coupling constants that can be renormalized to absorb all these divergences.

The renormalized coupling constants in $\Ell_4$ have been calculated for $\mu_I = 0$\autocite{gasserChiralPerturbationTheory1984}.
They are independent of $\mu_I$, and we can therefore use them in this calculation.
The renormalized coupling constants in the $\overline{\mathrm{MS}}$-scheme are related to the bare couplings through
%
\begin{align}
    l_i 
    & = 
    l_i^r 
    - \mu^{-2\epsilon}\frac{1}{2} \frac{\gamma_i }{(4 \pi)^2} 
    \left(\frac{1}{\epsilon} + 1 \right),
    \quad \quad
    i \in \{1, ... 7\},
    \\
    h_i 
    & = 
    h_i^r
    - \mu^{-2\epsilon} \frac{1}{2}  \frac{\delta_i }{(4 \pi)^2} 
    \left(\frac{1}{\epsilon} + 1 \right), 
    \quad \quad
    i \in \{1, ... 3\}.
\end{align}
%
Here, $\gamma_i$ and $\delta_i$ are numerical constants which are used to match the divergences.
The relevant terms are\footnote{Some authors~\autocite{adhikariTwoflavorChiralPerturbation2019,gerberHadronsChiralPhase1989} instead use $h_1' = h_1 - l_4$, with a corresponding $\delta_1' = \delta_1 - \gamma_1 = 0$.}
%
\begin{gather}
    \gamma_1 = \frac{1}{3}, \quad
    \gamma_2 = \frac{2}{3}, \quad
    \gamma_3 = - \frac{1}{2}, \quad
    \gamma_4 = 2, \\
    \delta_1 = 2, \quad
    \delta_3 = 0.
\end{gather}
%
The bare coupling constants $l_i$ and $h_i$, though infinite, are independent of our renormalization scale $\mu$.
From this we obtain the renormalization group equations for the running coupling constants,
\begin{equation}
    \mu \odv{l_i^r}{\mu } = - \mu^{-2\epsilon} \frac{\gamma_i }{(4 \pi)^2} + \mathcal{O}(\epsilon), \quad
    \mu \odv{h_i^r}{\mu } = -  \mu^{-2\epsilon}\frac{\delta_i}{(4 \pi)^2}+ \mathcal{O}(\epsilon).
\end{equation}
%
These have the general solutions
\begin{equation}
    l_i^r 
    = \frac{1}{2} \mu^{-2\epsilon} \frac{\gamma_i}{(4 \pi)^2} 
    \left( \bar l_i - \ln{\frac{\tilde \mu^2}{M^2}} \right)+ \mathcal{O}(\epsilon),
    \quad
    h_i^r 
    = \frac{1}{2} \mu^{-2\epsilon} \frac{\gamma_i}{(4 \pi)^2} 
    \left( \bar h_i - \ln{\frac{\tilde \mu^2}{M^2}} \right)+ \mathcal{O}(\epsilon),
\end{equation}
%
where $\bar l_i$ and $\bar h_i$ are the values of the coupling constants (times a constant) measured at the energy $M$.
This only applies if the numerical constants $\gamma_i$/$\delta_i$ are non-zero.
If they are zero, then the coupling is not running, and the measured value can be applied at all energies.
The bare couplings are thus given by
\begin{align}
    \label{bare couplings as functions of measured values}
    l_i &= \mu^{-2\epsilon} \frac{1}{2} \frac{\gamma_i}{(4 \pi)^2}
    \left(
        \bar l_i -1- \frac{1}{\epsilon} - \ln\frac{\tilde \mu^2}{M^2}
    \right)+ \mathcal{O}(\epsilon), \\
    h_i &= \mu^{-2\epsilon} \frac{1}{2} \frac{\delta_i}{(4 \pi)^2}
    \left(
        \bar h_i - 1 - \frac{1}{\epsilon} - \ln\frac{\tilde \mu^2}{M^2}
    \right)
    + \mathcal{O}(\epsilon).
\end{align}
%
The next-to-leading contribution to the free energy at tree-level is $\Eff_4^{0} = - \Ell_4^{(0)}$, which is given by \autoref{NLO-L0}.
When substituting \autoref{bare couplings as functions of measured values} into the bare couplings, we get
%
\begin{align*}
    \Eff^{(0)}_4
    & = 
    - (l_1 + l_2)\mu_I^4 \sin^4{\alpha}
    - (l_3 + l_4)\bar m^4 \cos^2{\alpha}
    - l_4 \bar m^2 \mu_I{}^2 \cos{\alpha} \sin^2{\alpha}
    -(h_1- l_4) \bar m^4
    - h_3 \Delta m^4
    \\
    & = 
    - \mu^{-2 \epsilon} \frac{1}{2} \frac{1}{(4 \pi)^2}
    \bigg[
        \frac{1}{3}
        \left( 
            \bar l_1 + 2 \bar l_2 - 3
        \right) \mu_I^4 \sin^4 \alpha
        +
        \frac{1}{2}
        \left(
            - \bar l_3 + 4 \bar l_4 - 3
        \right) \bar m^4 \cos^2\alpha
        \\
        & \quad \quad \quad \quad \quad \quad \quad
        + 2 \left(\bar l_4 - 1\right)
        \bar m^2 \mu_I^2 \cos\alpha \sin^2 \alpha
        + 2 (\bar l_4 - \bar h_1) \bar m^4
        + \bar h_3 \Delta m^4
        \\
        & \quad \quad \quad \quad \quad \quad \quad
        - 
        \left(\frac{1}{\epsilon} + \ln \frac{\tilde \mu^2}{M^2}\right) 
        \left(
            \mu_I^4\sin^4\alpha + \frac{3}{2} \bar m^4 \cos^2 \alpha
            + 2 \bar m^2 \mu_I^2 \cos\alpha \sin^2 \alpha
        \right) 
    \bigg] + \mathcal{O}(\epsilon).
\end{align*}
%
Notice that the term proportional to $\epsilon^{-1}$ cancel exactly with the divergent term from $\Eff^{(2)}$, as we expected.
Adding all the contribution to the free energy density, and taking the limit $\epsilon \rightarrow 0$, we get the next-to-leading order free energy density,
%
\begin{align}
    \nonumber
    \Eff_{\mathrm{NLO}} &=
    - f^2 \left(\bar m^2 \cos \alpha + \frac{1}{2}\mu_I^2 \sin^2 \alpha\right)
    + \Eff^{(1)}_{\mathrm{fin}, \pi_\pm}
    - \frac{1}{2}\frac{1}{(4 \pi)^2}
    \bigg[
        \frac{1}{3}
        \left( 
            \bar l_1 + 2 \bar l_2 + \frac{3}{2} + 3 \ln \frac{M^2}{m_3^2}
        \right) \mu_I^4 \sin^4 \alpha
        \\ %\nonumber
        &
        +
        \frac{1}{2}
        \left(
            - \bar l_3 + 4 \bar l_4 + \frac{3}{2} + 2\ln \frac{M^2}{m_3^2}
            + \ln \frac{M^2}{\tilde m^2_2}
        \right) \bar m^4 \cos^2\alpha 
        + 2 \left(\bar l_4 + \frac{1}{2} + \ln \frac{M^2}{m_3^2}\right)
        \bar m^2 \mu_I^2 \cos\alpha \sin^2 \alpha
        \label{NLO free energy}
    \bigg].
\end{align}
%
We have dropped the terms proportional to $\bar l_4 - \bar h_1$ and $\bar h_3$, as they only add an unobservable constant value to the free energy.


\subsection{Parameters and consistent expansion}

The coupling constants are free parameters of the effective theory of \chpt.
As we are not able to do calculations with QCD at low energies, these must be measured instead of calculated from first principles.
The values for the pion mass and pion decay constants used in this text are
\todo[]{use correct values}
%
\begin{equation}
    m_\pi = 131 \, \mathrm{MeV}, \quad 
    f_\pi = \frac{1}{\sqrt 2} 128 \, \mathrm{MeV} = 90.5 \, \mathrm{MeV}.
\end{equation}
%
The most up-to-date value of the pion masses from the Pardicle Data Book are~\cite{particledatagroupReviewParticlePhysics2020}
\todo[]{sammkjør verdier}
%
\begin{equation}
    m_0 = 139.57039\pm0.00018\, \text{MeV}, \quad
    m_\pm = 134.9768\pm0.0005\, \text{MeV}.
\end{equation}
%
These are the masses of the neutral and the charged pions and includes contribution due to electromagnetic and weak interactions, which are not included in this study.
The value for the pion decay constant is~\cite{particledatagroupReviewParticlePhysics2020}
%
\begin{equation}
    f_\pi = \frac{1}{\sqrt{2}} (130.2 \pm 1.2 \, \text{MeV})
    = 92.1 \pm 0.8 \, \text{MeV}.
\end{equation}
%

The physical mass, $m_\pi$, is defined as the pole of the propagator in the vacuum, and thus the zero of the inverse propagator,
%
\begin{equation}
    D^{-1}(p^2 = m_\pi^2) = 0.
\end{equation}
%
This relates it to the bare mass $\bar m$.
We found, in \autoref{m1}, that $m_\pi^2 = \bar m^2$ to leading order.
Similarly, $f_\pi = f$ to leading order.
However, in any NLO results we need the relationship between the bare and physical constants to NLO.
This is given by~\autocite{gasserChiralPerturbationTheory1984}
%
\begin{align}
    \label{equation bare mass}
    m_\pi^2 & = \bar m^2 + \frac{\bar l_3}{2 (4\pi)^2} \frac{\bar m^4}{f^2}, \\
    \label{equation bare decay constant}
    f_\pi^2 & = f^2 + \frac{2\bar l_4}{(4\pi)^2} \frac{\bar m^2}{f^2}.
\end{align}
%

\begin{table}[h]
    \centering
    \caption{The measured values and corresponding uncertainties of the relevant coupling constants, measured at $M = m_\pi$.}
    \begin{tabular}{c c c c}
        \hline \hline
        & value & uncertainty & source \\
        \hline \\[-1em]
        $\bar l_1$ & -0.4 & \pm 0.6   & \autocite{colangeloPpScattering2001}    \\
        $\bar l_2$ & 4.3  & \pm 0.1   & \autocite{colangeloPpScattering2001}   \\
        $\bar l_3$ & 2.9  & \pm 2.4   & \autocite{gasserChiralPerturbationTheory1984}\\
        $\bar l_4$ & 4.4  & \pm 0.2   & \autocite{colangeloPpScattering2001}    \\
        \hline
    \end{tabular}
    \label{table: coupling constants}
\end{table}
%
The values for the coupling constants used in this text are given in \autoref{table: coupling constants}.
The constants $\bar l_1$, $\bar l_2$ and $\bar l_4$ are estimated using data from $\pi \pi$-scattering~\autocite{colangeloPpScattering2001}, while $\bar l_3$ is estimated using three flavor chiral perturbation theory~\autocite{gasserChiralPerturbationTheory1984}.
These are the same values as those used in~\autocite{adhikariTwoflavorChiralPerturbation2019}.
In this text, we use the central values of the parameters.
Together with \autoref{equation bare mass}\autoref{equation bare decay constant}, the NLO results for the central value for the bare mass and decay constant are
%
\begin{align}
    \label{NLO m}
    \bar m  &= 1.01136\, m_\pi = 132.5 \, \mathrm{MeV}, \\
    \label{NLO f}
    f  &= 0.64835\, m_\pi = 85.9 \, \mathrm{MeV}.
\end{align}
%

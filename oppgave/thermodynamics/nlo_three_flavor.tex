\section{Next-to-leading order contribution}


The next-to-leading order contributions to free energy are of two types.
Contributions of the first type are the one-loop effects from the Lagrangian of the lowest chiral dimension $\Ell_2$, and the second type are tree-level effects from the next-to-leading order Lagrangian, $\Ell_4$.
We must include all these contributions to be able to renormalize the results.
When calculating loop integrals, the results will diverge, and they must be regularized and renormalized.
As laid out in \autoref{section: chiral perturbation theory}, the power counting scheme ensures that all terms in $\Ell_{2n}$ scales as $t^{2n}$ when the momenta $p$ are scaled as  $p \rightarrow t p$.\footnote{
    Remember that we scale pion mass $\bar m = B_0(m_u + m_d)$ as $t^2$, and the chemical potential and fundamental charge as $t$.
    }
The tree-level free energy from $\Ell_{2n}$ is thus of order $p^{2n}$.
The $m$-loop correction to the tree level result is then suppressed by $p^{2m}$~\autocite{gasserChiralPerturbationTheory1984,weinbergPhenomenologicalLagrangians1979}.
Our one-loop calculation using $\Ell_2$ therefore contains divergences of order $p^{4}$. 
Since $\Ell_4$ is, by construction, the most general possible Lagrangian at order $p^4$, it contains coupling constants that can be renormalized to absorb all these divergences.
This expansion in both loops and chiral dimension, which is used to evaluate all physical quantities, is elaborated on in \autoref{appendix: consisten expansion},


\subsection{One-loop contribution}

We now calculate the one-loop contribution to the free energy to leading order.
The loop integrals will diverge, and we must therefore regularize the integrals.
We will use dimensional regularization, in which the integral is generalized to $d$ dimensions, and employ the $\overline{\mathrm{MS}}$-scheme.
From the formula for the effective potential, \autoref{effective potential}, we have that the one loop-contribution to one-loop order is 
%
\begin{equation}
    \Eff^{(1)}
    =
    - \frac{i}{V T} \frac{1}{2}
    \Tr{\ln\left( - D_{ab}^{-1}\right)}.
\end{equation}
%
Where $D_{ab}^{-1}$ is the inverse propagator.
In \autoref{subsection: pion-condensed phase}, we found the inverse propagator to be on a block-diagonal form.
The trace is then a sum of the trace of each block.
%
\begin{align}
    \nonumber
    &\Tr{\ln\left( - D_{ab}^{-1}\right)}
    =\\\nonumber
    &\quad\quad
    \Tr{\ln(-p^2 + m_3^2)}
    + \Tr{\ln(-p^2 + m_8^2)}
    + \Tr{\ln(-D_{12}^{-1})}
    + \Tr{\ln(-D_{45}^{-1})}
    + \Tr{\ln(-D_{67}^{-1})}.
\end{align}
%
The trace, in this case, is not only a sum over the matrix-indices of the propagator, but is an operator trace, as discussed in \autoref{section:free scalar field}.
Writing the trace on integral form, the contributions from the neutral pion and the eta particle are
%
\begin{align}
    \Eff_{\pi^0}^{(1)}
    = -i\frac{1}{2} \int \frac{\dd^4 p}{(2\pi)^4}\, \ln(-p^2 + m_3^2), \quad
    \Eff_{\eta}^{(1)}
    = -i\frac{1}{2} \int \frac{\dd^4 p}{(2\pi)^4}\, \ln(-p^2 + m_8^2).
\end{align}
%
As shown in \autoref{section: integral}, we may rewrite integrals of this form,
%
\begin{equation}
    -i\frac{1}{2} \int \frac{\dd^4 p}{(2\pi)^4}\, \ln(-p_0^2 + E^2)
    = \frac{1}{2} \int  \frac{\dd^3 p}{(2\pi)^3 } \, E.
\end{equation}
%
We see that the result is what we would expect physically; the total energy is the integral of each mode's energy.
This also agrees with the result from \autoref{appendix: thermal field theory} in the zero-temperature limit $\beta \rightarrow \infty$.
With the integral on this form, we can regularize and evaluate it as described in  \autoref{section:free scalar field}, which gives
%
\begin{equation}
    \label{Free energy pi 0}
    \Eff^{(1)}_{\pi^0} 
    = 
    -  \frac{1}{4} \frac{ \mu^{-2 \epsilon}}{(4\pi)^2}
    \left( \frac{1}{\epsilon} + \frac{3}{2} + \ln \frac{\tilde \mu^2}{m_3^2} \right)
    m_3^4
    + \mathcal{O}(\epsilon), \quad
    \Eff^{(1)}_{\eta}
    = 
    - \frac{1}{4} \frac{\mu^{-2 \epsilon}}{(4\pi)^2} 
    \left( \frac{1}{\epsilon} + \frac{3}{2} + \ln \frac{\tilde \mu^2}{m_8^2} \right)
    m_8^4
    + \mathcal{O}(\epsilon).
\end{equation}
%
Here, $d = 3 - 2\epsilon$, and $\mu$ is the renormalization scale, a parameter with mass-dimension 1, introduced to ensure the action integral remains dimensionless during dimensional regularization. The scale $\tilde \mu$ is a related to $\mu$ as described in \autoref{definition mu tilde MS bar}.

Using the identity $\ln\{\det(A)\} = \Tr{\ln(A)}$, and spectrum we found in \autoref{subsection: pion-condensed phase}, we can write the charged pion contribution as
%
\begin{equation}
    \Eff_{\pi^\pm}^{(1)}
    = - \frac{i}{VT}\frac{1}{2}\Tr{\ln(-D_{12}^{-1})}
    =
    \frac{1}{2} \int  \frac{\dd^3 p}{(2\pi)^3 } \, (E_{\pi^+} + E_{\pi^-}).
\end{equation}
%
However, the similarities stop here, as the spectra of the charged pions are much more complicated,
%
\begin{equation}
    E_{\pi^\pm}
    = 
    \sqrt{
        |\vv p|^2 +
        \frac{1}{2}
        \left(
            m_1^2 + m_2^2 + m_{12}^2 
        \right)
        \mp 
        \frac{1}{2}
        \sqrt{
            4|\vv p|^2m_{12}^2 
            +
            \left(
                m_1^2 + m_2^2 + m_{12}^2
            \right)^2
            - 4 m_1^2 m_2^2
        }
    }.
\end{equation}
%
This is not an integral we can easily do in dimensional regularization.
Instead, we will seek a function $f(|\vv p|)$ with the same UV-behavior, that is, behavior for large $\vv p$, as $E_{\pi^+} + E_{\pi^-}$
If we then add $0 = f(|\vv p|) - f(|\vv p|)$ to the integrand, we can isolate the divergent behavior
%
\begin{equation}
    \Eff_{\pi^\pm}^{(1)}
    = 
    \frac{1}{2} \int \frac{\dd^3 p}{(2\pi)^3} 
    \left[E_{\pi^+} + E_{\pi^- } - f(|\vv p|)\right]
    + \frac{1}{2} \int \frac{\dd^3 p}{(2\pi)^3} f(|\vv p|)
    = \Eff^{(1)}_{\mathrm{fin}, \pipm } + \Eff^{(1)}_{\mathrm{div}, \pipm}.
\end{equation}
%
This results in a finite integral, $\Eff^{(1)}_{\mathrm{fin}, \pipm }$, which can be evaluated numerically, and a divergent integral $\Eff^{(1)}_{\mathrm{div}, \pipm }$, which we hope can be evaluated in dimensional regularization.

We can explore the UV-behavior of $E_{\pi^+} + E_{\pi^-}$ by expanding it in powers of $1 / \abs{\vv{p}}$,
%
\begin{align}
    \nonumber
    E_{\pi^+} + E_{\pi^-}
    & = 
    2  \abs{\vv{p}}
    + \frac{m_{12} + 2(m_1^2 + m_2^2)}{4} \, {\abs{\vv{p}}}^{-1}
    - \frac{m_{12}^4 + 4 m_{12}^2(m_1^2 + m_2^2) + 8(m_1^4 + m_2^4)}{64}
    {\abs{\vv{p}}}^{-3}
    + \Oh (\abs{\vv{p}}^{-5})
    \\
    & = 
    a_1  \abs{\vv{p}}
    + a_2 \, {\abs{\vv{p}}}^{-1}
    + a_3
    {\abs{\vv{p}}}^{-3}
    + \Oh (\abs{\vv{p}}^{-5}).
\end{align}
%
We have defined new constants $a_i$ for brevity of notation.
As
%
\begin{equation}
    \int_{\R^3} \frac{\dd^3 p}{(2 \pi)^3} |\vv p|^{n}
    = C \int_{0}^\infty \dd p \, p^{2 + n}
\end{equation}
%
is UV divergent for $n \geq -3$, $f$ need to match the expansion of $E_{\pi^+} + E_{\pi^-}$ up to and including $\Oh(|\vv{p}|^{-3})$ for $\Eff^{(1)}_{\mathrm{fin}, \pipm }$ to be finite.
The most obvious choice for $f$ is
%
\begin{equation}
    f(|\vv p|) 
    = a_1  \abs{\vv{p}} + a_2 \, {\abs{\vv{p}}}^{-1} + a_3 \, {\abs{\vv{p}}}^{-3}.
\end{equation}
%
However, this introduces a new problem.
$f$ has the same UV behavior as $E_{\pi^+} + E_{\pi^-}$, but the last term diverges in the IR, that is, for low $|\vv p|$.
This can be amended by introducing a mass term.
Let
%
\begin{equation}
    |\vv p|^{-3} 
    = 
    \left(\frac{1}{\sqrt{|\vv p|^2}}\right)^3 
    \longrightarrow 
    \left(\frac{1}{\sqrt{|\vv p|^2 + m^2}}\right)^3.
\end{equation}
%
For $|\vv p|^2 \rightarrow \infty$, this is asymptotic to $|\vv p|^{-3}$, so it retains its UV behavior.
However, for $|\vv p| \rightarrow 0$, it now approaches $m^{-3}$, so the IR-divergence is gone.
The cost of this technique is that we have introduced an arbitrary mass parameter.
Any final result must thus be independent of the value of $m$ to be acceptable.

We will instead regularize the integral by defining $E_i = \sqrt{|\vv{p}|^2 + \tilde m_i^2}$, and $\tilde m_i^2 = m_i^2 + \frac{1}{4} m_{12}^2$.
Then, we define $f(|\vv p|) = E_1 + E_2$, which differ from $E_{\pi^+} + E_{\pi^-}$ by $\Oh\left(|\vv p|^{-5}\right)$ and is well-behaved in the IR.
This leads to a divergent integral of the same form as in the case of a free scalar, and we may again use the result from \autoref{section: regualting free energy}.
With this, the divergent part of the free energy is
%
\begin{equation}
    \Eff^{(1)}_{\mathrm{div}, \pipm }
    =
    - \frac{1}{4} \frac{\mu^{-2 \epsilon}}{(4\pi)^2} 
    \left(
        \frac{1}{\epsilon} + \frac{3}{2} + \ln \frac{\tilde \mu^2}{\tilde m_1^2}
    \right) \tilde m_1^4
    - \frac{1}{4} \frac{\mu^{-2 \epsilon}}{(4\pi)^2} 
    \left(
        \frac{1}{\epsilon} + \frac{3}{2} + \ln \frac{\tilde \mu^2}{\tilde m_2^2}
    \right) \tilde m_2^4
    + \mathcal{O}(\epsilon).
\end{equation}
%
We then define the finite integral
%
\begin{equation}
    \Eff^{(1)}_{\mathrm{fin}, \pipm}
    = 
    \frac{1}{2} \int \frac{\dd^3 p}{(2\pi)^3} (E_{\pi^+} + E_{\pi^-} - E_1 - E_2),
\end{equation}

The charged kaon contribution is
%
\begin{equation}
    \Eff^{(1)}_{\Kpm}
    =
    -\frac{i}{VT}\frac{1}{2}
    \ln\left\{ 
        \det(-D_{45}^{-1})
    \right\}
    =
    -\frac{i}{VT}\frac{1}{2}
    \ln\left\{ 
        \det\left[(- p^2 + m_4)(- p^2 + m_5) - p_0^2m_{45}^2\right]
    \right\},
\end{equation}
%
where we have used the original form of the determinant wich we found in \autoref{subsection: pion-condensed phase}.
To performe the integral, we will use the following rewritings
%
\begin{align}
    (-p^2 + m_4^2)(-p^2 + m_5^2) - p_0^2 m_{45}^2
    &
    = \left[-p^2 + \frac{1}{2}(m_1^4 + m_5^2)\right]^2 
    - p_0^2 m_{45}^2 - \frac{1}{4}(m_4^2 - m_5^2)^2, \\
    \left[-p^2 + \frac{1}{2}(m_4^2 + m_5^2)\right]^2 - p_0^2 m_{45}^2
    &
    = \left[-p^2 + \frac{1}{2}(m_4^2 + m_5^2) - p_0 m_{45} \right]
    \left[-p^2 + \frac{1}{2}(m_1^4 + m_5^2) + p_0 m_{45} \right], \\
    - p^2 + \frac{1}{2}(m_4^2 + m_5^2) \pm p_0 m_{45}
    &= - \left(p_0 \mp \frac{1}{2}m_{45}\right)^2 + |\vv p|^2 
    + 
    \tilde m_{45}^2, 
\end{align}
%
where we have defined 
$
\tilde m_{45}^2 = \frac{1}{2}(m_4^2 + m_5^2) + \frac{1}{4} m_{45}^2.
$
As $m_4 = m_5$, we can then write the free energy integrals on the form
%
\begin{equation} 
    \nonumber
    \Eff_\Kpm^{(1)}
    =
    \frac{1}{2} \int \frac{\dd^4 p}{(2\pi^4)}
    \ln \left[
        - \left(p_0 + \frac{1}{2}m_{45}\right)^2 
        + |\vv p|^2 
        + \tilde m_{45}^2 
    \right]
    +
    \frac{1}{2}
    \int \frac{\dd^4 p}{(2\pi^4)}
    \ln \left[
        - \left(p_0 - \frac{1}{2}m_{45}\right)^2 
        + |\vv p|^2 
        + \tilde m_{45}^2 
    \right]
\end{equation}
%
After shifting the integration variable $p_0$, we can again us \autoref{dimreg integral}.
The result is therefore
%
\begin{equation}
    \Eff_\Kpm^{(1)}
    =
    - \frac{1}{2} \frac{\mu^{-2 \epsilon} }{(4\pi)^2} 
    \left(
        \frac{1}{\epsilon} + \frac{3}{2} + \ln \frac{\tilde \mu^2}{\tilde m_{45}^2}
    \right)\tilde m_{45}^4.
\end{equation}
%
The approach for the neutral kaon is the same, only with different masses.
The result is
%
\begin{equation}
    \Eff_\Kpm^{(1)}
    =
    - \frac{1}{2} \frac{ \mu^{-2 \epsilon}}{(4\pi)^2} 
    \left(
        \frac{1}{\epsilon} + \frac{3}{2} + \ln \frac{\tilde \mu^2}{\tilde m_{67}^2}
    \right)\tilde m_{67}^4.
\end{equation}
%
where
$
\tilde m_{67}^2 = \frac{1}{2}(m_6^2 + m_7^2) + \frac{1}{4} m_{67}^2.
$
The leading-order, one-loop contribution to the free energy is
%
\begin{align}
    \label{one-loop free energy}
    \nonumber
    \Eff^{(1)}_2
    =
    -\frac{1}{2} \frac{  \mu^{-2\epsilon}}{(4 \pi)^2} 
    \Bigg[&
        \left(
            \frac{1}{\epsilon} + \frac{3}{2} + \ln\frac{\tilde \mu^2}{m_3^2}
        \right)
        m_3^4
        +
        \frac{1}{2}
        \left(
            \frac{1}{\epsilon} + \frac{3}{2} + \ln\frac{\tilde \mu^2}{m_8^2} 
        \right)
        m_8^4+
        \frac{1}{2}
        \left(
            \frac{1}{\epsilon} + \frac{3}{2} + \ln\frac{\tilde \mu^2}{\tilde m_1^2}
        \right)
        \tilde m_1^4 \\ \nonumber
        & +
        \left(
            \frac{1}{\epsilon} + \frac{3}{2} + \ln\frac{\tilde \mu^2}{\tilde m_{45}^2}
        \right)
        \tilde m_{45}^4
        +
        \left(
            \frac{1}{\epsilon} + \frac{3}{2} + \ln\frac{\tilde \mu^2}{\tilde m_{67}^2} 
        \right)
        \tilde m_{67}^4
    \Bigg]
    + \Oh(\epsilon)
    + \Eff^{(1)}_{\pipm,\text{fin}}
\end{align}
%
This contribution, however, is not finite as $\epsilon\rightarrow 0$.
We must therefore include couter-terms from the higher order Lagrangian.


\subsection{Renormalization}


The second-order, tree-level contribution is given by second-order the static Lagrangian,
%
\begin{equation}
    \Eff^{(0)}_4 = - \Ell_4^{(0)},
\end{equation}
%
which we found in \autoref{nlo static lagrangian}.
In combination with the results from the last subsection, this gives the total next-to-leading order contribution to the free energy,
%
\begin{align}
    \nonumber
    \Eff_{\text{NLO}}
    ={}&
    -\frac{1}{2} f^2 
    \left( 2 \bar m^2 \cos\alpha + \mu_I^2 \sin^2\alpha + m_S^2\right)
    + \Eff_{\pipm,\text{fin}} \\ \nonumber
    &-2(2L_1^r + 2L_2^r + L_3^r) \mu_I^4 \sin^4\alpha
    - 4  L_4^r \left( 2 \bar m^2 \cos\alpha + m_S^2 \right) \mu_I^2\sin^2\alpha
    - 4 L_5^r \bar m^2 \mu_I^2 \cos\alpha \sin^2 \alpha 
    \\ \nonumber
    & 
    - 4 L_6^r (2\bar m^2\cos\alpha + m_S^2)^2
    - 2 L_8^r \left(2 \bar m^4 \cos2\alpha + 2 \Delta m^4 + m_S^4\right)
    - H_2^r \left(2\bar m^4 + 2\Delta m^4 + m_S^4 \right) \\\nonumber
    & - \frac{1}{2} \frac{1}{(4 \pi)^2}  
    \Bigg[
        \left(
            \frac{1}{2} + \ln\frac{M^2}{m_3^2}
        \right)
        m_3^2
        + 
        \frac{1}{2}
        \left(
            \frac{1}{2} + \ln\frac{M^2}{m_8^2} 
        \right)
        m_8^4
        +
        \frac{1}{2}
        \left(
            \frac{1}{2} + \ln\frac{M^2}{\tilde m_1^2}
        \right)
        \tilde m_1^4
        \\
        & 
        \quad\quad\quad\quad\quad
        +
        \left(
            \frac{1}{2} + \ln\frac{M^2}{\tilde m_{45}^2}
        \right)
        \tilde m_{45}^4
        +
        \left(
            \frac{1}{2} + \ln\frac{M^2}{\tilde m_{67}^2} 
        \right)
        \tilde m_{67}^4
    \Bigg].
\end{align}
%
Here, the mass terms are
%
\begin{align}
    \tilde m_1^2 
    & =
    \bar m^2 \cos\alpha \\
    \tilde m_2 &= m_3^2 = \bar m^2 \cos\alpha + \mu_I^2 \sin^2\alpha \\
    m_8^2 & = \frac{1}{3} (\bar m^2 \cos\alpha + 2m_S^2) \\
    \tilde m_{45}^2 & 
    = \frac{1}{2}(\bar m^2 \cos \alpha - \Delta m^2 + m_S^2)
    + \frac{1}{4} \mu_I^2\sin^2\alpha \\
    \tilde m_{67}^2 & 
    = \frac{1}{2}(\bar m^2 \cos \alpha + \Delta m^2 + m_S^2)
    + \frac{1}{4} \mu_I^2\sin^2\alpha
\end{align}

Notice that divergent $1/\epsilon$-terms cancel exactly, and the total NLO contribution is therefore finite even as $\epsilon\rightarrow 0$, as expected.
Furthermore, all dependence on the unphysical parameter $\mu$ has vanished, and only the renormalized coupling constants and the mass $M$ at which they are defined remain.

We also need the relationships between the bare constants that appear in the Lagrangian and their corresponding physical constants.
For numerical results, we will consider $\Delta m = 0$.
This is done in the QCD-lattice research and allows for easier comparison with those results~\autocite{brandtNewClassCompact2018}.
Furthermore, we expect this to have marginal effects on our results, as $\Delta m$ do not affect the results until NLO.
After setting $\Delta m = 0$, we are left with the bare masses $\bar m$ and $m_S$, as well as the bare decay constant $f_\pi$.
Given a choice of three physical constants to determine these, we will be able to predict other constants.
We must therefore use measured values of \emph{only} three constants, as more would lead to an overdetermined system of equations and inconsistencies.
As we work in with a pion condensate without electromagnetic interactions, it is natural to choose the neutral pion mass $m_\pio$ and the pion decay $f_\pi$, whose values can be found in \autoref{section: units}.
In addition, we choose the mass of the neutral kaon, $m_{\Ko}$, to determine $m_S$.
Now that $\Delta m = \Delta m_\text{EM} = 0$, the leading order kaon mass is $m_{K,0}^2 = m_{\Ko,0}^2 = m_{\Kpm,0}^2 = (\bar m + m_S)/2$.
The NLO relationships that determin the bare parameters are~\autocite{gasserChiralPerturbationTheory1985}
%
\begin{align}
    m_\pi^2 
    =&\, 
    m_{\pi,0}^2
    \left[
        1
        +
        \left(
            16L_8^r - 8L_5^r
            +
            \frac{1}{2(4\pi)^2}
            \ln\frac{m_{\pi,0}^2}{M^2}
        \right)\frac{m_{\pi,0}^2}{f^2}
        +
        \left(
            24L_6^r - 12L_4^r
            -
            \frac{1}{6(4\pi)^2}
            \ln\frac{m_{\eta,0}^2}{M^2}
        \right)\frac{m_{\eta,0}^2}{f^2}
    \right] \\
    m_{K}
    =&\,
    m_{K,0}^2
    \left[
        1
        + 8\left(2L_6^r - L_4^r\right) \frac{m_{\pi,0}^2}{f^2}
        + 8\left(2L_8^r - L_5^r + 4L_6^r- 2L_4^r\right) \frac{m_{K,0}^2}{f^2}
        +
        \left(        
            \frac{1}{3(4 \pi)^2} 
            \ln\frac{m_{\eta,0}^2}{M^2}
        \right)
        \frac{m_{\eta,0}^2}{f^2}
    \right]\\
    f_\pi^2
    =&\, f^2
    \left[
        1
        + 
        \left(
            8 L_4^r + 8 L_5^r - \frac{2}{(4\pi)^2} \ln\frac{m_{\pi,0}^2}{M^2}
        \right) \frac{m_{\pi,0}^2}{f^2}
        +
        \left(
            16 L_4^r
            - \frac{1}{(4\pi)^2} \ln\frac {m_{K,0}^2}{M^2}
        \right) \frac{m_{K,0}^2}{f^2}
    \right]
\end{align}
%
These three equations allow us to determine $\bar m$, $m_S$ and $f$ to next-to-leading order.
The solutions for this set of equations are shown in \autoref{table: nlo values}.

\begin{table}[!htb]
    \centering
    \caption{The leading order and next-to-leading order values for the bare masses and decay constant. Theses values are for $\Delta m = 0$.}
    \label{table: nlo values}
    \begin{tabular}{c c c c}
        \hline \hline
        quantity & LO[ MeV] & NLO [MeV] & NLO/LO \\
        \hline
        $f$ & 92.07 & 78.55 & 0.853\\
        $\bar m$ & 134.98 & 135.53 & 1.004 \\
        $m_S$ & 664.17 & 743.48 & 1.119
    \end{tabular}
\end{table}


The next-to-leading order relation between $\alpha$ and $\mu_I$ is given by, as explained in \autoref{appendix: consisten expansion},
%
\begin{equation}
    \pdv{\Eff(\mu_I, \alpha)}{\alpha} \Bigg|_{\alpha = \alpha_\text{NLO}} = 0.
\end{equation}
%
The result is compared to the leading order result in (Fig)



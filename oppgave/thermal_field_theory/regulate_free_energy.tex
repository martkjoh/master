\subsection{Low-temperature limit}

Using the result from \autoref{section: thermal sum} and the result for the free energy density of the free scalar field, \autoref{free scalar result 2}, we get an expression for the free energy density, $\Eff = F/V$, 
%
\begin{equation}
    \label{free scalar free enrgy}
    \Eff = \frac{\ln(Z)}{\beta V}
    = \frac{1}{2} \int_{\tilde V} \frac{\dd^3 k}{(2 \pi)^3}
    \left[
        \omega_k + \frac{2}{\beta }\ln(1 - e^{-\beta\omega_k})
    \right].
\end{equation}
%
The free energy density thus has two contributions from two parts; the first part is dependent on temperature, and the other part is a temperature-independent vacuum contribution.
Noticing that the integral is spherically symmetric, we may write the two contributions as
%
\begin{equation}
    \Eff_0 = \frac{1}{2} \frac{1}{2 \pi^2}\int_{\R} \dd k \, k^2 \sqrt{k^2 + m^2}, \quad
    \Eff_T = \frac{T^4}{2 \pi^2}\int_\R \dd x \, x^2  
    \ln\left(1 - \exp{-\sqrt{x^2 + (m/T)^2}}\right).
\end{equation}
%
The temperature-independent part, $\Eff_0$, is divergent, and we must impose a regulator and then add counter-terms.
$\Eff_T$, however, is convergent. 
To see this, we use the series expansion $\ln(1 + \epsilon) \sim \epsilon + \Oh(\epsilon)$ to find the leading part of the integrand for large $k$'s, 
%
\begin{equation}
    x^2 \ln\left(1 - \exp{-\sqrt{x^2 + (\beta m)^2}}\right) \sim - x^2 e^{-x}, 
\end{equation}
%
which is exponentially suppressed, making the integral convergent.
In the limit of $T \rightarrow 0$, we get
%
\begin{align}
    \nonumber
    \Eff_T & \sim 
    \frac{T^4}{2 \pi^2} \int_\R \dd x \, x^2 \ln(1 - e^{-x})
    = - \frac{T^4}{2 \pi^2} \sum_{n=1} \frac{1}{n} \pdv[2]{}{n} \int \dd x e^{-nx}
    = - \frac{T^4}{2 \pi^2} \sum_{n=1} \frac{2}{n^4}
    = - \frac{T^4}{\pi^2} \zeta(4),
\end{align}
%
where $\zeta$ is the Riemann-zeta function.
Using $\zeta(4) = \frac{\pi^4}{90}$, we get
%
\begin{equation}
    \Eff_T \sim - \frac{\pi^2}{90} T^4, \quad T \rightarrow 0.
\end{equation}



\subsection{Regularization and renormalization}
\label{subsection: renormalization}


Returning to the temperature-independent part, we use dimensional regularization to control its divergent behavior.
This is discussed in \autoref{section: integral}.
Using the definiton \autoref{def dimreg integral}, we can write $\Eff_0 = \Phi_3(m, 3, -1/2) / 2$.
This integral is calculated in \autoref{subsection: free energy integral}, with the result
%
\begin{equation}
    \Eff_0 =
    - \mu^{-2\epsilon} \frac{1}{4}\frac{m^4}{(4 \pi)^2}
    \left(
        \frac{1}{\epsilon} 
        + \frac{3}{2}
        + \ln \frac{\tilde\mu^2}{m^2}
    \right)
    + \Oh(\epsilon).
\end{equation}
%
As detailed in \autoref{section: integral}, this result uses the $\overline{\text{MS}}$-scheme.
Now that we have applied a regulator, we can handle the divergence in a well-defined way.
When $\epsilon \neq 0$, we can subtract terms that are proportional to $\epsilon^{-1}$, and be left with a term that is finite in the limit $\epsilon \rightarrow 0$.

We can subtract the divergences by performing \emph{renormalization}.
Consider an arbitrary Lagrangian, 
%
\begin{equation}
    \Ell[\varphi] = \sum_n \lambda_n \mathcal{O}_n[\varphi].
\end{equation}
%
Here, $\mathcal{O}_n[\varphi]$ are operators consisting of $\varphi$ and $\partial_\mu \varphi$, and $\lambda_n$ are coupling constants.
In $d$ dimensions, the action integral is
%
\begin{equation}
    S[\varphi] = \sum_n \int \dd^d x \, \lambda_n \mathcal{O}_n[\varphi].
\end{equation}
%
The action has mass dimension $0$.
This means that all terms $\lambda_n \mathcal O_n$ must have mass dimension $d$, as $[\dd^d x] = -d$.
We are free to choose the coupling constant corresponding to $\mathcal O_0 = \partial_\mu \varphi \partial^\mu \varphi$ to be of mass dimension 0, and therefore set $\lambda_0 = 1/2$ to get canonical normalization.
This allows us to deduce the dimensionality of $\varphi$.
As $[\partial_\mu] = 1$, we have that $[\varphi] = (d-2)/2$.
Assume $\mathcal O_n$ consists of $k_n$ factors of $\varphi$, and $l_n$ factors of $\partial_\mu \varphi$.
We must then have
%
\begin{align}
    [\lambda_n] + [\mathcal{O}_n] - d &= [\lambda_n] + (k_n + l_n)(d - 2) / 2 + l_n - d = 0, \\
    \implies D_n := [\lambda_n] &= d - k_n \frac{d - 2}{2} - l_n \frac{d}{2}.
\end{align}
%
From this formula, we recover that $[\lambda_0] = 0$, and if $\mathcal O_1 = \varphi^2$, then $[\lambda_1] = 2$, which we recognize as the mass squared term.
The mass dimensions of these coupling constants are independent of $d$.
However, the coupling constant for the interaction term
\begin{equation}
    - \frac{1}{4!} \lambda_3 \varphi^4
\end{equation}
has mass dimensions $[\lambda_3] = d -4(d-2)/2 = 4 - 2d$.
Our goal now is to exchange the bare coupling constants $\lambda_n$ with renormalized ones, $\lambda_n^r$, and remove the divergent terms proportional to $(d - 4)^{-m}$.
We can always define the renormalized coupling constants as dimensionless, i.e., $[\lambda_n^r] = 0$, by measuring them in units of a mass scale.
We therefore write 
%
\begin{align*}
    \lambda_n = \mu^{4 - D_n}
    \left[
        \lambda_n^r + \sum_{m=1} \frac{a_m(\lambda_n^r)}{(d - 4)^{m}}
    \right],
\end{align*}
%
where we have introduced the dimensionful parameter $\mu$ to ensure that $\lambda_n$ has the correct mass dimension for the action integral to stay dimensionless.
The functions $a_m$ are then determined to each order in perturbation theory by calculating Feynman diagrams.
As $\mu$ again is arbitrary, $\lambda_4'$ should not depend on this parameter.
In this case, we chose the same renormalization scale as we did when regulating the one-loop integral.
This is only for our own convenience.
This means that if we change $\mu \rightarrow \mu'$, then $\lambda_i^r$ and $a_m$ must adjust to compensate and keep $\lambda_n$ constant~\autocite{thooftDimensionalRegularizationRenormalization1973}.

The vacuum energy term absorbs the divergence in the one-loop contribution to the free energy density.
It is
%
\begin{equation}
    \label{scalar field static term}
    \lambda_4 \mathcal{O}_4 = \lambda_4 = m^4 \lambda_4'.
\end{equation}
%
Using the expansion in terms of the renormalized coupling, we have, 
%
\begin{equation}
    \lambda_4' = \mu^{- 2 \epsilon}\left[ \lambda_4^r + \frac{1}{2 \epsilon} a_1(\lambda_r^4) + ... \right],
\end{equation}
%
where $d = 4 - 2\epsilon$.
After adding \autoref{scalar field static term} to the Lagrangian of the free scalar, the temperature-independent free energy density becomes
%
\begin{equation}
    \Eff_0 \sim - \mu^{-2 \epsilon}  \frac{1}{4} \frac{m^4}{(4 \pi)^2}  
    \left[
        \frac{1}{\epsilon} + \frac{3}{2} + \ln{\frac{\tilde \mu^2}{m^2}}
        + 4 (4 \pi)^2 \left(\lambda_4^r + \frac{1}{2\epsilon} a_1(\lambda_4^r)\right)
    \right],
    \quad \epsilon \rightarrow 0.
\end{equation}
%
Thus, if we choose $a_1 = -8 (4\pi)^2 + \mathcal{O}(\lambda_4^r)$, and define $\lambda_4'^r = 4(4\pi)^2\lambda_4^r$, we are able to cancel the divergence, and may take the limit $\epsilon \rightarrow 0$ safely.
The free energy is now
%
\begin{equation}
    \Eff = -\frac{1}{4} \frac{m^4}{(4 \pi)^2} 
    \left(
        \frac{3}{2} + \lambda_4'^r + \ln \frac{\tilde \mu^2}{m^2}
    \right)
    +
    \frac{T^4}{2\pi^2} \int \dd x \, x^2 \ln\left(1 - \exp{-\sqrt{x^2 + \beta^2 m^2}}\right).
\end{equation}
%
Notice that all choices we have made up until now, such as defining $\lambda_4 = m^4 \lambda_4'$ and using the same renormalization scale $\mu$, have no impact on this result.
Different choices would force us to define $\lambda_4^r$ and $a_4$ differently, leaving us with the same result.


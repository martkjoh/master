In this appendix, we will give a brief overview of the basics of thermal field theory using imaginary-time formalism.
We will derive the relevant scalar path integral, discuss the operator determinant and regularization, derive the Feynman rules for a $\varphi^4$-theory and discuss the fermionic theory.
This appendix is based on~\autocite{kapustaFiniteTemperatureFieldTheory2006,laineBasicsThermalField2016}.



\section{*Statistical mechanics}
\label{section: statistical mechanics}

In classical mechanics, a thermal system at temperature $T = 1 / \beta$ is described as an ensemble state, which have a probability $P_n$ of being in state $n$, with energy $E_n$.
In the canonical ensemble, the probability is proportional to $e^{-\beta E_n}$.
The expectation value of some quantity $A$, with value $A_n$ in the state $n$, is
%
\begin{equation*}
    \ex{A} 
    = \sum_n A_n P_n = \frac{1}{Z} \sum_n A_n e^{- \beta E_n}, \quad 
    Z  = \sum_n e^{-\beta A_n}.
\end{equation*}
$Z$ is called the partition function. In quantum mechanics, an ensemble configuration is described by a non-pure density operator,
%
\begin{equation*}
    \hat \rho = \sum_n P_n \ket{n}\bra{n},
\end{equation*}
where $\ket{n}$ is some basis for the relevant Hilbert space. Assuming $\ket{n}$ are energy eigenvectors, i.e., $\hat H \ket{n} = E_n \ket{n}$, the density operator for the canonical ensemble, where $P_n = Ce^{-\beta E_n}$, is
%
\begin{equation*}
    \hat \rho 
    = \sum_n C e^{-\beta E_n} \ket{n}\bra{n} 
    = C e^{-\beta \hat H} \sum_n \ket{n}\bra{n} 
    = C e^{-\beta \hat H}.
\end{equation*}
The expectation value in the ensemble state of a quantity corresponding to the operator $\hat A$ is given by
%
\begin{align}
    \ex{A} = \frac{ \Tr{\hat \rho \hat A} }{\Tr{\hat \rho }}
    = \frac{1}{Z} \Tr{\hat A e^{-\beta \hat H}}.
\end{align}
%
The partition function $Z$ ensures that the probabilities add up to 1, and is defined as
%
\begin{equation}
    Z = \Tr{e^{-\beta \hat H}}.
\end{equation}
%
The grand canonical ensemble takes into account the conserved charges of the system, which are a result of Nöther's theorem, as discussed in 
\autoref{section: symmetry and goldstone's theorem}.
In the grand canonical ensemble, a system with $n$ conserved charges $Q_i$ has probability proportional to $e^{-\beta (H - \mu_i Q_i)}$.
Here, $\mu_i$ are the chemical potentials corresponding to conserved charge $Q_i$.
This leads to the partition function
%
\begin{equation}
    Z = \Tr{e^{-\beta(\hat H - \mu_i \hat Q_i)}}.
\end{equation}
%



\section{*Imaginary-time formalism}
\label{section: imaginary-time formalism}

The partition function may be calculated similarly to the path integral approach, in what is called the imaginary-time formalism. 
This formalism is restricted to time-independent problems and is used to study fields in a volume $V$.
This volume is taken to infinity in the thermodynamic limit.
As an example, take a scalar quantum field theory with the Hamiltonian
%
\begin{equation}
    \hat H
    = \int_V \dd^3 x \, \hat \He\left[\hat \varphi(\vec x), \hat \pi(\vec x)\right],
\end{equation}
%
where $\hat \varphi(\vec x)$ is the field operator, and $\hat \pi(\vec x)$ is the corresponding canonical momentum operator.
These field operators have time-independent eigenvectors, $\ket{\varphi}$ and $\ket \pi$, defined by
%
\begin{equation}
    \hat \varphi(\vec x) \ket{\varphi} = \varphi(\vec x) \ket{\varphi}, \quad
    \hat \pi(\vec x) \ket{\pi} = \pi(\vec x) \ket{\pi}.
\end{equation}
%
In analogy with regular quantum mechanics, they obey the relations
\footnote{Some authors write $\D\pi/2 \pi$. This extra factor $2\pi$ is a convention that we leave out for notational clarity.}
%
\begin{gather}
    \label{functional completness}
    \one
    = \int \D \varphi(\vec x) \ket{\varphi}\bra{\varphi} 
    = \int \D\pi(\vec x) \ket{\pi}\bra{\pi}, \\
     \braket{\varphi|\pi} 
    = \exp{i \int_V \dd^3 x \, \varphi(\vec x) \pi(\vec x)}, \\
    \braket{\pi_a|\pi_b}
    =  \delta(\phi_a - \phi_b), \quad
    \braket{\varphi_a|\varphi_b} 
    = \delta(\varphi_a - \varphi_b).
\end{gather}
%
The functional integral is defined by starting with $M$ degrees of freedom, $\{\varphi_m\}_{m=1}^M$ located at a finite grid $\{\vec x_m\}_{m=1}^M \subset V$.
The integral is then the limit of the integral over all degrees of freedom, as $M \rightarrow \infty$:
%
\begin{equation*}
    \lim_{M \rightarrow \infty} \int \left(\prod_{m=1}^M \dd \varphi_m\right)
    : = \int \D \varphi(\vec x) .
\end{equation*}
The functional Dirac-delta $\delta(f) = \prod_x\delta(f(x))$ is generalization of the familiar Dirac delta function.
Given a functional $\mathcal{F}[f]$, it is defined by the relation
%
\begin{equation}
    \int \D f(x)\, \mathcal{F}[f] \delta(f - g) = \mathcal{F}[g].
\end{equation}
%
The Hamiltonian is the limit of a sum of Hamiltonians $\hat H_m$ for each point $\vec x_m$
%
\begin{equation*}
    \hat H
    = \lim_{M \rightarrow \infty} \sum_{m=1}^M 
    \frac{V}{M} \hat H_m(\{\hat \varphi_m\}, \{\hat \pi_m\}).
\end{equation*}
$H_m$ may depend on the local degrees of freedom $\hat \varphi_m, \, \hat \pi_m$ as well as those at neighboring points.
By inserting the completeness relations \autoref{functional completness} $N$ times into the definition of the partition function, it may be written as
%
\begin{align*}
    Z& 
    = \int \D\varphi(\vec x) \, \inner{\varphi}{e^{- \beta \hat H}}{\varphi}\\
    &= 
    \prod_{n=1}^N
    \left(
        \int \D \varphi_n (\vec x) \int \D \pi_n(\vec x)
    \right) 
    \prod_{n=1}^N  \braket{\varphi_n}{\pi_n}
    \inner{\pi_n}{e^{- \epsilon \hat H}}{\varphi_{n+1}} \braket{\varphi_{1}}{\varphi_{N+1}},
\end{align*}
%
where $\epsilon = \beta / N$. The last term ensures that $\varphi_1 = \varphi_{N+1}$.
Bosons such as the scalar field $\varphi$, follow the periodic boundary condition $\varphi(0, \vec x) = \varphi(\beta, \vec x)$.
Fermions, as we will show later, follow the anti-periodic boundary condition $\psi(0, \vec x) = -\psi(\beta, \vec x)$.
We now want to exploit the fact that $\ket{\pi}$ and $\ket{\varphi}$ are the eigenvectors of the operators that define the Hamiltonian.
In our case, as the Hamiltonian density $\He$ can be written as a sum of functions of $\varphi$ and $\pi$ separately, $\He[\varphi(\vec x), \pi(\vec x)] = \mathcal{F}_1[\varphi(\vec x)] + \mathcal{F}_2[\pi(\vec x)]$ we may evaluate it as $\inner{\pi_n}{\He[\hat \varphi(\vec x), \hat \pi(\vec x)]}{\varphi_{n+1}} = \He[\varphi_{n+1}(\vec x), \pi_n(\vec x)] \braket{\pi_n | \varphi_{n+1}}$.
This relationship does not, however, hold for more general functions of the field operators.
In that case, one has to be more careful about the ordering of the operators, for example, by using \emph{Weyl ordering}~\cite{peskinIntroductionQuantumField1995}.
By series expanding $e^{-\epsilon \hat H}$ and exploiting this relationship, the partition function can be written as, to second order in $\epsilon$,
%
\begin{align*}
    Z = 
    \prod_{n=1}^N  
    &
    \left(
        \int \D \varphi_n (\vec x) \int \D \pi_n(\vec x)
    \right)
    \\ &
    \times \exp{-\epsilon \sum_{n=1}^N \int_V \dd^3x \,
    \left(
        \He[\varphi_n(\vec x), \pi_n(\vec x)] - i \pi_n(\vec x) \frac{\varphi_n(\vec x) - \varphi_{n+1}(\vec x)}{\epsilon}
    \right)
    }.
\end{align*}
%
We denote $\varphi_n(\vec x) = \varphi(\tau_n, \vec x) $, $\tau \in [0, \beta]$ and likewise with $\pi_n(\vec x)$. 
In the limit $N \rightarrow \infty$, the expression for the partition function becomes
%
\begin{flalign}
    \label{Thermal partition function}
    Z = \int_S \D \varphi(\tau, \vec x) \D \pi(\tau, \vec x) \,
    \exp{
        - \int_0^\beta \dd \tau \int_V \dd \vec x \, 
        \left(
            \He[\varphi(\tau, \vec x), \pi(\tau, \vec x)]
            - i \pi(\tau, \vec x) \dot \varphi(\tau, \vec x)
        \right)
        },
\end{flalign}
%
where $S$ is the set of field configurations $\varphi$ such that $\varphi(\beta, \vec x) = \varphi(0, \vec x)$.
With a Hamiltonian density of the form $\He = \frac{1}{2} \pi^2 + \frac{1}{2} (\nabla \varphi)^2 + \Ve(\varphi)$, we can evaluate the integral over the canonical momentum $\pi$ by discretizing $\pi(\tau_n, \vec x_m) = \pi_{n,m}$,
%
\begin{align*}
    & \int \D \pi \exp{-  \int_0^\beta\dd \tau \int_V \dd^3x 
    \left(
        \frac{1}{2} \pi^2 - i \pi \dot \varphi 
    \right)} \\
    & = \lim_{M,N \rightarrow \infty} \int \left(\prod_{m, n = 1}^{M, N} \frac{\dd \pi_{m, n}}{2 \pi}\right)
    \exp{
        - \sum_{m, n} \frac{V\beta}{MN}
        \left[
            \frac{1}{2}  (\pi_{m, n} - i \dot \varphi_{m, n})^2
            + \frac{1}{2} \dot \varphi_{m, n}^2
        \right]
    } \\
    & = \lim_{M,N \rightarrow \infty} \left( \frac{M N }{2 \pi V \beta} \right)^{MN/2}
    \exp{- \int_0^\beta\dd \tau \int_V \dd^3x \, \frac{1}{2}\dot \varphi^2},
\end{align*}
%
where $\dot \varphi_{m, n} = (\varphi_{m, n+1} - \varphi_{m, n})/\epsilon$.
The partition function is then, 
%
\begin{equation}
    Z = C \int \D \varphi
    \exp{
        - \int_0^\beta \dd \tau \int_V \dd^3 x
        \left[
            \frac{1}{2} \left(\dot \varphi^2 + \nabla \varphi^2\right) 
            + \Ve(\varphi)
        \right]
    }.
\end{equation}
%
Here, $C$ is the divergent constant that results from the $\pi$-integral.
In the last line, we exploited the fact that the variable of integration $\pi_{n,m}$ may be shifted by a constant without changing the integral, and used the Gaussian integral
%
\begin{equation*}
    \int_{-\infty}^{\infty} \dd x \, e^{-a x^2/2} = \sqrt{\frac{2 \pi}{a}}.
\end{equation*}


The partition function resulting from this procedure may also be obtained by starting with the ground state path integral
%
\begin{equation*}
    Z_g
    =\int \D\varphi\D\pi
    \exp{i \int_{\Omega'} \dd^4 x \, \left(\pi\dot\varphi - \He[\varphi, \pi]\right)}
    = C' \int \D \varphi(x)
    \exp{i \int_{\Omega'} \dd^4 x \, \Ell[\varphi, \partial_\mu \varphi]},
\end{equation*}
and follow a formal procedure.
First, the action integral is modified by performing a Wick-rotation of the time coordinate $t$.
This involves changing the domain of $t$ from the real line to the imaginary line by closing the contour at infinity and changing variable $it \rightarrow \tau$.
The new variable is then restricted to the interval $\tau\in [0, \beta]$, and the domain of the functional integral $\int \D \varphi$ is restricted from \emph{all} (smooth enough) field configurations $\varphi(t, \vec x)$, to only those that obey $\varphi(\beta, \vec x) = e^{i\theta} \varphi(0, \vec x) $, which is denoted $S$.
Here, $\theta \in \{0, \pi\}$, depending on if the particle is a boson of fermion.
This procedure motivates the introduction of  the Euclidean Lagrange density, 
$\Ell_E(\tau, \vec x) = -\Ell(-i \tau, \vec x)$, as well as the name ``imaginary-time formalism''.
The result is the same partition function as before,
%
\begin{align}
    Z & = C \int_S \D \varphi \int \D \pi
    \exp{
        - \int_0^\beta \dd \tau \int_V \dd^3x \, 
        \left[
            - i\dot \varphi \pi
            + \He(\varphi, \pi)
        \right]
    } \nonumber \\ \label{free scalar result 2}
    & =
    C' \int_S \D \varphi
    \exp{- \int_0^\beta \dd \tau \int_V \dd^3x \, \Ell_E(\varphi, \pi)}.
\end{align}
%



\subsection{Fourier series}

Due to the finite range of the imaginary-time coordinate $\tau \in [0, \beta]$, the momentum-space fields in imaginary-time formalism have a discrete coordinate. 
We define the Matsubara-frequencies as $\omega_n = 2 n \pi / \beta$ for bosons and $\omega_n = (2n + 1) \pi / \beta$ for fermions.
They together form the reciprocal space $\tilde \Omega = \{\omega_n\}\times \tilde V$, where $\tilde V$ is reciprocal to $V$.
To get a more economical notation, we denote the Euclidean real-space coordinates as $X = (\tau, \vec x)$ and the reciprocal space coordinates as $K = (\omega_n, \vec k)$.
The dot product is $X\cdot K = \omega_n \tau + \vec k \cdot \vec x$.
In the limit $V\rightarrow \infty$, we follow the prescription
%
\begin{equation*}
    \frac{1}{V} \sum_{\vec p \in \tilde V} \rightarrow \int_{\R^3} 
    \frac{\dd^3 p}{(2 \pi)^3}.
\end{equation*}
The sum over all degrees of freedom, and the corresponding integrals for the thermodynamic limit are
%
\begin{align*}
     \frac{\beta V}{NM}\sum_{n=1}^N \sum_{\vec x_m \in V} 
    & \xrightarrow{N,M\rightarrow \infty} \int_{0}^\beta \dd \tau \int_{\R^3} \dd^3 x
    = \int_\Omega \dd X, \\
     \frac{1}{V} \sum_{n=-\infty}^\infty \sum_{\vec k \in \tilde V}
    & \xrightarrow{V \rightarrow \infty} \sum_{n=-\infty}^\infty \int_{R^3} \frac{\dd^3 p}{(2 \pi)^3}
    = \int_{\tilde \Omega} \dd K.
\end{align*}
%
The convention used for the Fourier expansion of thermal fields is in accordance with ~\autocite{kapustaFiniteTemperatureFieldTheory2006}. 
The prefactor is chosen to make the Fourier components of the field dimensionless, which makes it easier to evaluate the trace correctly.
For bosons, the Fourier expansion is
%
\begin{align*}
    \varphi(X)
    = &
    \sqrt{V \beta} \int_{\tilde \Omega} \dd K \,  \tilde \varphi(K) e^{i X\cdot K}
    =
    \sqrt{\frac{\beta}{V}} \sum_{n=-\infty}^\infty \sum_{\vec k \in \tilde V}
    \tilde \varphi_n(\vec p)\, \exp{i(\omega_n \tau + \vec x \cdot \vec k)}, \\
    \tilde \varphi(K)
    = &
    \sqrt{\frac{1}{V \beta^3}} \int_{\tilde \Omega} \dd X \,  \tilde \varphi(X) e^{ - i X\cdot K},
\end{align*}
%
while for Fermions, it is
%
\begin{equation}
    \psi(X) 
    = \sqrt{V} \int_{\tilde \Omega} \dd K \, \tilde \psi(K) e^{i X\cdot K} 
    = \frac{1}{\sqrt{V}} \sum_{n = - \infty}^\infty \sum_{\vec k \in \tilde V}
    \psi(\omega_n, \vec k) \, \exp{i(\omega_n \tau + \vec x \cdot \vec k)}.
\end{equation}
%
Two often-used identities are
%
\begin{align}
    \label{thermal delta}
    \int_{\Omega} \dd X e^{i X\cdot(K - K')} 
    & = \beta \delta_{nn'} (2 \pi)^3 \delta^3(\vec k - \vec k') := \beta \delta(K - K'), \\\
    \int_{\tilde \Omega} \dd K \, e^{i K(X - X')} 
    & = \beta \delta (\tau - \tau') \delta^3(\vec x - \vec x') 
    := \beta \delta(X - X').
\end{align}
%

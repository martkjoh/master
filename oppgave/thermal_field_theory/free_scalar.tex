\section{*Free scalar field}
\label{section:free scalar field}

The procedure for obtaining the thermal properties of an interacting scalar field is similar to that used in scattering theory.
One starts with a free theory, which can be solved exactly.
Then an interaction term is added, which is accounted for perturbatively by using Feynman diagrams.
The Euclidean Lagrangian for a free scalar gas is, after integrating by parts,
%
\begin{equation}
    \Ell_E = \frac{1}{2} \varphi(X) \left( -\partial_E^2 + m^2 \right) \varphi(X)
\end{equation}
%
Here, $X = (\tau, \vec x)$ is the Euclidean coordinate resulting from the Wick-rotation as described in the last section.
We have also introduced the Euclidean Laplace operator, $\partial_E^2 = \partial_\tau^2 + \nabla^2$.
Following the procedure to obtain the thermal partition function yields
%
\begin{equation}
    Z = C \int_S \D \varphi(X) 
    \exp{
        - \int_\Omega \dd X \frac{1}{2} 
        \varphi(X) \left( -\partial_E^2 + m^2 \right) \varphi(X)
    }.
\end{equation}
%
Here, $\Omega$ is the domain $[0, \beta] \times V$.
We then insert the Fourier expansion of $\varphi$ and change the functional integration variable to the Fourier components.
The integration measures are related by
\begin{equation*}
    \D \varphi(X) = \det\left(\frac{\delta \varphi(X)}{\delta \tilde \varphi(K)}\right) \D\tilde \varphi(K),
\end{equation*}
where $K = (\omega_n, \vec k)$ is the Euclidean Fourier-space coordinate.
The determinant factor which appears may be absorbed into the constant $C$, as the integration variables are related by a linear transform.
The action becomes
%
\begin{align*}
    S & = - \int_\Omega \dd X\, \Ell_E 
    = - \frac{1}{2} V \beta \int_\Omega \dd X \int_{\tilde \Omega} \dd K \int_{\tilde \Omega} \dd K' \,
    \tilde \varphi(K') 
    \left(
        \omega_n^2 + \vec k^2 + m^2
    \right)
    \tilde \varphi(K)
    e^{iX\cdot(K - K')} \\
    & = - \frac{1}{2} V \beta^2 \, \int_{\tilde \Omega} \dd K \,
    \tilde \varphi(K)^*
    \left(
        \omega_n^2 + \omega_k^2
    \right)
    \tilde \varphi(K),
\end{align*}
%
where $\omega_k^2 = \vec k^2 + m^2$.
$\tilde \Omega$ is the reciprocal space corresponding to $\Omega$.
We used the fact that $\varphi$ is real, which implies that $\tilde \varphi(-K) = \tilde \varphi(K)^*$, as well as the identity \autoref{thermal delta}.
This gives the partition function 
%
\begin{equation}
    Z = C \int_{\tilde S} \D \tilde \varphi(K) 
    \exp{
        -  \frac{1}{2} V \int_{\tilde \Omega} \dd K \, 
        \tilde \varphi(K)^* \left[\beta^2 (\omega_n^2+ \omega_k^2)\right] \tilde \varphi(K)
    },
\end{equation}
%
Going back to before the continuum limit, this integral can be written as a product of Gaussian integrals and may therefore be evaluated
%
\begin{align*}
    Z = C \prod_{n=-\infty}^\infty \prod_{k \in \tilde V}
    \left(
        \int \dd \tilde \varphi_{n, \vec k} \,
        \exp{
            - \frac{1}{2} \tilde \varphi_{n, \vec k}^*
            \left[\beta^2 (\omega_n^2+ \omega_k^2)\right] 
            \tilde \varphi_{n, \vec k}
            }
    \right)
    = 
    C \prod_{n=-\infty}^\infty \prod_{k \in \tilde V} 
    \sqrt{\frac{2 \pi}{\beta^2 (\omega_n^2 + \omega_k^2)}}.
\end{align*}
%
The partition function is related to free energy $F$ through
%
\begin{equation}
    \label{result free scalar 1}
    \frac{F}{T V}= - \frac{\ln(Z)}{V} = \frac{1}{2} \int_{\tilde \Omega} \dd K \ln[\beta^2(\omega_n^2 + \omega_k^2)] + \frac{F_0}{TV},
\end{equation}
%
where $F_0$ is a constant.

A faster and more formal way to get to this result is to compare the partition function to the multidimensional version of the Gaussian integral~\autocite{kapustaFiniteTemperatureFieldTheory2006,peskinIntroductionQuantumField1995}.
The partition function has the form 
\begin{equation*}
    I_n = \int_{\R^n} \dd^n x\, \exp{- \frac{1}{2} \langle x, D_0^{-1} x\rangle },
\end{equation*}
where $D_0^{-1}$ is a linear operator, and $\langle \cdot , \cdot \rangle$ an inner product on the corresponding vector space.
By diagonalizing $D_0^{-1}$, we get the result
\begin{equation*}
    I_n = \sqrt{\frac{(2 \pi)^n}{\det(D_0^{-1})}}.
\end{equation*}
We may then use the identity
%
\begin{equation}
    \label{lndettrln}
    \det(D_0^{-1}) = \prod_i \lambda_i = \exp{\text{Tr}[\ln(D_0^{-1})]},
\end{equation}
%
where $\lambda_i$ are the eigenvalues of $D_0^{-1}$.
The trace in this context is defined by the vector space $D_0^{-1}$ acts on.
For given an orthonormal basis $x_n$, such that $\langle x_n, x_{n'}\rangle = \delta_{nn'}$ the trace can be evaluated as $\Tr{D_0^{-1}} = \sum_n \langle x_n, D_0^{-1} x_n \rangle$.
Identifying 
\begin{equation*}
    \langle x, D_0^{-1} x\rangle = \int_\Omega \dd X \varphi(X)\left(-\partial_E^2+m^2\right)\varphi(X),
\end{equation*}
we get the formal result
\begin{equation*}
    Z = \det(-\partial_E^2 + m^2)^{-1/2},
\end{equation*}
and 
\begin{equation*}
    \beta F = \frac{1}{2}\Tr{\ln(-\partial_E^2 + m^2)}.
\end{equation*}
The logarithm may then be evaluated by using the eigenvalues of the linear operator.
This is found by diagonalizing the operator,
\begin{equation*}
    \langle x, D_0^{-1} x \rangle 
    = \int_\Omega \dd X \varphi(X)\left(-\partial_E^2+m^2\right)\varphi(X)
    = V  \int_{\tilde \Omega} \dd K 
    \tilde \varphi(K)^* [\beta^2(\omega_k^2 +\omega_n^2)] \tilde \varphi(K),
\end{equation*}
leaving us with the same result as we obtained in 
\autoref{result free scalar 1},
\begin{equation*}
    \beta F 
    = \frac{1}{2} \Tr{\ln(-\partial_E^2 + m^2)} 
    = \frac{1}{2} V \int_{\tilde \Omega} \dd K \ln[\beta^2(\omega_n^2 + \omega_k^2)].
\end{equation*}
Sums similar to this show up a lot, and we show how to evaluate them in the next section.

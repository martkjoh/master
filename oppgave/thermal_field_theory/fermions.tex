\section{*Fermions}
\label{section: fermions}


The derivation of the fermionic path integral is similar to the scalar one, but with some slight but important differences.
If $\psi$ and $\bar \psi$ are fermionic spinor fields, relevant relations are~\autocite{laineBasicsThermalField2016}
%
\begin{align}
    \one 
    &=
    \int \D \psi \D \bar \psi \,
    \exp{- \int \dd X \, \bar \psi \psi}
    \ket{\psi} \bra{\psi}, \\
    Z 
    &= \Tr{\hat \rho} 
    = \int \D \psi \D \bar \psi\,
    \exp{- \int \dd X \, \bar \psi \psi} \braket{- \psi | e^{i\beta \hat H} | \psi},\\
    \braket{\bar \psi | \psi } 
    &= \exp{\int \dd X\, \bar \psi \psi}.
\end{align}
%
The anti-periodic nature of fermion-fields, as mentioned in \autoref{section: imaginary-time formalism}, is due to these differences.
We can verify them by studying the properties of the thermal Greens function.
The thermal Greens function may be written 
%
\begin{equation}
    D(X_1, X_2) = D(\vec x, \vec y, \tau_1, \tau_2) 
    = \Tr{e^{-\beta H} \text T_\tau [ \psi(\tau_1, \vec x) \bar \psi(\tau_2, \vec y) ] }.
\end{equation}
%
Here, $\text T_\tau$  is the imaginary-time-ordering operator, which sorts fields just as the regular time-ordering operator, only with respect to $\tau$ instead of $t$.
In the same way that $i \hat H$ generates the time translation of a quantum field operator through 
$
\hat\psi(x) = \hat\psi(t, \vec x) = e^{it\hat H} \hat \psi(0, \vec x) e^{-it\hat H} 
$, 
the imaginary-time formalism implies the relation
$
    \hat\psi(X) = \hat\psi(\tau, \vec x) 
    = e^{\tau\hat H} \hat \psi(0, \vec x) e^{-\tau \hat H}.
$
Using $\one = e^{\tau \hat H} e^{-\tau \hat H}$ and the cyclic property of the trace, we show that, assuming $\beta>\tau>0$,
\begin{align*}
    D(\vec x, \vec y, \tau, 0)
    % = \bra{\Omega}e^{-\beta \hat H} \T{\psi(\tau, \vec x) \psi(0, \vec y)}\ket{\Omega} 
    &= \frac{1}{Z} \Tr{
        e^{-\beta \hat H} \text T_\tau [ \psi(\tau, \vec x) \bar \psi(0, \vec y) ]
    } \\
    % & = \frac{1}{Z} \Tr{
    %     \psi(0, \vec y) e^{-\beta \hat H} \psi(\tau, \vec x)
    % } \\
    & = \frac{1}{Z} \Tr{
        \text T_\tau[
        e^{-\beta \hat H} e^{\beta \hat H} \bar \psi(0, \vec y) 
        e^{-\beta \hat H} \psi(\tau, \vec x)
        ]
    } \\
    & = \frac{1}{Z} \Tr{
        e^{-\beta \hat H} \text T_\tau [ \psi(\vec y, \beta) \bar \psi(\tau, \vec x)]
    } 
    = \nu D (\vec x, \vec y, \tau, \beta)
    % = \nu \bra{\Omega}        
    % e^{-\beta \hat H} \T{ \psi(\tau, \vec x) \psi( \beta, \vec y) }
    % \ket{\Omega}.
\end{align*}
This implies that $\psi(0, x) = \nu \psi(\beta, \psi)$, where $\nu = \pm 1$ for bosons and fermions respectively, which shows that bosons are periodic in time, as stated earlier, while fermions are anti-periodic.


The Lagrangian density of a free Dirac fermion is
%
\begin{equation}
    \Ell = \bar \psi \left( i \slashed{\partial} - m \right) \psi.
\end{equation}
%
This Lagrangian is invariant under the transformation $\psi \rightarrow e^{-i \alpha} \psi$, which by Nöther's theorem results in a conserved current
%
\begin{equation}
    j^\mu = \pdv{\Ell}{(\partial_\mu \psi)} \delta \psi=  \bar \psi \gamma^\mu \psi.
\end{equation}
%
The canonical momentum corresponding to $\psi$ is
%
\begin{equation}
    \pi = \pdv{\Ell}{(\partial_0 \psi)} = i \bar \psi \gamma^0,
\end{equation}
%
and the Hamiltonian density is 
%
\begin{equation}
    \He = \pi \dot\psi - \Ell
    = \bar \psi (-i\gamma^i\partial_i + m) \psi
\end{equation}
%
In the grand canonical ensemble, we substitute $\He \rightarrow \He - \mu \bar \psi \gamma^0 \psi$.
The Euclidian Lagrangian is then
%
\begin{equation}
    \Ell_E = 
    - \pi \dot\psi + \He(\psi, \pi) - \mu \bar \psi \gamma^0 \psi
    = \bar\psi[\gamma^0 (\partial_\tau - \mu) - i\gamma^i \partial_i + m] \psi,
\end{equation}
%
The partition function for this system is then~\autocite{laineBasicsThermalField2016}
%
\begin{align*}
    Z & = \int \D \psi \D \bar \psi
    \exp{
        - \int_\Omega \dd X \, \bar \psi
        \left[
            \gamma_0(\partial_\tau -\mu) -  i \gamma^i \partial_i + m
        \right]
        \psi
    }\\
    % & = C \int \D \tilde \psi\D \tilde {\bar \psi}\,
    % \exp{
    %     - \int_{\tilde \Omega} \dd K \, \tilde {\bar \psi}
    %     \left[
    %         i \gamma_0(\omega_n + i\mu) + i \gamma_i p_i + m
    %     \right]
    %     \tilde \psi
    % } \\
    & = C \int \D \psi \D {\bar \psi} \,
    e^{- \langle  {\bar \psi}, D^{-1} \psi\rangle} 
    = C \det(D^{-1}).
\end{align*}
In the second line, we have inserted the Fourier expansion of the field, as defined in \autoref{section: imaginary-time formalism}, and changed variable of integration, as we did for the scalar field.
We then used the Grassmann version of the Gaussian integral~\autocite{schwartzQuantumFieldTheory2013},
%
\begin{equation}
    \int \dd \bar \theta_i \dd \theta_i \exp{- \theta_i A_{ij}\theta_j} = \det(A).
\end{equation}
%
The linear operator in this case is 
\begin{equation}
    D^{-1} = i \gamma^0 (-i\partial_\tau + i\mu) - (- i \gamma^i) \partial_i + m
    = 
    \beta [i \tilde \gamma_a k_a + m ].
\end{equation}
This equality must be understood as an equality between linear operators, which are represented in different bases.
We introduced the notation $k_a = (\omega_n + i \mu, k_i)$ and use the Euclidean gamma matrices $\tilde \gamma_i$, as defined in \autoref{subsection: Pauli matrices}.
We use the fact that
%
\begin{equation*}
    \det(i\tilde\gamma_a k_a + m)
    = \det(\gamma^5 \gamma^5)
    \det(i\tilde\gamma_a k_a + m)
    = \det[\gamma^5 (i\tilde\gamma_a k_a + m) \gamma^5]
    = \det(-i\tilde\gamma_a k_a + m),
\end{equation*}
%
Let $\tilde D^{-1} = \beta[-i\tilde\gamma_a k_a + m]$, which means we can write
%
\begin{equation}
    Z = \sqrt{\det(D^{-1})\det(\tilde D^{-1})} = \sqrt{\det(D^{-1}\tilde D^{-1})} 
    = \det[\one \beta^2(k_a k_a + m^2)]^{1/2},
\end{equation}
%
where we have used the anti-commutation rule for the Euclidean gamma-matrices, $\{\tilde \gamma_a, \tilde  \gamma_b\} = 2 \delta_{ab}$.
It is important to keep in mind that the determinant here refers to linear operators on the space of spinor functions.
Thus,
%
\begin{align}
    \nonumber
    \ln(Z) 
    & = \ln\left\{\det[\one\beta^2(k_a k_a + m^2)]^{1/2}\right\}
    = \frac{1}{2} \Tr{\ln[\one\beta^2(k_a k_a + m^2)]} \\
    & =  2 V \int_{\tilde \Omega} \dd K \,  \ln\{ \beta^2[(\omega_n + i\mu)^2 + \omega_k^2]\}.
\end{align}
%
In the last step, we used the fact that the matrix within the logarithm is diagonal.
The matrix part of the trace is therefore trivial.
Using the fermionic version of the thermal sum from \autoref{section: thermal sum} gives the answer
%
\begin{equation}
    \label{free energy fermions}
    \Eff 
    = -\frac{2}{\beta} \int\frac{\dd^3 k}{(2\pi)^3} \, 
    \left[
        \beta \omega_k
        + \ln\left(1 + e^{-\beta(\omega_k-\mu)}\right)
        + \ln\left(1 + e^{-\beta(\omega_k+\mu)}\right)
    \right].
\end{equation}
%
We see again that the temperature-independent part of the integral diverges, and must be regulated.
There are two temperature-dependent terms, one from the particle and one from the anti-particle.


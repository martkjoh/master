\section{Constructing an effective field theory}
\label{section: effective field theories}

One of the most powerful concepts in quantum field theory is the notion of effective field theories.
The methods we have laid out for quantum field theory involve, in general, calculations where we must integrate over all possible momenta and thus all possible energies.
However, we do not profess to know how physics behaves at arbitrarily high energies, which at first glance seem to render our theories moot.
The fact that the standard model allows for such precise predictions suggests that the physics that happens at energies that are accessible to us can be described without knowledge of physics at the highest energies.
This is a familiar concept from our everyday life---we can describe billiard balls colliding or rocks falling with high precision without having an accurate microscopic description of these objects.
An effective field theory is a description of the physics of some underlying theory, some degrees of freedom $\varphi_a$ governed by a Lagrangian $\Ell[\varphi]$, in terms of a smaller set of degrees of freedom, $\pi_i$, governed by an effective Lagrangian $\Ell_\text{eff}[\pi]$.
As they describe the same physics, these two descriptions are related by 
%
\begin{equation}
    Z = \int \D \varphi \, \exp{i\int \dd^4 x \, \Ell[\varphi]}
    = \int \D \pi\, \exp{i\int \dd^4 x \, \Ell_\text{eff}[\pi] }.
\end{equation}
%
We say that the degrees of freedom not present in the effective description have been \emph{integrated out}.
An effective theory can come from integrating out all degrees of freedom above some energy cut off or integrating out a particle and describing the interactions it mediates as point-like.
In \autoref{section: effective action}, we found that the 1PI effective action resulted from integrating out all fluctuations away from the ground state, leaving us with an effective field theory in which all interactions are described entirely at the tree level.
Furthermore, the modern understanding of the Standard Model is that it is an effective field theory.
This would mean some more complete theory of physics---perhaps string theory, quantum loop gravity or something we have yet to think of---acts as an effective theory at low energies.
Low energies, in this context, include collisions at the LHC, which is why the Standard Model is so sucsessful~\autocite{pencoIntroductionEffectiveField2020}.

One of the pioneers of the philosophy of effective field theories was Steven Weinberg.
He proposed that quantum field theories, in themselves, have almost no content beyond some basic assumptions~\autocite{weinbergDevelopmentEffectiveField2021}.
This means that if we try to model a system by writing down the most general possible Lagrangian, we cannot be wrong.
This was formulated more precisely in---as Weinberg himself called it---a ``theorem'':
%
\begin{quote}
    ``[I]f one writes down the most general possible Lagrangian, including all terms consistent with assumed symmetry principles, and then calculates matrix elements with this Lagrangian to any given order of perturbation theory, the result will simply be the most general possible $S$-matrix consistent with analyticity, perturbative unitary, cluster decomposition and the assumed symmetry principles.''~\autocite{weinbergPhenomenologicalLagrangians1979a}
\end{quote}
%
The properties of ``analyticity, perturbative unitarity and cluster decomposition'' are basic properties we expect of good, effective theories.
Analyticity is an assumption about the poles of the $S$-matrix, and perturbative unitarity says that the theory should be unitary, as quantum theories should, for \emph{any order in perturbation theory}.
Cluster decomposition states that non-entangled processes far apart should be independent~\autocite{weinbergQuantumTheoryFields1995,weinbergQuantumTheoryFields1996}.
Such a ``most general possible Lagrangian'' will have the form
\begin{equation}
    \Ell_\text{eff}[\pi] = \sum_i \lambda_i \mathcal O_i[\pi],
\end{equation}
where $\mathcal O_i[\pi]$ are local functionals of the effective fields and their derivatives, and $\lambda_i$ are coupling constants.
The coupling constants are free parameters, which parametrizes the most general $S$-matrix consistent with foundational assumptions and the underlying theory.
A Lagrangian with an infinite amount of free parameters might seem useless.
However, if we can find a consistent series expansion, then only a finite number of terms are needed to calculate quantities to any given order in perturbation theory.
Furthermore, even though such a theory is called ``non-renormalizable''---renormalizing an arbitrary order in perturbation theory requires an arbitrary number of parameters---only a finite number of parameters are needed to renormalize any \emph{given} order.
Non-renormalizable theories can thus perfectly well be renormalized~\autocite{schwartzQuantumFieldTheory2013}.

In the last section, we showed that the Goldstone modes will always appear in the Lagrangian as the terms of the Mauer-Cartan form, $d_\mu$, and $e_\mu$.
Thus, the approach to creating an effective theory of Goldstone bosons, such as chiral perturbation theory, is to write down the most general Lagrangian, consistent with the underlying symmetries, made up of these terms.
Then, using Weinberg's power counting scheme, as we discuss in \autoref{subsection: Weinberg's power counting scheme}, we expand perturbatively in the Goldstone boson energies.
This will give us a self-consistent description of the Goldstone bosons of the theory, the pseudoscalar mesons.
The world is, of course, not only made up of only pseudoscalar mesons.
We also need to describe how these fields interact with other fields or external sources.
Furthermore, the global $\Lie{SU}{N_f}\times\Lie{SU}{N_f}$ symmetry of QCD with $N_f$ quarks is only approximate, as we will explore further in \autoref{chapter: chpt}.
We must extend the CCWZ construction to incorporate these effects, which we will do by introducing some new QFT tools in the next subsection.



\subsection{Schwinger-Dyson equations and Ward identities}
\label{subsection: ward identities}


Given a system of fields $\varphi_a$ governed by some action $S[\varphi]$, the expectation value of a functional of the fields, $F[ \varphi]$, is given by 
%
\begin{equation}
    \label{Most general schwinger dyson equation}
    \Braket{ F[\varphi] } = \int \D \varphi \, e^{iS[\varphi]} F[\varphi].
\end{equation}
%
If we perform a \emph{local} transformation of the field on the form $\varphi(x) \rightarrow \varphi(x) + \epsilon \eta(x)$, the integral measure will remain unchanged.
The expectation value, to first order in $\epsilon$, then changes as
%
\begin{equation}
    \Braket{F} \rightarrow
    \int \D \varphi \, e^{i(S+\epsilon \delta S)}(F + \epsilon \delta F)
    = \Braket{F} +  \epsilon\Braket{i(\delta_\eta S) F} + \epsilon\Braket{\delta_\eta F}.
\end{equation}
%
Where the variation $\delta_\eta S = \int \dd^n x\, \fdv{S[\varphi]}{\varphi(x)} \eta(x)$, as defined in \autoref{appendix: Functional derivatives}.
As this amounts to a change of integration variable, the expectation value should remain unchanged.
This gives us the important identity
%
\begin{equation}
    \Braket{i (\delta_\eta S[\varphi]) F[\varphi]} + \Braket{\delta_\eta F[\varphi]} = 0.
\end{equation}
%
Inserting the integral form of the variation, and using the fact that $\eta$ is independent of $\varphi$, we may write this identity as
%
\begin{equation}
    \label{Generalized schwinger dyson equation}
    \Braket{\fdv{S[\varphi]}{\varphi(x)} F[\varphi]} = i \Braket{\fdv{F[\varphi]}{\varphi(x)}}.
\end{equation}
%
The Schwinger-Dyson equations are important special cases of this identity.
They are the equations of motion of correlation functions.
They thus incorporate the dynamics of a theory.
We derive them by setting $F[\varphi] = \varphi(x_1)...\varphi(x_n)$.
If we have a Lagrangian on the form $\Ell = - \frac{1}{2}\varphi(\partial^2 + m^2)\varphi - V[\varphi]$, then \autoref{Generalized schwinger dyson equation} becomes
%
\begin{align*}
    (\partial^2_x + m^2)\braket{\varphi(x)\varphi(x_1)\dots \varphi(x_n)}
    = - \Braket{\Ve'[\varphi](x)\varphi(x_1)\dots\varphi(x_n)} 
    -i \sum_i\delta(x - x_i)
    \Braket{\varphi(x_1)\dots \widehat {\varphi(x_{i})} \dots \varphi(x_n)},
\end{align*}
%
where the hat denotes that the field is \emph{omitted}.
If $n = 1$ and $\Ve = 0$, we get the defining relation for the free Greens function,
%
\begin{equation}
    (\partial^2_x + m^2)\braket{\varphi(x)\varphi(y)}
    = -i \delta(x - y).
\end{equation}
%
We may also consider slightly more general transformations of $\varphi(x)$, such as local phase-transformations $\varphi(x) \rightarrow e^{i\epsilon(x)}\varphi(x)$, as long as they do not affect the measure of the path integral.
We will use this to derive identities related to the Schwinger-Dyson equations that incorporate the symmetries of a given theory.
If $\varphi(x) \rightarrow \varphi(x) + \delta \varphi(x)$ is a global symmetry transformation, so that $\delta \Ell = \partial_\mu K^\mu$ and the integration measure is unchanged, then $\varphi(x) \rightarrow \varphi(x) + \eta(x) \delta \varphi(x)$ is a corresponding local transformation.
We recover the global transformation for $\eta = 1$.
The variation of the action from this transformation will be
%
\begin{align}
    \nonumber
    \delta S 
    &= 
    \int \dd^4 x \,
    \left(
        \pdv{\Ell}{\varphi} \eta \delta \varphi
        + \pdv{\Ell}{(\partial_\mu\varphi) } \partial_\mu (\eta \delta \varphi)
    \right)\\
    \nonumber
    & =
    \int \dd^4 x \, \left( \pdv{\Ell}{(\partial_\mu \varphi)} \delta \varphi \right) \partial_\mu \eta
    + \int \dd^4 x \, \eta(x)
    \left( 
        \pdv{\Ell}{\varphi} \delta \varphi 
        + \pdv{\Ell}{(\partial_\mu \varphi)} \partial_\mu \delta \varphi  
    \right)\\
    \nonumber
    &=
    - \int \dd^4 x \, \eta(x) \partial_\mu 
    \left(  \pdv{\Ell}{(\partial_\mu \varphi)} \delta \varphi - K^\mu \right)
\end{align}
%
In the last line, we integrated by parts, and used $\delta \Ell = \partial_\mu K^\mu$.
From \autoref{subsection: nothers theorem}, we recognize the term within the parenthesis as precisely the Nöther current $J^\mu$, so
%
\begin{equation}
    \label{delta S}
    \delta S = - \int \dd^4 x \, \eta(x) \partial_\mu J^\mu.
\end{equation}
%
As $\varphi$ is an integration variable in the path integral, it is not necessarily on-shell.
We can therefore not use Nöther's theorem, $\partial_\mu J^\mu = 0$, as this relies on the equation of motion.
However, for $F = 1$ and thus $\delta F = 0$, we can insert \autoref{delta S} into \autoref{Most general schwinger dyson equation} to obtain the quantum version of the current conservation equation,
%
\begin{equation}
    \partial_\mu \Braket{J^\mu} = 0.
\end{equation}
%
With $F = \varphi(x_1) \varphi(x_2)$, we get~\autocite{schwartzQuantumFieldTheory2013,peskinIntroductionQuantumField1995}
%
\begin{equation}
    \partial_{x, \mu} \braket{J^\mu (x)\varphi(x_1) \varphi(x_2)}
    = -i \delta(x - x_1) \braket{\delta \varphi(x_1) \varphi(x_2)}
    - i \delta(x - x_2) \braket{\varphi(x_1) \delta \varphi(x_2)}.
\end{equation}
%
Identities of this form are called Ward-Takashi identities, often just Ward-identities, and encode the symmetries of a theory.
In case symmetry is only approximate, so $\delta \Ell = \partial_\mu K^\mu + \Delta$, where $\Delta$ is some small symmetry breaking operator, or it is subject to an anomaly, so $\D \varphi \rightarrow \D \varphi (1 + \Delta)$, then the conservation equation is modified to
%
\begin{equation}
    \partial_\mu \Braket{J^\mu} = \Braket{\Delta}.
\end{equation}

To create the generating functional, we must add external currents $j$.
However, these new terms in the Lagrangian might break the invariance under a symmetry transformation $\varphi \rightarrow \varphi'$.
If we transform the external currents as $j \rightarrow j'$ to counteract the transformation of the fields, then the theory should remain invariant.
As before, we make both these transformations local, making sure that they leave the measure invariant.
We can then perform a change of variable in the path integral, which relates generating functionals with different external currents, $Z[j] = Z[j']$.
This relation must not only be obeyed by the underlying theory but also by any effective theory, which significantly constrains the form of the effective Lagrangian.
As an illustration, we consider an example of spinor fields adapted from~\autocite{schererIntroductionChiralPerturbation2002}, as this is closely related to the construction of chiral perturbation theory.
Spinors and gauge theory, which are relevant for this example, are discussed in more depth in \autoref{chapter: chpt}.
Consider a massless spinor field with the Lagrangian,
%
\begin{equation}
    \Ell = i \bar \psi \slashed \partial \psi - \Ve[\psi, \bar \psi].
\end{equation}
%
Assume this theory has a global $\Lie{SU}{N}$ symmetry, so the $\Ve$ remains unchanged under the transformation $\psi \rightarrow U \psi$, $\bar \psi \rightarrow \bar \psi U^\dagger$.
The system then has a corresponding conserved current,
%
\begin{equation}
    J_\alpha^\mu = \bar \psi T_\alpha \gamma^\mu \psi,
\end{equation}
%
where $T_\alpha$ are the generators of $\Lie{SU}{N}$.
We then include spinor sources $\eta = \eta_\alpha T_\alpha$ and vector sources $v_\mu = v_\mu^\alpha T_\alpha$ by adding the terms $ \bar \eta \psi$, $\bar \psi \eta$, and $v_\mu^\alpha J^\mu_\alpha$ to the Lagrangian.
Under a local $\Lie{SU}{N}$ transformation, $\psi \rightarrow e^{i\theta_\alpha(x)T_\alpha} \psi$, the action changes as
%
\begin{equation}
    S \rightarrow 
    \int \dd^4 x \,
    \left[
        i \bar \psi \slashed \partial \psi 
        - \Ve[\psi, \bar \psi]
        + \bar \eta U \psi
        + \bar \psi U^\dagger \eta 
        + \bar \psi \gamma^\mu (U^\dagger v_\mu U + i U^\dagger \partial_\mu U)\psi
    \right].
\end{equation}
%
The last term corresponds to the change in action without sources, which we found earlier \autoref{delta S}.
We then define transformations of the external fields to counteract the transformation of $\psi$.
As these transformations are local, the sources now act as gauge fields.
The gauge transformation of the external sources are
%
\begin{equation}
    \eta \rightarrow U \eta, \quad
    \bar \eta \rightarrow \bar \eta U^\dagger,\quad
    v_\mu \rightarrow U(v_\mu + i \partial_\mu) U^\dagger.
\end{equation}
%
This gives the relation
$
    S[\psi', \bar \psi', \eta', \bar \eta', v'] =
    S[\psi, \bar \psi, \eta, \bar \eta, v],
$
where the prime indicates gauge transformed fields.
As we argued in the subsection on the Dyson-Schwinger equations, we can change the integration variables inside the path integral without changing the result.
Considering an infinitesimal transformation, and expanding to first order in $\theta$, we get
%
\begin{align}
    \nonumber
    0 &= Z[\eta', \bar \eta', v'] - Z[\eta, \bar \eta, v]\\
    \label{local invariance with external fields as gauge fields}
    &=
    i \int \dd^4 x 
    \Braket{
        i \theta_\alpha(x) \bar \psi T_\alpha \eta
        - i \theta_\alpha(x) \bar \eta T_\alpha \psi
        + i \bar \psi \gamma^\mu 
        (
            i\theta_\alpha(x) [T_\alpha, v_\mu] - i \partial_\mu \theta_\alpha(x) T_\alpha
        ) \psi
    }
\end{align}
%
As the transformation, and thus $\theta_\alpha$, is arbitrary, the integrand must vanish.
After partial integration, we are left with
%
\begin{equation}
    \Braket{
        \bar \psi T_\alpha \eta
        - \bar \eta T_\alpha \psi
        + D^{\alpha\beta}_\mu J^\mu_\beta
    }
    = 0.
\end{equation}
%
Here, $D^{\alpha\beta}_\mu$ is the covariant derivative in the adjoint representation,
$D^{\alpha\beta}_\mu J^\beta_\nu = (\delta_{\alpha\beta}\partial_\mu + i v_\mu^\gamma f^{\alpha \gamma \beta} )J^\nu_\beta $, and $f^{\alpha \beta \gamma}$ are the structure constants of $\lie{su}{N}$.
We can get a more general expression by writing this using the generating functional $W$,
%
\begin{equation}
    \label{generating equation ward identitites}
    \left( 
        \fdv{}{\eta_\alpha(x)} \eta - \bar \eta \fdv{}{\bar \eta_\alpha(x)}  
        + D_{\mu}^{\alpha \beta} \fdv{}{v_\mu^\beta(x)}
        \right) W[\eta, \bar \eta, v] = 0.
\end{equation}
%
If we evaluate this at $\eta = \bar \eta = v_\mu = 0$, we get the quantum conservation equation $\partial_\mu \Braket{J^\mu_\alpha} = 0$.
From \autoref{generating equation ward identitites}, we can also get more general Ward identities by taking functional derivatives with respect to the external sources.


We have now seen how the Ward identities encode the global symmetries of the theory, and that they may be derived by transforming external source fields as gauge fields to ensure the invariance of the action under the corresponding \emph{local} transformations.
This is the key insight behind the systematic development of chiral perturbation theory~\autocite{gasserChiralPerturbationTheory1984,gasserChiralPerturbationTheory1985,leutwylerFoundationsChiralPerturbation1994}.
With the CCWZ construction, we can create the Lagrangian of Goldstone bosons alone, given only the symmetry breaking pattern $G \rightarrow G/H$.
However, it does not tell us how they are coupled to external fields.
With the constraint \autoref{local invariance with external fields as gauge fields}, we know that the new action must be invariant under \emph{local} $G$ transformations, given that we transform the external fields as gauge fields.
When including external fields, the new effective Lagrangian $\Ell_\text{eff}$ must therefore not only be invariant under global $G$ transformations but rather local $G$ transformations.
If we modify the Mauer-Cartan form \autoref{Mauer-Cartan form} by introducing a covariant derivative,
%
\begin{equation}
    i\Sigma(x)^{-1} \partial_\mu \Sigma(x)
    \rightarrow i\Sigma(x)^{-1} \nabla_\mu \Sigma(x),
\end{equation}
%
then all terms that were invariant under global $G$ become locally so.
This is because, as in the case of covariant derivative in \autoref{subsection: goemetry and the metric}, the covariant derivative transforms as the object it acts on.
In addition to new and modified terms due to the covariant derivative, we can now also combine the external currents and $\Sigma$ into $G$-invariant terms.
This will allow us to take into account approximate symmetries as well.
By treating the symmetry-breaking parameter in the underlying Lagrangian, such as the mass of quarks in the case of chiral perturbation theory, as an external current, we can still apply this procedure.
Such fields are called \emph{spurion fields}.
We will apply this to derive chiral perturbation theory in \autoref{chapter: chpt}.


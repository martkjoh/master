\section{Tree-level, two flavor pion stars}


\subsection{Equation of state}
The free energy density of two-flavor chiral perturbation theory, to leading-order and at $T = 0$, is
%
\begin{equation}
    \Eff = - f^2 \left(\bar m^2 \cos \alpha + \frac{1}{2} \mu_I^2 \sin^2 \alpha\right).
\end{equation}
%
The $\alpha$ parameter is determined by minimizing $\Eff$ for a given value of $\mu_I$,
%
\begin{equation}
    \pdv{\Eff}{\alpha} = f^2 \left(\bar m^2 - \mu_I^2 \cos \alpha\right) \sin(\alpha) = 0.
\end{equation}
%
This gives an explicit formula for $\alpha$ in terms of $\mu_I$.
We are only interested in the phase where $\mu_I \geq \bar m$, where this solution is
%
\begin{align}
    \label{alpha as function of mu lowest order}
    \cos \alpha = \frac{\bar m^2}{\mu_I^2}.
\end{align}
%
We introduce a dimensionless variable $x^2 = \cos(\alpha) = \bar m^2 / \mu_I^2$.
This variable has the domain $[0, 1]$.
By an argument using a right triangle, we can vertify that $\cos a = b$ implies that $\sin^2 a = 1 - x^4$.
Substituting the dimen sionless varaible into the free energy density, we get 
%
\begin{equation}
    \Eff = - \frac{u_0}{2} \left( x^2 + \frac{1}{x^2} \right).
\end{equation}
%
We have introduced the characteristic energy density $u_0 = \bar m^2 f^2$.
As we found in \autoref{section: cold fermi star}, the pressure is given by negative the free energy density, normalized to $\mu_I = \bar m$, or $x = 1$.
We choose $p_0 = u_0$, so the dimensionless pressure can be written
%
\begin{equation}
    \tilde p = -\frac{1}{p_0} \left(\Eff - \Eff_{x=1}\right) 
    = \frac{1}{2} \left( x^2 + \frac{1}{x^2} - 2 \right).
\end{equation}
%
The charge density corresponding to a chemical potential is given by minus the derivative of the free energy with respect to that chemical potential. 
We must, however, no assume any dependence of $\alpha$ on $\mu_I$ when taking this derivative.
The isospin density therefore is
%
\begin{equation}
    n_I = -\pdv{\Eff}{\mu_I} = ^2 \mu_I^2 \sin^2 \alpha 
    = 
    \frac{u_0}{\mu_I} \left(\frac{1}{x^2} - x^2\right).
\end{equation}
%
With this, the dimensionless energy density at $T = 0$ is
%
\begin{equation}
    \tilde u = - \tilde p + \frac{\mu_I n_I}{u_0}
    = \frac{1}{2} \left( 2 + \frac{1}{x^2} - 3 x^2\right).
\end{equation}
%
This is illustrated in~\autoref{fig: equation of state pions tree level}.
The ratio of pressure to energy density is
%
\begin{equation}
    \frac{p}{u} = \frac{x^2 - 2 + \frac{1}{x^2}}{-3x^2 + 2 + \frac{1}{x^2}}.
\end{equation}
%
In the ultrarelativistic limit, where $\mu_I \rightarrow \infty$ and thus $x \rightarrow 0$, we get $p / u \rightarrow 1$, or $p = u$.
\todo[inline]{Er denne grensen gyldig? Er chpt fremdelse gyldig her?}



\begin{figure}[h]
    \centering
    \includegraphics[width=0.6\textwidth]{../scripts/figurer/pion_tree_eos.pdf}
    \caption{
        This plot shows the tree-level equation of state of two-flavor chiral perturbation theory. The $x$-axis shows the pressure normalized to $p_0$, while the $y$-axis shows the energy density normalized to $u_0$.
    }
    \label{fig: equation of state pions tree level}
\end{figure}


\subsection{Units}

The characteristic mass and length, as discussed in \autoref{section: TOV equation}, are found by setting $k_1 = k_2 = k_3 = 1$.
These are the dimensionless constants of the TOV equation, \autoref{dimensionless constants TOV}.
At tree-level, the bare constants $f$ and $\bar m$ are related to physical constants by $f = f_\pi$ and $m = m_\pi$, the pion decay constant and the pion mass.
Using the values for $f_\pi$ and $m_\pi$ as given in \autoref{section: units} and reinstating $c$ and $\hbar$, these quantities are give in SI-units by
%
\begin{align}
    u_0 & =m_\pi^2 f_\pi^2 \frac{c}{\hbar^3}
    = 3.216\cdot 10^{33} \, \text{J}\,\text{m}^{-3}, \\
    m_0 & = \frac{c^4}{\sqrt{\frac{4 \pi}{ 3} u_0 G}} = 64.21\, M_\odot, \\
    r_0 & = \frac{G}{c^2} m_0 = 94.79 \, \text{km}.
\end{align}
%
We therefore expect both the radius and mass of the pion star to be around one order of magnitude larger than the star made up of cold neutrons.


\subsection{Results}

The code used for obtaining numerical results is discussed in \autoref{appendix: code}.

\autoref{fig: pressure and mass for tree-level pion star} show the pressure and mass as a function of radius for a range of central pressures.
The quantities are normalized to the stellar radius, stellar mass, and central pressure, respectively.
The black dashed line corresponds to the configuration with the maximum mass.
We see that both the pressure and mass distribution are very similar for stars with a mass less than the maximum.

\autoref{fig: mass radius relation tree-level pion star} shows the mass-radius relationship for the pion star using the tree-level free energy density of two-flavor chiral perturbation theory.
As in the case of the neutron star, it has a maximum mass, in this case of $M_\text{max} 10.47\, M_\odot$.
What distinguishes this from the case of the neutron star, however, is the fact that in the limit of low central pressure, the distribution apparently approaches a maximum radius of $R = 85.95 \, \text{km}$.
\todo{Hvorofor? Fordi eos er så lite stiv?}

\begin{figure}[!htb]
    \centering
    \includegraphics[width=0.8\textwidth]{../scripts/figurer/pressure_mass_pion_star.pdf}
    \caption{
        Top: The pressure normalized to the central pressure, as a function of radius, normalized to the stellar radius.
    Bottom: The mass, normalized to stellar mass, within a radius $r$, normalized to the stellar radius.
    Both plots show a range of stars with different central pressures, indicated by the color range.
    The black dotted line corresponds to the star with the largest mass.}
    \label{fig: pressure and mass for tree-level pion star}
\end{figure}


\begin{figure}[!htb]
    \centering
    \includegraphics[width=0.95\textwidth]{../scripts/figurer/mass_radius_pion_star_tree.pdf}
    \caption{
        The plot shows the relationship between the mass and radius of a pion star. Mass is given in units of solar masses, while the radius is measured in kilometers.
        This line is parameterized by the central pressure $p_c$ of the star, as indicated by the color gradient.
        The dashed black line indicates the theoretical maximum mass for a given radius, and the configuration above it will collapse to a black hole.
        }
        \label{fig: mass radius relation tree-level pion star}
\end{figure}


\subsection{Including electromagnetic contributions}

From \autoref{static lagrangian with EM}, the free energy density including electromagnetic interactions is
%
\begin{equation}
    \Eff =
    - f^2 \left[
        \bar m^2 \cos \alpha 
        + \frac{1}{2} \mu_I^2 \sin^2 \alpha
        + \Delta m_\pm^2 \left(\cos^2 \alpha - \frac{4}{9}\right)
    \right],
\end{equation}
%
Free energy minimization now gives
%
\begin{equation}
    \frac{1}{u_0}\pdv{\Eff}{\alpha}
    = 
    \left[ \left( \frac{1}{x^2} - 2 \Delta \right) \cos \alpha - 1\right] \sin \alpha = 0.
\end{equation}
%
Here, $x$ is defined as before, and we introduced the new quatnity $\Delta = \Delta m_{\pm}^2 / \bar m^2= 0.06916 $.
This gives the new solution
%
\begin{equation}
    \cos \alpha = \frac{x^2}{1 - 2 \Delta x^2}.
\end{equation}
%
This reduces to our old solution for $\Delta = 0$, as it should.



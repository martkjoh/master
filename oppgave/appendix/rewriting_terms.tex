\section{*Rewriting terms}
\label{appendix: rewriting terms}

This section shows how to rewrite terms in the Lagrangian of Chiral perturbation theory.
These techniques and more are used to reduce the total number of terms and to change between different conventions.
Changing the field parametrization that appears in the Lagrangian does not affect any of the physics, as it corresponds to a change of variables in the path integral~\autocite{chisholmChangeVariablesQuantum1961,kamefuchiChangeVariablesEquivalence1961,schererIntroductionChiralPerturbation2002}.
However, a change of variables can result in new terms in the Lagrangian.
As a result of this, terms that appear independent on their face may be redundant.
These terms can be eliminated by using the classical equations of motion.
In this section, we show first the derivation of the equations of motion, then use this result to identify redundant terms which need not be included in the most general Lagrangian.

We derive the equations of motion for the leading order Lagrangian using the principle of least action.
Choosing the parametrization $\Sigma = \exp{i \pi_a \tau_a}$, a variation $\pi_a \rightarrow \pi_a + \delta \pi_a$ results in a variation in $\Sigma$, $\delta \Sigma = i \tau_a \delta \pi_a \Sigma $.
The variation of the leading order action,
\begin{equation}
    S_2 = \int_\Omega \dd^4x \, \Ell_2,
\end{equation}
when varying $\pi_a$ is 
\begin{align*}
    \delta S = \int_\Omega \dd x \, \frac{f^2}{4}
    \Tr{
        (\nabla_\mu \delta \Sigma) (\nabla^\mu \Sigma)^\dagger
        + (\nabla_\mu \Sigma) (\nabla^\mu \delta \Sigma)^\dagger
        + \chi \delta \Sigma^\dagger + \delta \Sigma \chi^\dagger
    }.
\end{align*}
Using the properties of the covariant derivative to do partial integration, as shown in \autoref{appendix: covariant derivative}, as well as $\delta(\Sigma \Sigma^\dagger) = (\delta\Sigma)\Sigma^\dagger + \Sigma (\delta \Sigma)^\dagger = 0$, the variation of the action can be written
\begin{align*}
    \delta S 
    & = \frac{f^2}{4} \int_\Omega \dd x\, 
    \Tr{
        - \delta \Sigma \nabla^2 \Sigma^\dagger
        + (\nabla^2 \Sigma) (\Sigma^\dagger \delta \Sigma \Sigma^\dagger)
        - \chi (\Sigma^\dagger \delta \Sigma \Sigma^\dagger)
        + \delta \Sigma \chi^\dagger
    } \\
    & = 
    \frac{f^2}{4} \int_\Omega \dd x\, 
    \Tr{
        \delta \Sigma \Sigma^\dagger 
        \left[
            (\nabla^2 \Sigma)\Sigma^\dagger
            - \Sigma \nabla^2 \Sigma^\dagger
            - \chi \Sigma^\dagger
            + \Sigma \chi^\dagger
        \right]
        } \\
    & = 
    i \frac{f^2}{4} \int_\Omega \dd x\, 
    \Tr{\tau_a 
    \left[
         (\nabla^2 \Sigma)\Sigma^\dagger
        - \Sigma \nabla^2 \Sigma^\dagger
        - \chi \Sigma^\dagger
        + \Sigma \chi^\dagger
    \right]
    } 
    \delta \pi_a = 0.
\end{align*}
%
As the variation is arbitrary, the equations of motion to leading order is
%
\begin{equation}
    \Tr{
        \tau_a 
        \left[
            (\nabla^2 \Sigma)\Sigma^\dagger
            - \Sigma \nabla^2 \Sigma^\dagger
            - \chi \Sigma^\dagger
            + \Sigma \chi^\dagger
        \right]
    } = 0.
\end{equation}
%
We define
%
\begin{equation}
    \mathcal O_\text{EOM}^{(2)}
    = 
    (\nabla^2 \Sigma)\Sigma^\dagger
    - \Sigma \nabla^2 \Sigma^\dagger
    - \chi \Sigma^\dagger
    + \Sigma \chi^\dagger.
\end{equation}


The next step in eliminating redundant terms is to change the parametrization of $\Sigma$ by $\Sigma(x) \rightarrow \Sigma'(x)$.
Here, $ \Sigma(x) = e^{iS(x)} \Sigma'(x), \, S(x) \in \lie{su}{2}$. This change leads to a new Lagrange density, $\Ell[\Sigma] = \Ell[\Sigma'] + \Delta \Ell[\Sigma']$.
We are free to choose $S(x)$, as long $\Sigma'$ still obey the required transformation properties.
Any terms in the Lagrangian $\Delta \Ell$ due to a reparametrization can be neglected, as argued earlier.
When demanding that $\Sigma'$ obey the same symmetries as $\Sigma$,
the most general transformation to second order in Weinberg's power counting scheme  is~\cite{schererIntroductionChiralPerturbation2002}
%
\begin{equation}
    \label{S reparametrization}
    S_{2} = 
    i \alpha_2 
    \left[
        (\nabla^2 \Sigma') \Sigma^\dagger - \Sigma' (\nabla^2 {\Sigma'})^\dagger
    \right]
    + i \alpha_2
    \left[
        \chi \Sigma'^\dagger - \Sigma' \chi^\dagger 
        - \frac{1}{2} \Tr{\chi \Sigma'^\dagger - \Sigma' \chi^\dagger}
    \right].
\end{equation}
%
$\alpha_1$ and $\alpha_2$ are arbitrary real numbers. As \autoref{S reparametrization} is to second order, $\Delta \Ell$ is fourth order in Weinberg's power counting scheme.
Inserting this gives
%
\begin{align*}
    \Ell_2\left[e^{i S_2}\Sigma '\right]
    & =
    \frac{f^2}{4}\Tr{[\nabla_\mu (1 +i S_2)\Sigma'][\nabla^\mu \Sigma'^\dagger  (1 - i S_2)]}
    + \frac{f^2}{4} \Tr{\chi\Sigma'^\dagger (1 - i S_2) + (1 +i S_2)\Sigma' \chi^\dagger} \\
    & = \Ell[\Sigma'] + 
    i \frac{f^2}{4}
    \Tr{[\nabla_\mu (S_2\Sigma')][\nabla^\mu\Sigma']^\dagger 
    -  [\nabla_\mu\Sigma'][\nabla^\mu (\Sigma'^\dagger  S_2) ]}
    - i \frac{f^2}{4} \Tr{\chi \Sigma'^\dagger S_2 - S_2 \Sigma' \chi^\dagger}
\end{align*}
%
Using the properties of the covariant derivative, we may use the product rule and partial integration to write the difference between the two Lagrangians to fourth-order as
%
\begin{align}
    \nonumber
    \Delta \Ell[\Sigma'] 
    & = 
    i \frac{f^2}{4}
    \Tr{
        (\nabla_\mu S_2)
        (\Sigma' \nabla^\mu \Sigma'^\dagger - (\nabla^\mu \Sigma') \Sigma'^\dagger) 
    }
    - i \frac{f^2}{4} \Tr{\chi \Sigma'^\dagger  S_2 - S_2 \Sigma' \chi^\dagger} \\
    & = 
    \label{Delta reparametrization}
    \frac{f^2}{4} \Tr{i S_2 \mathcal{O}_\mathrm{EOM}^{(2)}}.
\end{align}
%
Any term that can be written in the form of \autoref{Delta reparametrization} for arbitrary $\alpha_1, \alpha_2 \in \R$ is redundant, as we argued earlier, and may therefore be discarded.
$\Delta \Ell_2$ is of fourth order, and it can thus be used to remove terms from $\Ell_4$ or higher order.



\subsection{Rewriting NLO Lagrangian}
\label{subsection:rewriting NLO Lagrangian}

The NLO Lagrangian used in this text is given in \autoref{NLO Lagrangian}, and is
%
\begin{align}
    \notag
    \Ell_4 
    & = 
    \frac{l_1}{4} \Tr{\nabla_\mu \Sigma (\nabla^\mu \Sigma)^\dagger}^2
    + \frac{l_2}{4} \Tr{\nabla_\mu \Sigma (\nabla_\nu \Sigma)^\dagger} 
    \Tr{\nabla^\mu \Sigma (\nabla^\nu \Sigma)^\dagger} 
    +
    \frac{l_3 + l_4}{16} \Tr{\chi \Sigma^\dagger + \Sigma \chi^\dagger}^2
    \\ \notag
    &
    + \frac{l_4}{8}\Tr{\nabla_\mu \Sigma (\nabla^\mu \Sigma)^\dagger} \Tr{\chi \Sigma^\dagger + \Sigma \chi^\dagger}
    - \frac{l_7}{16} \Tr{\chi \Sigma^\dagger - \Sigma \chi^\dagger}^2
    + \frac{h_1 + h_3 - l_4}{4} \Tr{\chi \chi^\dagger} \\
    & +
    \frac{h_1 - h_3 - l_4}{16}
    \left[
        \Tr{\chi \Sigma^\dagger + \Sigma \chi^\dagger}^2
        + \Tr{\chi \Sigma^\dagger - \Sigma \chi^\dagger}^2
        -2 \Tr{\left(\chi \Sigma^\dagger\right)^2 + \left( \Sigma \chi^\dagger\right)^2}
    \right].
\end{align}
%
We can rewrite it to match the one used in~\autocite{adhikariTwoflavorChiralPerturbation2019,martinariaTwoflavorChiralPerturbation2020}, starting with
%
\begin{align*}
    & \frac{h_1 - h_3 - l_4}{16}
    \left(
        \Tr{\chi \Sigma^\dagger - \Sigma \chi^\dagger}^2
        - 2 \Tr{(\chi \Sigma^\dagger)^2 + (\Sigma \chi^\dagger)^2}
    \right) \\
    & = 
    \frac{h_1 - h_3 - l_4}{16}
    \left(
        \Tr{\chi \Sigma^\dagger}^2 - 2\Tr{\chi \Sigma^\dagger}\Tr{\Sigma \chi^\dagger}
        + \Tr{\Sigma \chi^\dagger}^2
        - 2 \Tr{(\chi \Sigma^\dagger)^2} -2 \Tr{(\Sigma \chi^\dagger)^2}
    \right).
\end{align*}
Using $\Tr{A^2} = \Tr{A}^2 - \det(A)\Tr{\one}$, we get
\begin{align*}
    &= - \frac{h_1 - h_3 - l_4}{16}
    \left(
        \Tr{\chi \Sigma^\dagger}^2 + 2\Tr{\chi \Sigma^\dagger}\Tr{\Sigma \chi^\dagger}
        + \Tr{\Sigma \chi^\dagger}^2
        - 4 \det(\chi \Sigma^\dagger)
        - 4 \det(\Sigma\chi^\dagger)
    \right) \\
    &= - \frac{h_1 - h_3 - l_4}{16}
    \left(
        \Tr{\chi \Sigma^\dagger + \Sigma \chi^\dagger}^2
        - 4 \det(\chi \Sigma^\dagger)
        - 4 \det(\Sigma\chi^\dagger)
    \right).
\end{align*}
%
Furthermore, as $\det(\Sigma)= 1$, 
% and $\chi = a_i \tau_i$, we have $\Tr{\chi \chi^\dagger} = a_i a_i^*$, $\det(\chi) + \det(\chi^\dagger) = a_i a_i + a_i^* a_i^*$
%
\begin{align}
    \notag
    \Ell_4 
    & = 
    \frac{l_1}{4} \Tr{\nabla_\mu \Sigma (\nabla^\mu \Sigma)^\dagger}^2
    + \frac{l_2}{4} \Tr{\nabla_\mu \Sigma (\nabla_\nu \Sigma)^\dagger} 
    \Tr{\nabla^\mu \Sigma (\nabla^\nu \Sigma)^\dagger} 
    +
    \frac{l_3 + l_4 }{16} \Tr{\chi \Sigma^\dagger + \Sigma \chi^\dagger}^2
    \\\notag
    &
    + \frac{l_4}{8}\Tr{\nabla_\mu \Sigma (\nabla^\mu \Sigma)^\dagger} \Tr{\chi \Sigma^\dagger + \Sigma \chi^\dagger}
    - \frac{l_7}{16} \Tr{\chi \Sigma^\dagger - \Sigma \chi^\dagger}^2
    + \frac{h_1 + h_3 -l_4}{4} \Tr{\chi \chi^\dagger} \\
    & +\frac{h_1 - h_3 - l_4}{4} (\det{\chi} + \det{\chi^\dagger}).
\end{align}
 %
For real $\chi$, we have $\Tr{\chi \chi^\dagger} = \det(\chi) + \det(\chi^\dagger)$, and we can define $h_1' = h_1 - l_4$ to get
%
\begin{align}
    \notag
    \Ell_4 
    & = 
    \frac{l_1}{4} \Tr{\nabla_\mu \Sigma (\nabla^\mu \Sigma)^\dagger}^2
    + \frac{l_2}{4} \Tr{\nabla_\mu \Sigma (\nabla_\nu \Sigma)^\dagger} 
    \Tr{\nabla^\mu \Sigma (\nabla^\nu \Sigma)^\dagger} 
    +
    \frac{l_3 + l_4 }{16} \Tr{\chi \Sigma^\dagger + \Sigma \chi^\dagger}^2
    \\
    &
    + \frac{l_4}{8}\Tr{\nabla_\mu \Sigma (\nabla^\mu \Sigma)^\dagger} \Tr{\chi \Sigma^\dagger + \Sigma \chi^\dagger}
    - \frac{l_7}{16} \Tr{\chi \Sigma^\dagger - \Sigma \chi^\dagger}^2
    + \frac{h_1'}{2} \Tr{\chi \chi^\dagger}.
\end{align}
%
If one assumes $\Delta m = 0$, i.e., what is called the chiral limit, then the term $l_7$ falls away, as $\chi = \chi^\dagger$.

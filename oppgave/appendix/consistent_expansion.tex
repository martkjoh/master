\section{*Consistent series expansion o thermodynamic quantities}
\label{appendix: consisten expansion}

As with all other quantities, we calculate using \chpt, the free energy density $\Eff$ must be expanded in chiral dimension, as explained in \autoref{subsection: Weinberg's power counting scheme}, as well as an expansion in loops.
We write 
%
\begin{equation}
    \Eff = \Eff^{(0)}_2 + \Eff^{(1)}_2 + \Eff^{(2)}_2 + \Eff^{(0)}_4,\dots
\end{equation}
%
where $\Eff^{(n)}_D$ is the $n$-loop contribution with chiral dimension $D$.
In \autoref{section: free energy at lowest order}, we found a relationship between $\alpha$ and $\mu_I$, using the lowest-order result for $\Eff$, given in \autoref{leading order minimization}.
To calculate any thermodynamic quantities to leading order, at tree-level, we must use this result.
When using the NLO result for the free energy, we must consistently calculate this and other quantities to the same order.
As we have seen in \autoref{1PI effective action}, when replacing the action by $S[\varphi] \rightarrow g^{-1}S[\varphi]$, the $L$-loop contribution is proportional to $g^{L-1}$.
In Weinberg's power counting scheme, we scale $p \rightarrow t p$ and $m_q \rightarrow t^2 m_q$.
Then, the $n$th term in the expansion is proportional to $t^{2n}$.
Using both these scalings, the expansion of the free energy becomes
%
\begin{equation}
    \Eff = t^2 g^{-1} \Eff_2^{(0)} + t^2 \Eff_2^{(1)} + t^4 g^{-1} \Eff_4^{(0)}
    + ...
\end{equation}
%
We consider terms where $k = L + n$ has the same value to be of same order.
This expansion can be written as
\begin{equation}
    \Eff = \sum_{k=0}^\infty \sum_{n+L=k} t^{2n} g^{L-1} \Eff_{2n}^{(L)}.
\end{equation}
%
If we now define
\begin{equation}
    \tilde \Eff_k = \sum_{n+L=k} t^{2n} g^{L-1} \Eff_{2n}^{(L)},
\end{equation}
%
then scale $t \rightarrow \sqrt{s} \, t$ and $g \rightarrow s g$, where $s$ is some real number, then $t^{2n}g^{L-1}$ scales as $s^{n+L-1} = s^{m-1}$.
All expansions are now done in this new parameter $s$.
The free energy expansion is
\begin{equation}
    \Eff = s^{-1}\sum_{k = 0}^\infty \tilde \Eff_k \, s^{k}.
\end{equation}
%
As argued earlier, $\alpha$ must minimize the free energy, and therefore satisfy
%
\begin{equation}
    \pdv{\Eff}{\alpha} = 0,
\end{equation}
%
to all orders.
We expand this solution in $s$,
\begin{equation}
    \alpha = \alpha_0 + \alpha_1 s + \dots.
\end{equation}
%
Combining this, we get
%
\begin{align}
    \nonumber
    0 &
    = 
    \pdv{}{\alpha}
    \left[
        s^{-1}\tilde \Eff_0
        + 
        \tilde \Eff_1
        +
        \Oh(s)
    \right]
    \bigg|_{\alpha=\alpha_0 + \alpha_1 s + \Oh{s}} 
    \\ \nonumber
    & = 
    s^{-1 }
    \left[
        \tilde \Eff_0'(\alpha_0)
        +
        (\alpha' - \alpha_0)
        \tilde \Eff_0''(\alpha_0)
        +
        \Oh(s^2)
    \right]
    +
    \tilde\Eff_1'(\alpha_0)
    +
    \Oh(s) \\
    &
    =
    s^{-1} \tilde\Eff_1'(\alpha_0)
    + s^{0}
    \left[
        \alpha_1
        \tilde \Eff_0''(\alpha_0)
        +
    \tilde\Eff_1'(\alpha_0)
    \right]
    +
    \Oh(s).
    \label{minimizing free energy expansion}
\end{align}
%
Here, the prime indicates partial derivatives with respect to $\alpha$.
The equality in \autoref{minimizing free energy expansion} has to hold term for term.
After setting $s = g = t = 1$, we get
%
\begin{align*}
    \pdv{\tilde \Eff_0}{\alpha}\bigg|_{\alpha=\alpha_0} = 0, \quad
    \tilde \Eff_0 = \Eff_2^{(0)},
\end{align*}
%
which is what we have used as the leading-order result. 
The first correction is on this result is
%
\begin{equation}
    \label{NLO correction alpha}
    \alpha_1 = - \frac{\tilde \Eff_1'(\alpha_0)}{{\tilde \Eff_0''(\alpha_0)}},
    \quad 
    \tilde \Eff_1 = \Eff_{4}^{(0)} + \Eff_{2}^{(1)}.
\end{equation}
%

The next to leading order results for the free energy and $\alpha$ are
%
\begin{equation}
    \Eff_\mathrm{NLO} = \tilde \Eff_0 + \tilde \Eff_1, \quad
    \alpha_\mathrm{NLO} = \alpha_0 + \alpha_1.
\end{equation}
%
We have that
%
\begin{equation}
    \Eff_\mathrm{NLO}'(\alpha_\mathrm{NLO})
    = \tilde \Eff_0'(\alpha_0) + \alpha_1 \tilde \Eff_0''(\alpha_0) + \dots
    + \tilde \Eff_1'(\alpha_0) + \dots,
\end{equation}
%
where the excluded terms are beyond next-to-leading order.
Using \autoref{NLO correction alpha}, we see that this vanishes to next-to-leading order.
We therefore use the criterion
%
\begin{equation}
    \label{alpha nlo method 2}
    \pdv{\Eff_\mathrm{NLO}}{\alpha} \bigg|_{\alpha=\alpha_{\mathrm{NLO}}} = 0
\end{equation}
%
to calculate $\alpha_\text{NLO}$.
The result is shown in \autoref{fig: alpha}, which compares the leading order and next-to-leading order results for $\alpha$.

\todo[]{fiks figur}
\begin{figure}
    \centering
    % \includegraphics[width=0.7\textwidth]{../scripts/numerikk/plots/alpha.pdf}
    \caption{The leading order and next-to-leading order results for $\alpha$ as a function of $\mu_I$, which is given in units of $m_\pi$. }
    \label{fig: alpha}
\end{figure}

We can use the expansion in $s$ to consistently evaluate any observable to any power in perturbation theory.
Assume that $f$ is some obserable, and a function of $\alpha$.
We then expand in $s$,
%
\begin{equation}
    f(\alpha) = f_0(\alpha) + s f_1(\alpha) + s^2 f_2(\alpha) + \dots
\end{equation}
%
When Taylor expanding around the leading order result for $\alpha$, we get
%
\begin{align*}
    f(\alpha) 
    & 
    = 
    f_0(\alpha_0)
    +
    s\alpha_1  f_0'(\alpha_0)
    +
    s f_1(\alpha_0)
    + \Oh(s^2)
    = f_0(\alpha_0 + s \alpha_1)
    + s f_1(\alpha_0)
    + \Oh(s^2).
\end{align*}
%
We see that, to get a consistent expansion, we must evaluate the leading order result for the function $f_0$ at next-to-leading order in $\alpha$, while next-to-leading order correction can be evaluated at leading order.
However, as
%
\begin{equation}
    f_1(\alpha_0 + s\alpha_1) = f_1(\alpha_0) + \Oh(s),
\end{equation}
%
we also get a consistent expansion if we evaluate the leading-order result and its correction at next-to-leading order in $\alpha$.
We will do this to obtain the results in the next section.

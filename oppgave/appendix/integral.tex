\section{Integrals in dimensional regularization}
\label{section: integral}


When applying dimensional regularization, we will encounter integrals on the form
%
\begin{equation}
    \label{def dimreg integral}
    \Phi_n(m, d, \alpha) 
    := \int_{\tilde \Omega} \frac{\dd^d p}{(2 \pi)^d} (k^2 + m^2)^{-\alpha},
\end{equation}
%
We will use the formula for integration of spherically symmetric function in $d$-dimensions,
%
\begin{equation}
    \int_{\R^d} \dd^d x \, f(r) 
    = \frac{2 \pi^{d/2}}{\Gamma(d/2)} \int_\R \dd r \, r^{d-1}f(r),
\end{equation}
where $r = \sqrt{x_i x_i}$ is the radial distance, and $\Gamma$ is the Gamma function.
The factor in the front of the integral is the solid angle.
By extending this formula from integer-valued $d$ to real numbers, we get
%
\begin{equation}
    \Phi_n
    = \frac{2 \pi^{d/2} }{\Gamma(d/2)} \int_\R \dd p \, 
    \frac{p^{d-1}}{(p^2 + m^2)^\alpha}
    = \frac{m^{n-2\alpha}m^{d-n} }{(4 \pi)^{d / 2}\Gamma(d/2)} 
    2 \int_\R \dd z \, \frac{z^{d - 1}}{(1 + z)^\alpha}, 
\end{equation}
%
where we have made the change of variables $m z = k$.
We make one more change of variable to the integral,
%
\begin{equation}
    I = 2 \int_\R \dd z \, \frac{z^{d - 1}}{(1 + z)^\alpha}.
\end{equation}
%
Let
%
\begin{equation}
    z^2 = \frac{1}{s} - 1 \implies 2 z \dd z = - \frac{\dd s}{s^2}
\end{equation}
%
Thus,
%
\begin{equation}
    I = \int_0^a \dd s \, s^{\alpha - d/2 - 1} (1 - z)^{d/2 - 1}.
\end{equation}
%
This is the beta function, which can be written in terms of Gamma functions~\autocite{peskinIntroductionQuantumField1995}
%
\begin{equation}
    I = B\left(\alpha - \frac{d}{2}, \frac{d}{2}\right) 
    = \frac{\Gamma\left(\alpha - \frac{d}{2}\right) \Gamma\left(\frac{d}{2}\right)}{\Gamma(\alpha)}.
\end{equation}
%
Combining this gives
%
\begin{equation}
    \label{result dimreg}
    \Phi_n(m, d, \alpha) 
    = \mu^{n-d} \frac{m^{n - 2\alpha}}{(4 \pi)^{d / 2}}
    \frac{
        \Gamma \left(\alpha - \frac{d}{2} \right) 
    }
    {\Gamma(\alpha)}
    \left(\frac{m^2}{\mu^2}\right)^{(d-n)/2 }
    .
\end{equation}
%
In the last step, we have introduced a parameter $\mu$ with mass dimension 1, that is, $[\mu] = [m]$.
This is done to be able to series expand around $d - n$ in a dimensionless variable. 
This parameter is arbitrary, and all physical quantities should therefore be independent of it.



\subsection{A special case}
\label{subsection: free energy integral}

A special case of this integral, which shows up in calculations of free energy, is 
%
\begin{equation}
    \Eff = \int \frac{\dd^3 p}{(2 \pi)^3} \sqrt{\vv p^2 + m^2},
\end{equation}
%
Which corresponds to $n=3$, $d = 3 - 2\epsilon$ and $\alpha = -1/2$, or $\Eff = \Phi_3(m, 3 - 2\epsilon, -1/2)$.
Inserting this into \autoref{result dimreg}, we get
%
\begin{equation}
    \Eff
    =
    \frac{m^4 \mu^{-2\epsilon}}{(4 \pi)^{d/2}\Gamma(-1/2)} \Gamma(-2 + \epsilon) \left(\frac{m^2}{\mu^2}\right)^{-\epsilon}
    =
    - \mu^{-2\epsilon} \frac{m^4}{(4 \pi)^{2}}
    \left(\frac{m^2}{4 \pi \mu^2}\right)^{- \epsilon}
    \frac{\Gamma(\epsilon)}{(\epsilon - 2)(\epsilon - 1)},
\end{equation}
%
where we have used the defining property $\Gamma(z + 1) = z\Gamma(z)$ and $\Gamma(1/2) = \sqrt \pi$.
Expanding around $\epsilon = 0$ gives
%
\begin{align}
    \left(\frac{m^2}{4 \pi \mu^2}\right)^{- \epsilon}
    &\sim 1 + \epsilon \ln\left(4 \pi \frac{\mu^2}{m^2}\right),\\
    \Gamma(\epsilon) 
    & \sim \frac{1}{\epsilon} - \gamma, \\
    \frac{1}{(\epsilon - 2)(\epsilon - 1)}
    &\sim \frac{1}{2}\left(1 + \frac{3}{2} \epsilon\right).
\end{align}
%
The result is therefore
%
\begin{align}
    \Eff =
    - \mu^{-2\epsilon} \frac{1}{4}\frac{m^4}{(4 \pi)^2}
    \left[
        \frac{1}{\epsilon} 
        - \gamma + \frac{3}{2}
        + \ln\left(4 \pi \frac{\mu^2}{m^2}\right)
    \right]
    + \Oh(\epsilon).
\end{align}
%
With this regulator, one can then add counter-terms to cancel the $\epsilon^{-1}$-divergence.
The exact form of the counter-term is convention.
One may also cancel the finite contribution due to the regulator.
The minimal subtraction ($\mathrm{MS}$) scheme involves only subtracting the divergent term, as the name suggests.
We will use the modified minimal subtraction, or $\overline{ \mathrm{MS}}$, scheme.
In this scheme, one also removes the $-\gamma$ and $\ln(4 \pi)$ term, by defining a new mass scale $\tilde \mu$ by
\begin{equation}
    \label{definition mu tilde MS bar}
    -\gamma + \ln\left(4\pi \frac{\mu^2}{m^2}\right) 
    = \ln\left(4\pi e^{-\gamma} \frac{\mu^2}{m^2}\right) 
    := \ln\left(\frac{\tilde\mu^2}{m^2}\right),
\end{equation}
The result is thus
\begin{equation}
    \Eff =
    - \mu^{-2\epsilon} \frac{1}{4}\frac{m^4}{(4 \pi)^2}
    \left(
        \frac{1}{\epsilon} 
        + \frac{3}{2}
        + \ln \frac{\tilde\mu^2}{m^2}
    \right)
    + \Oh(\epsilon).
\end{equation}



\subsection{Rewriting a real-time integral}
\label{subsection: rewriting integral}


We seek to rewrite the integral
%
\begin{equation}
    \label{free energy logarithmic integral}
    I = \int \frac{\dd^4 p}{(2 \pi)^2} \ln(-p_0^2 + E^2),
\end{equation}
%
where $E$ is some function of the 3-momentum $\vec p$, but not $p_0$.
We use the trick
%
\begin{equation}
    \pdv{}{\alpha} \left(-p_0^2 + E^2\right)^{-\alpha} \Big|_{\alpha=0}
    = \pdv{}{\alpha} \exp {-\alpha \ln\left(- p_0^2 + E^2\right)} \Big|_{\alpha=0}
    = \ln\left(- p_0^2 + E^2\right),
\end{equation}
%
and then perform a Wick-rotation of the $p_0$-integral to write the integral on the form 
%
\begin{equation}
    I = i \pdv{}{\alpha} \int \frac{\dd^4 p}{(2 \pi)^4} \left(p_0^2 + E^2\right)^{-\alpha} \Big|_{\alpha=0},
\end{equation}
%
where $p$ now is a Euclidean four-vector.
The $p_0$ integral equals $\Phi_1(E, 1, \alpha)$, as defined in \autoref{def dimreg integral}. 
The result is therefore given by \autoref{result dimreg},
%
\begin{equation}
    \int \frac{\dd p_0}{2 \pi} (p_0^2 + E)^{-\alpha} 
    = \frac{E^{1-2\alpha}}{\sqrt{4 \pi}} \frac{\Gamma(\alpha-\frac{1}{2})}{\Gamma(\alpha)}.
\end{equation}
%
The derivative of the Gamma function is $\Gamma'(\alpha) = \psi(\alpha)\Gamma(\alpha)$, where $\psi(\alpha)$ is the digamma function.
Using
%
\begin{align}
    \pdv{}{\alpha} & \frac{\Gamma(\alpha - \frac{1}{2}) }{\Gamma(\alpha)} \Big|_{\alpha=0}
    = \Gamma\left(\alpha - \frac{1}{2}\right) \frac{\psi(\alpha - \frac{1}{2}) - \psi(\alpha)}{\Gamma(\alpha)} \Big|_{\alpha=0}
    = \sqrt{4 \pi}, \\
    & \frac{\Gamma(\alpha - \frac{1}{2}) }{\Gamma(\alpha)}\Big|_{\alpha=0} = 0,
\end{align}
%
we get
%
\begin{equation}
    I = i \int \frac{\dd^3 p}{(2 \pi)^3} E.
\end{equation}
%
When the energy is on the simple form $E^2 = \vv p^2 + m^2$, we can use the result from \autoref{subsection: free energy integral}, 
%
\begin{equation}
    \label{dimreg integral}
    \frac{1}{2} \mu^{-2\epsilon}  \int \frac{\dd^d p}{(2 \pi)^d} \sqrt{\vv p^2 + m^2}
    = \frac{1}{2} \frac{\mu^{-2\epsilon}}{(4 \pi)^2} 
    \left(\frac{1}{\epsilon} + \frac{3}{2} + \ln \frac{\tilde \mu^2}{m^2}\right), 
\end{equation}
%
This integral is regulated using dimensional regularization, where $d = 3 - 2\epsilon $, and using the $\overline{\text{MS}} $-shceme, as described in \autoref{subsection: free energy integral}.
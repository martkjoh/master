\section{Algebra bases}
\label{section: algebra bases}

\subsection{Pauli matrices}
\label{subsection: Pauli matrices}

The Pauli matrices are
\begin{align}
    \tau_1 = 
    \begin{pmatrix}
        0 & 1 \\
        1 & 0 \\
    \end{pmatrix}
    , \quad 
    \tau_2 = 
    \begin{pmatrix}
        0 & -i \\
        i & 0 \\
    \end{pmatrix}, \quad 
    \tau_3 = 
    \begin{pmatrix}
        1 & 0 \\
        0 & -1 \\
    \end{pmatrix}.
\end{align}
They obey
\begin{align}
    [\tau_a, \tau_b] &= 2 i \varepsilon_{abc}\tau_c, \\
    \{\tau_a,\tau_b\} &= 2\delta_{ab} \one, \\
    \Tr{\tau_a} &= 0, \\
    \Tr{\tau_a \tau_b} &= 2 \delta_{ab}, \\
    \Tr{\tau_a\tau_b\tau_c\tau_d} 
    & = 2 (\delta_{ab}\delta_{cd} - \delta_{ac}\delta_{cb} + \delta_{ad}\delta_{cb}).
\end{align}
Together with the identity matrix $\one$, the Pauli matrices form a basis for the vector space of all 2-by-2 matrices.
An arbitrary 2-by-2 matrix $M$ may be written
\begin{equation}
    \label{2-by-2 matrix decomp}
    M = M_0 \one + M_a \tau_a, \quad 
    M_0 = \frac{1}{2} \Tr{M}, \,\, M_a = \frac{1}{2} \Tr{\tau_a M}.
\end{equation}


\subsection{Gell-Mann matrices}
\label{subsection: gell-mann matrices}


The Gell-Mann matrices are
%
\begin{align*}
    \lambda_1
    & = 
    \begin{pmatrix}
        0 & 1 & 0 \\
        1 & 0 & 0 \\
        0 & 0 & 0
    \end{pmatrix},\,
    \lambda_2
    = 
    \begin{pmatrix}
        0 & -i & 0 \\
        i & 0 & 0 \\
        0 & 0 & 0
    \end{pmatrix},\,
    \lambda_3
    = 
    \begin{pmatrix}
        1 & 0 & 0 \\
        0 & -1 & 0 \\
        0 & 0 & 0
    \end{pmatrix},
    \lambda_4
    = 
    \begin{pmatrix}
        0 & 0 & 1 \\
        0 & 0 & 0 \\
        1 & 0 & 0
    \end{pmatrix},\,\\
    \lambda_5
    &= 
    \begin{pmatrix}
        0 & 0 & -i \\
        0 & 0 & 0 \\
        i & 0 & 0
    \end{pmatrix},\,
    \lambda_6
    = 
    \begin{pmatrix}
        0 & 0 & 0 \\
        0 & 0 & 1 \\
        0 & 1 & 0
    \end{pmatrix},
    \lambda_7
    = 
    \begin{pmatrix}
        0 & 0 & 0 \\
        0 & 0 & -i \\
        0 & i & 0
    \end{pmatrix},\,
    \lambda_8
    =  \frac{1}{\sqrt 3}
    \begin{pmatrix}
        1 & 0 & 0 \\
        0 & 1 & 0 \\
        0 & 0 & -2
    \end{pmatrix}.
\end{align*}
%
They obey
%
\begin{align}
    [\lambda_a, \lambda_b] & = 2if^{abc} \lambda_c, \\
    \{\lambda_a, \lambda_b\} & = \frac{4}{3}\one \delta_{ab} + 2d_{abc} \lambda_c, \\
    \Tr{\lambda_a} & = 0, \\
    \Tr{\lambda_a \lambda_b} &= 2 \delta_{ab}, \\
    \Tr{\lambda_a \lambda_b \lambda_c \lambda_d} 
    &= \frac{4}{3} \delta_{ab}\delta_{cd} + 2 (d_{abe} + if_{abe})(d_{cde} + if_{cde}).
\end{align}
%
where
%
\begin{equation}
    f_{abc} = -\frac{i}{4}\Tr{\lambda_a[\lambda_b, \lambda_c]}, 
    \quad
    d_{abc} = -\frac{i}{4}\Tr{\lambda_a\{\lambda_b, \lambda_c\}}.
\end{equation}
%
where the non-zero elements of $f_{abc}$ and $d_{abc}$ are
%
\begin{align}
    \label{structure constants su(3)}
    f_{123} &= 1, \quad 
    f_{147} = f_{246} = f_{257} = f_{345} = -f_{156} =  -f_{367} = \frac{1}{2}, \quad
    f_{458} = f_{678} = \frac{\sqrt 3}{2}, \\ \nonumber
    d_{146} &= d_{157} = d_{256} = -d_{247} = d_{355} = - d_{366} = -d_{377} = \frac{1}{2} \\
    d_{118}& = d_{228} = d_{338} = - d_{888} = \frac{1}{\sqrt 3}, \quad
    d_{448} = d_{558} = d_{668} = d_{778} = - \frac{1}{2 \sqrt 3},
\end{align}
%
or a permutation of the indices.
The indices of $f$ are totally antisymmetric, while those of $d$ are totally symmetric~\autocite{borodulinCORECOmpendiumRElations2017}.




\subsection{Gamma matrices}
\label{subsection: gamma matrices}

The gamma matrices $\gamma^\mu$, $\mu \in \{0, 1, 2, 3\}$, obey
%
\begin{align}
    \{\gamma^\mu,\gamma^\nu\} = 2 g^{\mu \nu} \one,\\
    {\gamma^0}^\dagger = \gamma^0, \quad {\gamma^i}^\dagger = - \gamma^i.
\end{align}
%
These matrices, together with
%
\begin{align}
    \sigma^{\mu\nu} &= \frac{1}{2} [\gamma^\mu, \gamma^\mu], \\ 
    \gamma_A^\mu &= \gamma^\mu \gamma^5, \\
     \gamma^5 
    &= \frac{i}{4!}\epsilon_{\mu \nu \rho \sigma} \gamma^{\mu}\gamma^{\nu}\gamma^{\rho}\gamma^{\sigma},
\end{align}
%
form the Clifford algebra $\text{Cl}_{1,3}$, also known as the \emph{space-time algebra}.
The subscripts $(1, 3)$ denotes the signature of the metric.
The ``fifth $\gamma$-matrix'', which can be expressed as $\gamma^5 = \gamma^0\gamma^1\gamma^2\gamma^3$, obey
%
\begin{equation}
    \{\gamma^5,\gamma^\mu\} = 0, \quad (\gamma^5)^2 = \one.
\end{equation}


The Euclidian counterpart of the space-time algebra, $\text{Cl}_4$, is defined by the ``Euclidian gamma matrices'', which obey
%
\begin{equation}
    \{\tilde \gamma_a, \tilde \gamma_b\} = 2 \delta_{ab}\one.
\end{equation}
%
These can be related to the regular Minkowski-matrices by
%
\begin{equation}
    \tilde \gamma_0 = \gamma^0,\quad 
    \tilde \gamma_j = -i\gamma^j.
\end{equation}
%
These then obey
%
\begin{equation}
    {\tilde\gamma_a}^\dagger = \tilde\gamma_a.
\end{equation}
%
The Euclidean $\tilde \gamma_5$ is defined as
%
\begin{equation}
    \tilde \gamma_5 = \gamma_0\gamma_1\gamma_2\gamma_3 = i \gamma^0\gamma^1\gamma^2\gamma^3 = \gamma^5.
\end{equation}
It thus also anti-commutes with the Euclidean $\gamma$-matrices,
%
\begin{equation}
    \{\tilde \gamma_5, \tilde \gamma_a\} = 0.
\end{equation}


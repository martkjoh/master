\section{Functionals}
\label{appendix: Functional derivatives}
(TODO: INKLUDER KILER!!!!)

The principle of stationary action and the path integral method relies on functional calculus, where ordinary, $n$-dimensional calculus is generalized to an infinite-dimensional calculus on a space of functions.
A functional, $S$, takes in a function $\varphi(x)$, and returns a real number $S[\varphi]$.
We will be often be dealing with functionals of the form
%
\begin{equation}
    \label{general action functional}
    S[\varphi] = \int_\Em \dd^n x \, \Ell[\varphi](x),
\end{equation}
%
Here, $\Ell[\varphi](x)$, the Lagrangian density, is a functional which takes in a function $\varphi$, and returns a real number $\Ell[\varphi](x)$ \emph{for each point} $x \in \Em$.
Thus, $\Ell$ does, in fact, return a real-valued function, not just a number.
$\Em$ is the manifold, in our case space-time, of which both $\varphi(x)$ and $\Ell[\varphi](x)$ are functions.
The function $\varphi$ can, in general, take on the value of a scalar, complex number, spinor, vector, etc\dots, while $\Ell[\varphi](x)$ must be a scalar-valued function.
This strongly constraints the form of any Lagrangian and is an essential tool in constructing quantum field theories.
Although this section is written with a single scalar-valued function $\varphi$, this can easily be generalized by adding an index, $\varphi \rightarrow \varphi_\alpha$, enumerating all the degrees of freedom, then restating the arguments~\autocite{carrollSpacetimeGeometryIntroduction2019,schwartzQuantumFieldTheory2013}.


\subsection{Functional derivative}

The functional derivative is base on an arbitrary \emph{variation} $\eta$ of the function $\varphi$.
The variation $\eta$, often written $\delta \varphi$ is an arbitrary function only constrained to vanish \emph{quickly enough} at the boundary $\partial \Em$.\footnote{%
The condition of ``quickly enough'' is to ensure that we can integrate by parts and ignore the boundary condition, which we will do without remorse.
}
The variation of the functional $S$ is defined as
%
\begin{equation}
    \delta_\eta S[\varphi] = \lim_{\epsilon \rightarrow 0} \frac{1}{\epsilon}
    \left( S[\varphi + \epsilon \eta] - S[\varphi] \right) 
    = \odv{}{\epsilon} S[\varphi + \epsilon \eta] |_{\epsilon = 0}.
\end{equation}
%
We can regard the variation of a functional as the generalization of the differential of a function, \autoref{covectors i.e. one forms}, as the best linear approximation around a point.
In regular differential geometry, a function $f$ can be approximated around a point $x$ by
%
\begin{equation}
    f(x + \epsilon v) = f(x) + \epsilon \dd f(v),
\end{equation}
%
where $v$ is a vector in the tangent space at $x$.
In functional calculus, the functional $S$ is analogous to $f$, $\varphi$ to $x$, and $\eta$ to $v$.
We can more clearly see the resemblance by writing
%
\begin{equation}
    \odv{}{\epsilon} f(x + \epsilon v) = \dd f(v) = \pdv{f}{x^\mu} v^\mu.
\end{equation}
%
In the last line we expanded the differential using the basis-representation, $v = v^\mu\partial_\mu$.
To generalize this to functional, we define the \emph{functional derivative}, by
%
\begin{equation}
    \label{definition functional derivative}
    \delta_\eta S[\varphi] = \int_\Em \dd^n x \, \fdv{S[\varphi]}{\eta(x)} \eta(x).
\end{equation}
%
If we let $S[\varphi] = \varphi(x)$, for some fixed $x$, the variation becomes
%
\begin{equation}
    \delta_\eta S [\varphi] = \eta(x) = \int \dd^n y \, \delta(x - y) \eta(y),
\end{equation}
%
which leads to the identity
%
\begin{equation}
    \fdv{\varphi(x)}{\varphi(y)} = \delta(x - y).
\end{equation}
%
There is also a generalized chain rule for functional derivatives.
If $\psi$ is some new functional variable, then
%
\begin{equation}
    \fdv{S[\varphi]}{\varphi(x)}
    = \int_\Em \dd^n y \, 
    \fdv{S[\varphi]}{\psi(y)}
    \fdv{\psi(y)}{\varphi(x)}.
\end{equation}
%
Higher functional derivatives are defined in terms of higher-order variations,
%
\begin{equation}
    \delta^m_\eta S[\varphi]
    = \odv{}{\epsilon} \delta^{m-1}_\eta S[\varphi + \epsilon \eta]|_{\epsilon=0}
    = \int_\Em 
    \left(\prod_{i=1}^m \dd^n x_i \eta(x_i)\right) 
    \frac{\delta^m S[\varphi]}{\delta \varphi(x_1)...\delta\varphi(x_m)}.
\end{equation}
%
With this, we can write the functional Taylor expansion,
%
\begin{equation}
    S[\varphi_0 + \varphi]
    = S[\varphi_0]
    + \int_\Em \dd^n x \, \varphi(x) \fdv{S[\varphi_0]}{\varphi(x)}
    + \frac{1}{2} \int_\Em \dd^n x \dd^n y \, \varphi(x) \varphi(y) \fdv{S[\varphi_0]}{\varphi(x), \varphi(y)}
    +\dots
\end{equation}
%
Here, the notation $\fdv{S[\varphi_0]}{\varphi}$ indicate that $S[\varphi]$ is first differentiated with respect to $\varphi$, then evaluated at $\varphi = \varphi_0$~\autocite{peskinIntroductionQuantumField1995}.


\subsection{The Euler-Lagrange equation}

The Lagrangian may also be written as a scalar function of the field-values at $x$, $\varphi(x)$, as well as its derivatives, $\partial_\mu \varphi(x)$, for example
%
\begin{equation}
    \Ell(\varphi, \partial_\mu \varphi) = \frac{1}{2} \partial_\mu \varphi \partial^\mu\varphi - \frac{1}{2} m^2 \varphi^2 - \frac{1}{4!}\lambda \varphi^4+ \dots
\end{equation}
%
We have omitted the evaluation at $x$ for the brevity of notation.
We use this to evaluate the variation of a functional in the of \autoref{general action functional}, 
%
\begin{equation}
    \label{variation of action}
    \delta_\eta S[\varphi] = \odv{}{\epsilon}
    \int_\Em \dd^n x \, \Ell[\varphi + \epsilon \eta](x),
\end{equation}
%
by Taylor expanding the Lagrangian density as a function of $\varphi$ and its derivatives,
%
\begin{equation}
    \Ell[\varphi + \epsilon \eta]
    = \Ell
    \left(
        \varphi + \epsilon \eta, \partial_\mu\{\varphi + \epsilon \eta\}
    \right)
     = 
    \Ell[\varphi]
    +
    \epsilon
    \left(
        \pdv{\Ell}{\varphi} \eta 
        + \pdv{\Ell}{(\partial_\mu \varphi)}\partial_\mu\eta 
    \right) + \Oh(\epsilon^2).
\end{equation}
%
Inserting this into \autoref{variation of action} and partially integrating the last term allows us to write the variation in the form \autoref{definition functional derivative}, and the functional derivative is
%
\begin{equation}
    \fdv{S}{\varphi} = \pdv{\Ell}{\varphi} - \partial_\mu \pdv{\Ell}{(\partial_\mu \varphi)}.
\end{equation}
%
The principle of stationary action says that the equation of motion of a field obeys $\delta_\eta S = 0$.
As $\eta$ is arbitrary, this is equivalent to setting the functional derivative of $S$ equal to zero.
The result is the Euler-Lagrange equations of motion~\autocite{schwartzQuantumFieldTheory2013},
%
\begin{equation}
    \pdv{\Ell}{\varphi} 
    -
    \partial_\mu \pdv{\Ell}{(\partial_\mu \varphi)}
    = 0.
\end{equation}


\subsection{Functional calculus on a curved manifold}
\label{subsection: functional calculus on a curved manifold}

As discussed in \autoref{subsection: integration on manifolds}, when integrating a scalar on a curved manifold, we must include the $\sqrt{|g|}$-factor to get a coordinate-independent result.
The action in curved spacetime is therefore~\autocite{carrollSpacetimeGeometryIntroduction2019}
%
\begin{equation}
    S[g, \varphi] = \int_\Em \dd^n x \, \sqrt{|g|} \Ell[g, \varphi],
\end{equation}
%
where the action and the Lagrangian now is a functional of both the matter-field $\varphi$ and the metric $g_{\mu \nu}$.
Our example Lagrangian from last section now takes the form
\begin{equation}
    \label{Lagrangian curved spacetime}
    \Ell(g_{\mu \nu}, \varphi, \nabla_\mu \varphi) = \frac{1}{2} g^{\mu \nu} \nabla_\mu \varphi \nabla_\nu \varphi - \frac{1}{2}m^2 \varphi^2 - \frac{1}{4!}\lambda \varphi^4 \dots,
\end{equation}
%
where partial derivatives are substituted with covariant derivatives.
We define the functional derivative as
%
\begin{equation}
    \delta_\eta S = \int_\Em \dd^n x \sqrt{|g|} \fdv{S}{\eta(x)} \eta(x).
\end{equation}
%
If this is a variation in $\varphi$ only, this gives the same result as before.
However, in general relativity, the metric itself is a dynamic field, and we may thereofre vary it.
Consider $g_{\mu \nu} \rightarrow g_{\mu \nu} + \delta g_{\mu \nu}$.
The variation of the action is then
assuming $\Ell$ only depends on $g$ and not its derivatives, we get
%
\begin{equation}
    \label{result derivation of einstein field equation}
    \delta_{g} S = \int_\Em \dd^n x \, 
    \left[
        \left(\delta \sqrt{|g|}\right) \Ell[g] + \sqrt{|g|} \delta \Ell[g]
    \right]
\end{equation}
%
We have used
%
\begin{equation}
    \label{variation of metric factor}
    \delta \sqrt{|g|} = -\frac{1}{2} \sqrt{|g|}g_{\mu\nu} \delta g^{\mu \nu}
\end{equation}
%
The variation of the $\sqrt{|g|}$-factor can be evaluated using
Using the Levi-Civita symbol $\varepsilon_{\mu_1 \dots \mu_n}$, a determinant of a $n \times n$-matrix may be written as
%
\begin{equation}
    \det(A) = \frac{1}{n!} \varepsilon_{\mu_1\dots\mu_n}\varepsilon^{\nu_1\dots\nu_n}
    A^{\mu_1}{}_{\nu_1} \dots A^{\mu_n}{}_{\nu_n}.
\end{equation}
%
Using this, we can write for a matrix $M$
%
\begin{align}
    \det(\one + \varepsilon M) &
    = 
    \frac{1}{n!}
    \varepsilon_{\mu_1\dots}\varepsilon^{\nu_1\dots}
    (\one + \epsilon M)^{\mu_1}{}_{\nu_1}  (\one + \epsilon M)^{\mu_2}{}_{\nu_2} \dots\\
    & =
    \frac{1}{n!}\varepsilon_{\mu_1\dots} \varepsilon^{\nu_1\dots} 
    [
        \delta^{\mu_1}_{\nu_1} \dots 
        + 
        \epsilon(
            M^{\mu_1}{}_{\nu_1} \delta^{\mu_2}_{\nu_1}\dots 
            + M^{\mu_2}{}_{\nu_2}\delta^{\mu_1}_{\nu_1}\dots 
            +\dots)
    + \dots]\\
   & 
   = 1 + M^{\mu}{}_{\mu}  + \mathcal{O}(\epsilon^2)
\end{align}
%
Thus,
%
\begin{equation}
    \delta \sqrt{|g|} 
    = \sqrt{| \det[g_{\mu\nu}(\delta^\nu_\rho + g^{\nu \sigma}\delta g_{\sigma \rho})]|}
    -\sqrt{|g|}
    = \sqrt{|g|} 
    \left(
        \sqrt{|1 + g^{\mu \nu} \delta g_{\mu \nu}|} - 1
    \right)
    = -\frac{1}{2}\sqrt{|g|} g^{\mu \nu} \delta g_{\mu \nu}.
\end{equation}
%
The minus sign is included as the determinant of a Lorentzian metric is negative. 
Assuming the Lagrangian only depneds on the metric directly, and not its derivatives, the variation of the action is
%
\begin{equation}
    \delta_g S
    = 
    \int_\Em \dd^n x \, \sqrt{|g|}
    \left(
       \pdv{\Ell}{g^{\mu\nu}}
    - \frac{1}{2}g_{\mu \nu} \Ell 
    \right) \delta g^{\mu \nu}.
\end{equation}
%
   With the Lagrangian in \autoref{Lagrangian curved spacetime}, we get
%
\begin{equation}
    \label{functional derivative with respect to metric}
    \fdv{S}{g^{\mu \nu}}
    =
    \pdv{\Ell}{g^{\mu\nu}}
    - \frac{1}{2}g_{\mu \nu} \Ell 
    =
    - \frac{1}{2}
    \left(
        \frac{1}{2} \nabla_\mu \varphi \nabla_\nu \varphi + \frac{1}{2}m^2 \varphi^2 + \dots
    \right).
\end{equation}
%
We recognize the $(\mu, \nu )= (0, 0)$-component as negative half the Hamiltonian density, which supports the definition of the definition of the stress-energy tensor \autoref{definition stress energy densor}.
 


\subsection{Functional derivative of the Einstein-Hilbert action}
\label{subsection: functional derivative of the einstein-hilbert action}
(NEEDS MORE CLEANUP)

In the Einstein-Hilbert action, \autoref{Einstein-Hilbert action}, the Lagrangian density is $\Ell = k R = k g^{\mu \nu} R_{\mu \nu}$, where $k$ is a constant and $R_{\mu \nu}$ the Ricci tensor, \autoref{Ricci tensor}.
As the Ricci tensor is dependent on both the derivative and second derivative of the metric,  we can not use \autoref{functional derivative with respect to metric} directly.
Instead, we use the variation
%
\begin{equation}
    \delta S_{\text{EH}} = k \int_{\Em} \dd^n x \, \sqrt{|g|}
    \left( \delta R - \frac{1}{2} g_{\mu \nu} R \delta g^{\mu \nu} \right).
\end{equation}
%
The variation of the Ricci scalar is
%
\begin{equation}
    \delta R = R_{\mu \nu} \delta g^{\mu \nu} + g^{\mu \nu} \delta R_{\mu \nu},
\end{equation}
%
We can write the variation of the Ricci scalar, and thus the Riemann curvature tensor, in terms of variations in Christoffel symbols, $\delta \Gamma^{\rho}_{\mu \nu}$ using the explicit formula for a symmetric, metric-compatible covariant derivative, \autoref{riemann tensor in terms of christoffel symbols}.
As $\delta \Gamma = \Gamma - \Gamma'$, it is a tensor, and we may write
\todo{clean up}
%
\begin{align*}
    \delta R^\rho{}_{\sigma \mu \nu} 
    & = \delta(\partial_{[\mu} \Gamma^\rho_{\nu] \sigma} + \Gamma^\rho_{\lambda [\mu} \Gamma^\lambda_{\nu] \sigma})
    = \partial_{[\mu} \delta \Gamma^\rho_{\nu] \sigma} + (\delta \Gamma^\rho_{\lambda [\mu}) \Gamma^\lambda_{\nu] \sigma} + \Gamma^\rho_{\lambda [\mu}(\delta \Gamma^\lambda_{\nu] \sigma}) \\
    & = \partial_{\mu} \delta \Gamma^\rho_{\nu \sigma}  + \Gamma^\rho_{\lambda \mu}(\delta \Gamma^\lambda_{\nu \sigma}) - \Gamma^\lambda_{\mu \sigma}   (\delta \Gamma^\rho_{\lambda \nu}) - \left( \partial_{\nu} \delta \Gamma^\rho_{\mu \sigma}  + \Gamma^\rho_{\lambda \nu}(\delta \Gamma^\lambda_{\mu \sigma}) - \Gamma^\lambda_{\nu \sigma}   (\delta \Gamma^\rho_{\lambda \mu}) \right) + (\Gamma^{\lambda}_{\mu\nu}\delta\Gamma^\rho_{\lambda \sigma} - \Gamma^{\lambda}_{\mu\nu}\delta\Gamma^\rho_{\lambda \sigma}) \\
    & = \nabla_{\mu}\delta \Gamma^\rho_{\nu \sigma} - \nabla_{\nu}\delta \Gamma^\rho_{\mu \sigma}
     = \nabla_\eta \left(g^\eta{}_{\mu} \delta\Gamma^\rho_{\nu \sigma} - g^\eta{}_{\nu} \delta\Gamma^\rho_{\mu \sigma} \right) 
    = \nabla_\eta (K^\rho{}_{\sigma \mu \nu})^\eta,
\end{align*}
%
where $K$ is a tensorial quantity, which vanish at the boundary of our spacetime.
Using the generalized divergence theorem, \autoref{generalized divergence theorem}, we see that the contribution to the action from this quantity vansih.
The contribution comes from an integral over $g^{\mu \nu} \delta R_{\mu \nu} = g^{\mu \nu} \delta R^{\rho}_{\mu \rho \nu} = g^{\mu \nu} \nabla_\eta (K^\rho{}_{\mu \rho \nu})^\eta$
Using metric compatibility, we can exchange the covariant derivative and the metric, and we have $g^{\mu \nu} \delta R_{\mu \nu} = \nabla_\eta [g^{\mu \nu}K^{\eta \rho}{}_{\mu \rho \nu}]$.
The contribution to the action therefore becomes
%
\begin{equation}
    \int_\Em \dd^4 x \, \sqrt{|g|} g^{\mu \nu} \delta R_{\mu \nu} 
    = \int_\Em \dd^4 x \, \sqrt{|g|} \nabla_\eta [g^{\mu \nu}K^{\eta \rho}{}_{\mu \rho \nu}]
    = \int_{\partial \Em} \dd^3 y \, \sqrt{|\gamma|} n_\eta [g^{\mu \nu}K^{\eta \rho}{}_{\mu \rho \nu}] = 0,
\end{equation}
where we used the fact that $\delta g_{\mu \nu}$, and thus $K$, vanish at $\partial \Em$.
The variation of the action is therefore
%
\begin{equation}
    \delta S_{\text{EH}} = k \int_\Em \dd^n x \sqrt{|g|} \left[R_{\mu \nu} - \frac{1}{2} R g_{\mu \nu}\right] \delta g^{\mu \nu},
\end{equation}
%
and by the definition of the functional derivative, 
%
\begin{equation}
    \label{functional derivatie einstein-hilber action}
    \fdv{S_{\text{EH}}}{g^{\mu \nu}} 
    =
    k(R_{\mu \nu} - \frac{1}{2} R g_{\mu \nu}).
\end{equation}

\section{Constants and units}
\label{section: units}

All values in this section are from the Particle Data Group~\cite{particledatagroupReviewParticlePhysics2020}.
To obtain results in the SI-system, we use the following conversion factors, as given by
%
\begin{align}
    \label{speed of ligh}
    c       &= 2.998 \cdot 10^8     \, \text{m} \, \text{s}^{-1}, \\
    \label{hbar}
    \hbar   &= 1.055 \cdot 10^{-34} \, \text{J} \, \text{s}, \\
    \label{Boltzmanns constat}
    k_B     &= 1.380 \cdot 10^{-23} \, \text{J} \, \text K^{-1}, \\
    \label{Newtons gravitational constant}
    G       &= 6.674 \cdot 10^{-11} \, \text m^3 \, \text{kg}^{-1} \, \text s^{-2},
\end{align}
%
where $G$ is Newton's gravitational constant.
The conversion factor between $\text{MeV}$ and SI-units is
%
\begin{equation}
    \label{electronvolt}
    1 \, \text{MeV} = 1.60218\, \cdot 10^{-19} \, \text{J}. 
\end{equation}
%
The fine structure constant and the elementary charge is
%
\begin{align}
    \label{Fine structure constant}
    \alpha &= 7.297 \cdot 10^{-3}, \\
    \label{Elementary charge}
    e &:= \sqrt{4 \pi \alpha} =  3.028\cdot 10^{-1}.
\end{align}
%
In astronomical calculation, the solar mass is used, which is
%
\begin{equation}
    \label{solar mass}
    M_\odot = 1.988 \cdot 10^{30} \, \text{kg}.
\end{equation}
%
The physical parameters we use are
%
\begingroup
\allowdisplaybreaks % Make page break possbible
\begin{align}
    \label{pion decay constant}
    f_\pi & =  92.1 \, \text{MeV}, \\
    \label{pion mass}
    m_\pio & = 134.98 \, \text{MeV}, \\
    \label{charged pion mass}
    m_{\pipm} &= 139.57 \, \text{MeV}, \\
    m_{\Kpm} & = 493.68\,\text{MeV}, \\
    m_{\Ko} & = 497.61\,\text{MeV}, \\
    m_e &= 0.5110 \, \text{MeV}, \\
    m_\mu &= 105.7 \, \text{MeV}, \\
    \label{mass of neutron}
    m_N &= 939.57 \, \text{MeV}.
\end{align}
\endgroup
%
For the $\rho$-meson, we use the mass
%
\begin{equation}
    m_\rho = 770\, \text{MeV}.
\end{equation}


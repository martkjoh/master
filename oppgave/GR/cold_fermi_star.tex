\section{A star of cold, non-interacting fermions}
\label{section: cold fermi star}


Ekstra kilder: \autocite{andersenIntroductionStatisticalMechanics2012,glendenningCompactStarsNuclear2012}
\todo{Ha kilder på riktig sted}


\subsection{Thermodynamics and free energy}

A non-interacting Fermi gas is governed by the Dirac Lagrangian
%
\begin{equation}
    \Ell = \bar \psi (i \slashed \partial - m) \psi,
\end{equation}
%
as described in (APPENDIX thermal field theory).
\todo{add appendix}
In the grand canoncial ensamble, the density of a conserved charge is regulated by a chemical potential $\mu$.
The conserved current corresponding to the $\text U(1)$ phase symmetry of the Dirac Lagrangian, i.e., the transformation $\psi \rightarrow e^{i \alpha} \psi \approx 1 + \alpha \delta \psi$, is
%
\begin{equation}
    j^\mu = \diffp{\Ell}{(\partial_\mu \psi)} \delta \psi = \bar \psi \gamma^\mu \psi.
\end{equation}
%
The conserved charge is
%
\begin{equation}
    Q = \int \dd^3 x \, j^0 = \int \dd^3 x\, \bar \psi \gamma^0 \psi,
\end{equation}
%
the number of particles minus antiparticles.
The total energy $U$ is related to the grand canonical free energy $F$ by a Legendre transformation,
%
\begin{equation}
    F(T, V, \mu) = U - T S - \mu Q, \quad \dd F = - S \dd T - p \dd V - Q \dd \mu.
\end{equation}
%
Here, $p$ is pressure, $T = {1}/{\beta}$ is temperature, and $S$ entropy.
These thermodynamic variables are related to the free energy by
%
\begin{equation}
    S = - \diffp{F}{T} = \beta^2 \diffp{F}{\beta}, \quad
    Q = - \diffp{F}{\mu}, \quad
    p = - \diffp{F}{V}.
\end{equation}
%
When the free energy can be written as $F = V \Eff$, where the free energy density $\Eff$ is independent of the volume $V$, then $\Eff = - p$ and
%
\begin{equation}
    \dd (V \Eff) = V \dd \Eff - p \dd V,
\end{equation}
%
allowing us to write
%
\begin{equation}
    \Eff(T, \mu) = u - Ts - \mu n, \quad
    \dd \Eff = -s \dd T - n \dd \mu,
\end{equation}
% 
where $s$ and $n$ are entropy and charge density, defined by
%
\begin{equation}
    s = - \diffp{\Eff}{T} = \beta^2 \diffp{\Eff}{\beta}, \quad
    n = - \diffp{\Eff}{\mu}.
\end{equation}
%
With this, we can write the energy density as
%
\begin{equation}
    u = \diffp{}{\beta} \left(\beta \Eff\right) + \mu n.
\end{equation}


\subsection{Equation of state}
The free energy density of the Fermion gas is (REF: appendix felt teori)\todo{inkluder appendix om termisk felt-teori}
%
\begin{equation}
    \Eff = - 
    \frac{2}{\beta}\int \frac{\dd^3 p}{(2 \pi)^3} 
    \left[
        \beta \omega
        +
        \ln\left(1 + e^{-\beta(\omega - \mu)}\right)
        + 
        \ln\left(1 + e^{-\beta(\omega + \mu)}\right)
    \right],
\end{equation}
%
where $\omega = \sqrt{p^2 + m^2}$.
The first term in the integral is the divergent vacuum energy, which must be renormalized.
We can drop this term; it does not have any observable effects on our results, as we are interested in relative pressure and energy density.
With this, we find the charge density
%
\begin{equation}
    n = \frac{1}{\pi^2} \int \frac{\dd^3 p}{(2 \pi)^3} [n_f(\omega - \mu) - n_f(\omega + \mu)],
\end{equation}
%
where
%
\begin{equation}
    n_f(\omega) = \frac{1}{e^{\beta \omega} + 1}.
\end{equation}
%
is the Fermi-Dirac distribution.
The energy density is 
%
\begin{equation}
    \label{energy density}
    u = \frac{1}{\pi^2} \int_0^\infty \dd p\, p^2 \, \omega \, [n_f(\omega - \mu) + n_f(\omega + \mu)].
\end{equation}
%
As expected, this is the energy per mode times the density of states, integrated over all modes.
To write the pressure, $p = - \Eff$ in terms of an integral over the Fermi-Dirac distribution, we integrate by parts.
We have
%
\begin{equation}
    \int_0^\infty \dd p \, p^2 \ln\left[1 + e^{-\beta(\omega \pm m)}\right]
    = 
    \frac{1}{3} p^3\ln\left[1 + e^{-\beta(\omega \pm m)}\right] \bigg |_0^\infty
    + 
    \frac{1}{3} \int_0^\infty \dd p \, \frac{ \beta p^4}{\omega}n_f(\omega \pm \mu),
\end{equation}
%
where the boundary term vanish.
This allows us to write the pressure as 
%
\begin{equation}
    \label{pressure}
    p = \frac{1}{3} \int_0^\infty \dd p \, \frac{p^4}{\omega} [n_f(\omega - \mu) + n_f(\omega + \mu)]
\end{equation}



We are interested in the $T = 0$ limit.
In this case, the Fermi distribution becomes a step function, $n_f(\omega) = \theta(-\omega)$.
Without loss of generality, we assume that $\mu > 0$, i.e., we are dealing with an abundance of matter compared to anti-matter.
The dispersion relation $\omega = \sqrt{p^2 + m^2}$ is always positivive.
This means that the contribution to thermodynamic quantities from anti-particles vanish, as the integral is multiplied with $n_f(\omega + \mu) = \theta(-\omega - \mu)$, where the argument $-\omega - \mu$ is strictly negative on the domain of integration.
At zero temperature, the only dynamics are due to the degeneracy pressure of the fermions, that is, due to the Pauli exclusion principle.
There are no thermal fluctuations that can create a particle-antiparticle pair.
Thus, if the system has a positive chemical potential, it will contain no antiparticles.
Furthermore, if $\mu< m$, then integrand multiplied with $n_f(\omega - \mu)$ is also zero in the whole domain of integration.
It is only when $\mu\geq m$ that it is energetically favorable for the system to be in a state with particles.
We define the Fermi momentum $p_f$ by $\mu = \sqrt{\smash{p_f}^2 + m^2}$. 
In the zero-temperature limit, we can then rewrite any integral over the Fermi distribution as
%
%
\begin{equation}
    \int_0^\infty \dd p \, [f(p) n_f(\omega - \mu) + g(p) n_f(\omega - \mu)]= \int_0^{p_f} \dd p \, f(p).
\end{equation}
%
The charge density is thus,
%
\begin{equation}
    n = \frac{1}{\pi^2} \int_0^{p_f} \dd p\, p^2 = \frac{p_f^3}{3 \pi^2}.
\end{equation}
%
At $T = 0$, this is the particle number density, as there are no antiparticles.
This density is given by the chemical potential and vanishes when $\mu \leq m$, i.e. when the free energy cost of creating a particle is positive.
We can write the energy density and pressure integrals, \autoref{energy density} and \autoref{pressure}, as
%
\begin{align}
    u &= \frac{1}{\pi^2} \int_0^{p_f} \dd p \,
    p^2 \sqrt{p^2 + m^2}
    = \frac{m^4}{\pi^2} \int_0^{x_f} \dd x \, x^2 \sqrt{x^2 + 1}, \\
    p & = \frac{1}{3 \pi^2} \int_0^{p_f} \dd p \,  \frac{p^4}{\sqrt{p^2 + m^2}} 
    = \frac{m^4}{3 \pi^2} \int_0^{x_f} \frac{\dd x \, x^4}{\sqrt{x^2 + 1}}.
\end{align}
% 
We have defined $x = p / m$ and $x_f = p_f/m$.
These integrals can be evaluated exactly as
%
\begin{align}
    \int_0^a \dd x \, x^2 \sqrt{x^2 + 1} 
    & = \frac{1}{8} 
    \left[\sqrt{a^4 + 1}(2 a^3 + a) - \arcsinh\left(a\right)\right], \\
    \int_0^a \dd x \, \frac{x^4}{\sqrt{x^2 + 1} }
    & = \frac{1}{8} 
    \left[\sqrt{a^2 + 1}(2 a^3 - 3a) + 3\arcsinh\left(a\right)\right].
\end{align}
%
We introduce the characteristic energy and number density,
% 
\begin{equation}
    u_0 = \frac{m^4}{8 \pi^2}, \quad n_0 = \frac{u_0}{m},
\end{equation}
%
which allows us to write the thermodynamic variables as
%
\begin{align}
    n &= \frac{8}{3}  n_0 \,x_f^3 \\
    \label{Fermi gas energy density}
    u &= u_0
    \left[(2x_f^3 + x_f) \sqrt{1 + x_f^2} - \arcsinh\left(x_f\right)\right], \\
    \label{Fermi gas pressure}
    p &= \frac{u_0}{3}
    \left[(2x_f^3 - 3x_f) \sqrt{1 + x_f^2} + 3\arcsinh\left(x_f\right)\right].
\end{align}
%
We have thus chosen $u_0 = p_0$, or equivalently set $k_3 = 1$.
This is natural in the case of a relativistic fluid.




\subsection{Limits}

In the non-relativistic limit, as the chemical potential approaches $m$ and thus $p_f \ll m$, the lowest order contributions to the energy density and pressure are given by the Taylor series around $x_f = 0$,
%
\begin{align}
    \tilde u(x_f) &= \frac{8}{3}x_f^3 + \frac{4}{5} x_f^5 + \Oh(x_f^7),  \\
    \tilde p(x_f) &= \frac{8}{15}x_f^5 + \Oh(x_f^7).
\end{align}
%
By neglecting terms of order $x_f^7$ and higher, we can write this as
%
\begin{equation}
    \tilde u = \tilde n + \frac{4}{5} \left( \frac{8}{3} \tilde n \right)^{5/3},
    \quad
    \tilde p =  \frac{8}{15} \left( \frac{8}{3} \tilde n \right)^{5/3}.
\end{equation}
%
The leading order contribution to the energy density is the rest mass of the particles.
This term does not contribute to the pressure.
As discussed earlier, the non-relativistic limit corresponds to $k_3 \ll 1$, if we chose units so that $\tilde u \approx \tilde p$, or $\tilde u \gg \tilde p$ if we demand that $k_3 = 1$.
We see that $x_f \rightarrow 0$ corresponds to the latter case.
By including only the leading order term, we can eliminate the Fermi momentum and write the equation of state in the non-relativistic limit as $u_{\mathrm{rn}} = k p^{3/5}$ where $k = 8/3 \cdot (15/8)^{3/5}$.
Equations of state where $u \propto p^{\gamma}$ are called polytropes.
In the ultrarelativistic limit, where $p_f \gg m$, the leading order contributions to the pressure and energy density are
%
\begin{equation}
    \tilde u = 2 x_f^4, \quad \tilde p = \frac{2}{3} x_f^4, 
\end{equation}
%
and we get the particularly simple equation of state for the ultrarelativistic limit, $ u_{\mathrm{ur}} = 3 p $, which we recognize as the formula for radiation pressure.
The equation of state $\tilde u(\tilde p)$ in two different regimes is shown in \autoref{fig: equation of state fermi fluid}.
The full equation of state is compared to the non-relativistic and ultrarelativistic approximations.

\begin{figure}[h]
    \centering
    \includegraphics[width=\textwidth]{../scripts/figurer/fermi_eos.pdf}
    \caption{The equation of state of a cold Fermi gas. Both pressure and energy density is normalized to their characteristic quantities, $p_0$ and $u_0$. The equation of state is compared to the non-relativistic approximation, $\tilde u_{\mathrm{nr}}$ as well as the ultrarelativistic approximation, $\tilde u_{\mathrm{ur}}$, in two different regimes.}
    \label{fig: equation of state fermi fluid}
\end{figure}



\subsection{Units}

The equation of state has given us the characteristic energy density and pressure, $u_0$ and $p_0$. 
If we demand
%
\begin{equation}
    G \frac{m_0}{r_0} = \frac{4 \pi }{3}\frac{r_0^3 u_0}{m_0} = 1,
\end{equation}
%
we have two equations and two unknowns, $m_0$ and $r_0$.
This thus defines a complete set of units.
We are using the cold Fermi-gas as a model for a neutron star, an the mass of the fermion $m$ is therefore the neutron mass, \autoref{mass of neutron}, $m_N = 1.674 \cdot 10^{-27} \, \text{kg}$.
After reinstating $\hbar$ and $c$ in metric units, we get
%
\begin{align}
    u_0 &= p_0 = \frac{m^4}{8 \pi^2}\frac{c^5}{\hbar^3} 
    = 2.032\cdot10^{35}  \, \text{J}\,\text{m}^{-3}, \\
    m_0 &= \frac{c^4}{\sqrt{\frac{4 \pi}{3} u_0 G^3} }
    = 1.608 \cdot 10^{31} \, \text{kg}
    = 8.082 \, M_\odot \\
    r_0 &= \frac{G m_0}{c^2} = 11.93 \, \text{km}. % sjekk
\end{align}
%
From this, we expect our star to have a mass of the order of a solar mass, $M_\odot = 1.988 41\cdot 10^{30}\, \text{kg}$~\autocite{particledatagroupReviewParticlePhysics2020}, and a radius of the order of kilometers, without solving the TOV equation.



\subsection{Numerical results}

With the energy density, \autoref{Fermi gas energy density}, and pressure, \autoref{Fermi gas pressure}, we can numerically solve the TOV equation given a central pressure $p_c$. 
This is done using an adaptive Runge-Kutta method, with the stop criterion $p(r) = 0$.
Description of the code and where to find it is given in \autoref{appendix: code}.
With our choice of units, the form of the TOV equation is
%
\begin{align}
    \diff{\tilde m}{\tilde r} 
    = 3 \tilde r^2 \tilde u, \quad
    \diff{\tilde p}{\tilde r} 
     = - \frac{1}{\tilde r^2} \left(\tilde p + \tilde u\right) 
    \left(3  \tilde r^3 \tilde p + \tilde m\right) 
    \left(1 - \frac{2 \tilde m}{\tilde r}\right)^{-1}.
\end{align}
%
As $r \rightarrow 0$, parts of the TOV equation \autoref{TOV dimensionless} diverge, and we must take use an approximation for numeric evaluation.
The Taylor-expansion of the mass function around $\tilde r = 0$ is
%
\begin{equation}
    \tilde m(r) = \tilde m(0) + \tilde m'(0) \, \tilde r + \frac{1}{2!} \tilde m''(0) \tilde r^2
    + \frac{1}{3!} \tilde m'''(0) \tilde r^3 + \Oh\left(\tilde r^4\right).
\end{equation}
%
One of the boundary conditions is $\tilde m(0) = 0$.
We then use the differntial equation for $\tilde m$, \autoref{diff eq mass}, to find
%
\begin{equation}
    \tilde m'(0) = 0, \quad
    \tilde m''(0) = 0, \quad
    \tilde m'''(0) = 6 k_2 \tilde u_0,
\end{equation}
%
where $\tilde u_0 = \tilde u(r = 0)$.
We get an approximation of the TOV equation for $\tilde r \ll 1$ by substituting the $\tilde m$ for its Taylor expansion and including only the leading-order term, which gives
%
\begin{equation}
    \diff{\tilde p}{\tilde r}
    \sim - \tilde r \, \left(\tilde p + \tilde u\right)
    \left( 3 \tilde p + \tilde u_0  \right)
    \left(1 - 2 \tilde u_0 \tilde r^2\right)^{-1}, \quad r\rightarrow 0
\end{equation}
%
For the Newtonian approximation to the TOV equation, we get
%
\begin{equation}
    \diff{\tilde p}{\tilde r} = -\frac{\tilde u \tilde m}{\tilde r^2}
    \sim - \tilde u \tilde u_0 \tilde r,  \quad r\rightarrow 0.
\end{equation}


The top graph in \autoref{fig: pressure and mass as a function of radius} shows the pressure, normalized to the central pressure $p_c$, as a function of radius, normalized to the corresponding stellar radius $R$.
The stellar radius, $R$, is defined as the point where the pressure drops to zero, i.e., $p(R) = 0$.
The boundary conditions are logarithmically spaced.
The lower graph in \autoref{fig: pressure and mass as a function of radius} shows the mass, normalized to the total mass $M = m(R)$, as a function of the radius, again normalized to the stellar radius.
As in the case of the incompressible fluid, the pressure follows a half bell-shaped curve, with a peak that becomes narrower as the central pressure increases.
The black dashed line corresponds to the solution with the maximum mass.
We see that the shape of the pressure and mass plots changes most drastically for central pressures higher than that which corresponds to the most massive star.

\begin{figure}[h]
    \centering
    \includegraphics[width=\textwidth]{../scripts/figurer/pressure_mass.pdf}
    \caption{The pressure noramlized to central value (top) and the mass normalized to total mass (bottom), as a function of radius, nomralized to total radius. This is plotted for several different values of central pressure, which is indicated by the color shceme.}
    \label{fig: pressure and mass as a function of radius}
\end{figure}


In \autoref{fig: mass radius relationship fermi gas}, we see the relationship between the mass and radius.
This line is parameterized by the base-10 logarithm of the central pressure, $p(0)$, normalized by $p_0 = u_0$.
The cross marks the maximum mass, $M_\mathrm{max} = 0.711 \, M_\odot$, which corresponds to a radius of $R = 9.20 \, \mathrm{km}$.
This matches the results obtained by Oppenheimer and Volkoff~\cite{oppenheimerMassiveNeutronCores1939}, $M_\mathrm{max} = 0.71$.
In their 1939 paper, Oppenheimer and Volkoff computed five data points in the mass-radius space.
The results are marked by blue circles in \autoref{fig: mass radius relationship fermi gas}.
We find good agreement between the three points closest to the maximum value and our results.
However, the two results of Oppenheimer and Volkoff furthest away differ significantly from our results.
\todo{Hvorfor er to punkter så langt unna?}
The black dashed line is the absolute mass-radius constraint, \autoref{mass radius constraint}, and any stable configuration must be on the right side of this line.


\begin{figure}[h]
    \centering
    \includegraphics[width=\textwidth]{../scripts/figurer/mass_radius_neutron.pdf}
    \caption{The mass-radius relationship of a star made of a cold gas of neutrons. The line is parametrized by the central pressure $p(0)$. The corss indicate the maximum mass solution. The blue circles are results form the 1939 paper of Oppenheimer and Volkof~\autocite{oppenheimerMassiveNeutronCores1939}.}
    \label{fig: mass radius relationship fermi gas}
\end{figure}


In \autoref{fig: mass radius relationship comparison}, we compare the mass-radius relationship obtained from the full theory with results from approximations.
The lowest line is obtained by using both the full TOV equation and the equation of state.
The middle line is obtained using the non-relativistic equation of state together with the full TOV equation.
The upper line is from the Newtonian approximation to the TOV equation and the non-relativistic equation of state.
This last approximation does not seem to have an upper limit for the mass, as expected.


\begin{figure}[h]
    \centering
    \includegraphics[width=\textwidth]{../scripts/figurer/mass_radius_comparison.pdf}
    \caption{The mass-radius relationship of a cold gas of neutrons. The lowest line is obtained from the TOV equation and full equation of state. The middle line is from the TOV equation and the non-relativistic equation of state. The upper line is obtained from the Newtonian approximation of the TOV equation and the non-relativistic equation of state.}
    \label{fig: mass radius relationship comparison}
\end{figure}



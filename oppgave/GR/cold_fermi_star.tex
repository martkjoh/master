\section{A star of cold, non-interacting fermions}
Ekstra kilder: \autocite{andersenIntroductionStatisticalMechanics2012,glendenningCompactStarsNuclear2012}
\todo{Ha kilder på riktig sted}

\todo{Skrive om white dwarfs/neutron stars?}
A non-interacting Fermi gas is governed by the Dirac Lagrangian
%
\begin{equation}
    \Ell = \bar \psi (i \slashed \partial - m) \psi,
\end{equation}
%
as described in (APPENDIX thermal field theory).
\todo{add appendix}
In the grand canoncial ensamble, the density of a conserved charge is regulated by a chemical potential $\mu$.
The conserved current corresponding to the $\text U(1)$ phase symmetry of the Dirac Lagrangian, i.e., the transformation $\psi \rightarrow e^{i \alpha} \psi \approx 1 + \alpha \delta \psi$, is
\begin{equation}
    j^\mu = \diffp{\Ell}{(\partial_\mu \psi)} \delta \psi = \bar \psi \gamma^\mu \psi.
\end{equation}
%
The conserved charge is
%
\begin{equation}
    Q = \int \dd^3 \, j^0 = \int \dd^3 x\, \bar \psi \gamma^0 \psi,
\end{equation}
%
the number of particles minus antiparticles.
The energy density $u$ is related to the grand canonical free energy $F$ by a Legendre transformation,
%
\begin{equation}
    F(T, V, \mu) = U - T S - \mu Q, \quad \dd F = - S \dd T - p \dd V - Q \dd \mu.
\end{equation}
%
Here, $p$ is pressure, $T = {1}/{\beta}$ is temprature, and $S$ entropy.
These thermodynamic variables are related to the free energy by
%
\begin{equation}
    S = - \diffp{F}{T} = \beta^2 \diffp{F}{\beta}, \quad
    Q = - \diff{F}{\mu}, \quad
    p = - \diffp{F}{V}.
\end{equation}
%
When the free energy can be written as $F = V \Eff$, where the free energy density $\Eff$ is independent of the volume $V$, then $\Eff = - p$ and
%
\begin{equation}
    \dd (V \Eff) = V \dd \Eff - p \dd V,
\end{equation}
%
allowing us to write
%
\begin{equation}
    \Eff(T, \mu) = u - Ts - \mu n, \quad
    \dd \Eff = -s \dd T - n \dd \mu,
\end{equation}
% 
where $s$ and $n$ are entropy and charge density, defined by
%
\begin{equation}
    s = - \diffp{\Eff}{T} = \beta^2 \diffp{\Eff}{\beta}, \quad
    n = - \diffp{\Eff}{\mu}, \quad
\end{equation}
%
With this, we can write the energy density as
%
\begin{equation}
    u = \diffp{}{\beta} \left(\beta \Eff\right) + \mu n.
\end{equation}



The free energy density of the Fermion gas is (REF: appendix felt teori)
%
\begin{equation}
    \Eff = - 
    \frac{2}{\beta}\int \frac{\dd^3 p}{(2 \pi)^3} 
    \left[
        \beta \omega
        +
        \ln(1 + e^{-\beta(\omega - \mu)})
        + 
        \ln(1 + e^{-\beta(\omega + \mu)})
    \right],
\end{equation}
%
where $\omega = \sqrt{p^2 + m^2}$.
The first term in the integral is the divergent vacuum energy, which must be renormalized.
We can drop this term; it does not have any observable effects on our results, as we are interested in relative pressure and energy density.
With this, we find the charge density
%
\begin{equation}
    n = \frac{1}{\pi^8} \int \frac{\dd^3 p}{(2 \pi)^3} [n_f(\omega - \mu) - n_f(\omega + \mu)],
\end{equation}
%
where
%
\begin{equation}
    n_f(\omega) = \frac{1}{e^{\beta \omega} + 1}.
\end{equation}
%
is the Fermi distribution.
The energy density is 
%
\begin{equation}
    \label{energy density}
    u = \frac{1}{\pi^2} \int_0^\infty \dd p\, p^2 \, \omega [n_f(\omega - \mu) + n_f(\omega + \mu)].
\end{equation}
%
As expected, this is the energy per mode times the density of states, integrated over all modes.
To write the pressure, $p = - \Eff$ in terms of an integral over the Fermi distribution, we integrate by parts.
We have
%
\begin{equation}
    \int_0^\infty \dd p \, p^2 \ln\left[1 + e^{-\beta(\omega \pm m)}\right]
    = 
    \frac{1}{3} p^3\ln\left[1 + e^{-\beta(\omega \pm m)}\right] \bigg |_0^\infty
    + 
    \frac{1}{3} \int_0^\infty \dd p \, \frac{ \beta p^4}{\omega}n_f(\omega \pm \mu),
\end{equation}
%
where the boundary term vanish.
This allows us to write the pressure as 
%
\begin{equation}
    \label{pressure}
    p = \frac{2}{3} \int_0^\infty \dd p \, \frac{p^4}{\omega} [n_f(\omega - \mu) + n_f(\omega + \mu)]
\end{equation}



We are interested in the $T = 0$ limit.
In this case, the Fermi distribution becomes a step function, $n_f(\omega) = \theta(-\omega)$.
Without loss of generality, we assume that $\mu > 0$, i.e., we are dealing with an abundance of matter compared to anti-matter.
The dispersion relation $\omega = \sqrt{p^2 + m^2}$ is always positivive.
This means that the contribution to thermodynamic quantities from anti-particles vanish, as the integral is multiplied with $n_f(\omega + \mu) = \theta(-\omega - \mu)$, where the argument $-\omega - \mu$ is strictly negative on the domain of integration.
At zero temperature, the only dynamics are due to the degeneracy pressure of the fermions, that is, due to the Pauli exclusion principle.
There are no thermal fluctuations that can create a particle-antiparticle pair.
Thus, if the system has a positive chemical potential, it will contain no antiparticles.
In the zero-temperature limit, we can then rewrite any integral over the Fermi distribution as
%
\begin{equation}
    \int_0^\infty \dd p \, [f(p) n_f(\omega - \mu) + g(p) n_f(\omega - \mu)]= \int_0^{p_f} \dd p \, f(p),
\end{equation}
%
where Fermi momentum $p_f$ is defined by the chemical potential as $\mu = \sqrt{p_f^2 + m^2}$.
We thus assume the chemical potential is equal or greater than the mass $m$.
This allows us to evaluate the charge density exactly,
%
\begin{equation}
    n = \frac{1}{\pi^2} \int_0^{p_f} \dd p\, p^2 = \frac{p_f^3}{3 \pi^2}.
\end{equation}
%
At $T = 0$, this is the particle number density, as there are no antiparticles.
This density is given by the chemical potential and vanishes when $\mu \leq m$, i.e. when the free energy cost of creating a particle is positive.
We can write the energy density and pressure integrals, \autoref{energy density} and \autoref{pressure}, as
%
\begin{align}
    u &= \frac{1}{\pi^2} \int_0^{p_f} \dd p \,
    p^2 \sqrt{p^2 + m^2}
    = \frac{m^4}{\pi^2} \int_0^{x_f} \dd x \, x^2 \sqrt{x^2 + 1}, \\
    p & = \frac{1}{3 \pi^2} \int_0^{p_f} \dd p \,  \frac{p^4}{\sqrt{p^2 + m^2}} 
    = \frac{m^4}{3 \pi^2} \int_0^{x_f} \frac{\dd x \, x^4}{\sqrt{x^2 + 1}}.
\end{align}
% 
We have defined $x = p / m$ and $x_f = p_f/m$.
These integrals can be evaluated exactly as
%
\begin{align}
    \int_0^a \dd x \, x^2 \sqrt{x^2 + 1} 
    & = \frac{1}{8} 
    \left[\sqrt{a^4 + 1}(2 a^3 + a) - \arcsinh\left(a\right)\right], \\
    \int_0^a \dd x \, \frac{x^4}{\sqrt{x^2 + 1} }
    & = \frac{1}{8} 
    \left[\sqrt{a^2 + 1}(2 a^3 - 3a) + 3\arcsinh\left(a\right)\right].
\end{align}
%
We introduce the characteristic energy and number density, 
\begin{equation}
    u_0 = \frac{m^4}{8 \pi^2}, \quad n_0 = \frac{u_0}{m},
\end{equation}
which allows us to write the thermodynamic variables as
%
\begin{align}
    n &= n_0 \frac{8}{3} x_f^3 \\
    u &= u_0
    \left[(2x_f^3 + x_f) \sqrt{1 + x_f^2} - \arcsinh\left(x_f\right)\right], \\
    p &= \frac{u_0}{3}
    \left[(2x_f^3 - 3x_f) \sqrt{1 + x_f^2} + 3\arcsinh\left(x_f\right)\right],
\end{align}
%

If we demand
%
\begin{equation}
    \frac{G m_0}{r_0} = \frac{4 \pi }{3}\frac{r_0^3 u_0}{m_0} = 1,
\end{equation}
%
we have defined a complete set of units.
Inserting $\hbar$ and $c$ gives
%
\begin{align}
    u_0 &= \frac{m^4}{8 \pi^2}\frac{c^5}{\hbar^3} 
    = 2.032\cdot10^{35}  \, \text{J}\,\text{m}^{-3}, \\
    m_0 &= \frac{c^4}{\sqrt{\frac{4 \pi}{3} u_0 G^3} }
    = 1.608 \cdot 10^{31} \, \text{kg}
    = 8.082 \, M_\odot \\
    r_0 &= \frac{G m_0}{c^2} = 11.93 \, \text{km}. % sjekk
\end{align}
%
From this, we can expect our star to have a mass of the order of a solar mass, $M_\odot = 1.988 41\cdot 10^{30}\, \text{kg}$~\autocite{particledatagroupReviewParticlePhysics2020}, and a radius of the order of kilometers, without solving the TOV equation.


\subsection*{Limits}

In the non-relativistic limit, as the chemical potential approaches $m$ and thus $p_f \ll m$, the lowest order contributions to the energy density and pressure are given by the Taylor series around $x_f = 0$,
%
\begin{align}
    \tilde u(x_f) &= \frac{8}{3}x_f^3 + \frac{4}{5} x_f^5 + \Oh(x_f^7),  \\
    \tilde p(x_f) &= \frac{8}{5}x_f^5 + \Oh(x_f^7).
\end{align}
%
By neglecting terms of order $x_f^7$ and higher, we can write this as
%
\begin{equation}
    \tilde u = \tilde n + \frac{4}{5} \left( \frac{8}{3} \tilde n \right)^{5/3},
    \quad
    \tilde p =  \frac{8}{5} \left( \frac{8}{3} \tilde n \right)^{5/3}.
\end{equation}
%
The leading order contribution to the energy density is the rest mass of the particles.
This term does not contribute to the pressure.
By including only the leading order term, we can eliminate the Fermi momentum and write the equation of state as $u = k p^{3/5}$ for some constant $i$.
Equations of state where $u \propto p^{\gamma}$ are called polytropes.
In the ultrarelativistic limit, where $p_f \gg m$, the leading order contributions to the pressure and energy density are
%
\begin{equation}
    \tilde u = 2 x_f^4, \quad \tilde p = \frac{2}{3} x_f^4, 
\end{equation}
%
and we get the particularly simple equation of state $u = 3 p$, which we recognize as the formula for radiation pressure.

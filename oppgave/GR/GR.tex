General relativity describes how matter and energy curve the fabric of space and time.
Einstein first wrote down the theory more than a century ago, and it is still our most accurate theory of gravitational effects.
It makes precise and counterintuitive predictions, which experiments have borne out.
This chapter surveys the basics of general relativity.
We will then use this to derive the Tolman-Oppenheimer-Volkoff (TOV) equation, a differential equation used to model stars.
We start by summarizing the theory it succeeded.


\section{Newtonian Gravity}

This section is based on~\autocite{carrollSpacetimeGeometryIntroduction2019}.


Newton's famous law of gravity states that the force of gravity from an object of mass $M$ acting on another object of mass $m$ is proportional to both objects' masses and is inversely proportional to the distance between them squared.
This force is directed radially inwards.
Formulating this as an equation, with $\vv r$ as the vector from the object with mass $M$ to that with mass $m$ with length $|\vv r| = r$, gives
%
\begin{equation}
    \vv F_g = - \frac{G M m}{r^2} \hat r.
\end{equation}
%
Here, $G$ is Newton's gravitation constant, and $\hat r = \vv r / r$.
The vector $\vv r$ is a purely spatial three-vector, as space is separated from time in the Newtonian picture.
The law of gravitation, together with Newton's second law of motion
%
\begin{equation}
    \label{Newton's second law}
    \sum_i \vv F_i = m \vv a,
\end{equation}
%
where $\vv a$ is the acceleration of an object and $\vv F_i$ the forces acting upon it, can account for the motion of stellar objects.
These laws work well in all but the most extreme circumstances, involving very massive objects or very short distances.
As we will see, such extreme circumstances can be quantified by $2G M \approx r$.
Newtonian gravity works well as an approximation because $G$ is small in everyday units.
Highly precise measurements of the orbit of Mercury were needed before any deviation from it was noticed.

We restate Newtonian gravity in a field-theoretic language to better compare Newtonian gravity to its successor theory.
The gravitational potential from a mass $M$, which gives rise to Newton's force law, obeys the equation
%
\begin{equation}
    \nabla^2 \Phi_g = - 4 \pi G \rho.
\end{equation}
%
Here, $\rho$ is the mass density.
For a single point mass, $\rho(\vv r) = M \delta(\vv r)$, this has the solution $\Phi_g = -G M / r$.
The acceleration due to gravity is then
%
\begin{equation}
    \label{acceleration from Newtonian gravitational potential}
    \vv a = - \vec \nabla \Phi_g.
\end{equation}
%
We see that mass acts as a gravitational charge.
Due to the success of Newtonian gravity, we expect it to be a limit of any theory that succeeds it.
This gives us the ability to theoretically test any new theory of gravity, as well as to connect parameters in the new theory to old, known quantities.



\section{General relativity}
This section is based on~\cite{carrollSpacetimeGeometryIntroduction2019,glendenningCompactStarsNuclear2012}.

The derivation of the spherically symmetric metric is done using computer code, as described in \autoref{appendix: code}.



\subsection{Einstein's field equations}

General relativity describes spacetime as a smooth manifold $\Em$, with a (pseudo-Riemannian) metric, $g_{\mu \nu}$.
This metric is treated as a dynamical field, which is affected by the presence of matter and energy.
The matter and energy contents of spacetime are encoded in the stress-energy tensor $T_{\mu \nu}$, while the behavior of $g^{\mu \nu}$ is encoded in a scalar Lagrangian density.
We employ the minimal-coupling rule to reformulate laws from special relativity to curved spacetime.
This rule states that laws written in an inertial frame in a coordinate-independent way remain true in curved spacetime.
In an inertial frame, the Crisoffel-symbols vanish, so $\nabla_\mu = \partial_\mu$.
We can thus write any laws containing partial derivatives in a coordinate impenitent way by exchanging them for covariant derivatives.
To generalize Newton's second law \autoref{Newton's second law}, we must first make it relativistic by introducing a 4-force, $F^\mu = \odv{}{\tau} p^\mu $, where $p^\mu$ is the 4-momentum.
When applying the minimal coupling rule, Newton's second law then becomes \autoref{goedesic equation},~\autocite{hartleGravityIntroductionEinstein2021}
%
\begin{equation}
    \label{covariant version of second law}
    \sum_i F_i^\mu 
    = m \left(
        \odv[2]{x^\mu}{\tau} + \Gamma^\mu_{\nu \rho} \odv{x^\nu}{\tau} \odv{x^\rho}{\tau}
    \right).
\end{equation}
%
In the absence of any external forces, objects will follow geodesics in spacetime.
With this, we must now find the law governing $g^{\mu\nu}$.
As this is a field, we will do this by assigning it a Lagrangian density.
The most obvious---and correct---choice as the Lagrangian is the Ricci scalar, which results in the Einstein-Hilbert action,
%
\begin{equation}
    \label{Einstein-Hilbert action}
    S_{\text{EH}} = \frac{1}{2\kappa}\int_{\Em} \dd^n x \, \sqrt{|g|}\, R.
\end{equation}
%
The $\sqrt{|g|}$-factor is included for the integral to be coordinate-independent, as discussed in  \autoref{subsection: integration on manifolds}.\footnote{The gravitational action can also include a cosmological constant, modifying the Lagrangian to $R + 2 \Lambda$. This constant does not affect the subject of this thesis and is therefore not included here.}
The $\kappa$ is a constant and decides how strong the coupling of gravity to matter and energy is.
This constant can then be related to Newton's constant of gravitation $G$ by $\kappa = 8 \pi G$.
When including the contributions from other fields with an action $S_\text{m}$, the total action becomes 
%
\begin{equation}
    S = S_\text{EH} + S_\text{m}.
\end{equation}
%
The equations of motion of the dynamical field, which in this case is the metric, are given by Hamilton's principle of stationary action.
Using functional derivatives, as defined in \autoref{subsection: functional calculus on a curved manifold}, this is stated as
%
\begin{equation}
    \fdv{S}{g^{\mu\nu}} = 0.
\end{equation}
%
We define the stress-energy tensor as
%
\begin{equation}
    \label{definition stress energy densor}
    T_{\mu\nu} = - 2 \fdv{S_\text{m}}{g^{\mu \nu}}.
\end{equation}
%
The functional derivative of the Einstein-Hilbert action is evaluated in \autoref{subsection: functional derivative of the einstein-hilbert action}, and with the result, \autoref{functional derivatie einstein-hilber action}, we get the Einstein field equations
%
\begin{equation}
    \label{Einstein field equations}
    R_{\mu \nu} - \frac{1}{2} R g_{\mu \nu} = \kappa T_{\mu \nu}.
\end{equation}
%
The left-hand side of the Einstein field equations is called the Einstein tensor, $G_{\mu \nu} = R_{\mu \nu} - \frac{1}{2} R g_{\mu \nu}$. This tensor obeys the identity
%
\begin{equation}
    \label{Einstein tensor bianchi identity}
    \nabla^\mu G_{\mu \nu} = 0,
\end{equation}
%
as a consequence of the more general Bianchi identity, \autoref{Binachi identiy}.


\subsection{Spherically symmetric spacetime}

To model stars, we will assume that the metric is spherically symmetric and time-independent.
In this case, the most general metric can be written, at least locally, as~\autocite{carrollSpacetimeGeometryIntroduction2019}
%
\begin{equation}
    \dd s^2 
    = e^{2\alpha(r)} \dd t^2 - e^{2 \beta(r)} \dd r^2 - 
    r^2 (\dd \theta^2 + \sin^2 \theta \, \dd \varphi^2),
\end{equation}
%
where $\alpha$ and $\beta$ are general functions of the radial coordinate $r$.
In matrix form, this corresponds to 
%
\begin{equation}
    \label{spherically symmetric metric}
    g_{\mu \nu} =
    \left(
        \begin{matrix}
            e^{2 \alpha{\left(r \right)}} & 0 & 0 & 0\\
            0 & - e^{2 \beta{\left(r \right)}} & 0 & 0
            \\0 & 0 & - r^{2} & 0
            \\0 & 0 & 0 & - r^{2} \sin^{2}{\left(\theta \right)}
        \end{matrix}
     \right).
\end{equation}
%
Using \autoref{christoffel symbols from metric}, we can now compute the Christoffel symbols in terms of the unknown functions.
These computations in this subsection are done using computer code, which is shown in \autoref{appendix: code}.
The results are
%
\begin{align}
    \Gamma^t_{\mu \nu}
    & =
    \left(
        \begin{matrix}
            0 & \odv{}{r} \alpha{\left(r \right)} & 0 & 0\\\odv{}{r} \alpha{\left(r \right)} & 0 & 0 & 0\\0 & 0 & 0 & 0\\0 & 0 & 0 & 0
        \end{matrix}
    \right), \\
    \Gamma^r_{\mu \nu}
    &=
    \left(
        \begin{matrix}
            e^{2 \alpha{\left(r \right)}} e^{- 2 \beta{\left(r \right)}} \odv{}{r} \alpha{\left(r \right)} & 0 & 0 & 0\\0 & \odv{}{r} \beta{\left(r \right)} & 0 & 0\\0 & 0 & - r e^{- 2 \beta{\left(r \right)}} & 0\\0 & 0 & 0 & - r e^{- 2 \beta{\left(r \right)}} \sin^{2}{\left(\theta \right)}
        \end{matrix}
     \right), \\
     \Gamma^\theta_{\mu \nu} 
     & =
     \left(
         \begin{matrix}
            0 & 0 & 0 & 0\\0 & 0 & \frac{1}{r} & 0\\0 & \frac{1}{r} & 0 & 0\\0 & 0 & 0 & - \sin{\left(\theta \right)} \cos{\left(\theta \right)}
        \end{matrix}
    \right), \\
    \Gamma^\phi_{\mu \nu} 
    &=
    \left(
        \begin{matrix}
            0 & 0 & 0 & 0\\0 & 0 & 0 & \frac{1}{r}\\0 & 0 & 0 & \frac{\cos{\left(\theta \right)}}{\sin{\left(\theta \right)}}\\0 & \frac{1}{r} & \frac{\cos{\left(\theta \right)}}{\sin{\left(\theta \right)}} & 0
        \end{matrix}
    \right).
\end{align}
%
The symbols not included are zero.
Substituting these results into \autoref{riemann tensor in terms of christoffel symbols} gives the Riemann curvature tensor.
We can then obtain the Ricci tensor by taking the trace, as shown in \autoref{Ricci tensor}.
The results are
%
\begin{align}
    \label{tt component ricci tensor}
    R_{tt}
    & =
    \left[
        r \left(\odv{}{r} \alpha{\left(r \right)}\right)^{2} 
        - r \odv{}{r} \alpha{\left(r \right)} \odv{}{r} \beta{\left(r \right)} 
        + r \odv[2]{}{r} \alpha{\left(r \right)} 
        + 2 \odv{}{r} \alpha{\left(r \right)}
    \right]
    \frac{
         e^{2 \alpha{\left(r \right)}} e^{- 2 \beta{\left(r \right)}}}{r}, \\
    \label{rr component ricci tensor}
    R_{rr}
    & =
    - \frac{1}{r}
    \left[
        r \left(\odv{}{r} \alpha{\left(r \right)}\right)^{2} 
        - r \odv{}{r} \alpha{\left(r \right)} \odv{}{r} \beta{\left(r \right)} 
        + r \odv[2]{}{r} \alpha{\left(r \right)} - 2 \odv{}{r} \beta{\left(r \right)} 
    \right],\\
    \label{thetatheta component ricci tensor}
    R_{\theta \theta}
    &=
    -  
    \left[
        r \odv{}{r} \alpha{\left(r \right)} - r \odv{}{r} \beta{\left(r \right)} - e^{2 \beta{\left(r \right)}} + 1
    \right]
        e^{- 2 \beta{\left(r \right)}}, \\
    R_{\varphi \varphi} & = R_{\theta \theta} \sin^2( \theta).
\end{align}
%
All other components are zero.
Finally, the trace of the Ricci tensor gives the Ricci scalar,
%
\begin{align}
    \nonumber
    R =
    \, \frac{2 e^{- 2 \beta{\left(r \right)}}}{r^{2}}
    \bigg[
        &
        r^{2} \left(\odv{}{r} \alpha{\left(r \right)}\right)^{2} 
        - r^{2} \odv{}{r} \alpha{\left(r \right)} \odv{}{r} \beta{\left(r \right)}
        \\ &
        + r^{2} \frac{d^{2}}{d r^{2}} \alpha{\left(r \right)} 
        + 2 r \odv{}{r} \alpha{\left(r \right)} 
        - 2 r \odv{}{r} \beta{\left(r \right)} - e^{2 \beta{\left(r \right)}} + 1
    \bigg].
\end{align}
%
 The unknown functions $\alpha$ and $\beta$ are now determined by the matter and energy content of the universe, which is encoded in $T_{\mu \nu}$, through Einstein's field equation, \autoref{Einstein field equations}. 



\subsection{The Schwarzschild metric}


The simples case for a matter distribution in spacetime is $T_{\mu \nu} = 0$.
Although this might only seem to be useful to model a non-empty universe, it can be combined with a central point particle and empty space elsewhere.
In this case, the Einstein equations are simply $R_{\mu \nu} - \frac{1}{2}R g_{\mu \nu} = 0$.
We can show that the trace of the Ricci tensor is zero by taking the trace of this equation, simplifying it to $R_{\mu \nu} = 0$.
By combining \autoref{tt component ricci tensor} and \autoref{rr component ricci tensor}, we find
%
\begin{equation}
    R_{tt} + e^{2(\alpha - \beta)}R_{rr} = 2 \odv{}{r} \left(\alpha + \beta\right) = 0,
\end{equation}
%
which implies $\alpha = -\beta + \const$
The constant corresponds to rescaling the coordinates, which allows us to set it to zero.
From \autoref{thetatheta component ricci tensor}, we get
%
\begin{equation}
    e^{2 \beta} R_{\theta \theta} = - 2 r\odv{}{r} \alpha - e^{-2\alpha} + 1 = 0,
\end{equation}
%
which may be restated as
%
\begin{equation}
    \odv{}{r} \left(r e^{2 \alpha}\right) = 1.
\end{equation}
%
This equation has the solution
%
\begin{equation}
    e^{2\alpha(r)} = e^{-2 \beta(r)} = \left( 1- \frac{R_s}{r} \right),
\end{equation}
%
where $R_s$, the Schwarzschild radius, is a constant.
With this, we should obtain Newton's law of gravity with a small perturbation, $g_{\mu \nu} = \eta_{\mu \nu} + h_{\mu \nu}$, and at slow speeds, $\odv*{x^i}{\tau} \ll \odv*{t}{\tau}$.
Inserting this into the geodesic equation \autoref{covariant version of second law} with $F_i = 0$, using $\partial_0 g_{\mu \nu} = 0$ and expanding to leading order, we get
%
\begin{equation}
    \odv[2]{x^i}{t} = \frac{1}{2}\eta^{i j}\partial_j h_{00}.
\end{equation}
%
We now obtain Newtonian gravity, as formulated in \autoref{acceleration from Newtonian gravitational potential}, if we identify $h_{00} = 1 -  e^{2\alpha} = - 2 \Phi_g$.
The Schwarzschild radius is thus $R_s = 2 G M$, and we see that this solution corresponds to a point-mass $M$ located at $\vv r = 0$.
We will extend our discussion to finite densities in the next section.
Inserting our results into the metric, we get the Schwarzschild metric
%
\begin{equation}
    \label{Schwarzchild metric}
    \dd s^2 
    = 
    \left( 1 - \frac{2 G M}{r} \right) \dd t^2
    -\left( 1 - \frac{2 G M}{r} \right)^{-1} \dd r^2
    - r^2 \left(\dd \theta^2 + \sin^2 \theta \dd \varphi^2\right).
\end{equation}
%






\section{The TOV equation}

We will model a star as being made up of a \emph{perfect fulid}, with energy density $\rho$ and pressure $p$.
The relationship between the pressure and energy density of a substance is called the \emph{equation of state}, or EOS, and has the form
\begin{equation}
    \label{EOS}
    f(p, \rho, \{\xi_i\}) = 0,
\end{equation}
where $\{\xi_i\}$ are possible other thermodynamic variables.
We will be working at zero temperature, in which case there are no other free thermodynamic variables.
This allows us to, at least locally, express the pressure as a function of the energy density, $p = p(\rho)$.
The stress-energy tensor of a perfect fluid is\todo{Forklar}
%
\begin{equation}
    T_{\mu \nu} = (\rho + p) u_\mu u_\nu - p g_{\mu \nu},
\end{equation} 
where $u_\mu$ is the 4-velocity of the fluid.
In the rest frame of the fluid, we may write 
\begin{equation}
    u_\mu = \left(u_0, 0, 0, 0\right).
\end{equation}
This, together with the normalization condition of 4-velocities, $u_\mu u^\mu = 1$, allows us to calculate that
%
\begin{equation}
    u_\mu u^\mu = g^{\mu \nu} u_\mu u_\nu = g^{00} (u_0)^2 = 1.
\end{equation}
%
Using \autoref{spherically symmetric metric}, we see that
\begin{equation}
    u_0 = e^{\alpha(r)}.
\end{equation}
%
This gives us the stress-energy tensor of the perfect fluid in its rest frame,
%
\begin{equation}
    T_{\mu \nu} 
    =
    \left(
        \begin{matrix}
            \rho{\left(r \right)} e^{2 \alpha{\left(r \right)}} & 0 & 0 & 0\\0 & 
            p{\left(r \right)} e^{2 \beta{\left(r \right)}} & 0 & 0\\
            0 & 0 & p{\left(r \right)} r^{2} & 0\\
            0 & 0 & 0 & p{\left(r \right)} r^{2} \sin^{2}{\left(\theta \right)}
        \end{matrix}
    \right).
\end{equation}
%
We will use the $tt$ and $rr$ components of the Einstein field equations, which are
%
\begin{align}
    \label{tt equation}
    8 \pi G r^{2} \rho{\left(r \right)} e^{2 \beta{\left(r \right)}} 
    & =   2 r \frac{d}{d r} \beta{\left(r \right)} + e^{2 \beta{\left(r \right)}} - 1 \\
    \label{rr equation}
    8 \pi G r^{2} p{\left(r \right)} e^{2 \beta{\left(r \right)}} 
    & = 2 r \frac{d}{d r} \alpha{\left(r \right)} - e^{2 \beta{\left(r \right)}} + 1.
\end{align}
%
In analogy with the Schwartzhild metmric, we define the function $m(r)$ by
\begin{equation}
    e^{2 \beta(r)} = \left(1 - \frac{2 G m(r)}{r} \right)^{-1}. 
\end{equation}
Substituting this into \autoref{tt equation} yields 
\begin{equation}
    \diff{m(r)}{r} = 4 \pi r^2 \, \rho(r).
\end{equation}
The solution is simply
\begin{equation}
    \label{mass relation}
    m(r) = 4 \pi \int_0^r \dd r' \, {r'}^2 \rho(r').
\end{equation}
We see that $m(r)$ is the matter content contained within a radius $r$.
If $\rho = 0$ for $r > R$ and $m(r>R) = M$, then the metric on a constant-time surface, i.e. $\dd t = 0$, is
%
\begin{equation}
    \dd s^2
    = 
    \left( 1 - \frac{2 G M}{r^2} \right)^{-1} \dd r^2 
    + r^2 (\dd \theta^2 + \sin^2 \theta \, \dd\varphi^2),
\end{equation} 
%
which is the same as the for the Schwartzhild solution.

Using the Bianchi identity, \autoref{Einstein tensor bianchi identity}, together with EInstein's equation, we find that
%
\begin{equation}
    \nabla^\mu G_{\mu \nu} = \nabla^\mu T_{\mu \nu} = 0.
\end{equation}
%
The $r$-component of this equation is
%
\begin{equation}
    \nabla_\mu T^{\mu r} 
    =
    \partial_r T^{rr} 
    + \Gamma^\mu_{\mu \nu} T^{\nu r} 
    + \Gamma^r_{\mu \nu} T^{\mu \nu},
\end{equation}
%
where we have used the particular form of $T_{\mu \nu}$ and the Christoffel symbols to eliminate vanishing terms.
We calculate
%
\begin{align*}
    \nabla_\mu T^{\mu r} 
    & = 
    \partial_r \left(p e^{-2\beta}\right)
    + (2 \Gamma^r_{rr} + \Gamma^t_{tr}) T^{rr} 
    + \Gamma^r_{tt}T^{tt} \\ 
    &=   e^{-2\beta} \left( \partial_r p + p \partial_r \alpha + \rho \partial_r \alpha \right) = 0.
\end{align*} 
%
This allows us to relate $\alpha$ to $p$ and $\rho$, via
\begin{equation}
    \partial_r \alpha = - \frac{\partial_r p}{p + \rho}
\end{equation}
When we substitute this, together with the definition of $m(r)$, into \autoref{rr equation}, we obtain
\begin{equation}
    \label{TOV}
    \diff{p}{r}
    =
    -
    \frac{
        G \left(4 \pi r^{3} p + m \right) \left(\rho + p \right)
        }
        {r \left(r- 2 G m\right)},
\end{equation}
the Tolman-Oppenheimer-Volkoff (TOV) equation.

To summarize, we have three functions describeing the star, $\rho$, $p$, and $m$, as well as thre equations, \autoref{EOS}, \autoref{mass relation} and \autoref{TOV}.
Together with inital conditions, such as $p(0) = p_0$, this is enough to determin the unknown funcions.
For completness, the equations are
\begin{gather*} 
    f(p, \rho) = 0, \\
    m(r) = 4 \pi \int_0^r \dd r \, {r'}^2 \rho(r'),\\
    \diff{p}{r} =
    -
    \frac{G \left(4 \pi r^{3} p + m \right) \left(\rho + p \right)}{r \left(r- 2 G m\right)}.
\end{gather*}

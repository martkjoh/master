\documentclass{book}

\usepackage[left=2.5cm, right=2.5cm, top=2cm, bottom=2cm]{geometry}

\usepackage{tikz-cd}
\usepackage[compat=1.1.0]{tikz-feynman}


\tikzfeynmanset{ Fermion/.style = {
   decoration={
     markings,
     mark=at position 0.5
          with {\arrow[xshift=.7mm, scale=1.2]{latex}}
     },
   postaction=decorate
   }
}


% % math mode additions
\usepackage{amsmath}
\usepackage{amsfonts}
\usepackage{amssymb}
\usepackage{mathtools} 

% \usepackage{physics}
\usepackage{braket}
% nice derivatives
\usepackage{derivative}
% I should not have to do this
\DeclareOdvVariant{\odv}{d}[style-inf=\mathrm]
\usepackage{diffcoeff}
\renewcommand{\fdv}[2]{\diff.delta.{#1}{#2}}

\usepackage{slashed}

% nice 3-vectors
\usepackage{esvect}

% appendix
\usepackage[title]{appendix}
\usepackage{pdfpages}
 
\usepackage[export]{adjustbox}

% in-document references
\usepackage{hyperref}
% This should have always been used...
\usepackage[capitalize]{cleveref}

% Nice table of contents
\usepackage{tocloft}

% Get floats right
\usepackage[section]{placeins}
\usepackage{float}

% adjust fig captions
\usepackage{caption}
\usepackage{subcaption}
\captionsetup{width=.9\textwidth}

\usepackage{titlepic}
\usepackage[prependcaption,textsize=footnotesize]{todonotes}

\usepackage{setspace}

% Quote in introduction
\usepackage{epigraph} 

\usepackage[%
    style=numeric-comp,
    sorting=none,
    sortcites=true,
    doi=true,
    url=false,
    giveninits=true,
    hyperref
    ]{biblatex}

\setlength{\marginparwidth}{2.2cm}
% \setlength\cftparskip{1pt}
% \setlength\cftbeforechapskip{0pt}

% space instead of indentation for paragraphs

\def\equationautorefname~#1\null{Eq.~(#1)\null}
\renewcommand{\chapterautorefname}{Chapter}

\listfiles

% Chiral pertubation theory
\newcommand{\chpt}[0]{\ensuremath{\chi \text{PT}}} 
% Lagrangian L
\newcommand{\Ell}{\mathcal{L}}
% Hamiltonian H
\newcommand{\He}{\mathcal{H}}
% Potential V
\newcommand{\Ve}{\mathcal{V}}
% effective potential
\newcommand{\Veff}{\mathcal{V}_{\mathrm{eff}}}
% Manifold M
\newcommand{\Em}{\mathcal{M}}
% Fancy f
\newcommand{\Eff}{\mathcal{F}}
\newcommand{\Oh}{\mathcal O}
% Real numbers
\newcommand{\R}{\mathbb{R}}
% Funcitonal integral D
\newcommand{\D}{\mathcal D}
\newcommand{\id}{\mathrm{id}}
\newcommand{\const}{\mathrm{const.}}
% Identity matice/operation
\newcommand{\one}{\text{\usefont{U}{bbold}{m}{n}1}}
\MakeRobust{\one}
\newcommand{\hc}{\text{h.c.}}
% Differential d
\newcommand{\dd}{\text{d}}

%! Fix these
\newcommand{\Olie}{\text O}
\newcommand{\SO}{\text{SO}}

\newcommand{\Lie}[2]{\text{#1}\left(#2\right)}
\newcommand{\lie}[2]{\mathfrak{#1}\left(#2\right)}
\newcommand{\curly}[1]{\left\{ #1 \right\}}

% operator in braket
\newcommand{\inner}[3]{\left\langle #1 {\left| #2 \right|} #3 \right\rangle}
% expectationvalue
\newcommand{\ex}[1]{\left\langle #1 \right\rangle}

% Set builder notation
\newcommand{\setbuilder}[2]{\left\{\, #1 \mid #2 \,\right\}}
% Time ordering operator
\newcommand{\T}[1]{\textrm{T} \left\{ #1 \right\}}

% Using diffcoeff instead
% \newcommand{\}{\fdv}
 

% Floor function
\DeclarePairedDelimiter\floor{\lfloor}{\rfloor}
\DeclareMathOperator{\arcsinh}{arcsinh}

\newcommand{\Tr}[1]{ \text{Tr} \left\{ #1 \right\}}
\let\exp\relax %undefine exp
\newcommand{\exp}[1]{ \text{exp} \left\{ #1 \right\}}
\newcommand{\abs}[1]{\left| #1 \right|}


\newcommand{\pio}{{\pi^0}}
\newcommand{\pipm}{{\pi^\pm}}
\newcommand{\Ko}{{K^0}}
\newcommand{\Kpm}{{K^\pm}}

\DeclareMathOperator{\sgn}{sgn}


\title{\huge{Gravgård}}
\author{
    \large{Martin Kjøllesdal Johnsrud }
    }


\bibliography{master}


\begin{document}

\maketitle
\setlength{\parindent}{0em}
\setlength{\parskip}{0.8em}


Alt som ikke kom med...

\section{Tree level pion star}


We introduce the new dimensionless variable $1 + x^2 = \mu_I^2 / \bar m^2$.
This is reminicent of the dimensionless Fermi momentum $x_f = p_f / m$ in \{section: cold fermi star\}.
By an argument using a right triangle, we can vertify that $\cos a = b$ implies $\sin a = \sqrt{1 - b^2}$.
Substituting the dimensionless variable into the free energy density, we get
%
\begin{equation}
    \Eff = - \frac{u_0}{2}\left(1 + x^2 +\frac{1}{1 + x^2}\right).
\end{equation}
%
We have introduced the characteristic energy density $u_0 = \bar m^2 f^2$.
As we found in \{section: cold fermi star\}, pressure is given by negative the free energy density, normalized to $\mu_I = \bar m$, or $x = 0$.
We choose $p_0 = u_0$, so the dimensionless pressure can be written
%
\begin{equation}
    \tilde p = -\frac{1}{u_0}(\Eff - \Eff_{x = 0}) 
    = \frac{1}{2} \frac{x^4}{1 + x^2}.
\end{equation}
%

The charge density corresponding to a chemical potential is given by minus the derivative of the free energy with respect to that chemical potential.
We must, however, not assume any dependence of $\alpha$ on $\mu_I$.
The isospin density therefore is
%
\begin{equation}
    n_I = -\pdv{\Eff}{\mu_I} = f^2 \mu_I \sin^2 \alpha
    = \frac{u_0}{\mu_I} \frac{2x^2 + x^4}{1 + x^2}.
    ,
\end{equation}
%
With this, the dimensionless energy density
%
\begin{equation}
    \tilde u = -\tilde p + \frac{1}{u_0} n_I \mu_I
    = \frac{1}{2} \frac{4x^2 + x^4}{1 + x^4}
\end{equation}
% 


\section{Feil analyse av three-flavor vakum}

%
\begin{align}
    \frac{1}{4} \Tr{\nabla_\mu \Sigma_\alpha \nabla^\mu \Sigma_\alpha^\dagger}
    & = -\frac{1}{2} \sin^2 \alpha
    \left\{
        \mu_I^2 \left[n_1^2 + n_2^2 + \frac{1}{4}(n_6^2 + n_7^2)\right]
        + \frac{1}{4} [\mu_I^2 + 3 \mu_8^2][n_4^2 + n_5^2] 
    \right\} \\
    \frac{1}{4} \Tr{\chi \Sigma^\dagger + \Sigma \chi^\dagger} 
    & = M_1^2 \cos \alpha
\end{align}


\begin{align}
    \He
    &= -\frac{1}{8} \sin^2 \alpha 
    \left[
        \mu_I^2 \left(4a^2 + b^2 + c^2\right) 
        + 3 \mu_8^2 (b^2 + c^2)
        + 2 \sqrt{3} \mu_8 \mu_I (b^2 - c^2)
    \right]
    + M_1^2 \cos\alpha\\
    &= -\frac{1}{8} \sin^2\alpha
    \left[
        4 \mu_I^2 a^2  + (\mu_I + \sqrt 3 \mu_8 )^2b^2 + (\mu_I - \sqrt 3 \mu_8)^2c^2  
    \right]
\end{align}



Choose, without loss of generality, $n_1 = n_4 = 0$, which leaves $n_2 = \cos\beta$, $n_5 = \sin\beta$, and thus
%
\begin{equation}
    \Sigma = \exp{i \alpha (\cos\beta \lambda_2 + \sin\beta \lambda_5) }
    = \cos\alpha + i (\lambda_2 \cos\beta + \lambda_5 \sin\beta)\sin\alpha.
\end{equation}
%


\section{Feil three-flavor chpt beregning}

We need to parametrize the ground state, as we did in {subsection: parametrization}, and define
Let
%
\begin{equation}
    \Sigma_\alpha 
    = \exp{i \alpha n_a \lambda_a} = \cos \alpha + i n_a \lambda_a \sin \alpha,
    \quad \alpha = \frac{1}{f} \sqrt{\pi_a^0 \pi_a^0}, \quad n_a = \frac{\pi_a^0}{\sqrt{\pi_b^0 \pi_b^0}}. 
\end{equation}
%
The relevant terms are then
%
\begin{align}
    \frac{1}{4} \Tr{\nabla_\mu \Sigma_\alpha \nabla^\mu \Sigma_\alpha^\dagger}
    & = \frac{1}{2} \sin^2\alpha
    \left[
        \mu_I^2 (n_1^2 + n_2^2) 
        + \frac{1}{4} (\mu_I + 2\mu_s)^2(n_4^2 + n_5^2)
        + \frac{1}{4} (\mu_I - 2\mu_s)^2(n_6^2 + n_7^2)
    \right]\\
    \frac{1}{4} \Tr{\chi \Sigma^\dagger + \Sigma \chi^\dagger} 
    & = M_1^2 \cos \alpha
\end{align}
%
We notice that both terms are independent of $\mu_B$.
With this, the static Hamiltonian is
%
\begin{align}
    \He_0
    = -\frac{1}{2}f^2\sin^2\alpha
    \left[
        \mu_I^2a^2 + \mu_\Kpm^2 b^2  + \mu_\Ko^2 c^2
    \right]
    - f^2M_1^2 \cos\alpha
\end{align}
%
We have defined the chemical potentials $\mu_{K^{\pm}} = \frac{1}{2}(\mu_I + 2 \mu_S) = \mu_u - \mu_s$ and $\mu_{K^{\pm}} = \frac{1}{2}(\mu_I - 2 \mu_S) = -\mu_d + \mu_s$, 
and
%
\begin{equation}
    a^2 = n_1^2 + n_2^2, \quad
    b^2 = n_4^2 + n_5^2, \quad
    c^2 = n_6^2 + n_7^2, \quad
    a^2 + b^2 + c^2 = 1 - n_3^2 - n_8^2.
\end{equation}
%
All the terms with a square chemical potential factors are positive definite, which means that the Hamiltonian will always be minimized by $n_3 = n_8 = 0$.
Furhtermore, we can without loss of generality chose $n_1 = n_4 = n_6 = 0$.
This corresponds to changing basis of $\lie{su}{3}$.
Depending on the signs of $\mu_I$ and $\mu_S$, we must have either $b = 0$ or $c = 0$
If $\sgn(\mu_I) = \sgn(\mu_S)$, then $\mu_\Kpm > \mu_\Ko$, and $c = 0$.
Likewise, if $\sgn(\mu_I) = - \sgn(\mu_S)$, then $\mu_\Kpm < \mu_\Ko$, and $b = 0$.
To begin with, we assume the former.
Define $a^2 = \cos^2\beta$, which implies $b^2 = \sin^2\beta$.
The Hamiltonian density is then
%
\begin{equation}
    \He_0 =
    -\frac{1}{2} f^2
    \left[
        \mu_I^2 \cos^2\beta + \mu_\Kpm^2\sin^2\beta
    \right]
    \sin^2\alpha
    -
    f^2 M_1^2 \cos\alpha.
\end{equation}


The $\beta$ parameter is set, as $\alpha$, by minimizing $\He$.
We have
%
\begin{equation}
    \pdv{\He}{\beta} = \frac{1}{2} (\mu_I^2 - \mu_\Kpm^2) f^2 \sin^2\alpha\cos2\beta, \quad
    \pdv[2]{\He}{\beta} = f^2 (\mu_I^2 - \mu_\Kpm^2) \sin^2\alpha\sin2\beta.
\end{equation}
%
We see that, if we are in the pion condensate phase where $\alpha \neq 0$, the stationary points for $\beta$ are $0$ and $\pi/2$.
However, which one these that is a minimum depends on the sign of $\mu_I^2 - \mu_\Kpm^2$, as this determines the sign of the second derivative.
For $\mu_I^2 > \mu_\Kpm^2$, $\beta = 0$, while for $\mu_I^2 < \mu_\Kpm^2$ we have $\beta = \pi/2$.
The analysis for $\sgn(\mu_I) = -\sgn(\mu_S)$ is the same, only with $\mu_\Kpm$ changed to $\mu_\Ko$.
The different ground states are then
%
\begin{equation}
    \Sigma_0 = \one, \quad
    \Sigma_\pi = \exp{i\alpha \lambda_2}, \quad
    \Sigma_\Kpm = \exp{i\alpha \lambda_5}, \quad
    \Sigma_\Ko = \exp{i\alpha \lambda_7}.
\end{equation}
%
As we found for two flavors, this corresponds to a transformation of the vacuum to a new ground state by, $\Sigma_0 \rightarrow A_\alpha \Sigma_0 A_\alpha$.
We must therefore transform the excitations around ground state in the same way.
However, now the transformation depend on which phase we are in.
We therefore parametrize the fields as
%
\begin{align}
    \Sigma = A^i_\alpha [U(x) \Sigma_0 U(x)] A_\alpha^i, \quad
    U(x) = \exp{i \frac{\pi_a \lambda_a}{2 f}}, \quad
    A_\alpha^i = \exp{i \alpha \lambda_i}.
\end{align}
%
Here, there is no sum over $i$.
Rather, $i = 2$, $5$, or $7$, dependent on if we are in the pion condensate, the charged kaon condensate or neutral kaon condensate.


\subsection{failed Leading order}


We work in the pion condensate, with $e = 0$.
The relevant terms are then
%
\begin{align}
    \nonumber
    \frac{f^2}{8B_0}\Tr{\chi \Sigma^\dagger + \Sigma \chi^\dagger}
    & =
    - \frac{1}{4}(m_u + m_d)\cos\alpha (\pi_1^2 + \pi_2^2 + \pi_3^2)
    - \frac{1}{4} 
    \left[
        (m_u + m_s)\cos^2\frac{\alpha}{2} - m_d \sin^2\frac{\alpha}{2}
    \right](\pi_4^2 + \pi_5^2)\\ \nonumber
    &- 
    \left[
        (m_d + m_s)\cos^2\frac{\alpha}{2} - m_u \sin^2\frac{\alpha}{2}
    \right](\pi_6^2 + \pi_7^2) 
    + \frac{1}{12} 
    \left[
        (m_u + m_d + 2m_s) \cos\alpha + 2m_s
    \right] \pi_8^2 \\
    &-\frac{1}{2 \sqrt{3}} (m_u - m_d) \pi_3 \pi_8
    - \frac{1}{2}(m_u + m_d)\sin\alpha \pi_2
    + \frac{1}{2}(m_u + m_d)\cos\alpha + \frac{1}{4}m_s(\cos\alpha + 1)
\end{align}

\section{Two-flavor electromagnetic effecets}


When including contribution from a dynamical photon field, the leading order Lagrangian is~\autocite{eckerRoleResonancesChiral1989,urechVirtualPhotonsChiral1995}
%
\begin{equation}
    \label{leading order lagrangian EM}
    \Ell_2^{\text{EM}}
    = 
    \frac{1}{4}f^2 
    \Tr{
        \nabla_\mu \Sigma \nabla^\mu \Sigma^\dagger
    }
    +
    \frac{1}{4}f^2 
    \Tr{
        \chi \Sigma^\dagger + \Sigma\chi^\dagger
    }
    +
    e^2 C
    \Tr{Q \Sigma Q \Sigma^\dagger}
\end{equation}
%
$Q$ is the quark charge matrix, which for $N_f = 2$ is
%
\begin{equation}
    \label{two-flavor charge matrix}
    Q
    =
    \frac{1}{3} 
    \begin{pmatrix}
        2 & 0 \\
        0 & -1
    \end{pmatrix}
    = 
    \frac{1}{2} \one + \frac{1}{6} \tau_3.
\end{equation}
%
$C$ and dimensionfull constant, and $\chi = 2B_0 m$, where $m$ is the quark mass matrix \autoref{two-flavor mass matrix}.
To find the electromagnetic effect on the pion mass, we assume $\mu_I = 0$.
We use the parametrization $\Sigma = \exp{i \pi_a \tau_a / f}$, and the covariant derivative is in this case
%
\begin{equation}
    \nabla_\mu \Sigma = \partial_\mu \Sigma - i e \mathcal A_\mu [Q, \Sigma].
\end{equation}
%
We expand to second order in $\pi_a/f$, which gives
%
\begin{align}
    \frac{1}{4}f^2 \Tr{\nabla_\mu \Sigma \nabla^\mu \Sigma}
    &=
    \frac{1}{2}\partial_\mu \pi_a\partial^\mu \pi_a
    + e \mathcal A^\mu (\pi_1 \partial_\mu \pi_2 -\pi_2 \partial_\mu \pi_1)
    + e^2 \mathcal A^2 (\pi_1^2 + \pi_2^2), \\
    \frac{1}{4}f^2 \Tr{\chi \Sigma^\dagger + \Sigma \chi^\dagger}
    & = \bar m^2\left(f^2 - \frac{1}{2} \pi_a \pi_a\right), \\
    \Tr{Q \Sigma Q \Sigma^\dagger}
    & = \frac{5}{9} - \frac{\pi_1^2 + \pi_2^2}{f^2}.
\end{align}
%
Inserting this into \autoref{leading order lagrangian EM}, we get
%
\begin{equation}
    \Ell_2^\text{EM}
    = \bar m^2 f^2 + \frac{5}{9}e^2 C
    + \frac{1}{2}\partial_\mu \pi_a \partial^\mu \pi_a
    - \frac{1}{2} \bar m_\pm^2 (\pi_1^2 + \pi_2^2) 
    - \frac{1}{2}\bar m^2 \pi_3^2
    + e \mathcal A^\mu (\pi_1 \partial_\mu \pi_2 -\pi_2 \partial_\mu \pi_1)
    + e^2 \mathcal A^2 (\pi_1^2 + \pi_2^2).
\end{equation}
%
where
\begin{equation}
    \bar m_\pm^2 = \bar m^2 + 2\frac{e^2}{f^2}C.
\end{equation}
%
This is the leading order electromagnetic contribution to the mass.
It only affects the $\pi_1, \pi_2$ pions, as they are linear combinations of the charged pions $\pi^\pm$, while $\pi_3 = \pi^0$, the neutral pion.
To leading order, $\bar m = m_\pi$, the neutral pion mass, and $\bar m_{\pm} = m_{\pi^{\pm}}$
From the values listed in \autoref{section: units}, we find
%
\begin{equation}
    \label{EM mass contribtuion leading order}
    \Delta m_{\pm} := \frac{e}{f}\sqrt{2C} = \sqrt{m_{\pi_\pm}^2 - m_{\pi}^2} = 35.50 \, \text{MeV}.
\end{equation}
%
This corresponds to $C = 0.3771 \, u_0 = 5.824 \cdot 10^{-5} \, \text{GeV}^4$.
\todo[]{Compare/use Urec's results: C = 61.1 × 10 −6 (GeV)${}^4$}
We now no longer assume $\mu_I = 0$.
The zeroth-order expansion in $\pi/f$ is
%
\begin{equation}
    \Sigma = e^{i \alpha \tau_1} = \sin \alpha + i \tau_1 \cos \alpha.
\end{equation}
%
This gives the contributions
%
\begin{align}
    \Tr{\nabla_\mu \Sigma \nabla^\mu \Sigma^\dagger}
    & = 2 \sin^2\alpha\left(\mu_I^2 + 2 e \mu \mathcal A_0 + e^2\mathcal A^2 \right), \\
    \Tr{\chi \Sigma^\dagger + \Sigma \chi^\dagger}
    & =4 \bar m^2 \cos \alpha,\\
    \Tr{Q \Sigma Q \Sigma^\dagger}
    & =  \cos^2 \alpha - \frac{4}{9}.
\end{align}
%
We are interested in the static Lagrangian, the Lagrangian for $\pi_a = \mathcal A_\mu = 0$.
Inserting these terms into \autoref{leading order lagrangian EM}, we get
%
\begin{equation}
    \label{static lagrangian with EM}
    \Ell^{\text{EM}, 0}_2
    = f^2 \left[
        \frac{1}{2}\mu_I^2 \sin^2 \alpha + \bar m^2 \cos \alpha 
        + \frac{1}{2} \Delta m^2_{\pi_\pm} \left(\cos^2 \alpha - \frac{4}{9}\right)
    \right].
\end{equation}
%




\end{document}

\documentclass{book}

\usepackage[left=2.5cm, right=2.5cm, top=2cm, bottom=2cm]{geometry}

\usepackage{tikz-cd}
\usepackage[compat=1.1.0]{tikz-feynman}


\tikzfeynmanset{ Fermion/.style = {
   decoration={
     markings,
     mark=at position 0.5
          with {\arrow[xshift=.7mm, scale=1.2]{latex}}
     },
   postaction=decorate
   }
}


% % math mode additions
\usepackage{amsmath}
\usepackage{amsfonts}
\usepackage{amssymb}
\usepackage{mathtools} 

% \usepackage{physics}
\usepackage{braket}
% nice derivatives
\usepackage{derivative}
% I should not have to do this
\DeclareOdvVariant{\odv}{d}[style-inf=\mathrm]
\usepackage{diffcoeff}
\renewcommand{\fdv}[2]{\diff.delta.{#1}{#2}}

\usepackage{slashed}

% nice 3-vectors
\usepackage{esvect}

% appendix
\usepackage[title]{appendix}
\usepackage{pdfpages}
 
\usepackage[export]{adjustbox}

% in-document references
\usepackage{hyperref}
% This should have always been used...
\usepackage[capitalize]{cleveref}

% Nice table of contents
\usepackage{tocloft}

% Get floats right
\usepackage[section]{placeins}
\usepackage{float}

% adjust fig captions
\usepackage{caption}
\usepackage{subcaption}
\captionsetup{width=.9\textwidth}

\usepackage{titlepic}
\usepackage[prependcaption,textsize=footnotesize]{todonotes}

\usepackage{setspace}

% Quote in introduction
\usepackage{epigraph} 

\usepackage[%
    style=numeric-comp,
    sorting=none,
    sortcites=true,
    doi=true,
    url=false,
    giveninits=true,
    hyperref
    ]{biblatex}

\setlength{\marginparwidth}{2.2cm}
% \setlength\cftparskip{1pt}
% \setlength\cftbeforechapskip{0pt}

% space instead of indentation for paragraphs

\def\equationautorefname~#1\null{Eq.~(#1)\null}
\renewcommand{\chapterautorefname}{Chapter}

\listfiles

% Chiral pertubation theory
\newcommand{\chpt}[0]{\ensuremath{\chi \text{PT}}} 
% Lagrangian L
\newcommand{\Ell}{\mathcal{L}}
% Hamiltonian H
\newcommand{\He}{\mathcal{H}}
% Potential V
\newcommand{\Ve}{\mathcal{V}}
% effective potential
\newcommand{\Veff}{\mathcal{V}_{\mathrm{eff}}}
% Manifold M
\newcommand{\Em}{\mathcal{M}}
% Fancy f
\newcommand{\Eff}{\mathcal{F}}
\newcommand{\Oh}{\mathcal O}
% Real numbers
\newcommand{\R}{\mathbb{R}}
% Funcitonal integral D
\newcommand{\D}{\mathcal D}
\newcommand{\id}{\mathrm{id}}
\newcommand{\const}{\mathrm{const.}}
% Identity matice/operation
\newcommand{\one}{\text{\usefont{U}{bbold}{m}{n}1}}
\MakeRobust{\one}
\newcommand{\hc}{\text{h.c.}}
% Differential d
\newcommand{\dd}{\text{d}}

%! Fix these
\newcommand{\Olie}{\text O}
\newcommand{\SO}{\text{SO}}

\newcommand{\Lie}[2]{\text{#1}\left(#2\right)}
\newcommand{\lie}[2]{\mathfrak{#1}\left(#2\right)}
\newcommand{\curly}[1]{\left\{ #1 \right\}}

% operator in braket
\newcommand{\inner}[3]{\left\langle #1 {\left| #2 \right|} #3 \right\rangle}
% expectationvalue
\newcommand{\ex}[1]{\left\langle #1 \right\rangle}

% Set builder notation
\newcommand{\setbuilder}[2]{\left\{\, #1 \mid #2 \,\right\}}
% Time ordering operator
\newcommand{\T}[1]{\textrm{T} \left\{ #1 \right\}}

% Using diffcoeff instead
% \newcommand{\}{\fdv}
 

% Floor function
\DeclarePairedDelimiter\floor{\lfloor}{\rfloor}
\DeclareMathOperator{\arcsinh}{arcsinh}

\newcommand{\Tr}[1]{ \text{Tr} \left\{ #1 \right\}}
\let\exp\relax %undefine exp
\newcommand{\exp}[1]{ \text{exp} \left\{ #1 \right\}}
\newcommand{\abs}[1]{\left| #1 \right|}


\newcommand{\pio}{{\pi^0}}
\newcommand{\pipm}{{\pi^\pm}}
\newcommand{\Ko}{{K^0}}
\newcommand{\Kpm}{{K^\pm}}

\DeclareMathOperator{\sgn}{sgn}


\title{\huge{Gravgård}}
\author{
    \large{Martin Kjøllesdal Johnsrud }
    }


\bibliography{master}


\begin{document}

\maketitle
\setlength{\parindent}{0em}
\setlength{\parskip}{0.8em}


Alt som ikke kom med...

\section{Tree level pion star}


We introduce the new dimensionless variable $1 + x^2 = \mu_I^2 / \bar m^2$.
This is reminicent of the dimensionless Fermi momentum $x_f = p_f / m$ in \{section: cold fermi star\}.
By an argument using a right triangle, we can vertify that $\cos a = b$ implies $\sin a = \sqrt{1 - b^2}$.
Substituting the dimensionless variable into the free energy density, we get
%
\begin{equation}
    \Eff = - \frac{u_0}{2}\left(1 + x^2 +\frac{1}{1 + x^2}\right).
\end{equation}
%
We have introduced the characteristic energy density $u_0 = \bar m^2 f^2$.
As we found in \{section: cold fermi star\}, pressure is given by negative the free energy density, normalized to $\mu_I = \bar m$, or $x = 0$.
We choose $p_0 = u_0$, so the dimensionless pressure can be written
%
\begin{equation}
    \tilde p = -\frac{1}{u_0}(\Eff - \Eff_{x = 0}) 
    = \frac{1}{2} \frac{x^4}{1 + x^2}.
\end{equation}
%

The charge density corresponding to a chemical potential is given by minus the derivative of the free energy with respect to that chemical potential.
We must, however, not assume any dependence of $\alpha$ on $\mu_I$.
The isospin density therefore is
%
\begin{equation}
    n_I = -\pdv{\Eff}{\mu_I} = f^2 \mu_I \sin^2 \alpha
    = \frac{u_0}{\mu_I} \frac{2x^2 + x^4}{1 + x^2}.
    ,
\end{equation}
%
With this, the dimensionless energy density
%
\begin{equation}
    \tilde u = -\tilde p + \frac{1}{u_0} n_I \mu_I
    = \frac{1}{2} \frac{4x^2 + x^4}{1 + x^4}
\end{equation}
% 


\end{document}
